\documentclass[12pt]{article}
\usepackage{amsmath}
\usepackage{amssymb}
\usepackage{graphicx}
\usepackage{enumitem}
\usepackage{hyperref}
\usepackage{xcolor}

\title{View problem}
\author{USACO Contest: JAN22 Contest - Silver Division}
\date{\today}

\begin{document}

\maketitle

\section*{Problem Statement}

Farmer John's cows like nothing more than cereal for breakfast!  In fact, the
cows have such large appetites that they will each eat an entire box of cereal
for a single meal.

The farm has recently received a shipment with $M$ different types of cereal
$(2\le M\le 10^5)$. Unfortunately, there is only one box of each cereal!  Each
of the $N$ cows $(1\le N\le 10^5)$ has a favorite cereal and a second favorite
cereal. When given a selection of cereals to choose from, a cow performs the
following process:

If the box of her favorite cereal is still available, take it and
leave.Otherwise, if the box of her second-favorite cereal is still available, 
take it and leave.Otherwise, she will moo with disappointment and leave without taking any
cereal.
Find the minimum number of cows that go hungry if you permute them optimally.
Also, find any permutation of the $N$ cows that achieves this minimum.

INPUT FORMAT (input arrives from the terminal / stdin):
The first line contains two space-separated integers $N$ and $M.$

For each $1\le i\le N,$ the $i$-th line contains two space-separated integers
$f_i$ and $s_i$ ($1\le f_i,s_i\le M$ and $f_i\neq s_i$) denoting the favorite
and second-favorite cereals of the $i$-th cow.

OUTPUT FORMAT (print output to the terminal / stdout):
Print the minimum number of cows that go hungry, followed by any permutation of
$1\ldots N$ that achieves this minimum. If there are multiple permutations, any
one will be accepted.

SAMPLE INPUT:
8 10
2 1
3 4
2 3
6 5
7 8
6 7
7 5
5 8
SAMPLE OUTPUT: 
1
1
3
2
8
4
6
5
7

In this example, there are $8$ cows and $10$ types of cereal. 

Note that we can effectively solve for the first three cows independently of 
the last five, since they share no favorite cereals in common.

If the first three cows choose in the order $[1,2,3]$, then cow $1$ will choose
cereal $2$, cow $2$ will choose cereal $3$, and cow $3$ will go hungry. 

If the first three cows choose in the order $[1,3,2]$, then cow $1$ will choose
cereal $2$, cow $3$ will choose cereal $3$, and cow $2$ will choose cereal $4$;
none of these cows will go hungry.

Of course, there are other permutations that result in none of the first three
cows going hungry. For example, if the first three cows choose in the order
$[3,1,2]$ then cow $3$ will choose cereal $2$, cow $1$ will choose cereal $1$,
and cow $2$ will choose cereal $3$; again, none of cows $[1,2,3]$ will go
hungry.

It can be shown that out of the last five cows, at least one must go hungry.

SCORING:
In $4$ out of $14$ test cases, $N,M\le 100$.In $10$ out of $14$ test cases, no additional constraints.


Problem credits: Dhruv Rohatgi



\section*{Input Format}
For each $1\le i\le N,$ the $i$-th line contains two space-separated integers
$f_i$ and $s_i$ ($1\le f_i,s_i\le M$ and $f_i\neq s_i$) denoting the favorite
and second-favorite cereals of the $i$-th cow.

OUTPUT FORMAT (print output to the terminal / stdout):Print the minimum number of cows that go hungry, followed by any permutation of
$1\ldots N$ that achieves this minimum. If there are multiple permutations, any
one will be accepted.SAMPLE INPUT:8 10
2 1
3 4
2 3
6 5
7 8
6 7
7 5
5 8SAMPLE OUTPUT:1
1
3
2
8
4
6
5
7In this example, there are $8$ cows and $10$ types of cereal.Note that we can effectively solve for the first three cows independently of 
the last five, since they share no favorite cereals in common.If the first three cows choose in the order $[1,2,3]$, then cow $1$ will choose
cereal $2$, cow $2$ will choose cereal $3$, and cow $3$ will go hungry.If the first three cows choose in the order $[1,3,2]$, then cow $1$ will choose
cereal $2$, cow $3$ will choose cereal $3$, and cow $2$ will choose cereal $4$;
none of these cows will go hungry.Of course, there are other permutations that result in none of the first three
cows going hungry. For example, if the first three cows choose in the order
$[3,1,2]$ then cow $3$ will choose cereal $2$, cow $1$ will choose cereal $1$,
and cow $2$ will choose cereal $3$; again, none of cows $[1,2,3]$ will go
hungry.It can be shown that out of the last five cows, at least one must go hungry.SCORING:In $4$ out of $14$ test cases, $N,M\le 100$.In $10$ out of $14$ test cases, no additional constraints.Problem credits: Dhruv Rohatgi

\section*{Output Format}
SAMPLE INPUT:8 10
2 1
3 4
2 3
6 5
7 8
6 7
7 5
5 8SAMPLE OUTPUT:1
1
3
2
8
4
6
5
7In this example, there are $8$ cows and $10$ types of cereal.Note that we can effectively solve for the first three cows independently of 
the last five, since they share no favorite cereals in common.If the first three cows choose in the order $[1,2,3]$, then cow $1$ will choose
cereal $2$, cow $2$ will choose cereal $3$, and cow $3$ will go hungry.If the first three cows choose in the order $[1,3,2]$, then cow $1$ will choose
cereal $2$, cow $3$ will choose cereal $3$, and cow $2$ will choose cereal $4$;
none of these cows will go hungry.Of course, there are other permutations that result in none of the first three
cows going hungry. For example, if the first three cows choose in the order
$[3,1,2]$ then cow $3$ will choose cereal $2$, cow $1$ will choose cereal $1$,
and cow $2$ will choose cereal $3$; again, none of cows $[1,2,3]$ will go
hungry.It can be shown that out of the last five cows, at least one must go hungry.SCORING:In $4$ out of $14$ test cases, $N,M\le 100$.In $10$ out of $14$ test cases, no additional constraints.Problem credits: Dhruv Rohatgi

\section*{Sample Input}
\begin{verbatim}
8 10
2 1
3 4
2 3
6 5
7 8
6 7
7 5
5 8
\end{verbatim}

\section*{Sample Output}
\begin{verbatim}
1
1
3
2
8
4
6
5
7

In this example, there are $8$ cows and $10$ types of cereal.Note that we can effectively solve for the first three cows independently of 
the last five, since they share no favorite cereals in common.If the first three cows choose in the order $[1,2,3]$, then cow $1$ will choose
cereal $2$, cow $2$ will choose cereal $3$, and cow $3$ will go hungry.If the first three cows choose in the order $[1,3,2]$, then cow $1$ will choose
cereal $2$, cow $3$ will choose cereal $3$, and cow $2$ will choose cereal $4$;
none of these cows will go hungry.Of course, there are other permutations that result in none of the first three
cows going hungry. For example, if the first three cows choose in the order
$[3,1,2]$ then cow $3$ will choose cereal $2$, cow $1$ will choose cereal $1$,
and cow $2$ will choose cereal $3$; again, none of cows $[1,2,3]$ will go
hungry.It can be shown that out of the last five cows, at least one must go hungry.SCORING:In $4$ out of $14$ test cases, $N,M\le 100$.In $10$ out of $14$ test cases, no additional constraints.Problem credits: Dhruv Rohatgi

Note that we can effectively solve for the first three cows independently of 
the last five, since they share no favorite cereals in common.If the first three cows choose in the order $[1,2,3]$, then cow $1$ will choose
cereal $2$, cow $2$ will choose cereal $3$, and cow $3$ will go hungry.If the first three cows choose in the order $[1,3,2]$, then cow $1$ will choose
cereal $2$, cow $3$ will choose cereal $3$, and cow $2$ will choose cereal $4$;
none of these cows will go hungry.Of course, there are other permutations that result in none of the first three
cows going hungry. For example, if the first three cows choose in the order
$[3,1,2]$ then cow $3$ will choose cereal $2$, cow $1$ will choose cereal $1$,
and cow $2$ will choose cereal $3$; again, none of cows $[1,2,3]$ will go
hungry.It can be shown that out of the last five cows, at least one must go hungry.SCORING:In $4$ out of $14$ test cases, $N,M\le 100$.In $10$ out of $14$ test cases, no additional constraints.Problem credits: Dhruv Rohatgi

If the first three cows choose in the order $[1,2,3]$, then cow $1$ will choose
cereal $2$, cow $2$ will choose cereal $3$, and cow $3$ will go hungry.If the first three cows choose in the order $[1,3,2]$, then cow $1$ will choose
cereal $2$, cow $3$ will choose cereal $3$, and cow $2$ will choose cereal $4$;
none of these cows will go hungry.Of course, there are other permutations that result in none of the first three
cows going hungry. For example, if the first three cows choose in the order
$[3,1,2]$ then cow $3$ will choose cereal $2$, cow $1$ will choose cereal $1$,
and cow $2$ will choose cereal $3$; again, none of cows $[1,2,3]$ will go
hungry.It can be shown that out of the last five cows, at least one must go hungry.SCORING:In $4$ out of $14$ test cases, $N,M\le 100$.In $10$ out of $14$ test cases, no additional constraints.Problem credits: Dhruv Rohatgi

If the first three cows choose in the order $[1,3,2]$, then cow $1$ will choose
cereal $2$, cow $3$ will choose cereal $3$, and cow $2$ will choose cereal $4$;
none of these cows will go hungry.Of course, there are other permutations that result in none of the first three
cows going hungry. For example, if the first three cows choose in the order
$[3,1,2]$ then cow $3$ will choose cereal $2$, cow $1$ will choose cereal $1$,
and cow $2$ will choose cereal $3$; again, none of cows $[1,2,3]$ will go
hungry.It can be shown that out of the last five cows, at least one must go hungry.SCORING:In $4$ out of $14$ test cases, $N,M\le 100$.In $10$ out of $14$ test cases, no additional constraints.Problem credits: Dhruv Rohatgi

Of course, there are other permutations that result in none of the first three
cows going hungry. For example, if the first three cows choose in the order
$[3,1,2]$ then cow $3$ will choose cereal $2$, cow $1$ will choose cereal $1$,
and cow $2$ will choose cereal $3$; again, none of cows $[1,2,3]$ will go
hungry.It can be shown that out of the last five cows, at least one must go hungry.SCORING:In $4$ out of $14$ test cases, $N,M\le 100$.In $10$ out of $14$ test cases, no additional constraints.Problem credits: Dhruv Rohatgi

It can be shown that out of the last five cows, at least one must go hungry.SCORING:In $4$ out of $14$ test cases, $N,M\le 100$.In $10$ out of $14$ test cases, no additional constraints.Problem credits: Dhruv Rohatgi

SCORING:In $4$ out of $14$ test cases, $N,M\le 100$.In $10$ out of $14$ test cases, no additional constraints.Problem credits: Dhruv Rohatgi
\end{verbatim}

\section*{Solution}


\section*{Problem URL}
https://usaco.org/index.php?page=viewproblem2&cpid=1184

\section*{Source}
USACO JAN22 Contest, Silver Division, Problem ID: 1184

\end{document}
