\documentclass[12pt]{article}
\usepackage{amsmath}
\usepackage{amssymb}
\usepackage{graphicx}
\usepackage{enumitem}
\usepackage{hyperref}
\usepackage{xcolor}

\title{View problem}
\author{USACO Contest: FEB22 Contest - Silver Division}
\date{\today}

\begin{document}

\maketitle

\section*{Problem Statement}

Bessie is learning how to control a robot she has recently received as a gift.

The robot begins at point $(0, 0)$ on the coordinate plane and Bessie wants the
robot to end at point $(x_g, y_g)$. Bessie initially has a list of $N$
($1\le N\le 40$)  instructions to give to the robot, the $i$-th of which will
move the robot $x_i$ units right and $y_i$ units up (or left or down when $x_i$
and $y_i$ are negative, respectively).

For each $K$ from $1$ to $N$, help Bessie count the number of ways she can
select $K$ instructions from the original $N$ such that after the $K$
instructions are executed, the robot will end at point $(x_g, y_g)$.

**Note: the time and memory limits for this problem are 4s and 512MB, twice
the defaults.**
INPUT FORMAT (input arrives from the terminal / stdin):
The first line contains $N$. The next line contains $x_g$ and $y_g$, each in the
range $-10^9 \ldots 10^9$. The final $N$ lines describe the instructions. Each
line has two integers $x_i$ and $y_i$, also in the range $-10^9 \ldots 10^9$. 

It is guaranteed that $(x_g,y_g)\neq (0,0)$ and $(x_i,y_i)\neq (0,0)$ for all
$i$.

OUTPUT FORMAT (print output to the terminal / stdout):
Print $N$ lines, the number of ways Bessie can select $K$ instructions from the
original $N$ for each $K$ from $1$ to $N$.

SAMPLE INPUT:
7
5 10
-2 0
3 0
4 0
5 0
0 10
0 -10
0 10
SAMPLE OUTPUT: 
0
2
0
3
0
1
0

In this example, there are six ways Bessie can select the instructions:


(-2,0) (3,0) (4,0) (0,10) (0,-10) (0,10) (1 2 3 5 6 7)
(-2,0) (3,0) (4,0) (0,10) (1 2 3 5)
(-2,0) (3,0) (4,0) (0,10) (1 2 3 7)
(5,0) (0,10) (0,-10) (0,10) (4 5 6 7)
(5,0) (0,10) (4 5)
(5,0) (0,10) (4 7)

For the first way, the robot's path looks as follows:


(0,0) -> (-2,0) -> (1,0) -> (5,0) -> (5,10) -> (5,0) -> (5,10)


SCORING:
Test cases 2-4 satisfy $N\le 20$.Test cases 5-16 satisfy no additional constraints.


Problem credits: Alex Liang



\section*{Input Format}
It is guaranteed that $(x_g,y_g)\neq (0,0)$ and $(x_i,y_i)\neq (0,0)$ for all
$i$.

OUTPUT FORMAT (print output to the terminal / stdout):Print $N$ lines, the number of ways Bessie can select $K$ instructions from the
original $N$ for each $K$ from $1$ to $N$.SAMPLE INPUT:7
5 10
-2 0
3 0
4 0
5 0
0 10
0 -10
0 10SAMPLE OUTPUT:0
2
0
3
0
1
0In this example, there are six ways Bessie can select the instructions:(-2,0) (3,0) (4,0) (0,10) (0,-10) (0,10) (1 2 3 5 6 7)
(-2,0) (3,0) (4,0) (0,10) (1 2 3 5)
(-2,0) (3,0) (4,0) (0,10) (1 2 3 7)
(5,0) (0,10) (0,-10) (0,10) (4 5 6 7)
(5,0) (0,10) (4 5)
(5,0) (0,10) (4 7)For the first way, the robot's path looks as follows:(0,0) -> (-2,0) -> (1,0) -> (5,0) -> (5,10) -> (5,0) -> (5,10)SCORING:Test cases 2-4 satisfy $N\le 20$.Test cases 5-16 satisfy no additional constraints.Problem credits: Alex Liang

\section*{Output Format}
SAMPLE INPUT:7
5 10
-2 0
3 0
4 0
5 0
0 10
0 -10
0 10SAMPLE OUTPUT:0
2
0
3
0
1
0In this example, there are six ways Bessie can select the instructions:(-2,0) (3,0) (4,0) (0,10) (0,-10) (0,10) (1 2 3 5 6 7)
(-2,0) (3,0) (4,0) (0,10) (1 2 3 5)
(-2,0) (3,0) (4,0) (0,10) (1 2 3 7)
(5,0) (0,10) (0,-10) (0,10) (4 5 6 7)
(5,0) (0,10) (4 5)
(5,0) (0,10) (4 7)For the first way, the robot's path looks as follows:(0,0) -> (-2,0) -> (1,0) -> (5,0) -> (5,10) -> (5,0) -> (5,10)SCORING:Test cases 2-4 satisfy $N\le 20$.Test cases 5-16 satisfy no additional constraints.Problem credits: Alex Liang

\section*{Sample Input}
\begin{verbatim}
7
5 10
-2 0
3 0
4 0
5 0
0 10
0 -10
0 10
\end{verbatim}

\section*{Sample Output}
\begin{verbatim}
0
2
0
3
0
1
0

In this example, there are six ways Bessie can select the instructions:(-2,0) (3,0) (4,0) (0,10) (0,-10) (0,10) (1 2 3 5 6 7)
(-2,0) (3,0) (4,0) (0,10) (1 2 3 5)
(-2,0) (3,0) (4,0) (0,10) (1 2 3 7)
(5,0) (0,10) (0,-10) (0,10) (4 5 6 7)
(5,0) (0,10) (4 5)
(5,0) (0,10) (4 7)For the first way, the robot's path looks as follows:(0,0) -> (-2,0) -> (1,0) -> (5,0) -> (5,10) -> (5,0) -> (5,10)SCORING:Test cases 2-4 satisfy $N\le 20$.Test cases 5-16 satisfy no additional constraints.Problem credits: Alex Liang

(-2,0) (3,0) (4,0) (0,10) (0,-10) (0,10) (1 2 3 5 6 7)
(-2,0) (3,0) (4,0) (0,10) (1 2 3 5)
(-2,0) (3,0) (4,0) (0,10) (1 2 3 7)
(5,0) (0,10) (0,-10) (0,10) (4 5 6 7)
(5,0) (0,10) (4 5)
(5,0) (0,10) (4 7)For the first way, the robot's path looks as follows:(0,0) -> (-2,0) -> (1,0) -> (5,0) -> (5,10) -> (5,0) -> (5,10)SCORING:Test cases 2-4 satisfy $N\le 20$.Test cases 5-16 satisfy no additional constraints.Problem credits: Alex Liang

(-2,0) (3,0) (4,0) (0,10) (0,-10) (0,10) (1 2 3 5 6 7)
(-2,0) (3,0) (4,0) (0,10) (1 2 3 5)
(-2,0) (3,0) (4,0) (0,10) (1 2 3 7)
(5,0) (0,10) (0,-10) (0,10) (4 5 6 7)
(5,0) (0,10) (4 5)
(5,0) (0,10) (4 7)

For the first way, the robot's path looks as follows:(0,0) -> (-2,0) -> (1,0) -> (5,0) -> (5,10) -> (5,0) -> (5,10)SCORING:Test cases 2-4 satisfy $N\le 20$.Test cases 5-16 satisfy no additional constraints.Problem credits: Alex Liang

(0,0) -> (-2,0) -> (1,0) -> (5,0) -> (5,10) -> (5,0) -> (5,10)SCORING:Test cases 2-4 satisfy $N\le 20$.Test cases 5-16 satisfy no additional constraints.Problem credits: Alex Liang

(0,0) -> (-2,0) -> (1,0) -> (5,0) -> (5,10) -> (5,0) -> (5,10)

SCORING:Test cases 2-4 satisfy $N\le 20$.Test cases 5-16 satisfy no additional constraints.Problem credits: Alex Liang

SCORING:Test cases 2-4 satisfy $N\le 20$.Test cases 5-16 satisfy no additional constraints.Problem credits: Alex Liang
\end{verbatim}

\section*{Solution}


\section*{Problem URL}
https://usaco.org/index.php?page=viewproblem2&cpid=1207

\section*{Source}
USACO FEB22 Contest, Silver Division, Problem ID: 1207

\end{document}
