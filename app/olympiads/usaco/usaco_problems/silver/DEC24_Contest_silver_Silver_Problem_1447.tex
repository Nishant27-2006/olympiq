\documentclass[12pt]{article}
\usepackage{amsmath}
\usepackage{amssymb}
\usepackage{graphicx}
\usepackage{enumitem}
\usepackage{hyperref}
\usepackage{xcolor}

\title{View problem}
\author{USACO Contest: DEC24 Contest - Silver Division}
\date{\today}

\begin{document}

\maketitle

\section*{Problem Statement}


Farmer John is expanding his farm! He has identified the perfect location in
the Red-Black Forest, which consists of $N$ trees ($1 \le N \le 10^5$) on a
number line, with the $i$-th tree at position $x_i$
($-10^9 \le x_i \le 10^9$).

Environmental protection laws restrict which trees Farmer John can
cut down to make space for his farm. There are $K$ restrictions ($1
\leq K \leq 10^5$) specifying that there must always be at least $t_i$
trees ($1 \leq t_i \leq N$) in the line segment $[l_i, r_i]$,
including the endpoints ($-10^9 \le l_i \leq r_i \le 10^9$). It is
guaranteed that the Red-Black Forest initially satisfies these
restrictions.

Farmer John wants to make his farm as big as possible. Please help him compute
the maximum number of trees he can cut down while still satisfying all the
restrictions!

INPUT FORMAT (input arrives from the terminal / stdin):
Each input consists of $T$ ($1 \le T \le 10$) independent test cases. It is
guaranteed that the sums of all $N$ and of all $K$ within an input each do not
exceed $3 \cdot 10^5$.

The first line of input contains $T$. Each test case is then formatted as
follows:

The first line contains integers $N$ and $K$.The next line
contains the $N$ integers $x_1, \dots, x_N$.Each of the next $K$ lines
contains three space-separated integers: $l_i$, $r_i$ and $t_i$.

OUTPUT FORMAT (print output to the terminal / stdout):
For each test case, output a single line with an integer denoting the maximum
number of trees Farmer John can cut down.


SAMPLE INPUT:
3
7 1
8 4 10 1 2 6 7
2 9 3
7 2
8 4 10 1 2 6 7
2 9 3
1 10 1
7 2
8 4 10 1 2 6 7
2 9 3
1 10 4
SAMPLE OUTPUT: 
4
4
3

For the first test case, Farmer John can cut down the first $4$ trees, leaving
trees at $x_i = 2, 6, 7$ to satisfy the restriction.

For the second test case, the additional restriction does not affect which trees
Farmer John can cut down, so he can cut down the same trees and satisfy both
restrictions.

For the third test case, Farmer John can only cut down at most $3$ trees because
there are initially $7$ trees but the second restriction requires him to leave
at least $4$ trees uncut.

SCORING:
Input 2: $N, K \le 16$Inputs 3-5: $N, K \le 1000$Inputs 6-7: $t_i = 1$ for all $i = 1, \dots, K$.Inputs 8-11: No additional constraints.


Problem credits: Tina Wang, Jiahe Lu, Benjamin Qi



\section*{Input Format}
Each input consists of $T$ ($1 \le T \le 10$) independent test cases. It is
guaranteed that the sums of all $N$ and of all $K$ within an input each do not
exceed $3 \cdot 10^5$.The first line of input contains $T$. Each test case is then formatted as
follows:The first line contains integers $N$ and $K$.The next line
contains the $N$ integers $x_1, \dots, x_N$.Each of the next $K$ lines
contains three space-separated integers: $l_i$, $r_i$ and $t_i$.

The first line of input contains $T$. Each test case is then formatted as
follows:The first line contains integers $N$ and $K$.The next line
contains the $N$ integers $x_1, \dots, x_N$.Each of the next $K$ lines
contains three space-separated integers: $l_i$, $r_i$ and $t_i$.

The first line contains integers $N$ and $K$.The next line
contains the $N$ integers $x_1, \dots, x_N$.Each of the next $K$ lines
contains three space-separated integers: $l_i$, $r_i$ and $t_i$.

OUTPUT FORMAT (print output to the terminal / stdout):For each test case, output a single line with an integer denoting the maximum
number of trees Farmer John can cut down.SAMPLE INPUT:3
7 1
8 4 10 1 2 6 7
2 9 3
7 2
8 4 10 1 2 6 7
2 9 3
1 10 1
7 2
8 4 10 1 2 6 7
2 9 3
1 10 4SAMPLE OUTPUT:4
4
3For the first test case, Farmer John can cut down the first $4$ trees, leaving
trees at $x_i = 2, 6, 7$ to satisfy the restriction.For the second test case, the additional restriction does not affect which trees
Farmer John can cut down, so he can cut down the same trees and satisfy both
restrictions.For the third test case, Farmer John can only cut down at most $3$ trees because
there are initially $7$ trees but the second restriction requires him to leave
at least $4$ trees uncut.SCORING:Input 2: $N, K \le 16$Inputs 3-5: $N, K \le 1000$Inputs 6-7: $t_i = 1$ for all $i = 1, \dots, K$.Inputs 8-11: No additional constraints.Problem credits: Tina Wang, Jiahe Lu, Benjamin Qi

\section*{Output Format}
For each test case, output a single line with an integer denoting the maximum
number of trees Farmer John can cut down.

SAMPLE INPUT:3
7 1
8 4 10 1 2 6 7
2 9 3
7 2
8 4 10 1 2 6 7
2 9 3
1 10 1
7 2
8 4 10 1 2 6 7
2 9 3
1 10 4SAMPLE OUTPUT:4
4
3For the first test case, Farmer John can cut down the first $4$ trees, leaving
trees at $x_i = 2, 6, 7$ to satisfy the restriction.For the second test case, the additional restriction does not affect which trees
Farmer John can cut down, so he can cut down the same trees and satisfy both
restrictions.For the third test case, Farmer John can only cut down at most $3$ trees because
there are initially $7$ trees but the second restriction requires him to leave
at least $4$ trees uncut.SCORING:Input 2: $N, K \le 16$Inputs 3-5: $N, K \le 1000$Inputs 6-7: $t_i = 1$ for all $i = 1, \dots, K$.Inputs 8-11: No additional constraints.Problem credits: Tina Wang, Jiahe Lu, Benjamin Qi

\section*{Sample Input}
\begin{verbatim}
3
7 1
8 4 10 1 2 6 7
2 9 3
7 2
8 4 10 1 2 6 7
2 9 3
1 10 1
7 2
8 4 10 1 2 6 7
2 9 3
1 10 4
\end{verbatim}

\section*{Sample Output}
\begin{verbatim}
4
4
3

For the first test case, Farmer John can cut down the first $4$ trees, leaving
trees at $x_i = 2, 6, 7$ to satisfy the restriction.For the second test case, the additional restriction does not affect which trees
Farmer John can cut down, so he can cut down the same trees and satisfy both
restrictions.For the third test case, Farmer John can only cut down at most $3$ trees because
there are initially $7$ trees but the second restriction requires him to leave
at least $4$ trees uncut.SCORING:Input 2: $N, K \le 16$Inputs 3-5: $N, K \le 1000$Inputs 6-7: $t_i = 1$ for all $i = 1, \dots, K$.Inputs 8-11: No additional constraints.Problem credits: Tina Wang, Jiahe Lu, Benjamin Qi

For the second test case, the additional restriction does not affect which trees
Farmer John can cut down, so he can cut down the same trees and satisfy both
restrictions.For the third test case, Farmer John can only cut down at most $3$ trees because
there are initially $7$ trees but the second restriction requires him to leave
at least $4$ trees uncut.SCORING:Input 2: $N, K \le 16$Inputs 3-5: $N, K \le 1000$Inputs 6-7: $t_i = 1$ for all $i = 1, \dots, K$.Inputs 8-11: No additional constraints.Problem credits: Tina Wang, Jiahe Lu, Benjamin Qi

For the third test case, Farmer John can only cut down at most $3$ trees because
there are initially $7$ trees but the second restriction requires him to leave
at least $4$ trees uncut.SCORING:Input 2: $N, K \le 16$Inputs 3-5: $N, K \le 1000$Inputs 6-7: $t_i = 1$ for all $i = 1, \dots, K$.Inputs 8-11: No additional constraints.Problem credits: Tina Wang, Jiahe Lu, Benjamin Qi

SCORING:Input 2: $N, K \le 16$Inputs 3-5: $N, K \le 1000$Inputs 6-7: $t_i = 1$ for all $i = 1, \dots, K$.Inputs 8-11: No additional constraints.Problem credits: Tina Wang, Jiahe Lu, Benjamin Qi
\end{verbatim}

\section*{Solution}


\section*{Problem URL}
https://usaco.org/index.php?page=viewproblem2&cpid=1447

\section*{Source}
USACO DEC24 Contest, Silver Division, Problem ID: 1447

\end{document}
