\documentclass[12pt]{article}
\usepackage{amsmath}
\usepackage{amssymb}
\usepackage{graphicx}
\usepackage{enumitem}
\usepackage{hyperref}
\usepackage{xcolor}

\title{View problem}
\author{USACO Contest: FEB25 Contest - Silver Division}
\date{\today}

\begin{document}

\maketitle

\section*{Problem Statement}


Bessie the brilliant bovine has discovered a new fascination—mathematical
magic! One day, while trotting through the fields of Farmer John’s ranch, she
stumbles upon two enchanted piles of hay. The first pile contains $a$ bales, and
the second pile contains $b$ bales ($1\le a,b\le 10^{18}$).

Next to the hay, half-buried in the dirt, she discovers an ancient scroll. As
she unfurls it, glowing letters reveal a prophecy:

To fulfill the decree of the Great Grasslands, the chosen one must transform
these two humble hay piles into exactly $c$ and $d$ bales—no more, no less.

Bessie realizes she can only perform the following two spells:
She can magically conjure new bales to increase the first pile’s size by
the amount currently in the second pile.She can magically conjure new
bales to increase the second pile’s size by the amount currently in the first
pile.
She must perform these operations sequentially, but she can perform them any
number of times and in any order. She must reach exactly $c$ bales in the first
pile and $d$ bales in the second pile
($1\le c,d\le 10^{18}$).

For each of $T$ ($1\le T\le 10^4$) independent test cases, output the minimum 
number of operations needed to fulfill the prophecy, or if it is impossible to
do so, output -1.

INPUT FORMAT (input arrives from the terminal / stdin):
The first line contains $T$.

The next $T$ lines each contain four integers $a,b,c,d$.

OUTPUT FORMAT (print output to the terminal / stdout):
Output $T$ lines, the answer to each test case.

SAMPLE INPUT:
4
5 3 5 2
5 3 8 19
5 3 19 8
5 3 5 3
SAMPLE OUTPUT: 
-1
3
-1
0

In the first test case, it is impossible since $b>d$ initially, but operations
can only increase $b$.

In the second test case, initially the two piles have $(5, 3)$ bales.  Bessie
can first increase the first pile by the amount in the second pile, resulting in
$(8, 3)$ bales.  Bessie can then increase the second pile by the new amount in
the first pile, and do this operation twice, resulting in $(8, 11)$ and finally
$(8, 19)$ bales.  This matches $c$ and $d$ and is the minimum number of
operations to get there.

Note that the third test case has a different answer than the second because $c$
and $d$ are swapped (the order of the piles matters).

In the fourth test case, no operations are necessary.

SAMPLE INPUT:
1
1 1 1 1000000000000000000
SAMPLE OUTPUT: 
999999999999999999

SCORING:
Inputs 3-4: $\max(c, d) \le 20 \cdot\min(a, b)$ Inputs 5-7: $T \le 10$ and $a,b,c,d\le 10^6$ Inputs 8-12: No additional constraints


Problem credits: Benjamin Qi



\section*{Input Format}
The next $T$ lines each contain four integers $a,b,c,d$.

OUTPUT FORMAT (print output to the terminal / stdout):Output $T$ lines, the answer to each test case.SAMPLE INPUT:4
5 3 5 2
5 3 8 19
5 3 19 8
5 3 5 3SAMPLE OUTPUT:-1
3
-1
0In the first test case, it is impossible since $b>d$ initially, but operations
can only increase $b$.In the second test case, initially the two piles have $(5, 3)$ bales.  Bessie
can first increase the first pile by the amount in the second pile, resulting in
$(8, 3)$ bales.  Bessie can then increase the second pile by the new amount in
the first pile, and do this operation twice, resulting in $(8, 11)$ and finally
$(8, 19)$ bales.  This matches $c$ and $d$ and is the minimum number of
operations to get there.Note that the third test case has a different answer than the second because $c$
and $d$ are swapped (the order of the piles matters).In the fourth test case, no operations are necessary.SAMPLE INPUT:1
1 1 1 1000000000000000000SAMPLE OUTPUT:999999999999999999SCORING:Inputs 3-4: $\max(c, d) \le 20 \cdot\min(a, b)$Inputs 5-7: $T \le 10$ and $a,b,c,d\le 10^6$Inputs 8-12: No additional constraintsProblem credits: Benjamin Qi

\section*{Output Format}
SAMPLE INPUT:4
5 3 5 2
5 3 8 19
5 3 19 8
5 3 5 3SAMPLE OUTPUT:-1
3
-1
0In the first test case, it is impossible since $b>d$ initially, but operations
can only increase $b$.In the second test case, initially the two piles have $(5, 3)$ bales.  Bessie
can first increase the first pile by the amount in the second pile, resulting in
$(8, 3)$ bales.  Bessie can then increase the second pile by the new amount in
the first pile, and do this operation twice, resulting in $(8, 11)$ and finally
$(8, 19)$ bales.  This matches $c$ and $d$ and is the minimum number of
operations to get there.Note that the third test case has a different answer than the second because $c$
and $d$ are swapped (the order of the piles matters).In the fourth test case, no operations are necessary.SAMPLE INPUT:1
1 1 1 1000000000000000000SAMPLE OUTPUT:999999999999999999SCORING:Inputs 3-4: $\max(c, d) \le 20 \cdot\min(a, b)$Inputs 5-7: $T \le 10$ and $a,b,c,d\le 10^6$Inputs 8-12: No additional constraintsProblem credits: Benjamin Qi

\section*{Sample Input}
\begin{verbatim}
4
5 3 5 2
5 3 8 19
5 3 19 8
5 3 5 3

1
1 1 1 1000000000000000000
\end{verbatim}

\section*{Sample Output}
\begin{verbatim}
-1
3
-1
0

In the first test case, it is impossible since $b>d$ initially, but operations
can only increase $b$.In the second test case, initially the two piles have $(5, 3)$ bales.  Bessie
can first increase the first pile by the amount in the second pile, resulting in
$(8, 3)$ bales.  Bessie can then increase the second pile by the new amount in
the first pile, and do this operation twice, resulting in $(8, 11)$ and finally
$(8, 19)$ bales.  This matches $c$ and $d$ and is the minimum number of
operations to get there.Note that the third test case has a different answer than the second because $c$
and $d$ are swapped (the order of the piles matters).In the fourth test case, no operations are necessary.SAMPLE INPUT:1
1 1 1 1000000000000000000SAMPLE OUTPUT:999999999999999999SCORING:Inputs 3-4: $\max(c, d) \le 20 \cdot\min(a, b)$Inputs 5-7: $T \le 10$ and $a,b,c,d\le 10^6$Inputs 8-12: No additional constraintsProblem credits: Benjamin Qi

In the second test case, initially the two piles have $(5, 3)$ bales.  Bessie
can first increase the first pile by the amount in the second pile, resulting in
$(8, 3)$ bales.  Bessie can then increase the second pile by the new amount in
the first pile, and do this operation twice, resulting in $(8, 11)$ and finally
$(8, 19)$ bales.  This matches $c$ and $d$ and is the minimum number of
operations to get there.Note that the third test case has a different answer than the second because $c$
and $d$ are swapped (the order of the piles matters).In the fourth test case, no operations are necessary.SAMPLE INPUT:1
1 1 1 1000000000000000000SAMPLE OUTPUT:999999999999999999SCORING:Inputs 3-4: $\max(c, d) \le 20 \cdot\min(a, b)$Inputs 5-7: $T \le 10$ and $a,b,c,d\le 10^6$Inputs 8-12: No additional constraintsProblem credits: Benjamin Qi

Note that the third test case has a different answer than the second because $c$
and $d$ are swapped (the order of the piles matters).In the fourth test case, no operations are necessary.SAMPLE INPUT:1
1 1 1 1000000000000000000SAMPLE OUTPUT:999999999999999999SCORING:Inputs 3-4: $\max(c, d) \le 20 \cdot\min(a, b)$Inputs 5-7: $T \le 10$ and $a,b,c,d\le 10^6$Inputs 8-12: No additional constraintsProblem credits: Benjamin Qi

In the fourth test case, no operations are necessary.SAMPLE INPUT:1
1 1 1 1000000000000000000SAMPLE OUTPUT:999999999999999999SCORING:Inputs 3-4: $\max(c, d) \le 20 \cdot\min(a, b)$Inputs 5-7: $T \le 10$ and $a,b,c,d\le 10^6$Inputs 8-12: No additional constraintsProblem credits: Benjamin Qi

SAMPLE INPUT:1
1 1 1 1000000000000000000SAMPLE OUTPUT:999999999999999999SCORING:Inputs 3-4: $\max(c, d) \le 20 \cdot\min(a, b)$Inputs 5-7: $T \le 10$ and $a,b,c,d\le 10^6$Inputs 8-12: No additional constraintsProblem credits: Benjamin Qi

999999999999999999

SCORING:Inputs 3-4: $\max(c, d) \le 20 \cdot\min(a, b)$Inputs 5-7: $T \le 10$ and $a,b,c,d\le 10^6$Inputs 8-12: No additional constraintsProblem credits: Benjamin Qi
\end{verbatim}

\section*{Solution}


\section*{Problem URL}
https://usaco.org/index.php?page=viewproblem2&cpid=1496

\section*{Source}
USACO FEB25 Contest, Silver Division, Problem ID: 1496

\end{document}
