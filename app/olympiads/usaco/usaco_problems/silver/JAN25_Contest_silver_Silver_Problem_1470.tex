\documentclass[12pt]{article}
\usepackage{amsmath}
\usepackage{amssymb}
\usepackage{graphicx}
\usepackage{enumitem}
\usepackage{hyperref}
\usepackage{xcolor}

\title{View problem}
\author{USACO Contest: JAN25 Contest - Silver Division}
\date{\today}

\begin{document}

\maketitle

\section*{Problem Statement}


Farmer John's $N$ ($1 \leq N \leq 5 \cdot 10^5$) cows are standing in a line,
with cow $1$ at the front of the line and cow $N$ at the back of the line. FJ's
cows also come in many different species. He denotes each species with an
integer from $1$ to $N$. The $i$'th cow from the front of the line is of species
$a_i$ ($1 \leq a_i \leq N$). 

FJ is taking his cows to a checkup at a local bovine hospital. However, the
bovine veterinarian is very picky and wants to perform a checkup on the $i$'th
cow in the line, only if it is species $b_i$ ($1 \leq b_i \leq N$). 

FJ is lazy and does not want to completely reorder his cows.  He will perform
the following operation exactly once.

 Select two integers $l$ and $r$ such that $1 \leq l \le r \leq N$. Reverse
the order of the cows that are between the $l$-th cow and the $r$-th cow in the
line, inclusive. 
FJ wants to measure how effective this approach is. Find the sum of the number
of cows  that are checked by the veterinarian over all $N(N+1)/2$ possible
operations.

INPUT FORMAT (input arrives from the terminal / stdin):
The first line contains an integer $N$.

The second line contains $a_1, a_2, \ldots, a_N$.

The third line contains $b_1, b_2, \ldots, b_N$.

OUTPUT FORMAT (print output to the terminal / stdout):
Output one line with the sum of the number of cows that are checked by the
veterinarian  over all possible operations.

SAMPLE INPUT:
3
1 3 2
3 2 1
SAMPLE OUTPUT: 
3

If FJ chooses ($l=1,r=1$), ($l=2,r=2$), or ($l=3,r=3$) then no cows will be
checked. Note that those operations do not modify the location of the cows.

The following operations result in one cow being checked.
$l=1,r=2$: FJ reverses the order of the first and second cows so the species
of each cow in the new lineup will be $[3,1,2]$. The first cow will be checked.
$l=2,r=3$: FJ reverses the order of the second and third cows so the species
of each cow in the new lineup will be $[1,2,3]$. The second cow will be checked.
$l=1,r=3$: FJ reverses the order of the first, second, and third cows so the
species of each cow in the new lineup will be $[2,3,1]$. The third cow will be
checked. 
The total number of cows checked over all six operations is $0+0+0+1+1+1=3$.

SAMPLE INPUT:
3
1 2 3
1 2 3
SAMPLE OUTPUT: 
12

There are three possible operations that cause $3$ cows to be checked:
($l=1,r=1$), ($l=2,r=2$), and ($l=3,r=3$). The remaining operations each result
in $1$ cow being checked. The total number of cows checked over all six
operations is $3+3+3+1+1+1=12$.

SAMPLE INPUT:
7
1 3 2 2 1 3 2
3 2 2 1 2 3 1
SAMPLE OUTPUT: 
60

SCORING:
Input 4: $N\le 100$Input 5: $N\le 5000$Inputs 6-9: $a_i, b_i$ are all generated uniformly at random in the range
$[1,N]$Inputs 10-15: $a_i, b_i$ are all generated uniformly at random in the range
$[1,2]$Inputs 16-23: No additional constraints.


Problem credits: Chongtian Ma, Haokai Ma, and Alex Liang



\section*{Input Format}
The second line contains $a_1, a_2, \ldots, a_N$.The third line contains $b_1, b_2, \ldots, b_N$.

The third line contains $b_1, b_2, \ldots, b_N$.

OUTPUT FORMAT (print output to the terminal / stdout):Output one line with the sum of the number of cows that are checked by the
veterinarian  over all possible operations.SAMPLE INPUT:3
1 3 2
3 2 1SAMPLE OUTPUT:3If FJ chooses ($l=1,r=1$), ($l=2,r=2$), or ($l=3,r=3$) then no cows will be
checked. Note that those operations do not modify the location of the cows.The following operations result in one cow being checked.$l=1,r=2$: FJ reverses the order of the first and second cows so the species
of each cow in the new lineup will be $[3,1,2]$. The first cow will be checked.$l=2,r=3$: FJ reverses the order of the second and third cows so the species
of each cow in the new lineup will be $[1,2,3]$. The second cow will be checked.$l=1,r=3$: FJ reverses the order of the first, second, and third cows so the
species of each cow in the new lineup will be $[2,3,1]$. The third cow will be
checked.The total number of cows checked over all six operations is $0+0+0+1+1+1=3$.SAMPLE INPUT:3
1 2 3
1 2 3SAMPLE OUTPUT:12There are three possible operations that cause $3$ cows to be checked:
($l=1,r=1$), ($l=2,r=2$), and ($l=3,r=3$). The remaining operations each result
in $1$ cow being checked. The total number of cows checked over all six
operations is $3+3+3+1+1+1=12$.SAMPLE INPUT:7
1 3 2 2 1 3 2
3 2 2 1 2 3 1SAMPLE OUTPUT:60SCORING:Input 4: $N\le 100$Input 5: $N\le 5000$Inputs 6-9: $a_i, b_i$ are all generated uniformly at random in the range
$[1,N]$Inputs 10-15: $a_i, b_i$ are all generated uniformly at random in the range
$[1,2]$Inputs 16-23: No additional constraints.Problem credits: Chongtian Ma, Haokai Ma, and Alex Liang

\section*{Output Format}
SAMPLE INPUT:3
1 3 2
3 2 1SAMPLE OUTPUT:3If FJ chooses ($l=1,r=1$), ($l=2,r=2$), or ($l=3,r=3$) then no cows will be
checked. Note that those operations do not modify the location of the cows.The following operations result in one cow being checked.$l=1,r=2$: FJ reverses the order of the first and second cows so the species
of each cow in the new lineup will be $[3,1,2]$. The first cow will be checked.$l=2,r=3$: FJ reverses the order of the second and third cows so the species
of each cow in the new lineup will be $[1,2,3]$. The second cow will be checked.$l=1,r=3$: FJ reverses the order of the first, second, and third cows so the
species of each cow in the new lineup will be $[2,3,1]$. The third cow will be
checked.The total number of cows checked over all six operations is $0+0+0+1+1+1=3$.SAMPLE INPUT:3
1 2 3
1 2 3SAMPLE OUTPUT:12There are three possible operations that cause $3$ cows to be checked:
($l=1,r=1$), ($l=2,r=2$), and ($l=3,r=3$). The remaining operations each result
in $1$ cow being checked. The total number of cows checked over all six
operations is $3+3+3+1+1+1=12$.SAMPLE INPUT:7
1 3 2 2 1 3 2
3 2 2 1 2 3 1SAMPLE OUTPUT:60SCORING:Input 4: $N\le 100$Input 5: $N\le 5000$Inputs 6-9: $a_i, b_i$ are all generated uniformly at random in the range
$[1,N]$Inputs 10-15: $a_i, b_i$ are all generated uniformly at random in the range
$[1,2]$Inputs 16-23: No additional constraints.Problem credits: Chongtian Ma, Haokai Ma, and Alex Liang

\section*{Sample Input}
\begin{verbatim}
3
1 3 2
3 2 1

3
1 2 3
1 2 3

7
1 3 2 2 1 3 2
3 2 2 1 2 3 1
\end{verbatim}

\section*{Sample Output}
\begin{verbatim}
3

If FJ chooses ($l=1,r=1$), ($l=2,r=2$), or ($l=3,r=3$) then no cows will be
checked. Note that those operations do not modify the location of the cows.The following operations result in one cow being checked.$l=1,r=2$: FJ reverses the order of the first and second cows so the species
of each cow in the new lineup will be $[3,1,2]$. The first cow will be checked.$l=2,r=3$: FJ reverses the order of the second and third cows so the species
of each cow in the new lineup will be $[1,2,3]$. The second cow will be checked.$l=1,r=3$: FJ reverses the order of the first, second, and third cows so the
species of each cow in the new lineup will be $[2,3,1]$. The third cow will be
checked.The total number of cows checked over all six operations is $0+0+0+1+1+1=3$.SAMPLE INPUT:3
1 2 3
1 2 3SAMPLE OUTPUT:12There are three possible operations that cause $3$ cows to be checked:
($l=1,r=1$), ($l=2,r=2$), and ($l=3,r=3$). The remaining operations each result
in $1$ cow being checked. The total number of cows checked over all six
operations is $3+3+3+1+1+1=12$.SAMPLE INPUT:7
1 3 2 2 1 3 2
3 2 2 1 2 3 1SAMPLE OUTPUT:60SCORING:Input 4: $N\le 100$Input 5: $N\le 5000$Inputs 6-9: $a_i, b_i$ are all generated uniformly at random in the range
$[1,N]$Inputs 10-15: $a_i, b_i$ are all generated uniformly at random in the range
$[1,2]$Inputs 16-23: No additional constraints.Problem credits: Chongtian Ma, Haokai Ma, and Alex Liang

The following operations result in one cow being checked.$l=1,r=2$: FJ reverses the order of the first and second cows so the species
of each cow in the new lineup will be $[3,1,2]$. The first cow will be checked.$l=2,r=3$: FJ reverses the order of the second and third cows so the species
of each cow in the new lineup will be $[1,2,3]$. The second cow will be checked.$l=1,r=3$: FJ reverses the order of the first, second, and third cows so the
species of each cow in the new lineup will be $[2,3,1]$. The third cow will be
checked.The total number of cows checked over all six operations is $0+0+0+1+1+1=3$.SAMPLE INPUT:3
1 2 3
1 2 3SAMPLE OUTPUT:12There are three possible operations that cause $3$ cows to be checked:
($l=1,r=1$), ($l=2,r=2$), and ($l=3,r=3$). The remaining operations each result
in $1$ cow being checked. The total number of cows checked over all six
operations is $3+3+3+1+1+1=12$.SAMPLE INPUT:7
1 3 2 2 1 3 2
3 2 2 1 2 3 1SAMPLE OUTPUT:60SCORING:Input 4: $N\le 100$Input 5: $N\le 5000$Inputs 6-9: $a_i, b_i$ are all generated uniformly at random in the range
$[1,N]$Inputs 10-15: $a_i, b_i$ are all generated uniformly at random in the range
$[1,2]$Inputs 16-23: No additional constraints.Problem credits: Chongtian Ma, Haokai Ma, and Alex Liang

The total number of cows checked over all six operations is $0+0+0+1+1+1=3$.SAMPLE INPUT:3
1 2 3
1 2 3SAMPLE OUTPUT:12There are three possible operations that cause $3$ cows to be checked:
($l=1,r=1$), ($l=2,r=2$), and ($l=3,r=3$). The remaining operations each result
in $1$ cow being checked. The total number of cows checked over all six
operations is $3+3+3+1+1+1=12$.SAMPLE INPUT:7
1 3 2 2 1 3 2
3 2 2 1 2 3 1SAMPLE OUTPUT:60SCORING:Input 4: $N\le 100$Input 5: $N\le 5000$Inputs 6-9: $a_i, b_i$ are all generated uniformly at random in the range
$[1,N]$Inputs 10-15: $a_i, b_i$ are all generated uniformly at random in the range
$[1,2]$Inputs 16-23: No additional constraints.Problem credits: Chongtian Ma, Haokai Ma, and Alex Liang

SAMPLE INPUT:3
1 2 3
1 2 3SAMPLE OUTPUT:12There are three possible operations that cause $3$ cows to be checked:
($l=1,r=1$), ($l=2,r=2$), and ($l=3,r=3$). The remaining operations each result
in $1$ cow being checked. The total number of cows checked over all six
operations is $3+3+3+1+1+1=12$.SAMPLE INPUT:7
1 3 2 2 1 3 2
3 2 2 1 2 3 1SAMPLE OUTPUT:60SCORING:Input 4: $N\le 100$Input 5: $N\le 5000$Inputs 6-9: $a_i, b_i$ are all generated uniformly at random in the range
$[1,N]$Inputs 10-15: $a_i, b_i$ are all generated uniformly at random in the range
$[1,2]$Inputs 16-23: No additional constraints.Problem credits: Chongtian Ma, Haokai Ma, and Alex Liang

12

There are three possible operations that cause $3$ cows to be checked:
($l=1,r=1$), ($l=2,r=2$), and ($l=3,r=3$). The remaining operations each result
in $1$ cow being checked. The total number of cows checked over all six
operations is $3+3+3+1+1+1=12$.SAMPLE INPUT:7
1 3 2 2 1 3 2
3 2 2 1 2 3 1SAMPLE OUTPUT:60SCORING:Input 4: $N\le 100$Input 5: $N\le 5000$Inputs 6-9: $a_i, b_i$ are all generated uniformly at random in the range
$[1,N]$Inputs 10-15: $a_i, b_i$ are all generated uniformly at random in the range
$[1,2]$Inputs 16-23: No additional constraints.Problem credits: Chongtian Ma, Haokai Ma, and Alex Liang

SAMPLE INPUT:7
1 3 2 2 1 3 2
3 2 2 1 2 3 1SAMPLE OUTPUT:60SCORING:Input 4: $N\le 100$Input 5: $N\le 5000$Inputs 6-9: $a_i, b_i$ are all generated uniformly at random in the range
$[1,N]$Inputs 10-15: $a_i, b_i$ are all generated uniformly at random in the range
$[1,2]$Inputs 16-23: No additional constraints.Problem credits: Chongtian Ma, Haokai Ma, and Alex Liang

60

SCORING:Input 4: $N\le 100$Input 5: $N\le 5000$Inputs 6-9: $a_i, b_i$ are all generated uniformly at random in the range
$[1,N]$Inputs 10-15: $a_i, b_i$ are all generated uniformly at random in the range
$[1,2]$Inputs 16-23: No additional constraints.Problem credits: Chongtian Ma, Haokai Ma, and Alex Liang
\end{verbatim}

\section*{Solution}


\section*{Problem URL}
https://usaco.org/index.php?page=viewproblem2&cpid=1470

\section*{Source}
USACO JAN25 Contest, Silver Division, Problem ID: 1470

\end{document}
