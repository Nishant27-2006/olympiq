\documentclass[12pt]{article}
\usepackage{amsmath}
\usepackage{amssymb}
\usepackage{graphicx}
\usepackage{enumitem}
\usepackage{hyperref}
\usepackage{xcolor}

\title{View problem}
\author{USACO Contest: JAN22 Contest - Silver Division}
\date{\today}

\begin{document}

\maketitle

\section*{Problem Statement}

Farmer John's $N$ cows ($N \leq 3 \times 10^5)$ have heights $1, 2, \ldots, N$.  One day, the cows are
standing in a line in some order playing frisbee; let $h_1 \ldots h_N$ denote
the heights of the cows in this order (so the $h$'s are a permutation of
$1 \ldots N$). 

Two cows at positions $i$ and $j$ in the line can successfully throw the frisbee
back and forth if and only if every cow between them has height lower than
$\min(h_i, h_j)$.  

Please compute the sum of distances between all pairs of locations $i<j$ at
which there resides a pair of cows that can successfully throw the frisbee  back
and forth.  The distance between locations $i$ and $j$ is $j-i+1$.

INPUT FORMAT (input arrives from the terminal / stdin):
The first line of input contains a single integer $N$. The next line of input
contains $h_1 \ldots h_N$, separated by spaces.

OUTPUT FORMAT (print output to the terminal / stdout):
Output the sum of distances of all pairs of locations at which there  are cows
that can throw the frisbee back and forth.  Note that the large  size of
integers involved in this problem may require the use of 64-bit integer data
types (e.g., a "long long" in C/C++).

SAMPLE INPUT:
7
4 3 1 2 5 6 7
SAMPLE OUTPUT: 
24

The pairs of successful locations in this example are as follows:


(1, 2), (1, 5), (2, 3), (2, 4), (2, 5), (3, 4), (4, 5), (5, 6), (6, 7)

SCORING
Test cases 1-3 satisfy $N\le 5000$.Test cases 4-11 satisfy no
additional constraints.


Problem credits: Quanquan Liu



\section*{Input Format}
OUTPUT FORMAT (print output to the terminal / stdout):Output the sum of distances of all pairs of locations at which there  are cows
that can throw the frisbee back and forth.  Note that the large  size of
integers involved in this problem may require the use of 64-bit integer data
types (e.g., a "long long" in C/C++).SAMPLE INPUT:7
4 3 1 2 5 6 7SAMPLE OUTPUT:24The pairs of successful locations in this example are as follows:(1, 2), (1, 5), (2, 3), (2, 4), (2, 5), (3, 4), (4, 5), (5, 6), (6, 7)SCORINGTest cases 1-3 satisfy $N\le 5000$.Test cases 4-11 satisfy no
additional constraints.Problem credits: Quanquan Liu

\section*{Output Format}
SAMPLE INPUT:7
4 3 1 2 5 6 7SAMPLE OUTPUT:24The pairs of successful locations in this example are as follows:(1, 2), (1, 5), (2, 3), (2, 4), (2, 5), (3, 4), (4, 5), (5, 6), (6, 7)SCORINGTest cases 1-3 satisfy $N\le 5000$.Test cases 4-11 satisfy no
additional constraints.Problem credits: Quanquan Liu

\section*{Sample Input}
\begin{verbatim}
7
4 3 1 2 5 6 7
\end{verbatim}

\section*{Sample Output}
\begin{verbatim}
24

The pairs of successful locations in this example are as follows:(1, 2), (1, 5), (2, 3), (2, 4), (2, 5), (3, 4), (4, 5), (5, 6), (6, 7)SCORINGTest cases 1-3 satisfy $N\le 5000$.Test cases 4-11 satisfy no
additional constraints.Problem credits: Quanquan Liu

(1, 2), (1, 5), (2, 3), (2, 4), (2, 5), (3, 4), (4, 5), (5, 6), (6, 7)SCORINGTest cases 1-3 satisfy $N\le 5000$.Test cases 4-11 satisfy no
additional constraints.Problem credits: Quanquan Liu

(1, 2), (1, 5), (2, 3), (2, 4), (2, 5), (3, 4), (4, 5), (5, 6), (6, 7)

SCORINGTest cases 1-3 satisfy $N\le 5000$.Test cases 4-11 satisfy no
additional constraints.Problem credits: Quanquan Liu
\end{verbatim}

\section*{Solution}


\section*{Problem URL}
https://usaco.org/index.php?page=viewproblem2&cpid=1183

\section*{Source}
USACO JAN22 Contest, Silver Division, Problem ID: 1183

\end{document}
