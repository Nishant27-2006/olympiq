\documentclass[12pt]{article}
\usepackage{amsmath}
\usepackage{amssymb}
\usepackage{graphicx}
\usepackage{enumitem}
\usepackage{hyperref}
\usepackage{xcolor}

\title{View problem}
\author{USACO Contest: FEB22 Contest - Silver Division}
\date{\today}

\begin{document}

\maketitle

\section*{Problem Statement}

Farmer John has fallen behind on organizing his inbox. The way his screen is
organized, there is a vertical list of folders on the left side of the screen
and a vertical list of emails on the right side of the screen. There are $M$
total folders, numbered $1 \ldots M$ ($1 \le M \le 10^4)$. His inbox currently
contains $N$ emails numbered $1\ldots N$ ($1 \le N \le 10^5$); the $i$th email
needs to be filed into folder $f_i$ ($1\le f_i\le M$). 

FJ's screen is rather small, so he can only view $K$ ($1\le K\le \min(N,M)$)
folders and $K$ emails at once.  Initially, his screen starts out displaying folders
$1 \ldots K$ on the left and emails $1 \ldots K$ on the right.  To access other
folders and emails, he needs to scroll through these respective lists.  For
example, if he scrolls down one position in the list of folders, his screen will display
folders $2 \ldots K+1$, and then scrolling down one position further it will display folders
$3 \ldots K+2$.  When FJ drags an email into a folder, the email disappears from
the email list, and the emails after the one that disappeared shift up by one
position.  For example, if emails $1, 2, 3, 4, 5$ are currently displayed and FJ
drags email 3 into its appropriate folder, the email list will now show emails
$1, 2, 4, 5, 6$.  FJ can only drag an email into the folder to which it needs to
be filed.

Unfortunately, the scroll wheel on FJ's mouse is broken, and he can only scroll
downwards, not upwards.  The only way he can achieve some semblance of upward
scrolling is if he is viewing the last set of $K$ emails in his email list, and
he files one of these.  In this case, the email list will again show the last
$K$ emails that haven't yet been filed, effectively scrolling the top email up
by one. If there are fewer than $K$ emails remaining, then all of them will be
displayed. 

Please help FJ determine if it is possible to file all of his emails.

INPUT FORMAT (input arrives from the terminal / stdin):
The first line of input contains $T$ ($1 \le T \le 10$), the number of subcases
in this input,  all of which must be solved correctly to solve the input case.
The $T$ subcases then follow.  For each subcase, the first line of input
contains $M$, $N$, and $K$. The next line contains $f_1 \ldots f_N$.

It is guaranteed that the sum of $M$ over all subcases does not exceed $10^4$,
and that the sum of $N$ over all subcases does not exceed $10^5$.

OUTPUT FORMAT (print output to the terminal / stdout):
Output $T$ lines, each one either containing either YES or NO, specifying
whether FJ can successfully file all his emails in each of the $T$ subcases.

SAMPLE INPUT:
6
5 5 1
1 2 3 4 5
5 5 1
1 2 3 5 4
5 5 1
1 2 4 5 3
5 5 2
1 2 4 5 3
3 10 2
1 3 2 1 3 2 1 3 2 1
3 10 1
1 3 2 1 3 2 1 3 2 1
SAMPLE OUTPUT: 
YES
YES
NO
YES
YES
NO

SCORING:
In inputs 2-10, the sum of $M$ over all subcases does not exceed
$10^3$.In inputs 11-12, no additional constraints.


Problem credits: Brian Dean



\section*{Input Format}
It is guaranteed that the sum of $M$ over all subcases does not exceed $10^4$,
and that the sum of $N$ over all subcases does not exceed $10^5$.

OUTPUT FORMAT (print output to the terminal / stdout):Output $T$ lines, each one either containing either YES or NO, specifying
whether FJ can successfully file all his emails in each of the $T$ subcases.SAMPLE INPUT:6
5 5 1
1 2 3 4 5
5 5 1
1 2 3 5 4
5 5 1
1 2 4 5 3
5 5 2
1 2 4 5 3
3 10 2
1 3 2 1 3 2 1 3 2 1
3 10 1
1 3 2 1 3 2 1 3 2 1SAMPLE OUTPUT:YES
YES
NO
YES
YES
NOSCORING:In inputs 2-10, the sum of $M$ over all subcases does not exceed
$10^3$.In inputs 11-12, no additional constraints.Problem credits: Brian Dean

\section*{Output Format}
SAMPLE INPUT:6
5 5 1
1 2 3 4 5
5 5 1
1 2 3 5 4
5 5 1
1 2 4 5 3
5 5 2
1 2 4 5 3
3 10 2
1 3 2 1 3 2 1 3 2 1
3 10 1
1 3 2 1 3 2 1 3 2 1SAMPLE OUTPUT:YES
YES
NO
YES
YES
NOSCORING:In inputs 2-10, the sum of $M$ over all subcases does not exceed
$10^3$.In inputs 11-12, no additional constraints.Problem credits: Brian Dean

\section*{Sample Input}
\begin{verbatim}
6
5 5 1
1 2 3 4 5
5 5 1
1 2 3 5 4
5 5 1
1 2 4 5 3
5 5 2
1 2 4 5 3
3 10 2
1 3 2 1 3 2 1 3 2 1
3 10 1
1 3 2 1 3 2 1 3 2 1
\end{verbatim}

\section*{Sample Output}
\begin{verbatim}
YES
YES
NO
YES
YES
NO

SCORING:In inputs 2-10, the sum of $M$ over all subcases does not exceed
$10^3$.In inputs 11-12, no additional constraints.Problem credits: Brian Dean
\end{verbatim}

\section*{Solution}


\section*{Problem URL}
https://usaco.org/index.php?page=viewproblem2&cpid=1208

\section*{Source}
USACO FEB22 Contest, Silver Division, Problem ID: 1208

\end{document}
