\documentclass[12pt]{article}
\usepackage{amsmath}
\usepackage{amssymb}
\usepackage{graphicx}
\usepackage{enumitem}
\usepackage{hyperref}
\usepackage{xcolor}

\title{View problem}
\author{USACO Contest: FEB18 Contest - Silver Division}
\date{\today}

\begin{document}

\maketitle

\section*{Problem Statement}

Farmer John and his personal trainer Bessie are hiking up Mount Vancowver. For
their purposes (and yours), the mountain can be represented as a long straight
trail of length $L$ meters ($1 \leq L \leq 10^6$). Farmer John will hike the
trail at a constant travel rate of $r_F$ seconds per meter
($1 \leq r_F \leq 10^6$). Since he is working on his stamina, he will not take
any rest stops along the way.

Bessie, however, is allowed to take rest stops, where she might find some tasty
grass. Of course, she cannot stop just anywhere! There are $N$ rest stops along
the trail ($1 \leq N \leq 10^5$); the $i$-th stop is $x_i$ meters from the start
of the trail ($0 < x_i < L$) and has a tastiness value $c_i$
($1 \leq c_i \leq 10^6$). If Bessie rests at stop $i$ for $t$ seconds, she
receives $c_i \cdot t$ tastiness units.

When not at a rest stop, Bessie will be hiking at a fixed travel rate of $r_B$
seconds per meter ($1 \leq r_B \leq 10^6$). Since Bessie is young and fit, $r_B$
is strictly less than $r_F$.

Bessie would like to maximize her consumption of tasty grass. But she is worried
about Farmer John; she thinks that if at any point along the hike she is behind
Farmer John on the trail, he might lose all motivation to continue!

Help Bessie find the maximum total tastiness units she can obtain while making
sure that Farmer John completes the hike.

INPUT FORMAT (file reststops.in):
The first line of input contains four integers: $L$, $N$, $r_F$, and $r_B$. The
next $N$ lines describe the rest stops. For each $i$ between $1$ and $N$, the
$i+1$-st line contains two integers $x_i$ and $c_i$, describing the position of
the $i$-th rest stop and the tastiness of the grass there.

It is guaranteed that $r_F > r_B$, and $0 < x_1 < \dots < x_N < L $.  Note
that $r_F$ and $r_B$ are given in seconds per meter! 

OUTPUT FORMAT (file reststops.out):
A single integer: the maximum total tastiness units Bessie can obtain.

SAMPLE INPUT:
10 2 4 3
7 2
8 1
SAMPLE OUTPUT: 
15

In this example, it is optimal for Bessie to stop for $7$ seconds at the $x=7$ rest stop (acquiring $14$ tastiness units) and then stop for an additional $1$ second at the $x=8$ rest stop (acquiring $1$ more tastiness unit, for a total of $15$ tastiness units).


Problem credits: Dhruv Rohatgi



\section*{Input Format}
It is guaranteed that $r_F > r_B$, and $0 < x_1 < \dots < x_N < L $.Note
that $r_F$ and $r_B$ are given in seconds per meter!

OUTPUT FORMAT (file reststops.out):A single integer: the maximum total tastiness units Bessie can obtain.SAMPLE INPUT:10 2 4 3
7 2
8 1SAMPLE OUTPUT:15In this example, it is optimal for Bessie to stop for $7$ seconds at the $x=7$ rest stop (acquiring $14$ tastiness units) and then stop for an additional $1$ second at the $x=8$ rest stop (acquiring $1$ more tastiness unit, for a total of $15$ tastiness units).Problem credits: Dhruv Rohatgi

\section*{Output Format}
SAMPLE INPUT:10 2 4 3
7 2
8 1SAMPLE OUTPUT:15In this example, it is optimal for Bessie to stop for $7$ seconds at the $x=7$ rest stop (acquiring $14$ tastiness units) and then stop for an additional $1$ second at the $x=8$ rest stop (acquiring $1$ more tastiness unit, for a total of $15$ tastiness units).Problem credits: Dhruv Rohatgi

\section*{Sample Input}
\begin{verbatim}
10 2 4 3
7 2
8 1
\end{verbatim}

\section*{Sample Output}
\begin{verbatim}
15

In this example, it is optimal for Bessie to stop for $7$ seconds at the $x=7$ rest stop (acquiring $14$ tastiness units) and then stop for an additional $1$ second at the $x=8$ rest stop (acquiring $1$ more tastiness unit, for a total of $15$ tastiness units).Problem credits: Dhruv Rohatgi

Problem credits: Dhruv Rohatgi

Problem credits: Dhruv Rohatgi
\end{verbatim}

\section*{Solution}


\section*{Problem URL}
https://usaco.org/index.php?page=viewproblem2&cpid=810

\section*{Source}
USACO FEB18 Contest, Silver Division, Problem ID: 810

\end{document}
