\documentclass[12pt]{article}
\usepackage{amsmath}
\usepackage{amssymb}
\usepackage{graphicx}
\usepackage{enumitem}
\usepackage{hyperref}
\usepackage{xcolor}

\title{View problem}
\author{USACO Contest: OPEN25 Contest - Silver Division}
\date{\today}

\begin{document}

\maketitle

\section*{Problem Statement}


Lately, the cows on Farmer John's farm have been infatuated with watching the
show Apothecowry Dairies. The show revolves around a clever bovine sleuth CowCow
solving problems of various kinds. Bessie found a new problem from the show, but
the solution won't be revealed until the next episode in a week! Please solve
the problem for her.

You are given integers $M$ and $K$ $(1 \leq M \leq 10 ^ 9, 1 \leq K \leq 31)$.
Please choose a positive integer $N$ and construct a sequence $a$ of $N$
non-negative integers such that the following conditions are satisfied:
 $1 \le N \le 100$  $a_1 + a_2 + \dots + a_N = M$ 
$\text{popcount}(a_1) \oplus \text{ popcount}(a_2) \oplus \dots \oplus \text{ popcount}(a_N) = K$

If no such sequence exists, print $-1$.

$\dagger \text{ popcount}(x)$ is the number of bits equal to $1$ in the binary
representation of the integer $x$. For instance, the popcount of $11$ is $3$ and
the popcount of $16$ is $1$.

$\dagger \oplus$ is the bitwise xor operator.

The input will consist of $T$ ($1 \le T \le 5 \cdot 10^3$) independent test
cases.

INPUT FORMAT (input arrives from the terminal / stdin):
The first line contains $T$.

The first and only line of each test case has $M$ and $K$. 

It is guaranteed that all test cases are unique.


OUTPUT FORMAT (print output to the terminal / stdout):
Output the solutions for $T$ test cases as follows:

If no answer exists, the only line for that test case should be $-1$.

Otherwise, the first line for that test case should be a single integer $N$, the
length of the sequence -- ($1 \le  N \le 100$).

The second line for that test case should contain $N$ space-separated integers
that satisfy the conditions -- ($0 \le a_i \le M$).

SAMPLE INPUT:
3
2 1
33 5
10 5
SAMPLE OUTPUT: 
2
2 0
3
3 23 7 
-1

In the first test case, the elements in the array $a = [2, 0]$ sum to $2$. The
xor sum of popcounts is $1 \oplus 0 = 1$. Thus, all the conditions are
satisfied.

In the second test case, the elements in the array $a = [3, 23, 7]$ sum to $33$.
The xor sum of the popcounts is $2 \oplus 4 \oplus 3 = 5$. Thus, all conditions
are satisfied.

Other valid arrays are $a = [4, 2, 15, 5, 7]$ and $a = [1, 4, 0, 27, 1]$.

It can be shown that no valid arrays exist for the third test case.

SCORING:
 Input 2: $M \leq 8, K \leq 8$ Inputs 3-5: $M > 2^K$ 
Inputs 6-18: No additional constraints.


Problem credits: Aakash Gokhale



\section*{Input Format}
The first line contains $T$.The first and only line of each test case has $M$ and $K$.It is guaranteed that all test cases are unique.

The first and only line of each test case has $M$ and $K$.It is guaranteed that all test cases are unique.

It is guaranteed that all test cases are unique.

OUTPUT FORMAT (print output to the terminal / stdout):Output the solutions for $T$ test cases as follows:If no answer exists, the only line for that test case should be $-1$.Otherwise, the first line for that test case should be a single integer $N$, the
length of the sequence -- ($1 \le  N \le 100$).The second line for that test case should contain $N$ space-separated integers
that satisfy the conditions -- ($0 \le a_i \le M$).SAMPLE INPUT:3
2 1
33 5
10 5SAMPLE OUTPUT:2
2 0
3
3 23 7 
-1In the first test case, the elements in the array $a = [2, 0]$ sum to $2$. The
xor sum of popcounts is $1 \oplus 0 = 1$. Thus, all the conditions are
satisfied.In the second test case, the elements in the array $a = [3, 23, 7]$ sum to $33$.
The xor sum of the popcounts is $2 \oplus 4 \oplus 3 = 5$. Thus, all conditions
are satisfied.Other valid arrays are $a = [4, 2, 15, 5, 7]$ and $a = [1, 4, 0, 27, 1]$.It can be shown that no valid arrays exist for the third test case.SCORING:Input 2: $M \leq 8, K \leq 8$Inputs 3-5: $M > 2^K$Inputs 6-18: No additional constraints.Problem credits: Aakash Gokhale

\section*{Output Format}
Output the solutions for $T$ test cases as follows:If no answer exists, the only line for that test case should be $-1$.Otherwise, the first line for that test case should be a single integer $N$, the
length of the sequence -- ($1 \le  N \le 100$).The second line for that test case should contain $N$ space-separated integers
that satisfy the conditions -- ($0 \le a_i \le M$).

If no answer exists, the only line for that test case should be $-1$.Otherwise, the first line for that test case should be a single integer $N$, the
length of the sequence -- ($1 \le  N \le 100$).The second line for that test case should contain $N$ space-separated integers
that satisfy the conditions -- ($0 \le a_i \le M$).

Otherwise, the first line for that test case should be a single integer $N$, the
length of the sequence -- ($1 \le  N \le 100$).The second line for that test case should contain $N$ space-separated integers
that satisfy the conditions -- ($0 \le a_i \le M$).

The second line for that test case should contain $N$ space-separated integers
that satisfy the conditions -- ($0 \le a_i \le M$).

SAMPLE INPUT:3
2 1
33 5
10 5SAMPLE OUTPUT:2
2 0
3
3 23 7 
-1In the first test case, the elements in the array $a = [2, 0]$ sum to $2$. The
xor sum of popcounts is $1 \oplus 0 = 1$. Thus, all the conditions are
satisfied.In the second test case, the elements in the array $a = [3, 23, 7]$ sum to $33$.
The xor sum of the popcounts is $2 \oplus 4 \oplus 3 = 5$. Thus, all conditions
are satisfied.Other valid arrays are $a = [4, 2, 15, 5, 7]$ and $a = [1, 4, 0, 27, 1]$.It can be shown that no valid arrays exist for the third test case.SCORING:Input 2: $M \leq 8, K \leq 8$Inputs 3-5: $M > 2^K$Inputs 6-18: No additional constraints.Problem credits: Aakash Gokhale

\section*{Sample Input}
\begin{verbatim}
3
2 1
33 5
10 5
\end{verbatim}

\section*{Sample Output}
\begin{verbatim}
2
2 0
3
3 23 7 
-1

In the first test case, the elements in the array $a = [2, 0]$ sum to $2$. The
xor sum of popcounts is $1 \oplus 0 = 1$. Thus, all the conditions are
satisfied.In the second test case, the elements in the array $a = [3, 23, 7]$ sum to $33$.
The xor sum of the popcounts is $2 \oplus 4 \oplus 3 = 5$. Thus, all conditions
are satisfied.Other valid arrays are $a = [4, 2, 15, 5, 7]$ and $a = [1, 4, 0, 27, 1]$.It can be shown that no valid arrays exist for the third test case.SCORING:Input 2: $M \leq 8, K \leq 8$Inputs 3-5: $M > 2^K$Inputs 6-18: No additional constraints.Problem credits: Aakash Gokhale

In the second test case, the elements in the array $a = [3, 23, 7]$ sum to $33$.
The xor sum of the popcounts is $2 \oplus 4 \oplus 3 = 5$. Thus, all conditions
are satisfied.Other valid arrays are $a = [4, 2, 15, 5, 7]$ and $a = [1, 4, 0, 27, 1]$.It can be shown that no valid arrays exist for the third test case.SCORING:Input 2: $M \leq 8, K \leq 8$Inputs 3-5: $M > 2^K$Inputs 6-18: No additional constraints.Problem credits: Aakash Gokhale

Other valid arrays are $a = [4, 2, 15, 5, 7]$ and $a = [1, 4, 0, 27, 1]$.It can be shown that no valid arrays exist for the third test case.SCORING:Input 2: $M \leq 8, K \leq 8$Inputs 3-5: $M > 2^K$Inputs 6-18: No additional constraints.Problem credits: Aakash Gokhale

It can be shown that no valid arrays exist for the third test case.SCORING:Input 2: $M \leq 8, K \leq 8$Inputs 3-5: $M > 2^K$Inputs 6-18: No additional constraints.Problem credits: Aakash Gokhale

SCORING:Input 2: $M \leq 8, K \leq 8$Inputs 3-5: $M > 2^K$Inputs 6-18: No additional constraints.Problem credits: Aakash Gokhale
\end{verbatim}

\section*{Solution}


\section*{Problem URL}
https://usaco.org/index.php?page=viewproblem2&cpid=1518

\section*{Source}
USACO OPEN25 Contest, Silver Division, Problem ID: 1518

\end{document}
