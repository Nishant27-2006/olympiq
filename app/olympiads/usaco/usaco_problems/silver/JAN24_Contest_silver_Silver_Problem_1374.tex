\documentclass[12pt]{article}
\usepackage{amsmath}
\usepackage{amssymb}
\usepackage{graphicx}
\usepackage{enumitem}
\usepackage{hyperref}
\usepackage{xcolor}

\title{View problem}
\author{USACO Contest: JAN24 Contest - Silver Division}
\date{\today}

\begin{document}

\maketitle

\section*{Problem Statement}


Farmer John is hiring a new herd leader for his cows. To that end, he has
interviewed $N$ ($2 \leq N \leq 10^5$) cows for the position. After interviewing
the $i$th candidate, he assigned the candidate an integer "cowmpetency" score
$c_i$ ranging from $1$ to $C$ inclusive ($1 \leq C \leq 10^9$) that is
correlated with their leadership abilities.

Because he has interviewed so many cows, Farmer John does not remember all of
their cowmpetency scores. However, he does remembers $Q$ ($1 \leq Q < N$) pairs
of numbers $(a_j, h_j)$ where cow $h_j$ was the first cow with a strictly
greater cowmpetency score than cows $1$ through $a_j$ (so
$1 \leq a_j < h_j \leq N$).

Farmer John now tells you the sequence $c_1, \dots, c_N$ (where $c_i = 0$ means
that he has forgotten cow $i$'s cowmpetency score) and the $Q$ pairs of
$(a_j, h_j)$. Help him determine the lexicographically smallest sequence
of cowmpetency scores consistent with this information, or that no such sequence
exists! A sequence of scores is lexicographically smaller than another sequence
of scores if it assigns a smaller score to the first cow at which the two
sequences differ.

Each input contains $T$ $(1 \leq T \leq 20)$ independent test cases. The sum of
$N$ across all test cases is guaranteed to not exceed  $3 \cdot 10^5$. 

INPUT FORMAT (input arrives from the terminal / stdin):
The first line contains $T$, the number of independent test cases.  Each test
case is described as follows:
First, a line containing $N$, $Q$, and $C$.Next, a line containing
the sequence $c_1, \dots, c_N$ $(0 \leq c_i \leq C)$.Finally, $Q$
lines each containing a pair $(a_j, h_j)$. It is guaranteed that all $a_j$
within a test case are distinct.

OUTPUT FORMAT (print output to the terminal / stdout):
For each test case, output a single line containing the lexicographically
smallest sequence of cowmpetency scores consistent with what Farmer John remembers, or $-1$ if such a sequence does not
exist.

SAMPLE INPUT:
1
7 3 5
1 0 2 3 0 4 0
1 2
3 4
4 5
SAMPLE OUTPUT: 
1 2 2 3 4 4 1

We can see that the given output satisfies all of Farmer John's remembered
pairs.

$\max(c_1) = 1$, $c_2 = 2$ and $1<2$ so the first pair is satisfied$\max(c_1,c_2,c_3) = 2$, $c_4 = 3$ and $2<3$ so the second pair is
satisfied$\max(c_1,c_2,c_3,c_4) = 3$, $c_5 = 4$ and $3<4$ so the third
pair is satisfied
There are several other sequences consistent with Farmer John's memory, such as

1 2 2 3 5 4 1
1 2 2 3 4 4 5

However, none of these are lexicographically smaller than the given output.

SAMPLE INPUT:
5
7 6 10
0 0 0 0 0 0 0
1 2
2 3
3 4
4 5
5 6
6 7
8 4 9
0 0 0 0 1 6 0 6
1 3
6 7
4 7
2 3
2 1 1
0 0
1 2
10 4 10
1 2 0 2 1 5 8 6 0 3
4 7
1 2
5 7
3 7
10 2 8
1 0 0 0 0 5 7 0 0 0
4 6
6 9
SAMPLE OUTPUT: 
1 2 3 4 5 6 7
1 1 2 6 1 6 7 6
-1
1 2 5 2 1 5 8 6 1 3
-1

In test case 3, since $C=1$, the only potential sequence is

1 1

However, in this case, cow 2 does not have a greater score than cow 1, so we
cannot satisfy the condition.

In test case 5, $a_1$ and $h_1$ tell us that cow 6 is the first cow to have a
strictly greater score than cows 1 through 4. Therefore, the largest score for
cows 1 through 6 is that of cow 6:  5. Since cow 7 has a score of 7, cow 7 is
the first cow to have a greater score than cows 1 through 6. So, the second
statement that cow 9 is the first cow to have a greater score than cows 1
through 6 cannot be true.

SCORING:
Input 3 satisfies $N \leq 10$ and $Q, C \leq 4$.Inputs 4-8 satisfy
$N \leq 1000$.Inputs 9-12 satisfy no additional constraints.


Problem credits: Suhas Nagar



\section*{Input Format}
OUTPUT FORMAT (print output to the terminal / stdout):For each test case, output a single line containing the lexicographically
smallest sequence of cowmpetency scores consistent with what Farmer John remembers, or $-1$ if such a sequence does not
exist.SAMPLE INPUT:1
7 3 5
1 0 2 3 0 4 0
1 2
3 4
4 5SAMPLE OUTPUT:1 2 2 3 4 4 1We can see that the given output satisfies all of Farmer John's remembered
pairs.$\max(c_1) = 1$, $c_2 = 2$ and $1<2$ so the first pair is satisfied$\max(c_1,c_2,c_3) = 2$, $c_4 = 3$ and $2<3$ so the second pair is
satisfied$\max(c_1,c_2,c_3,c_4) = 3$, $c_5 = 4$ and $3<4$ so the third
pair is satisfiedThere are several other sequences consistent with Farmer John's memory, such as1 2 2 3 5 4 1
1 2 2 3 4 4 5However, none of these are lexicographically smaller than the given output.SAMPLE INPUT:5
7 6 10
0 0 0 0 0 0 0
1 2
2 3
3 4
4 5
5 6
6 7
8 4 9
0 0 0 0 1 6 0 6
1 3
6 7
4 7
2 3
2 1 1
0 0
1 2
10 4 10
1 2 0 2 1 5 8 6 0 3
4 7
1 2
5 7
3 7
10 2 8
1 0 0 0 0 5 7 0 0 0
4 6
6 9SAMPLE OUTPUT:1 2 3 4 5 6 7
1 1 2 6 1 6 7 6
-1
1 2 5 2 1 5 8 6 1 3
-1In test case 3, since $C=1$, the only potential sequence is1 1However, in this case, cow 2 does not have a greater score than cow 1, so we
cannot satisfy the condition.In test case 5, $a_1$ and $h_1$ tell us that cow 6 is the first cow to have a
strictly greater score than cows 1 through 4. Therefore, the largest score for
cows 1 through 6 is that of cow 6:  5. Since cow 7 has a score of 7, cow 7 is
the first cow to have a greater score than cows 1 through 6. So, the second
statement that cow 9 is the first cow to have a greater score than cows 1
through 6 cannot be true.SCORING:Input 3 satisfies $N \leq 10$ and $Q, C \leq 4$.Inputs 4-8 satisfy
$N \leq 1000$.Inputs 9-12 satisfy no additional constraints.Problem credits: Suhas Nagar

\section*{Output Format}
SAMPLE INPUT:1
7 3 5
1 0 2 3 0 4 0
1 2
3 4
4 5SAMPLE OUTPUT:1 2 2 3 4 4 1We can see that the given output satisfies all of Farmer John's remembered
pairs.$\max(c_1) = 1$, $c_2 = 2$ and $1<2$ so the first pair is satisfied$\max(c_1,c_2,c_3) = 2$, $c_4 = 3$ and $2<3$ so the second pair is
satisfied$\max(c_1,c_2,c_3,c_4) = 3$, $c_5 = 4$ and $3<4$ so the third
pair is satisfiedThere are several other sequences consistent with Farmer John's memory, such as1 2 2 3 5 4 1
1 2 2 3 4 4 5However, none of these are lexicographically smaller than the given output.SAMPLE INPUT:5
7 6 10
0 0 0 0 0 0 0
1 2
2 3
3 4
4 5
5 6
6 7
8 4 9
0 0 0 0 1 6 0 6
1 3
6 7
4 7
2 3
2 1 1
0 0
1 2
10 4 10
1 2 0 2 1 5 8 6 0 3
4 7
1 2
5 7
3 7
10 2 8
1 0 0 0 0 5 7 0 0 0
4 6
6 9SAMPLE OUTPUT:1 2 3 4 5 6 7
1 1 2 6 1 6 7 6
-1
1 2 5 2 1 5 8 6 1 3
-1In test case 3, since $C=1$, the only potential sequence is1 1However, in this case, cow 2 does not have a greater score than cow 1, so we
cannot satisfy the condition.In test case 5, $a_1$ and $h_1$ tell us that cow 6 is the first cow to have a
strictly greater score than cows 1 through 4. Therefore, the largest score for
cows 1 through 6 is that of cow 6:  5. Since cow 7 has a score of 7, cow 7 is
the first cow to have a greater score than cows 1 through 6. So, the second
statement that cow 9 is the first cow to have a greater score than cows 1
through 6 cannot be true.SCORING:Input 3 satisfies $N \leq 10$ and $Q, C \leq 4$.Inputs 4-8 satisfy
$N \leq 1000$.Inputs 9-12 satisfy no additional constraints.Problem credits: Suhas Nagar

\section*{Sample Input}
\begin{verbatim}
1
7 3 5
1 0 2 3 0 4 0
1 2
3 4
4 5

5
7 6 10
0 0 0 0 0 0 0
1 2
2 3
3 4
4 5
5 6
6 7
8 4 9
0 0 0 0 1 6 0 6
1 3
6 7
4 7
2 3
2 1 1
0 0
1 2
10 4 10
1 2 0 2 1 5 8 6 0 3
4 7
1 2
5 7
3 7
10 2 8
1 0 0 0 0 5 7 0 0 0
4 6
6 9
\end{verbatim}

\section*{Sample Output}
\begin{verbatim}
1 2 2 3 4 4 1

$\max(c_1) = 1$, $c_2 = 2$ and $1<2$ so the first pair is satisfied$\max(c_1,c_2,c_3) = 2$, $c_4 = 3$ and $2<3$ so the second pair is
satisfied$\max(c_1,c_2,c_3,c_4) = 3$, $c_5 = 4$ and $3<4$ so the third
pair is satisfiedThere are several other sequences consistent with Farmer John's memory, such as1 2 2 3 5 4 1
1 2 2 3 4 4 5However, none of these are lexicographically smaller than the given output.SAMPLE INPUT:5
7 6 10
0 0 0 0 0 0 0
1 2
2 3
3 4
4 5
5 6
6 7
8 4 9
0 0 0 0 1 6 0 6
1 3
6 7
4 7
2 3
2 1 1
0 0
1 2
10 4 10
1 2 0 2 1 5 8 6 0 3
4 7
1 2
5 7
3 7
10 2 8
1 0 0 0 0 5 7 0 0 0
4 6
6 9SAMPLE OUTPUT:1 2 3 4 5 6 7
1 1 2 6 1 6 7 6
-1
1 2 5 2 1 5 8 6 1 3
-1In test case 3, since $C=1$, the only potential sequence is1 1However, in this case, cow 2 does not have a greater score than cow 1, so we
cannot satisfy the condition.In test case 5, $a_1$ and $h_1$ tell us that cow 6 is the first cow to have a
strictly greater score than cows 1 through 4. Therefore, the largest score for
cows 1 through 6 is that of cow 6:  5. Since cow 7 has a score of 7, cow 7 is
the first cow to have a greater score than cows 1 through 6. So, the second
statement that cow 9 is the first cow to have a greater score than cows 1
through 6 cannot be true.SCORING:Input 3 satisfies $N \leq 10$ and $Q, C \leq 4$.Inputs 4-8 satisfy
$N \leq 1000$.Inputs 9-12 satisfy no additional constraints.Problem credits: Suhas Nagar

1 2 2 3 5 4 1
1 2 2 3 4 4 5

SAMPLE INPUT:5
7 6 10
0 0 0 0 0 0 0
1 2
2 3
3 4
4 5
5 6
6 7
8 4 9
0 0 0 0 1 6 0 6
1 3
6 7
4 7
2 3
2 1 1
0 0
1 2
10 4 10
1 2 0 2 1 5 8 6 0 3
4 7
1 2
5 7
3 7
10 2 8
1 0 0 0 0 5 7 0 0 0
4 6
6 9SAMPLE OUTPUT:1 2 3 4 5 6 7
1 1 2 6 1 6 7 6
-1
1 2 5 2 1 5 8 6 1 3
-1In test case 3, since $C=1$, the only potential sequence is1 1However, in this case, cow 2 does not have a greater score than cow 1, so we
cannot satisfy the condition.In test case 5, $a_1$ and $h_1$ tell us that cow 6 is the first cow to have a
strictly greater score than cows 1 through 4. Therefore, the largest score for
cows 1 through 6 is that of cow 6:  5. Since cow 7 has a score of 7, cow 7 is
the first cow to have a greater score than cows 1 through 6. So, the second
statement that cow 9 is the first cow to have a greater score than cows 1
through 6 cannot be true.SCORING:Input 3 satisfies $N \leq 10$ and $Q, C \leq 4$.Inputs 4-8 satisfy
$N \leq 1000$.Inputs 9-12 satisfy no additional constraints.Problem credits: Suhas Nagar

1 2 3 4 5 6 7
1 1 2 6 1 6 7 6
-1
1 2 5 2 1 5 8 6 1 3
-1

In test case 3, since $C=1$, the only potential sequence is1 1However, in this case, cow 2 does not have a greater score than cow 1, so we
cannot satisfy the condition.In test case 5, $a_1$ and $h_1$ tell us that cow 6 is the first cow to have a
strictly greater score than cows 1 through 4. Therefore, the largest score for
cows 1 through 6 is that of cow 6:  5. Since cow 7 has a score of 7, cow 7 is
the first cow to have a greater score than cows 1 through 6. So, the second
statement that cow 9 is the first cow to have a greater score than cows 1
through 6 cannot be true.SCORING:Input 3 satisfies $N \leq 10$ and $Q, C \leq 4$.Inputs 4-8 satisfy
$N \leq 1000$.Inputs 9-12 satisfy no additional constraints.Problem credits: Suhas Nagar

1 1

In test case 5, $a_1$ and $h_1$ tell us that cow 6 is the first cow to have a
strictly greater score than cows 1 through 4. Therefore, the largest score for
cows 1 through 6 is that of cow 6:  5. Since cow 7 has a score of 7, cow 7 is
the first cow to have a greater score than cows 1 through 6. So, the second
statement that cow 9 is the first cow to have a greater score than cows 1
through 6 cannot be true.SCORING:Input 3 satisfies $N \leq 10$ and $Q, C \leq 4$.Inputs 4-8 satisfy
$N \leq 1000$.Inputs 9-12 satisfy no additional constraints.Problem credits: Suhas Nagar

SCORING:Input 3 satisfies $N \leq 10$ and $Q, C \leq 4$.Inputs 4-8 satisfy
$N \leq 1000$.Inputs 9-12 satisfy no additional constraints.Problem credits: Suhas Nagar
\end{verbatim}

\section*{Solution}


\section*{Problem URL}
https://usaco.org/index.php?page=viewproblem2&cpid=1374

\section*{Source}
USACO JAN24 Contest, Silver Division, Problem ID: 1374

\end{document}
