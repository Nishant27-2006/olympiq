\documentclass[12pt]{article}
\usepackage{amsmath}
\usepackage{amssymb}
\usepackage{graphicx}
\usepackage{enumitem}
\usepackage{hyperref}
\usepackage{xcolor}

\title{View problem}
\author{USACO Contest: JAN25 Contest - Silver Division}
\date{\today}

\begin{document}

\maketitle

\section*{Problem Statement}


It is another cold and boring day on Farmer John's farm. To pass the time, Farmer
John has invented a fun leisure activity involving performing operations on an
integer array.

Farmer John has an array $a$ of $N$ ($1 \leq N \leq 2 \cdot 10^5$)
non-negative integers and an integer $M$ ($1 \leq M \leq 10^9$). Then,
FJ will ask Bessie for an integer $x$. In one operation, FJ can pick
an index $i$ and subtract or add $1$ to $a_i$. FJ's boredom value is
the minimum number of operations he must perform so that $a_i-x$ is
divisible by $M$ for all $1 \leq i \leq N$.

Among all possible $x$, output FJ's minimum possible boredom value.

INPUT FORMAT (input arrives from the terminal / stdin):
The first line contains $T$ ($1 \leq T \leq 10$), the number of independent test
cases to solve.

The first line of each test case contains $N$ and $M$.

The second line of each test case contains $a_1, a_2, ..., a_N$
($0 \leq a_i \leq 10^9$).

It is guaranteed that the sum of $N$ over all test cases does not exceed
$5 \cdot 10^5$.


OUTPUT FORMAT (print output to the terminal / stdout):
For each test case, output an integer on a new line containing FJ's minimum
possible boredom value among all possible values of $x$.

SAMPLE INPUT:
2
5 9
15 12 18 3 8
3 69
1 988244353 998244853
SAMPLE OUTPUT: 
10
21

In the first test case, one optimal choice of $x$ is $3$. FJ can perform $10$
operations to make
$a = [12, 12, 21, 3, 12]$.

SCORING:
Input 2: $N \le 1000$ and $M \le 1000$.Input 3:
$N\le 1000$.Inputs 4-5: $M\le 10^5$.Inputs 6-16: No
additional constraints.


Problem credits: Chongtian Ma



\section*{Input Format}
The first line contains $T$ ($1 \leq T \leq 10$), the number of independent test
cases to solve.The first line of each test case contains $N$ and $M$.The second line of each test case contains $a_1, a_2, ..., a_N$
($0 \leq a_i \leq 10^9$).It is guaranteed that the sum of $N$ over all test cases does not exceed
$5 \cdot 10^5$.

The first line of each test case contains $N$ and $M$.The second line of each test case contains $a_1, a_2, ..., a_N$
($0 \leq a_i \leq 10^9$).It is guaranteed that the sum of $N$ over all test cases does not exceed
$5 \cdot 10^5$.

The second line of each test case contains $a_1, a_2, ..., a_N$
($0 \leq a_i \leq 10^9$).It is guaranteed that the sum of $N$ over all test cases does not exceed
$5 \cdot 10^5$.

It is guaranteed that the sum of $N$ over all test cases does not exceed
$5 \cdot 10^5$.

OUTPUT FORMAT (print output to the terminal / stdout):For each test case, output an integer on a new line containing FJ's minimum
possible boredom value among all possible values of $x$.SAMPLE INPUT:2
5 9
15 12 18 3 8
3 69
1 988244353 998244853SAMPLE OUTPUT:10
21In the first test case, one optimal choice of $x$ is $3$. FJ can perform $10$
operations to make
$a = [12, 12, 21, 3, 12]$.SCORING:Input 2: $N \le 1000$ and $M \le 1000$.Input 3:
$N\le 1000$.Inputs 4-5: $M\le 10^5$.Inputs 6-16: No
additional constraints.Problem credits: Chongtian Ma

\section*{Output Format}
SAMPLE INPUT:2
5 9
15 12 18 3 8
3 69
1 988244353 998244853SAMPLE OUTPUT:10
21In the first test case, one optimal choice of $x$ is $3$. FJ can perform $10$
operations to make
$a = [12, 12, 21, 3, 12]$.SCORING:Input 2: $N \le 1000$ and $M \le 1000$.Input 3:
$N\le 1000$.Inputs 4-5: $M\le 10^5$.Inputs 6-16: No
additional constraints.Problem credits: Chongtian Ma

\section*{Sample Input}
\begin{verbatim}
2
5 9
15 12 18 3 8
3 69
1 988244353 998244853
\end{verbatim}

\section*{Sample Output}
\begin{verbatim}
10
21

In the first test case, one optimal choice of $x$ is $3$. FJ can perform $10$
operations to make
$a = [12, 12, 21, 3, 12]$.SCORING:Input 2: $N \le 1000$ and $M \le 1000$.Input 3:
$N\le 1000$.Inputs 4-5: $M\le 10^5$.Inputs 6-16: No
additional constraints.Problem credits: Chongtian Ma

SCORING:Input 2: $N \le 1000$ and $M \le 1000$.Input 3:
$N\le 1000$.Inputs 4-5: $M\le 10^5$.Inputs 6-16: No
additional constraints.Problem credits: Chongtian Ma
\end{verbatim}

\section*{Solution}


\section*{Problem URL}
https://usaco.org/index.php?page=viewproblem2&cpid=1471

\section*{Source}
USACO JAN25 Contest, Silver Division, Problem ID: 1471

\end{document}
