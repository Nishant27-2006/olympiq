\documentclass[12pt]{article}
\usepackage{amsmath}
\usepackage{amssymb}
\usepackage{graphicx}
\usepackage{enumitem}
\usepackage{hyperref}
\usepackage{xcolor}

\title{View problem}
\author{USACO Contest: JAN17 Contest - Silver Division}
\date{\today}

\begin{document}

\maketitle

\section*{Problem Statement}

The cows are experimenting with secret codes, and have devised a method for
creating an infinite-length string to be used as part of one of their codes.

Given a string $s$, let $F(s)$ be $s$ followed by $s$ "rotated" one character to
the right (in a right rotation, the last character of $s$ rotates around and
becomes the new first character).  Given an initial string $s$, the cows build
their infinite-length code string by repeatedly applying $F$; each step
therefore doubles the length of the current string.

Given the initial string and an index $N$, please help the cows compute the
character at the $N$th position within the infinite code string.

INPUT FORMAT (file cowcode.in):
The input consists of a single line containing a string followed by $N$.  The
string consists of at most 30 uppercase characters, and $N \leq 10^{18}$.

Note that $N$ may be too large to fit into a standard 32-bit integer, so you may
want to use a 64-bit integer type (e.g., a "long long" in C/C++).

OUTPUT FORMAT (file cowcode.out):
Please output the $N$th character of the infinite code built from the initial
string.  The first character is $N=1$.

SAMPLE INPUT:
COW 8
SAMPLE OUTPUT: 
C

In this example, the initial string COW expands as follows:

COW -> COWWCO -> COWWCOOCOWWC
                 12345678


Problem credits: Brian Dean



\section*{Input Format}
Note that $N$ may be too large to fit into a standard 32-bit integer, so you may
want to use a 64-bit integer type (e.g., a "long long" in C/C++).

OUTPUT FORMAT (file cowcode.out):Please output the $N$th character of the infinite code built from the initial
string.  The first character is $N=1$.SAMPLE INPUT:COW 8SAMPLE OUTPUT:CIn this example, the initial string COW expands as follows:COW -> COWWCO -> COWWCOOCOWWC
                 12345678Problem credits: Brian Dean

\section*{Output Format}
SAMPLE INPUT:COW 8SAMPLE OUTPUT:CIn this example, the initial string COW expands as follows:COW -> COWWCO -> COWWCOOCOWWC
                 12345678Problem credits: Brian Dean

\section*{Sample Input}
\begin{verbatim}
COW 8
\end{verbatim}

\section*{Sample Output}
\begin{verbatim}
C

In this example, the initial string COW expands as follows:COW -> COWWCO -> COWWCOOCOWWC
                 12345678Problem credits: Brian Dean

COW -> COWWCO -> COWWCOOCOWWC
                 12345678

Problem credits: Brian Dean

Problem credits: Brian Dean
\end{verbatim}

\section*{Solution}


\section*{Problem URL}
https://usaco.org/index.php?page=viewproblem2&cpid=692

\section*{Source}
USACO JAN17 Contest, Silver Division, Problem ID: 692

\end{document}
