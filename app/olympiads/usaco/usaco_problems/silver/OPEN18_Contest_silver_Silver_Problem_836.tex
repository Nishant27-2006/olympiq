\documentclass[12pt]{article}
\usepackage{amsmath}
\usepackage{amssymb}
\usepackage{graphicx}
\usepackage{enumitem}
\usepackage{hyperref}
\usepackage{xcolor}

\title{View problem}
\author{USACO Contest: OPEN18 Contest - Silver Division}
\date{\today}

\begin{document}

\maketitle

\section*{Problem Statement}

The cows have come up with a creative new game, surprisingly giving it the least
creative name possible: "Moo".

The game of Moo is played on an $N \times N$ grid of square cells, where a cow
claims a  grid cell by yelling "moo!" and writing her numeric ID number in the
cell.  

At the end of the game, every cell contains a number.  At this point, a cow wins
the game if she has created a region of connected cells as least as large as any
other region.  A "region" is defined as a group of cells all with the same ID
number, where every cell in the region is directly adjacent to some other cell
in the same region either above, below, left, or to the right (diagonals don't
count).  

Since it is a bit boring to play as individuals, the cows are also interested in
pairing up to play as teams.  A team of two cows can create a region as before,
but now the cells in the region can belong to either of the two cows on the
team.

Given the final state of the game board, please help the cows compute the number
of cells  belonging to the largest region that any one cow owns, and the number
of cells belonging  to the largest region that can be claimed by a two-cow team.
A region claimed by a two-cow team only counts if it contains the ID numbers of
both cows on the team, not just one of the cows.

INPUT FORMAT (file multimoo.in):
The first line of input contains $N$ ($1 \leq N \leq 250$).  The next $N$ lines
each contain $N$ integers (each in the range $0 \ldots 10^6$), describing the
final state of the game board.  At least two distinct ID numbers will be present
in the board.

OUTPUT FORMAT (file multimoo.out):
The first line of output should describe the largest region size claimed by any
single cow, and the second line of output should describe the largest region
size claimed by any team of two cows.  

SAMPLE INPUT:
4
2 3 9 3
4 9 9 1
9 9 1 7
2 1 1 9
SAMPLE OUTPUT: 
5
10

In this example, the largest region for a single cow consists of five 9s.  If 
cows with IDs 1 and 9 team up, they can form a region of size 10.


Problem credits: Brian Dean



\section*{Input Format}
OUTPUT FORMAT (file multimoo.out):The first line of output should describe the largest region size claimed by any
single cow, and the second line of output should describe the largest region
size claimed by any team of two cows.SAMPLE INPUT:4
2 3 9 3
4 9 9 1
9 9 1 7
2 1 1 9SAMPLE OUTPUT:5
10In this example, the largest region for a single cow consists of five 9s.  If 
cows with IDs 1 and 9 team up, they can form a region of size 10.Problem credits: Brian Dean

\section*{Output Format}
SAMPLE INPUT:4
2 3 9 3
4 9 9 1
9 9 1 7
2 1 1 9SAMPLE OUTPUT:5
10In this example, the largest region for a single cow consists of five 9s.  If 
cows with IDs 1 and 9 team up, they can form a region of size 10.Problem credits: Brian Dean

\section*{Sample Input}
\begin{verbatim}
4
2 3 9 3
4 9 9 1
9 9 1 7
2 1 1 9
\end{verbatim}

\section*{Sample Output}
\begin{verbatim}
5
10

In this example, the largest region for a single cow consists of five 9s.  If 
cows with IDs 1 and 9 team up, they can form a region of size 10.Problem credits: Brian Dean

Problem credits: Brian Dean

Problem credits: Brian Dean
\end{verbatim}

\section*{Solution}


\section*{Problem URL}
https://usaco.org/index.php?page=viewproblem2&cpid=836

\section*{Source}
USACO OPEN18 Contest, Silver Division, Problem ID: 836

\end{document}
