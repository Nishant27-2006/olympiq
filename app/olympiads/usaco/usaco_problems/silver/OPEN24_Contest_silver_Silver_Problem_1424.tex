\documentclass[12pt]{article}
\usepackage{amsmath}
\usepackage{amssymb}
\usepackage{graphicx}
\usepackage{enumitem}
\usepackage{hyperref}
\usepackage{xcolor}

\title{View problem}
\author{USACO Contest: OPEN24 Contest - Silver Division}
\date{\today}

\begin{document}

\maketitle

\section*{Problem Statement}


**Note: The memory limit for this problem is 512MB, twice the default.**
After years of hosting games and watching Bessie get first place over and over,
Farmer John has realized that this can't be accidental. Instead, he concludes
that Bessie must have winning coded into her DNA so he sets out to find this
"winning" gene.

He devises a process to identify possible candidates for this "winning" gene. He
takes Bessie's genome, which is a string $S$ of length $N$ where
$1 \leq N \leq 3000$. He picks some pair $(K,L)$ where $1 \leq L \leq K \leq N$
representing that the "winning" gene candidates will have length $L$ and will be
found within a larger $K$ length substring. To identify the gene, he takes all
$K$ length substrings from $S$ which we will call a $k$-mer. For a given
$k$-mer, he takes all length $L$ substrings, identifies the lexicographically
minimal substring as a winning gene candidate (choosing the leftmost such
substring if there is a tie),  and then writes down the $0$-indexed position
$p_i$ where that substring starts in $S$ to a set $P$. 

Since he hasn't picked $K$ and $L$ yet, he wants to know how many candidates
there will be for every pair of $(K,L)$.

For each $v$ in $1\dots N$, help him determine the number of $(K,L)$ pairs with
$|P|=v$.

INPUT FORMAT (input arrives from the terminal / stdin):
$N$ representing the length of the string. $S$ representing the given string.
All characters are guaranteed to be uppercase characters where $s_i \in A-Z$
since bovine genetics are far more advanced than ours.

OUTPUT FORMAT (print output to the terminal / stdout):
For each $v$ in $1\dots N$, output the number of $(K,L)$ pairs with $|P|=v$,
with each number on a separate line.

SAMPLE INPUT:
8
AGTCAACG
SAMPLE OUTPUT: 
11
10
5
4
2
2
1
1

In this test case, the third line of the output is 5 because we see that there are exactly 5 pairs of $K$ and $L$ that allow for
three "winning" gene candidates.  These candidates are (where $p_i$ is $0$-indexed):


(4,2) -> P = [0,3,4]
(5,3) -> P = [0,3,4]
(6,4) -> P = [0,3,4]
(6,5) -> P = [0,1,3]
(6,6) -> P = [0,1,2]

To see how (4,2) leads to these results, we take all $4$-mers

AGTC
GTCA
TCAA
CAAC
AACG

For each $4$-mer, we identify the lexicographically minimal length 2 substring

AGTC -> AG
GTCA -> CA
TCAA -> AA
CAAC -> AA
AACG -> AA

We take the positions of all these substrings in the original string and add
them to a set $P$ to get $P = [0,3,4]$.

On the other hand, if we focus on the pair $(4,1)$, we see that this only leads
to $2$ total "winning" gene candidates. If we take all $4$-mers and identify the
lexicographically minimum length $1$ substring (using A and A' and A* to
distinguish the different As), we get

AGTC -> A
GTCA' -> A'
TCA'A* -> A'
CA'A*C -> A'
A'A*CG -> A'

While both A' and A* are lexicographically minimal in the last 3 cases, the
leftmost substring takes precedence so A' is counted as the only candidate in
all of these cases. This means that $P = [0,4]$.

SCORING:
Inputs 2-4: $N \leq 100$ Inputs 5-7: $N \leq 500$ Inputs 8-16: No additional constraints. 


Problem credits: Suhas Nagar



\section*{Input Format}
OUTPUT FORMAT (print output to the terminal / stdout):For each $v$ in $1\dots N$, output the number of $(K,L)$ pairs with $|P|=v$,
with each number on a separate line.SAMPLE INPUT:8
AGTCAACGSAMPLE OUTPUT:11
10
5
4
2
2
1
1In this test case, the third line of the output is 5 because we see that there are exactly 5 pairs of $K$ and $L$ that allow for
three "winning" gene candidates.  These candidates are (where $p_i$ is $0$-indexed):(4,2) -> P = [0,3,4]
(5,3) -> P = [0,3,4]
(6,4) -> P = [0,3,4]
(6,5) -> P = [0,1,3]
(6,6) -> P = [0,1,2]To see how (4,2) leads to these results, we take all $4$-mersAGTC
GTCA
TCAA
CAAC
AACGFor each $4$-mer, we identify the lexicographically minimal length 2 substringAGTC -> AG
GTCA -> CA
TCAA -> AA
CAAC -> AA
AACG -> AAWe take the positions of all these substrings in the original string and add
them to a set $P$ to get $P = [0,3,4]$.On the other hand, if we focus on the pair $(4,1)$, we see that this only leads
to $2$ total "winning" gene candidates. If we take all $4$-mers and identify the
lexicographically minimum length $1$ substring (using A and A' and A* to
distinguish the different As), we getAGTC -> A
GTCA' -> A'
TCA'A* -> A'
CA'A*C -> A'
A'A*CG -> A'While both A' and A* are lexicographically minimal in the last 3 cases, the
leftmost substring takes precedence so A' is counted as the only candidate in
all of these cases. This means that $P = [0,4]$.SCORING:Inputs 2-4: $N \leq 100$Inputs 5-7: $N \leq 500$Inputs 8-16: No additional constraints.Problem credits: Suhas Nagar

\section*{Output Format}
SAMPLE INPUT:8
AGTCAACGSAMPLE OUTPUT:11
10
5
4
2
2
1
1In this test case, the third line of the output is 5 because we see that there are exactly 5 pairs of $K$ and $L$ that allow for
three "winning" gene candidates.  These candidates are (where $p_i$ is $0$-indexed):(4,2) -> P = [0,3,4]
(5,3) -> P = [0,3,4]
(6,4) -> P = [0,3,4]
(6,5) -> P = [0,1,3]
(6,6) -> P = [0,1,2]To see how (4,2) leads to these results, we take all $4$-mersAGTC
GTCA
TCAA
CAAC
AACGFor each $4$-mer, we identify the lexicographically minimal length 2 substringAGTC -> AG
GTCA -> CA
TCAA -> AA
CAAC -> AA
AACG -> AAWe take the positions of all these substrings in the original string and add
them to a set $P$ to get $P = [0,3,4]$.On the other hand, if we focus on the pair $(4,1)$, we see that this only leads
to $2$ total "winning" gene candidates. If we take all $4$-mers and identify the
lexicographically minimum length $1$ substring (using A and A' and A* to
distinguish the different As), we getAGTC -> A
GTCA' -> A'
TCA'A* -> A'
CA'A*C -> A'
A'A*CG -> A'While both A' and A* are lexicographically minimal in the last 3 cases, the
leftmost substring takes precedence so A' is counted as the only candidate in
all of these cases. This means that $P = [0,4]$.SCORING:Inputs 2-4: $N \leq 100$Inputs 5-7: $N \leq 500$Inputs 8-16: No additional constraints.Problem credits: Suhas Nagar

\section*{Sample Input}
\begin{verbatim}
8
AGTCAACG
\end{verbatim}

\section*{Sample Output}
\begin{verbatim}
11
10
5
4
2
2
1
1

In this test case, the third line of the output is 5 because we see that there are exactly 5 pairs of $K$ and $L$ that allow for
three "winning" gene candidates.  These candidates are (where $p_i$ is $0$-indexed):(4,2) -> P = [0,3,4]
(5,3) -> P = [0,3,4]
(6,4) -> P = [0,3,4]
(6,5) -> P = [0,1,3]
(6,6) -> P = [0,1,2]To see how (4,2) leads to these results, we take all $4$-mersAGTC
GTCA
TCAA
CAAC
AACGFor each $4$-mer, we identify the lexicographically minimal length 2 substringAGTC -> AG
GTCA -> CA
TCAA -> AA
CAAC -> AA
AACG -> AAWe take the positions of all these substrings in the original string and add
them to a set $P$ to get $P = [0,3,4]$.On the other hand, if we focus on the pair $(4,1)$, we see that this only leads
to $2$ total "winning" gene candidates. If we take all $4$-mers and identify the
lexicographically minimum length $1$ substring (using A and A' and A* to
distinguish the different As), we getAGTC -> A
GTCA' -> A'
TCA'A* -> A'
CA'A*C -> A'
A'A*CG -> A'While both A' and A* are lexicographically minimal in the last 3 cases, the
leftmost substring takes precedence so A' is counted as the only candidate in
all of these cases. This means that $P = [0,4]$.SCORING:Inputs 2-4: $N \leq 100$Inputs 5-7: $N \leq 500$Inputs 8-16: No additional constraints.Problem credits: Suhas Nagar

(4,2) -> P = [0,3,4]
(5,3) -> P = [0,3,4]
(6,4) -> P = [0,3,4]
(6,5) -> P = [0,1,3]
(6,6) -> P = [0,1,2]To see how (4,2) leads to these results, we take all $4$-mersAGTC
GTCA
TCAA
CAAC
AACGFor each $4$-mer, we identify the lexicographically minimal length 2 substringAGTC -> AG
GTCA -> CA
TCAA -> AA
CAAC -> AA
AACG -> AAWe take the positions of all these substrings in the original string and add
them to a set $P$ to get $P = [0,3,4]$.On the other hand, if we focus on the pair $(4,1)$, we see that this only leads
to $2$ total "winning" gene candidates. If we take all $4$-mers and identify the
lexicographically minimum length $1$ substring (using A and A' and A* to
distinguish the different As), we getAGTC -> A
GTCA' -> A'
TCA'A* -> A'
CA'A*C -> A'
A'A*CG -> A'While both A' and A* are lexicographically minimal in the last 3 cases, the
leftmost substring takes precedence so A' is counted as the only candidate in
all of these cases. This means that $P = [0,4]$.SCORING:Inputs 2-4: $N \leq 100$Inputs 5-7: $N \leq 500$Inputs 8-16: No additional constraints.Problem credits: Suhas Nagar

(4,2) -> P = [0,3,4]
(5,3) -> P = [0,3,4]
(6,4) -> P = [0,3,4]
(6,5) -> P = [0,1,3]
(6,6) -> P = [0,1,2]

AGTC
GTCA
TCAA
CAAC
AACG

AGTC -> AG
GTCA -> CA
TCAA -> AA
CAAC -> AA
AACG -> AA

On the other hand, if we focus on the pair $(4,1)$, we see that this only leads
to $2$ total "winning" gene candidates. If we take all $4$-mers and identify the
lexicographically minimum length $1$ substring (using A and A' and A* to
distinguish the different As), we getAGTC -> A
GTCA' -> A'
TCA'A* -> A'
CA'A*C -> A'
A'A*CG -> A'While both A' and A* are lexicographically minimal in the last 3 cases, the
leftmost substring takes precedence so A' is counted as the only candidate in
all of these cases. This means that $P = [0,4]$.SCORING:Inputs 2-4: $N \leq 100$Inputs 5-7: $N \leq 500$Inputs 8-16: No additional constraints.Problem credits: Suhas Nagar

AGTC -> A
GTCA' -> A'
TCA'A* -> A'
CA'A*C -> A'
A'A*CG -> A'

SCORING:Inputs 2-4: $N \leq 100$Inputs 5-7: $N \leq 500$Inputs 8-16: No additional constraints.Problem credits: Suhas Nagar
\end{verbatim}

\section*{Solution}


\section*{Problem URL}
https://usaco.org/index.php?page=viewproblem2&cpid=1424

\section*{Source}
USACO OPEN24 Contest, Silver Division, Problem ID: 1424

\end{document}
