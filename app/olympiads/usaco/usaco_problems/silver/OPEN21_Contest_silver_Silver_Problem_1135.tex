\documentclass[12pt]{article}
\usepackage{amsmath}
\usepackage{amssymb}
\usepackage{graphicx}
\usepackage{enumitem}
\usepackage{hyperref}
\usepackage{xcolor}

\title{View problem}
\author{USACO Contest: OPEN21 Contest - Silver Division}
\date{\today}

\begin{document}

\maketitle

\section*{Problem Statement}

Farmer John's cows have been holding a daily online gathering on the "mooZ" 
video meeting platform.  For fun, they have invented a simple number game to
play during the meeting to keep themselves entertained.

Elsie has three positive integers $A$, $B$, and $C$ ($1\le A\le B\le C$). These
integers are supposed to be secret, so she will not directly reveal them to her
sister Bessie.  Instead, she tells Bessie $N$ ($4\le N\le 7$) distinct
integers $x_1,x_2,\ldots,x_N$ ($1\le x_i\le 10^9$), claiming that each $x_i$ is
one of $A$, $B$, $C$, $A+B$, $B+C$, $C+A$, or $A+B+C$. However, Elsie may be
lying; the integers $x_i$ might not correspond to any valid triple $(A,B,C)$.

This is too hard for Bessie to wrap her head around, so it is up to you to
determine the number of triples $(A,B,C)$ that are consistent with the numbers Elsie
presented (possibly zero).

Each input file will contain $T$ ($1\le T\le 100$) test cases that should be
solved independently.

INPUT FORMAT (input arrives from the terminal / stdin):
The first input line contains $T$.

Each test case starts with $N$, the number of integers Elsie gives to Bessie.

The second line of each test case contains $N$ distinct integers
$x_1,x_2,\ldots,x_N$.

OUTPUT FORMAT (print output to the terminal / stdout):
For each test case, output the number of triples $(A,B,C)$ that are consistent
with the numbers Elsie presented.

SAMPLE INPUT:
10
7
1 2 3 4 5 6 7
4
4 5 7 8
4
4 5 7 9
4
4 5 7 10
4
4 5 7 11
4
4 5 7 12
4
4 5 7 13
4
4 5 7 14
4
4 5 7 15
4
4 5 7 16
SAMPLE OUTPUT: 
1
3
5
1
4
3
0
0
0
1

For $x=\{4,5,7,9\}$, the five possible triples are as follows:

$$(2, 2, 5), (2, 3, 4), (2, 4, 5), (3, 4, 5), (4, 5, 7).$$
SCORING:
In test cases 1-4, all $x_i$ are at most $50$.Test cases 5-6 satisfy $N=7$.Test cases 7-15 satisfy no additional constraints.


Problem credits: Benjamin Qi



\section*{Input Format}
Each test case starts with $N$, the number of integers Elsie gives to Bessie.The second line of each test case contains $N$ distinct integers
$x_1,x_2,\ldots,x_N$.

The second line of each test case contains $N$ distinct integers
$x_1,x_2,\ldots,x_N$.

OUTPUT FORMAT (print output to the terminal / stdout):For each test case, output the number of triples $(A,B,C)$ that are consistent
with the numbers Elsie presented.SAMPLE INPUT:10
7
1 2 3 4 5 6 7
4
4 5 7 8
4
4 5 7 9
4
4 5 7 10
4
4 5 7 11
4
4 5 7 12
4
4 5 7 13
4
4 5 7 14
4
4 5 7 15
4
4 5 7 16SAMPLE OUTPUT:1
3
5
1
4
3
0
0
0
1For $x=\{4,5,7,9\}$, the five possible triples are as follows:$$(2, 2, 5), (2, 3, 4), (2, 4, 5), (3, 4, 5), (4, 5, 7).$$SCORING:In test cases 1-4, all $x_i$ are at most $50$.Test cases 5-6 satisfy $N=7$.Test cases 7-15 satisfy no additional constraints.Problem credits: Benjamin Qi

\section*{Output Format}
SAMPLE INPUT:10
7
1 2 3 4 5 6 7
4
4 5 7 8
4
4 5 7 9
4
4 5 7 10
4
4 5 7 11
4
4 5 7 12
4
4 5 7 13
4
4 5 7 14
4
4 5 7 15
4
4 5 7 16SAMPLE OUTPUT:1
3
5
1
4
3
0
0
0
1For $x=\{4,5,7,9\}$, the five possible triples are as follows:$$(2, 2, 5), (2, 3, 4), (2, 4, 5), (3, 4, 5), (4, 5, 7).$$SCORING:In test cases 1-4, all $x_i$ are at most $50$.Test cases 5-6 satisfy $N=7$.Test cases 7-15 satisfy no additional constraints.Problem credits: Benjamin Qi

\section*{Sample Input}
\begin{verbatim}
10
7
1 2 3 4 5 6 7
4
4 5 7 8
4
4 5 7 9
4
4 5 7 10
4
4 5 7 11
4
4 5 7 12
4
4 5 7 13
4
4 5 7 14
4
4 5 7 15
4
4 5 7 16
\end{verbatim}

\section*{Sample Output}
\begin{verbatim}
1
3
5
1
4
3
0
0
0
1

For $x=\{4,5,7,9\}$, the five possible triples are as follows:$$(2, 2, 5), (2, 3, 4), (2, 4, 5), (3, 4, 5), (4, 5, 7).$$SCORING:In test cases 1-4, all $x_i$ are at most $50$.Test cases 5-6 satisfy $N=7$.Test cases 7-15 satisfy no additional constraints.Problem credits: Benjamin Qi

$$(2, 2, 5), (2, 3, 4), (2, 4, 5), (3, 4, 5), (4, 5, 7).$$SCORING:In test cases 1-4, all $x_i$ are at most $50$.Test cases 5-6 satisfy $N=7$.Test cases 7-15 satisfy no additional constraints.Problem credits: Benjamin Qi

SCORING:In test cases 1-4, all $x_i$ are at most $50$.Test cases 5-6 satisfy $N=7$.Test cases 7-15 satisfy no additional constraints.Problem credits: Benjamin Qi
\end{verbatim}

\section*{Solution}


\section*{Problem URL}
https://usaco.org/index.php?page=viewproblem2&cpid=1135

\section*{Source}
USACO OPEN21 Contest, Silver Division, Problem ID: 1135

\end{document}
