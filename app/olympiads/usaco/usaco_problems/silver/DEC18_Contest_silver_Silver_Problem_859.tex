\documentclass[12pt]{article}
\usepackage{amsmath}
\usepackage{amssymb}
\usepackage{graphicx}
\usepackage{enumitem}
\usepackage{hyperref}
\usepackage{xcolor}

\title{View problem}
\author{USACO Contest: DEC18 Contest - Silver Division}
\date{\today}

\begin{document}

\maketitle

\section*{Problem Statement}

Despite long delays in airport pickups, Farmer John's convention for cows
interested in  eating grass has been going well so far.  It has attracted cows
from all over the world.

The main event of the conference, however, is looking like it might cause Farmer
John some further scheduling woes.  A very small pasture on his farm features a
rare form of grass that is supposed to be the tastiest in the world, according
to discerning cows.  As a result, all of the $N$ cows at the conference
($1 \leq N \leq 10^5$) want to sample this grass.  This will likely cause long
lines to form, since the pasture is so small it can only accommodate one cow at
a time.

Farmer John knows the time $a_i$ that each cow $i$ plans to arrive at the
special pasture, as well as the amount of time $t_i$ she plans to spend sampling
the special grass, once it becomes her turn.  Once cow $i$ starts eating the
grass, she spends her full time of $t_i$ before leaving, during which other
arriving cows need to wait.  If multiple cows are waiting when the pasture
becomes available again, the cow with the highest seniority is the next to be
allowed to sample the grass.  For this purpose, a cow who arrives right as 
another cow is finishing is considered "waiting".  Similarly, if a number of 
cows all arrive at exactly the same time while no cow is currently eating,
then the one with highest seniority is the next to eat.

Please help FJ compute the maximum amount of time any cow might possibly have to
wait in line (between time $a_i$ and the time the cow begins eating).  

INPUT FORMAT (file convention2.in):
The first line of input contains $N$.  Each of the next $N$ lines specify the
details of the $N$ cows in order of seniority (the most senior cow being first).
Each line contains $a_i$ and $t_i$ for one cow.  The $t_i$'s are positive
integers each at most $10^4$, and the $a_i$'s are positive integers at most
$10^9$.  

OUTPUT FORMAT (file convention2.out):
Please print the longest potential waiting time over all the cows.  

SAMPLE INPUT:
5
25 3
105 30
20 50
10 17
100 10
SAMPLE OUTPUT: 
10

In this example, we have 5 cows (numbered 1..5 according to their order in the
input). Cow 4 is the first to arrive (at time 10), and before she can finish
eating (at time 27) cows 1 and 3 both arrive.  Since cow 1 has higher seniority,
she gets to eat next, having waited 2 units of time beyond her arrival time. 
She finishes at time 30, and then cow 3 starts eating, having waited for 10
units of time beyond her starting time.  After a gap where no cow eats, cow 5
arrives and then while she is eating cow 2 arrives, eating 5 units of time
later.  The cow who is delayed the most relative to her arrival time is cow 3.


Problem credits: Brian Dean



\section*{Input Format}
OUTPUT FORMAT (file convention2.out):Please print the longest potential waiting time over all the cows.SAMPLE INPUT:5
25 3
105 30
20 50
10 17
100 10SAMPLE OUTPUT:10In this example, we have 5 cows (numbered 1..5 according to their order in the
input). Cow 4 is the first to arrive (at time 10), and before she can finish
eating (at time 27) cows 1 and 3 both arrive.  Since cow 1 has higher seniority,
she gets to eat next, having waited 2 units of time beyond her arrival time. 
She finishes at time 30, and then cow 3 starts eating, having waited for 10
units of time beyond her starting time.  After a gap where no cow eats, cow 5
arrives and then while she is eating cow 2 arrives, eating 5 units of time
later.  The cow who is delayed the most relative to her arrival time is cow 3.Problem credits: Brian Dean

\section*{Output Format}
SAMPLE INPUT:5
25 3
105 30
20 50
10 17
100 10SAMPLE OUTPUT:10In this example, we have 5 cows (numbered 1..5 according to their order in the
input). Cow 4 is the first to arrive (at time 10), and before she can finish
eating (at time 27) cows 1 and 3 both arrive.  Since cow 1 has higher seniority,
she gets to eat next, having waited 2 units of time beyond her arrival time. 
She finishes at time 30, and then cow 3 starts eating, having waited for 10
units of time beyond her starting time.  After a gap where no cow eats, cow 5
arrives and then while she is eating cow 2 arrives, eating 5 units of time
later.  The cow who is delayed the most relative to her arrival time is cow 3.Problem credits: Brian Dean

\section*{Sample Input}
\begin{verbatim}
5
25 3
105 30
20 50
10 17
100 10
\end{verbatim}

\section*{Sample Output}
\begin{verbatim}
10

In this example, we have 5 cows (numbered 1..5 according to their order in the
input). Cow 4 is the first to arrive (at time 10), and before she can finish
eating (at time 27) cows 1 and 3 both arrive.  Since cow 1 has higher seniority,
she gets to eat next, having waited 2 units of time beyond her arrival time. 
She finishes at time 30, and then cow 3 starts eating, having waited for 10
units of time beyond her starting time.  After a gap where no cow eats, cow 5
arrives and then while she is eating cow 2 arrives, eating 5 units of time
later.  The cow who is delayed the most relative to her arrival time is cow 3.Problem credits: Brian Dean

Problem credits: Brian Dean

Problem credits: Brian Dean
\end{verbatim}

\section*{Solution}


\section*{Problem URL}
https://usaco.org/index.php?page=viewproblem2&cpid=859

\section*{Source}
USACO DEC18 Contest, Silver Division, Problem ID: 859

\end{document}
