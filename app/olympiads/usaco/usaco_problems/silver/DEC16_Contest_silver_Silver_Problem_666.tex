\documentclass[12pt]{article}
\usepackage{amsmath}
\usepackage{amssymb}
\usepackage{graphicx}
\usepackage{enumitem}
\usepackage{hyperref}
\usepackage{xcolor}

\title{View problem}
\author{USACO Contest: DEC16 Contest - Silver Division}
\date{\today}

\begin{document}

\maketitle

\section*{Problem Statement}

Farmer John has just arranged his $N$ haybales ($1 \leq N \leq 100,000$) at
various points along the one-dimensional road running across his farm.  To make
sure they are spaced out appropriately, please help him answer $Q$ queries
($1 \leq Q \leq 100,000$), each asking for the number of haybales within a
specific interval along the road.

INPUT FORMAT (file haybales.in):
The first line contains $N$ and $Q$. 

The next line contains $N$ distinct integers, each in the range
$0 \ldots 1,000,000,000$, indicating that there is a haybale at each of those
locations.

Each of the next $Q$ lines contains two integers $A$ and $B$
($0 \leq A \leq B \leq 1,000,000,000$) giving a query for the number of haybales
between $A$ and $B$, inclusive.

OUTPUT FORMAT (file haybales.out):
You should write $Q$ lines of output.  For each query, output the number of
haybales in its respective interval.

SAMPLE INPUT:
4 6
3 2 7 5
2 3
2 4
2 5
2 7
4 6
8 10
SAMPLE OUTPUT: 
2
2
3
4
1
0


Problem credits: Nick Wu



\section*{Input Format}
The next line contains $N$ distinct integers, each in the range
$0 \ldots 1,000,000,000$, indicating that there is a haybale at each of those
locations.Each of the next $Q$ lines contains two integers $A$ and $B$
($0 \leq A \leq B \leq 1,000,000,000$) giving a query for the number of haybales
between $A$ and $B$, inclusive.

Each of the next $Q$ lines contains two integers $A$ and $B$
($0 \leq A \leq B \leq 1,000,000,000$) giving a query for the number of haybales
between $A$ and $B$, inclusive.

OUTPUT FORMAT (file haybales.out):You should write $Q$ lines of output.  For each query, output the number of
haybales in its respective interval.SAMPLE INPUT:4 6
3 2 7 5
2 3
2 4
2 5
2 7
4 6
8 10SAMPLE OUTPUT:2
2
3
4
1
0Problem credits: Nick Wu

\section*{Output Format}
SAMPLE INPUT:4 6
3 2 7 5
2 3
2 4
2 5
2 7
4 6
8 10SAMPLE OUTPUT:2
2
3
4
1
0Problem credits: Nick Wu

\section*{Sample Input}
\begin{verbatim}
4 6
3 2 7 5
2 3
2 4
2 5
2 7
4 6
8 10
\end{verbatim}

\section*{Sample Output}
\begin{verbatim}
2
2
3
4
1
0

Problem credits: Nick Wu

Problem credits: Nick Wu
\end{verbatim}

\section*{Solution}


\section*{Problem URL}
https://usaco.org/index.php?page=viewproblem2&cpid=666

\section*{Source}
USACO DEC16 Contest, Silver Division, Problem ID: 666

\end{document}
