\documentclass[12pt]{article}
\usepackage{amsmath}
\usepackage{amssymb}
\usepackage{graphicx}
\usepackage{enumitem}
\usepackage{hyperref}
\usepackage{xcolor}

\title{View problem}
\author{USACO Contest: OPEN17 Contest - Silver Division}
\date{\today}

\begin{document}

\maketitle

\section*{Problem Statement}

Always known for being quite tech-savy, Farmer John is testing out his new
automated drone-mounted cow locator camera, which supposedly can take a picture
of his field and automatically figure out the location of cows.  Unfortunately,
the camera does not include a  very good algorithm for finding cows, so FJ needs
your help developing a better one.

The overhead image of his farm taken by the camera is described by an
$N \times N$ grid of characters, each  in the range $A \ldots Z$, representing
one of 26 possible colors.  Farmer John figures the best way to define a
potential cow location (PCL) is as follows: A PCL is a rectangular sub-grid
(possibly the entire image) with sides parallel to the image sides, not
contained within any other PCL (so no smaller subset of a PCL is also a PCL). 
Furthermore, a PCL must satisfy the following property: focusing on just the
contents of the rectangle and ignoring the rest of the image, exactly two colors
must be present, one forming a contiguous region and one forming two or more
contiguous regions.

For example, a rectangle with contents


AAAAA
ABABA
AAABB

would constitute a PCL, since the A's form a single contiguous region and the
B's form more than one contiguous region.  The interpretation is a cow of color
A with spots of color B.  

A region is "contiguous" if you can traverse the entire region by moving
repeatedly from one cell in the region to another cell in the region taking
steps up, down, left, or right.  

Given the image returned by FJ's camera, please count the number of PCLs.

INPUT FORMAT (file where.in):
The first line of input contains $N$, the size of the grid ($1 \leq N \leq 20$).
The next $N$ lines describe the image, each consisting of $N$ characters.

OUTPUT FORMAT (file where.out):
Print a count of the number of PCLs in the image.

SAMPLE INPUT:
4
ABBC
BBBC
AABB
ABBC
SAMPLE OUTPUT: 
2

In this example, the two PCLs are the rectangles with contents


ABB
BBB
AAB
ABB

and


BC
BC
BB
BC

Problem credits: Brian Dean



\section*{Input Format}
OUTPUT FORMAT (file where.out):Print a count of the number of PCLs in the image.SAMPLE INPUT:4
ABBC
BBBC
AABB
ABBCSAMPLE OUTPUT:2In this example, the two PCLs are the rectangles with contentsABB
BBB
AAB
ABBandBC
BC
BB
BCProblem credits: Brian Dean

\section*{Output Format}
SAMPLE INPUT:4
ABBC
BBBC
AABB
ABBCSAMPLE OUTPUT:2In this example, the two PCLs are the rectangles with contentsABB
BBB
AAB
ABBandBC
BC
BB
BCProblem credits: Brian Dean

\section*{Sample Input}
\begin{verbatim}
4
ABBC
BBBC
AABB
ABBC
\end{verbatim}

\section*{Sample Output}
\begin{verbatim}
2

In this example, the two PCLs are the rectangles with contentsABB
BBB
AAB
ABBandBC
BC
BB
BCProblem credits: Brian Dean

ABB
BBB
AAB
ABBandBC
BC
BB
BCProblem credits: Brian Dean

ABB
BBB
AAB
ABB

andBC
BC
BB
BCProblem credits: Brian Dean

BC
BC
BB
BCProblem credits: Brian Dean

BC
BC
BB
BC

Problem credits: Brian Dean
\end{verbatim}

\section*{Solution}


\section*{Problem URL}
https://usaco.org/index.php?page=viewproblem2&cpid=740

\section*{Source}
USACO OPEN17 Contest, Silver Division, Problem ID: 740

\end{document}
