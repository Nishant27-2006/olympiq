\documentclass[12pt]{article}
\usepackage{amsmath}
\usepackage{amssymb}
\usepackage{graphicx}
\usepackage{enumitem}
\usepackage{hyperref}
\usepackage{xcolor}

\title{View problem}
\author{USACO Contest: FEB16 Contest - Silver Division}
\date{\today}

\begin{document}

\maketitle

\section*{Problem Statement}

Farmer John has received an order for exactly $M$ units of milk
($1 \leq M \leq 200$) that he needs to fill right away.  Unfortunately, his
fancy milking machine has just become broken, and all he has are two milk pails
of integer sizes $X$ and $Y$ ($1 \leq X, Y \leq 100$) with which he can measure
milk.  Both pails are initially empty.  Using these two pails, he can perform up
to $K$ of the following types of operations ($1 \leq K \leq 100$):

- He can fill either pail completely to the top. 

- He can empty either pail. 

- He can pour the contents of one pail into the other, stopping when the former
becomes empty or the latter becomes full (whichever of these happens first).

Although FJ realizes he may not be able to end up with exactly $M$ total units
of milk in the two pails, please help him compute the minimum amount of error
between $M$ and the total amount of milk in the two pails.  That is, please
compute the minimum value of $|M-M'|$ such that FJ can construct $M'$ units of
milk collectively between the two  pails.

INPUT FORMAT (file pails.in):
The first, and only line of input, contains $X$, $Y$, $K$, and $M$.

OUTPUT FORMAT (file pails.out):
Output the smallest distance from $M$ to an amount of milk FJ can produce.

SAMPLE INPUT:
14 50 2 32
SAMPLE OUTPUT: 
18

In two steps FJ can be left with the following quanities in his pails


(0, 0) = 0 units
(14, 0) = 14 units
(0, 50) = 50 units
(0, 14) = 14 units
(14, 36) = 50 units
(14, 50) = 64 units

The closest we can come to 32 units is 14 for a difference of 18.  Note that it
would require an extra step to pour out the first pail to end up with (0, 36).

Problem credits: Brian Dean



\section*{Input Format}
OUTPUT FORMAT (file pails.out):Output the smallest distance from $M$ to an amount of milk FJ can produce.SAMPLE INPUT:14 50 2 32SAMPLE OUTPUT:18In two steps FJ can be left with the following quanities in his pails(0, 0) = 0 units
(14, 0) = 14 units
(0, 50) = 50 units
(0, 14) = 14 units
(14, 36) = 50 units
(14, 50) = 64 unitsThe closest we can come to 32 units is 14 for a difference of 18.  Note that it
would require an extra step to pour out the first pail to end up with (0, 36).Problem credits: Brian Dean

\section*{Output Format}
SAMPLE INPUT:14 50 2 32SAMPLE OUTPUT:18In two steps FJ can be left with the following quanities in his pails(0, 0) = 0 units
(14, 0) = 14 units
(0, 50) = 50 units
(0, 14) = 14 units
(14, 36) = 50 units
(14, 50) = 64 unitsThe closest we can come to 32 units is 14 for a difference of 18.  Note that it
would require an extra step to pour out the first pail to end up with (0, 36).Problem credits: Brian Dean

\section*{Sample Input}
\begin{verbatim}
14 50 2 32
\end{verbatim}

\section*{Sample Output}
\begin{verbatim}
18

In two steps FJ can be left with the following quanities in his pails(0, 0) = 0 units
(14, 0) = 14 units
(0, 50) = 50 units
(0, 14) = 14 units
(14, 36) = 50 units
(14, 50) = 64 unitsThe closest we can come to 32 units is 14 for a difference of 18.  Note that it
would require an extra step to pour out the first pail to end up with (0, 36).Problem credits: Brian Dean

(0, 0) = 0 units
(14, 0) = 14 units
(0, 50) = 50 units
(0, 14) = 14 units
(14, 36) = 50 units
(14, 50) = 64 unitsThe closest we can come to 32 units is 14 for a difference of 18.  Note that it
would require an extra step to pour out the first pail to end up with (0, 36).Problem credits: Brian Dean

(0, 0) = 0 units
(14, 0) = 14 units
(0, 50) = 50 units
(0, 14) = 14 units
(14, 36) = 50 units
(14, 50) = 64 units

The closest we can come to 32 units is 14 for a difference of 18.  Note that it
would require an extra step to pour out the first pail to end up with (0, 36).Problem credits: Brian Dean

Problem credits: Brian Dean
\end{verbatim}

\section*{Solution}


\section*{Problem URL}
https://usaco.org/index.php?page=viewproblem2&cpid=620

\section*{Source}
USACO FEB16 Contest, Silver Division, Problem ID: 620

\end{document}
