\documentclass[12pt]{article}
\usepackage{amsmath}
\usepackage{amssymb}
\usepackage{graphicx}
\usepackage{enumitem}
\usepackage{hyperref}
\usepackage{xcolor}

\title{View problem}
\author{USACO Contest: DEC14 Contest - Silver Division}
\date{\today}

\begin{document}

\maketitle

\section*{Problem Statement}

Problem 3: Cow Jog [Mark Gordon, 2014]



The cows are out exercising their hooves again!  There are N cows

jogging on an infinitely-long single-lane track (1 <= N <= 100,000).

Each cow starts at a distinct position on the track, and some cows jog

at different speeds.



With only one lane in the track, cows cannot pass each other.  When a

faster cow catches up to another cow, she has to slow down to avoid

running into the other cow, becoming part of the same running group.



Eventually, no more cows will run into each other.  Farmer John

wonders how many groups will be left when this happens.  Please help

him compute this number.



INPUT: (file cowjog.in)



The first line of input contains the integer N.



The following N lines each contain the initial position and speed of a

single cow.  Position is a nonnegative integer and speed is a positive

integer; both numbers are at most 1 billion.  All cows start at 

distinct positions, and these will be given in increasing order in

the input.



SAMPLE INPUT:



5

0 1

1 2

2 3

3 2

6 1



OUTPUT: (file cowjog.out)



A single integer indicating how many groups remain.



SAMPLE OUTPUT:



2




\section*{Input Format}


\section*{Output Format}


\section*{Sample Input}
\begin{verbatim}

\end{verbatim}

\section*{Sample Output}
\begin{verbatim}

\end{verbatim}

\section*{Solution}


\section*{Problem URL}
https://usaco.org/index.php?page=viewproblem2&cpid=489

\section*{Source}
USACO DEC14 Contest, Silver Division, Problem ID: 489

\end{document}
