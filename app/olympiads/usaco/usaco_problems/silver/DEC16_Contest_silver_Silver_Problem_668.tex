\documentclass[12pt]{article}
\usepackage{amsmath}
\usepackage{amssymb}
\usepackage{graphicx}
\usepackage{enumitem}
\usepackage{hyperref}
\usepackage{xcolor}

\title{View problem}
\author{USACO Contest: DEC16 Contest - Silver Division}
\date{\today}

\begin{document}

\maketitle

\section*{Problem Statement}

Farmer John's $N$ cows ($1 \leq N \leq 200$) want to organize an emergency
"moo-cast" system for broadcasting important messages among themselves.  

Instead of mooing at each-other over long distances, the cows decide to equip
themselves with walkie-talkies, one for each cow.  These walkie-talkies each
have a limited transmission radius -- a walkie-talkie of power $P$ can only
transmit to other cows up to a distance of $P$ away (note that cow A might
be able to transmit to cow B even if cow B cannot transmit back, due to cow
A's power being larger than that of cow B).  Fortunately, cows can relay messages
to one-another along a path consisting of several hops, so it is not necessary
for every cow to be able to transmit directly to every other cow.

Due to the asymmetrical nature of the walkie-talkie transmission, broadcasts
from some cows may be more effective than from other cows in their ability to
reach large numbers of recipients (taking relaying into account).  Please  help
the cows determine the maximum number of cows that can be reached by a broadcast
originating from a single cow.

INPUT FORMAT (file moocast.in):
The first line of input contains $N$.

The next $N$ lines each contain the $x$ and $y$ coordinates of a single cow ( 
integers in the range $0 \ldots 25,000$) followed by $p$, the power of the
walkie-talkie held by this cow.

OUTPUT FORMAT (file moocast.out):
Write a single line of output containing the maximum number of cows a broadcast
from a single cow can reach.  The originating cow is included in this number.

SAMPLE INPUT:
4
1 3 5
5 4 3
7 2 1
6 1 1
SAMPLE OUTPUT: 
3

In the example above, a broadcast from cow 1 can reach 3 total cows, including
cow 1.

Problem credits: Brian Dean



\section*{Input Format}
The next $N$ lines each contain the $x$ and $y$ coordinates of a single cow ( 
integers in the range $0 \ldots 25,000$) followed by $p$, the power of the
walkie-talkie held by this cow.

OUTPUT FORMAT (file moocast.out):Write a single line of output containing the maximum number of cows a broadcast
from a single cow can reach.  The originating cow is included in this number.SAMPLE INPUT:4
1 3 5
5 4 3
7 2 1
6 1 1SAMPLE OUTPUT:3In the example above, a broadcast from cow 1 can reach 3 total cows, including
cow 1.Problem credits: Brian Dean

\section*{Output Format}
SAMPLE INPUT:4
1 3 5
5 4 3
7 2 1
6 1 1SAMPLE OUTPUT:3In the example above, a broadcast from cow 1 can reach 3 total cows, including
cow 1.Problem credits: Brian Dean

\section*{Sample Input}
\begin{verbatim}
4
1 3 5
5 4 3
7 2 1
6 1 1
\end{verbatim}

\section*{Sample Output}
\begin{verbatim}
3

In the example above, a broadcast from cow 1 can reach 3 total cows, including
cow 1.Problem credits: Brian Dean

Problem credits: Brian Dean
\end{verbatim}

\section*{Solution}


\section*{Problem URL}
https://usaco.org/index.php?page=viewproblem2&cpid=668

\section*{Source}
USACO DEC16 Contest, Silver Division, Problem ID: 668

\end{document}
