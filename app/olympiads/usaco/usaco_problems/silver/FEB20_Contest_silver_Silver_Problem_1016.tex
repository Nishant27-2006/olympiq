\documentclass[12pt]{article}
\usepackage{amsmath}
\usepackage{amssymb}
\usepackage{graphicx}
\usepackage{enumitem}
\usepackage{hyperref}
\usepackage{xcolor}

\title{View problem}
\author{USACO Contest: FEB20 Contest - Silver Division}
\date{\today}

\begin{document}

\maketitle

\section*{Problem Statement}

Farmer John's new barn has a truly strange design: it consists of $N$ rooms
($2 \leq N \leq 2500$), conveniently numbered $1 \ldots N$, and $N-1$ corridors.
Each corridor connects a pair of rooms, in such a way that it is possible to
walk from any room to any other room along a series of corridors.

Every room in the barn has a circular clock on the wall with the standard
integers $1 \ldots 12$ around its face. However, these clocks only have one
hand, which always points directly at one of the integers on the clock face (it
never points between two of these integers).  

Bessie the cow wants to synchronize all the clocks in the barn so they all point
to the integer 12.  However, she is somewhat simple-minded, and as she walks
around the barn, every time she enters a room, she moves the hand on its clock
ahead by one position. For example, if the clock pointed at 5, it would now
point at 6, and if the clock pointed at 12, it would now point at 1.  If Bessie
enters the same room multiple times, she advances the clock in that room every
time she enters.

Please determine the number of rooms in which Bessie could start walking around
the barn such that she could conceivably set all the clocks to point to 12. 
Note that Bessie does not initially advance the clock in her starting room, but
she would advance the clock in that room any time she re-entered it.  Clocks do
not advance on their own; a clock only advances if Bessie enters its room. 
Furthermore, once Bessie enters a corridor she must exit through the other end
(it is not allowed to walk partially through the corridor and loop back around
to the same room).

SCORING:
Test cases 2-7 satisfy $N\le 100$.Test cases 8-15 satisfy no additional constraints.

INPUT FORMAT (file clocktree.in):
The first line of input contains $N$.  The next line contains $N$ integers, each
in the range $1 \ldots 12$, specifying the initial clock setting in each room. 
The next $N-1$ lines each describe a corridor in terms of two integers $a$ and
$b$, each in the range $1 \ldots N$, giving the room numbers connected by the
corridor.

OUTPUT FORMAT (file clocktree.out):
Print the number of rooms in which Bessie could start, such that it is possible
for her to set all clocks to point to 12.

SAMPLE INPUT:
4
11 10 11 11
1 2
2 3
2 4
SAMPLE OUTPUT: 
1

In this example, Bessie can set all the clocks to point to 12 if and only if she starts
in room 2 (for example, by moving to room 1, 2, 3, 2, and finally 4).


Problem credits: Brian Dean



\section*{Input Format}
OUTPUT FORMAT (file clocktree.out):Print the number of rooms in which Bessie could start, such that it is possible
for her to set all clocks to point to 12.SAMPLE INPUT:4
11 10 11 11
1 2
2 3
2 4SAMPLE OUTPUT:1In this example, Bessie can set all the clocks to point to 12 if and only if she starts
in room 2 (for example, by moving to room 1, 2, 3, 2, and finally 4).Problem credits: Brian Dean

\section*{Output Format}
SAMPLE INPUT:4
11 10 11 11
1 2
2 3
2 4SAMPLE OUTPUT:1In this example, Bessie can set all the clocks to point to 12 if and only if she starts
in room 2 (for example, by moving to room 1, 2, 3, 2, and finally 4).Problem credits: Brian Dean

\section*{Sample Input}
\begin{verbatim}
4
11 10 11 11
1 2
2 3
2 4
\end{verbatim}

\section*{Sample Output}
\begin{verbatim}
1

In this example, Bessie can set all the clocks to point to 12 if and only if she starts
in room 2 (for example, by moving to room 1, 2, 3, 2, and finally 4).Problem credits: Brian Dean

Problem credits: Brian Dean

Problem credits: Brian Dean
\end{verbatim}

\section*{Solution}


\section*{Problem URL}
https://usaco.org/index.php?page=viewproblem2&cpid=1016

\section*{Source}
USACO FEB20 Contest, Silver Division, Problem ID: 1016

\end{document}
