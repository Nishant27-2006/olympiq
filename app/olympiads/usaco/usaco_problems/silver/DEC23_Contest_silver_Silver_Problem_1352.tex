\documentclass[12pt]{article}
\usepackage{amsmath}
\usepackage{amssymb}
\usepackage{graphicx}
\usepackage{enumitem}
\usepackage{hyperref}
\usepackage{xcolor}

\title{View problem}
\author{USACO Contest: DEC23 Contest - Silver Division}
\date{\today}

\begin{document}

\maketitle

\section*{Problem Statement}


Bessie is a robovine, also known as a cowborg. She is on a number line trying to
shoot a series of $T$ $(1 \leq T \leq 10^5)$ targets located at distinct
positions. Bessie  starts at position $0$ and follows a string of $C$
$(1 \leq C \leq 10^5)$ commands,  each one of L, F, or R:

L: Bessie moves one unit to the left.R: Bessie moves one unit to the right.F: Bessie fires. If there is a target at Bessie's current position, it is
hit and destroyed, and cannot be hit again.
If you are allowed to change up to one command in the string  to a different
command before Bessie starts following it, what is the maximum number of targets
that Bessie can hit?

INPUT FORMAT (input arrives from the terminal / stdin):
The first line contains $T$ and $C$.

The next line contains the locations of the $T$ targets, distinct integers in the range
$[-C,C]$.

The next line contains the command string of length $C$, containing only the
characters F, L, and R.

OUTPUT FORMAT (print output to the terminal / stdout):
Print the maximum number of targets that Bessie can hit after changing up to one
command  in the string.

SAMPLE INPUT:
3 7
0 -1 1
LFFRFRR
SAMPLE OUTPUT: 
3

If you make no changes to the string, Bessie will hit two targets:


Command | Position | Total Targets Hit
--------+----------+-------------------
Start   |  0       | 0 
L       | -1       | 0
F       | -1       | 1
F       | -1       | 1 (can't destroy target more than once)
R       |  0       | 1
F       |  0       | 2
R       |  1       | 2
R       |  2       | 2

If you change the last command from R to F, Bessie will hit all three targets:


Command | Position | Total Targets Hit
--------+----------+-------------------
Start   |  0       | 0 
L       | -1       | 0
F       | -1       | 1
F       | -1       | 1 (can't destroy target more than once)
R       |  0       | 1
F       |  0       | 2
R       |  1       | 2
F       |  1       | 3

SAMPLE INPUT:
1 5
0
FFFFF
SAMPLE OUTPUT: 
1

If the commands are left unchanged, the only target at 0 will be destroyed.
Since a target cannot be destroyed multiple times, the answer is 1.

SAMPLE INPUT:
5 6
1 2 3 4 5
FFRFRF
SAMPLE OUTPUT: 
3

SCORING:
Inputs 4-6: $T,C \le 1000$Inputs 7-15: No additional constraints.


Problem credits: Suhas Nagar



\section*{Input Format}
The next line contains the locations of the $T$ targets, distinct integers in the range
$[-C,C]$.The next line contains the command string of length $C$, containing only the
characters F, L, and R.

The next line contains the command string of length $C$, containing only the
characters F, L, and R.

OUTPUT FORMAT (print output to the terminal / stdout):Print the maximum number of targets that Bessie can hit after changing up to one
command  in the string.SAMPLE INPUT:3 7
0 -1 1
LFFRFRRSAMPLE OUTPUT:3If you make no changes to the string, Bessie will hit two targets:Command | Position | Total Targets Hit
--------+----------+-------------------
Start   |  0       | 0 
L       | -1       | 0
F       | -1       | 1
F       | -1       | 1 (can't destroy target more than once)
R       |  0       | 1
F       |  0       | 2
R       |  1       | 2
R       |  2       | 2If you change the last command from R to F, Bessie will hit all three targets:Command | Position | Total Targets Hit
--------+----------+-------------------
Start   |  0       | 0 
L       | -1       | 0
F       | -1       | 1
F       | -1       | 1 (can't destroy target more than once)
R       |  0       | 1
F       |  0       | 2
R       |  1       | 2
F       |  1       | 3SAMPLE INPUT:1 5
0
FFFFFSAMPLE OUTPUT:1If the commands are left unchanged, the only target at 0 will be destroyed.
Since a target cannot be destroyed multiple times, the answer is 1.SAMPLE INPUT:5 6
1 2 3 4 5
FFRFRFSAMPLE OUTPUT:3SCORING:Inputs 4-6: $T,C \le 1000$Inputs 7-15: No additional constraints.Problem credits: Suhas Nagar

\section*{Output Format}
SAMPLE INPUT:3 7
0 -1 1
LFFRFRRSAMPLE OUTPUT:3If you make no changes to the string, Bessie will hit two targets:Command | Position | Total Targets Hit
--------+----------+-------------------
Start   |  0       | 0 
L       | -1       | 0
F       | -1       | 1
F       | -1       | 1 (can't destroy target more than once)
R       |  0       | 1
F       |  0       | 2
R       |  1       | 2
R       |  2       | 2If you change the last command from R to F, Bessie will hit all three targets:Command | Position | Total Targets Hit
--------+----------+-------------------
Start   |  0       | 0 
L       | -1       | 0
F       | -1       | 1
F       | -1       | 1 (can't destroy target more than once)
R       |  0       | 1
F       |  0       | 2
R       |  1       | 2
F       |  1       | 3SAMPLE INPUT:1 5
0
FFFFFSAMPLE OUTPUT:1If the commands are left unchanged, the only target at 0 will be destroyed.
Since a target cannot be destroyed multiple times, the answer is 1.SAMPLE INPUT:5 6
1 2 3 4 5
FFRFRFSAMPLE OUTPUT:3SCORING:Inputs 4-6: $T,C \le 1000$Inputs 7-15: No additional constraints.Problem credits: Suhas Nagar

\section*{Sample Input}
\begin{verbatim}
3 7
0 -1 1
LFFRFRR

1 5
0
FFFFF

5 6
1 2 3 4 5
FFRFRF
\end{verbatim}

\section*{Sample Output}
\begin{verbatim}
3

If you make no changes to the string, Bessie will hit two targets:Command | Position | Total Targets Hit
--------+----------+-------------------
Start   |  0       | 0 
L       | -1       | 0
F       | -1       | 1
F       | -1       | 1 (can't destroy target more than once)
R       |  0       | 1
F       |  0       | 2
R       |  1       | 2
R       |  2       | 2If you change the last command from R to F, Bessie will hit all three targets:Command | Position | Total Targets Hit
--------+----------+-------------------
Start   |  0       | 0 
L       | -1       | 0
F       | -1       | 1
F       | -1       | 1 (can't destroy target more than once)
R       |  0       | 1
F       |  0       | 2
R       |  1       | 2
F       |  1       | 3SAMPLE INPUT:1 5
0
FFFFFSAMPLE OUTPUT:1If the commands are left unchanged, the only target at 0 will be destroyed.
Since a target cannot be destroyed multiple times, the answer is 1.SAMPLE INPUT:5 6
1 2 3 4 5
FFRFRFSAMPLE OUTPUT:3SCORING:Inputs 4-6: $T,C \le 1000$Inputs 7-15: No additional constraints.Problem credits: Suhas Nagar

Command | Position | Total Targets Hit
--------+----------+-------------------
Start   |  0       | 0 
L       | -1       | 0
F       | -1       | 1
F       | -1       | 1 (can't destroy target more than once)
R       |  0       | 1
F       |  0       | 2
R       |  1       | 2
R       |  2       | 2If you change the last command from R to F, Bessie will hit all three targets:Command | Position | Total Targets Hit
--------+----------+-------------------
Start   |  0       | 0 
L       | -1       | 0
F       | -1       | 1
F       | -1       | 1 (can't destroy target more than once)
R       |  0       | 1
F       |  0       | 2
R       |  1       | 2
F       |  1       | 3SAMPLE INPUT:1 5
0
FFFFFSAMPLE OUTPUT:1If the commands are left unchanged, the only target at 0 will be destroyed.
Since a target cannot be destroyed multiple times, the answer is 1.SAMPLE INPUT:5 6
1 2 3 4 5
FFRFRFSAMPLE OUTPUT:3SCORING:Inputs 4-6: $T,C \le 1000$Inputs 7-15: No additional constraints.Problem credits: Suhas Nagar

Command | Position | Total Targets Hit
--------+----------+-------------------
Start   |  0       | 0 
L       | -1       | 0
F       | -1       | 1
F       | -1       | 1 (can't destroy target more than once)
R       |  0       | 1
F       |  0       | 2
R       |  1       | 2
R       |  2       | 2

If you change the last command from R to F, Bessie will hit all three targets:Command | Position | Total Targets Hit
--------+----------+-------------------
Start   |  0       | 0 
L       | -1       | 0
F       | -1       | 1
F       | -1       | 1 (can't destroy target more than once)
R       |  0       | 1
F       |  0       | 2
R       |  1       | 2
F       |  1       | 3SAMPLE INPUT:1 5
0
FFFFFSAMPLE OUTPUT:1If the commands are left unchanged, the only target at 0 will be destroyed.
Since a target cannot be destroyed multiple times, the answer is 1.SAMPLE INPUT:5 6
1 2 3 4 5
FFRFRFSAMPLE OUTPUT:3SCORING:Inputs 4-6: $T,C \le 1000$Inputs 7-15: No additional constraints.Problem credits: Suhas Nagar

Command | Position | Total Targets Hit
--------+----------+-------------------
Start   |  0       | 0 
L       | -1       | 0
F       | -1       | 1
F       | -1       | 1 (can't destroy target more than once)
R       |  0       | 1
F       |  0       | 2
R       |  1       | 2
F       |  1       | 3SAMPLE INPUT:1 5
0
FFFFFSAMPLE OUTPUT:1If the commands are left unchanged, the only target at 0 will be destroyed.
Since a target cannot be destroyed multiple times, the answer is 1.SAMPLE INPUT:5 6
1 2 3 4 5
FFRFRFSAMPLE OUTPUT:3SCORING:Inputs 4-6: $T,C \le 1000$Inputs 7-15: No additional constraints.Problem credits: Suhas Nagar

Command | Position | Total Targets Hit
--------+----------+-------------------
Start   |  0       | 0 
L       | -1       | 0
F       | -1       | 1
F       | -1       | 1 (can't destroy target more than once)
R       |  0       | 1
F       |  0       | 2
R       |  1       | 2
F       |  1       | 3

SAMPLE INPUT:1 5
0
FFFFFSAMPLE OUTPUT:1If the commands are left unchanged, the only target at 0 will be destroyed.
Since a target cannot be destroyed multiple times, the answer is 1.SAMPLE INPUT:5 6
1 2 3 4 5
FFRFRFSAMPLE OUTPUT:3SCORING:Inputs 4-6: $T,C \le 1000$Inputs 7-15: No additional constraints.Problem credits: Suhas Nagar

1

If the commands are left unchanged, the only target at 0 will be destroyed.
Since a target cannot be destroyed multiple times, the answer is 1.SAMPLE INPUT:5 6
1 2 3 4 5
FFRFRFSAMPLE OUTPUT:3SCORING:Inputs 4-6: $T,C \le 1000$Inputs 7-15: No additional constraints.Problem credits: Suhas Nagar

SAMPLE INPUT:5 6
1 2 3 4 5
FFRFRFSAMPLE OUTPUT:3SCORING:Inputs 4-6: $T,C \le 1000$Inputs 7-15: No additional constraints.Problem credits: Suhas Nagar

3

SCORING:Inputs 4-6: $T,C \le 1000$Inputs 7-15: No additional constraints.Problem credits: Suhas Nagar
\end{verbatim}

\section*{Solution}


\section*{Problem URL}
https://usaco.org/index.php?page=viewproblem2&cpid=1352

\section*{Source}
USACO DEC23 Contest, Silver Division, Problem ID: 1352

\end{document}
