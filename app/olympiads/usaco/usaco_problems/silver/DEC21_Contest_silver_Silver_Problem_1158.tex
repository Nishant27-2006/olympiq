\documentclass[12pt]{article}
\usepackage{amsmath}
\usepackage{amssymb}
\usepackage{graphicx}
\usepackage{enumitem}
\usepackage{hyperref}
\usepackage{xcolor}

\title{View problem}
\author{USACO Contest: DEC21 Contest - Silver Division}
\date{\today}

\begin{document}

\maketitle

\section*{Problem Statement}

Farmer John owns a long farm along the highway that can be considered somewhat
like a one-dimensional number line.  Along the farm, there are $K$ grassy 
patches ($1 \leq K \leq 2\cdot 10^5$); the $i$-th patch is located at position
$p_i$ and  has an associated tastiness  value $t_i$ ($0\le t_i\le 10^9$). 
Farmer John's nemesis, Farmer Nhoj, has  already situated his $M$ cows
($1 \leq M \leq 2\cdot 10^5$) at locations $f_1 \ldots f_M$. All $K+M$ of these
locations are distinct integers in the range $[0,10^9]$.

Farmer John needs to pick $N$ ($1\le N\le 2\cdot 10^5$) locations (not
necessarily integers) for his cows to be located.  These must be distinct from
those already occupied by the cows of Farmer Nhoj, but it is possible for 
Farmer John to place his cows at the same locations as grassy patches.

Whichever farmer owns a cow closest to a grassy patch can claim ownership of
that patch.  If there are two cows from rival farmers equally close to the
patch, then Farmer Nhoj claims the patch. 

Given the locations of Farmer Nhoj's cows and the locations and tastiness values
of the grassy patches, determine the maximum total tastiness Farmer John's cows
can claim if optimally positioned. 

INPUT FORMAT (input arrives from the terminal / stdin):
The first line contains $K$, $M$, and $N$.

The next $K$ lines each contain two space-separated integers $p_i$ and $t_i$.

The next $M$ lines each contain a single integer $f_i$.

OUTPUT FORMAT (print output to the terminal / stdout):
An integer denoting the maximum total tastiness. Note that the answer to this
problem can be too large to fit into a 32-bit integer, so you probably want to
use 64-bit integers (e.g., "long long"s in C or C++).

SAMPLE INPUT:
6 5 2
0 4
4 6
8 10
10 8
12 12
13 14
2
3
5
7
11
SAMPLE OUTPUT: 
36

If Farmer John places cows at positions $11.5$ and $8$ then he can claim a total tastiness
of
$10+12+14=36$.


Problem credits: Brian Dean



\section*{Input Format}
The next $K$ lines each contain two space-separated integers $p_i$ and $t_i$.The next $M$ lines each contain a single integer $f_i$.

The next $M$ lines each contain a single integer $f_i$.

OUTPUT FORMAT (print output to the terminal / stdout):An integer denoting the maximum total tastiness. Note that the answer to this
problem can be too large to fit into a 32-bit integer, so you probably want to
use 64-bit integers (e.g., "long long"s in C or C++).SAMPLE INPUT:6 5 2
0 4
4 6
8 10
10 8
12 12
13 14
2
3
5
7
11SAMPLE OUTPUT:36If Farmer John places cows at positions $11.5$ and $8$ then he can claim a total tastiness
of
$10+12+14=36$.Problem credits: Brian Dean

\section*{Output Format}
SAMPLE INPUT:6 5 2
0 4
4 6
8 10
10 8
12 12
13 14
2
3
5
7
11SAMPLE OUTPUT:36If Farmer John places cows at positions $11.5$ and $8$ then he can claim a total tastiness
of
$10+12+14=36$.Problem credits: Brian Dean

\section*{Sample Input}
\begin{verbatim}
6 5 2
0 4
4 6
8 10
10 8
12 12
13 14
2
3
5
7
11
\end{verbatim}

\section*{Sample Output}
\begin{verbatim}
36

If Farmer John places cows at positions $11.5$ and $8$ then he can claim a total tastiness
of
$10+12+14=36$.Problem credits: Brian Dean

Problem credits: Brian Dean

Problem credits: Brian Dean
\end{verbatim}

\section*{Solution}


\section*{Problem URL}
https://usaco.org/index.php?page=viewproblem2&cpid=1158

\section*{Source}
USACO DEC21 Contest, Silver Division, Problem ID: 1158

\end{document}
