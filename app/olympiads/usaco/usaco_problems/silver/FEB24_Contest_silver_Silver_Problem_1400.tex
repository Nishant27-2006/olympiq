\documentclass[12pt]{article}
\usepackage{amsmath}
\usepackage{amssymb}
\usepackage{graphicx}
\usepackage{enumitem}
\usepackage{hyperref}
\usepackage{xcolor}

\title{View problem}
\author{USACO Contest: FEB24 Contest - Silver Division}
\date{\today}

\begin{document}

\maketitle

\section*{Problem Statement}


Bessie and Elsie are playing a game of Moorbles. The game works as follows:
Bessie and Elsie each start out with some amount of marbles. Bessie holds out
$A$ of her marbles in her hoof and Elsie guesses if $A$ is Even or Odd. If Elsie
is correct, she wins the $A$ marbles from Bessie and if she guesses incorrectly,
she loses $A$ of her marbles to Bessie (if Elsie has less than $A$ marbles, she
loses all her marbles). A player loses when they lose all of their marbles.

After some amount of turns in the game, Elsie has $N$ $(1 \leq N \leq 10^9)$
marbles. She thinks it is hard to win, but she is playing to not lose. After
being around Bessie enough, Elsie has a good read on Bessie's habits and
recognizes that on turn $i$, there are only $K$ $(1 \leq K \leq 4)$ different
amounts of marbles that Bessie may put out. There are only $M$
$(1 \leq M \leq 3 \cdot 10^5)$ turns before Bessie gets bored and stops playing.
Can you identify a lexicographically minimum turn sequence such that Elsie will
not lose, regardless of how Bessie plays?

INPUT FORMAT (input arrives from the terminal / stdin):
The first line contains a single integer $T$ ($1 \leq T \leq 10$) representing
the number of test cases. Each test case is described as follows:
 First, one line containing three integers $N$, $M$, and $K$, representing
the number of marbles Elsie has, the number of turns, and the number of
potential moves Bessie can make respectively.Then, $M$ lines where
line $i$ contains $K$ distinct space separated integers
$a_{i,1} \; a_{i,2} \ldots a_{i,K}$ ($1 \leq a_{i, j} \leq 10^3$) representing
the possible amounts of marbles that Bessie might play on turn $i$.
It is guaranteed that the sum of $M$ over all test cases is at most
$3 \cdot 10^5$.

OUTPUT FORMAT (print output to the terminal / stdout):
For each test case, output the lexicographically minimum move sequence for Elsie
to guarantee not losing, or $-1$ if she will lose. The move sequence should be
on a single line and consist of $M$ space-separated tokens each equal to either
"Even" or "Odd".

Note: "Even" is lexicographically smaller than "Odd".

SAMPLE INPUT:
2
10 3 2
2 5
1 3
1 3
10 3 3
2 7 5
8 3 4
2 5 6
SAMPLE OUTPUT: 
Even Even Odd
-1

In the first case, the only lexicographically smaller sequence of moves is "Even
Even Even", but Bessie can make Elsie lose in that case by first playing $5$,
which reduces Elsie's number of marbles from $10$ to $5$, then playing $3$, which
reduces Elsie's number of marbles from $5$ to $2$, then playing $3$, which wipes out
all of her marbles.

If Elsie instead plays the correct move sequence "Even Even Odd", then if Bessie
plays the same way, at the end when she plays $3$, Elsie will gain those $3$
marbles, increasing her number of marbles to $5$. It can further be shown that
Bessie cannot play in a different way to take all of Elsie's marbles given that
Elsie plays "Even Even Odd".

In the second case, it can be shown that for any move sequence that Elsie could
choose, Bessie can play in a way to take all of Elsie's marbles.

SAMPLE INPUT:
1
20 8 2
3 5
3 5
3 5
3 5
3 5
3 5
3 5
3 5
SAMPLE OUTPUT: 
Even Even Even Odd Even Odd Even Odd

SCORING:
Input 3: $M \leq 16$.Inputs 4-6: $M \leq 1000$.Inputs
7-12: No further constraints.


Problem credits: Suhas Nagar



\section*{Input Format}
OUTPUT FORMAT (print output to the terminal / stdout):For each test case, output the lexicographically minimum move sequence for Elsie
to guarantee not losing, or $-1$ if she will lose. The move sequence should be
on a single line and consist of $M$ space-separated tokens each equal to either
"Even" or "Odd".Note: "Even" is lexicographically smaller than "Odd".SAMPLE INPUT:2
10 3 2
2 5
1 3
1 3
10 3 3
2 7 5
8 3 4
2 5 6SAMPLE OUTPUT:Even Even Odd
-1In the first case, the only lexicographically smaller sequence of moves is "Even
Even Even", but Bessie can make Elsie lose in that case by first playing $5$,
which reduces Elsie's number of marbles from $10$ to $5$, then playing $3$, which
reduces Elsie's number of marbles from $5$ to $2$, then playing $3$, which wipes out
all of her marbles.If Elsie instead plays the correct move sequence "Even Even Odd", then if Bessie
plays the same way, at the end when she plays $3$, Elsie will gain those $3$
marbles, increasing her number of marbles to $5$. It can further be shown that
Bessie cannot play in a different way to take all of Elsie's marbles given that
Elsie plays "Even Even Odd".In the second case, it can be shown that for any move sequence that Elsie could
choose, Bessie can play in a way to take all of Elsie's marbles.SAMPLE INPUT:1
20 8 2
3 5
3 5
3 5
3 5
3 5
3 5
3 5
3 5SAMPLE OUTPUT:Even Even Even Odd Even Odd Even OddSCORING:Input 3: $M \leq 16$.Inputs 4-6: $M \leq 1000$.Inputs
7-12: No further constraints.Problem credits: Suhas Nagar

\section*{Output Format}
Note: "Even" is lexicographically smaller than "Odd".

SAMPLE INPUT:2
10 3 2
2 5
1 3
1 3
10 3 3
2 7 5
8 3 4
2 5 6SAMPLE OUTPUT:Even Even Odd
-1In the first case, the only lexicographically smaller sequence of moves is "Even
Even Even", but Bessie can make Elsie lose in that case by first playing $5$,
which reduces Elsie's number of marbles from $10$ to $5$, then playing $3$, which
reduces Elsie's number of marbles from $5$ to $2$, then playing $3$, which wipes out
all of her marbles.If Elsie instead plays the correct move sequence "Even Even Odd", then if Bessie
plays the same way, at the end when she plays $3$, Elsie will gain those $3$
marbles, increasing her number of marbles to $5$. It can further be shown that
Bessie cannot play in a different way to take all of Elsie's marbles given that
Elsie plays "Even Even Odd".In the second case, it can be shown that for any move sequence that Elsie could
choose, Bessie can play in a way to take all of Elsie's marbles.SAMPLE INPUT:1
20 8 2
3 5
3 5
3 5
3 5
3 5
3 5
3 5
3 5SAMPLE OUTPUT:Even Even Even Odd Even Odd Even OddSCORING:Input 3: $M \leq 16$.Inputs 4-6: $M \leq 1000$.Inputs
7-12: No further constraints.Problem credits: Suhas Nagar

\section*{Sample Input}
\begin{verbatim}
2
10 3 2
2 5
1 3
1 3
10 3 3
2 7 5
8 3 4
2 5 6

1
20 8 2
3 5
3 5
3 5
3 5
3 5
3 5
3 5
3 5
\end{verbatim}

\section*{Sample Output}
\begin{verbatim}
Even Even Odd
-1

In the first case, the only lexicographically smaller sequence of moves is "Even
Even Even", but Bessie can make Elsie lose in that case by first playing $5$,
which reduces Elsie's number of marbles from $10$ to $5$, then playing $3$, which
reduces Elsie's number of marbles from $5$ to $2$, then playing $3$, which wipes out
all of her marbles.If Elsie instead plays the correct move sequence "Even Even Odd", then if Bessie
plays the same way, at the end when she plays $3$, Elsie will gain those $3$
marbles, increasing her number of marbles to $5$. It can further be shown that
Bessie cannot play in a different way to take all of Elsie's marbles given that
Elsie plays "Even Even Odd".In the second case, it can be shown that for any move sequence that Elsie could
choose, Bessie can play in a way to take all of Elsie's marbles.SAMPLE INPUT:1
20 8 2
3 5
3 5
3 5
3 5
3 5
3 5
3 5
3 5SAMPLE OUTPUT:Even Even Even Odd Even Odd Even OddSCORING:Input 3: $M \leq 16$.Inputs 4-6: $M \leq 1000$.Inputs
7-12: No further constraints.Problem credits: Suhas Nagar

If Elsie instead plays the correct move sequence "Even Even Odd", then if Bessie
plays the same way, at the end when she plays $3$, Elsie will gain those $3$
marbles, increasing her number of marbles to $5$. It can further be shown that
Bessie cannot play in a different way to take all of Elsie's marbles given that
Elsie plays "Even Even Odd".In the second case, it can be shown that for any move sequence that Elsie could
choose, Bessie can play in a way to take all of Elsie's marbles.SAMPLE INPUT:1
20 8 2
3 5
3 5
3 5
3 5
3 5
3 5
3 5
3 5SAMPLE OUTPUT:Even Even Even Odd Even Odd Even OddSCORING:Input 3: $M \leq 16$.Inputs 4-6: $M \leq 1000$.Inputs
7-12: No further constraints.Problem credits: Suhas Nagar

In the second case, it can be shown that for any move sequence that Elsie could
choose, Bessie can play in a way to take all of Elsie's marbles.SAMPLE INPUT:1
20 8 2
3 5
3 5
3 5
3 5
3 5
3 5
3 5
3 5SAMPLE OUTPUT:Even Even Even Odd Even Odd Even OddSCORING:Input 3: $M \leq 16$.Inputs 4-6: $M \leq 1000$.Inputs
7-12: No further constraints.Problem credits: Suhas Nagar

SAMPLE INPUT:1
20 8 2
3 5
3 5
3 5
3 5
3 5
3 5
3 5
3 5SAMPLE OUTPUT:Even Even Even Odd Even Odd Even OddSCORING:Input 3: $M \leq 16$.Inputs 4-6: $M \leq 1000$.Inputs
7-12: No further constraints.Problem credits: Suhas Nagar

Even Even Even Odd Even Odd Even Odd

SCORING:Input 3: $M \leq 16$.Inputs 4-6: $M \leq 1000$.Inputs
7-12: No further constraints.Problem credits: Suhas Nagar
\end{verbatim}

\section*{Solution}


\section*{Problem URL}
https://usaco.org/index.php?page=viewproblem2&cpid=1400

\section*{Source}
USACO FEB24 Contest, Silver Division, Problem ID: 1400

\end{document}
