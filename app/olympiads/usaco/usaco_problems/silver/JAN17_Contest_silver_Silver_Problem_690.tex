\documentclass[12pt]{article}
\usepackage{amsmath}
\usepackage{amssymb}
\usepackage{graphicx}
\usepackage{enumitem}
\usepackage{hyperref}
\usepackage{xcolor}

\title{View problem}
\author{USACO Contest: JAN17 Contest - Silver Division}
\date{\today}

\begin{document}

\maketitle

\section*{Problem Statement}

After several months of rehearsal, the cows are just about ready to put  on
their annual dance performance; this year they are performing the famous bovine ballet
"Cowpelia".

The only aspect of the show that remains to be determined is the size of the
stage.  A stage of size $K$ can support $K$ cows dancing simultaneously.   The
$N$ cows in the herd ($1 \leq N \leq 10,000$) are conveniently  numbered
$1 \ldots N$ in the order in which they must appear in the  dance.  Each cow $i$
plans to dance for a specific duration of time $d(i)$.  Initially, cows
$1 \ldots K$ appear on stage and start dancing.  When the first of these cows
completes her part, she leaves the stage and cow $K+1$ immediately starts
dancing, and so on, so there are always $K$ cows dancing (until the end of the
show, when we start to run out of cows).  The show ends when the last cow
completes  her dancing part, at time $T$.

Clearly, the larger the value of $K$, the smaller the value of $T$. Since the
show cannot last too long, you are given as input an upper bound $T_{max}$
specifying the largest possible value of $T$.  Subject to this constraint,
please determine the smallest possible value of $K$.

INPUT FORMAT (file cowdance.in):
The first line of input contains $N$ and $T_{max}$, where $T_{max}$ is an
integer of value at most 1 million.

The next $N$ lines give the durations $d(1) \ldots d(N)$ of the dancing parts
for cows $1 \ldots N$.  Each $d(i)$ value is an integer in the range
$1 \ldots 100,000$.

It is guaranteed that if $K=N$, the show will finish in time.

OUTPUT FORMAT (file cowdance.out):
Print out the smallest possible value of $K$ such that the dance performance
will take no more than $T_{max}$ units of time.

SAMPLE INPUT:
5 8
4
7
8
6
4
SAMPLE OUTPUT: 
4


Problem credits: Delphine and Brian Dean



\section*{Input Format}
The next $N$ lines give the durations $d(1) \ldots d(N)$ of the dancing parts
for cows $1 \ldots N$.  Each $d(i)$ value is an integer in the range
$1 \ldots 100,000$.It is guaranteed that if $K=N$, the show will finish in time.

It is guaranteed that if $K=N$, the show will finish in time.

OUTPUT FORMAT (file cowdance.out):Print out the smallest possible value of $K$ such that the dance performance
will take no more than $T_{max}$ units of time.SAMPLE INPUT:5 8
4
7
8
6
4SAMPLE OUTPUT:4Problem credits: Delphine and Brian Dean

\section*{Output Format}
SAMPLE INPUT:5 8
4
7
8
6
4SAMPLE OUTPUT:4Problem credits: Delphine and Brian Dean

\section*{Sample Input}
\begin{verbatim}
5 8
4
7
8
6
4
\end{verbatim}

\section*{Sample Output}
\begin{verbatim}
4

Problem credits: Delphine and Brian Dean

Problem credits: Delphine and Brian Dean
\end{verbatim}

\section*{Solution}


\section*{Problem URL}
https://usaco.org/index.php?page=viewproblem2&cpid=690

\section*{Source}
USACO JAN17 Contest, Silver Division, Problem ID: 690

\end{document}
