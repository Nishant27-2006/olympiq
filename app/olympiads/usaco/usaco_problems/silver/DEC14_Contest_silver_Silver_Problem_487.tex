\documentclass[12pt]{article}
\usepackage{amsmath}
\usepackage{amssymb}
\usepackage{graphicx}
\usepackage{enumitem}
\usepackage{hyperref}
\usepackage{xcolor}

\title{View problem}
\author{USACO Contest: DEC14 Contest - Silver Division}
\date{\today}

\begin{document}

\maketitle

\section*{Problem Statement}

Problem 1: Marathon [Nick Wu, 2014]



Unhappy with the poor health of his cows, Farmer John enrolls them in

an assortment of different physical fitness activities.  His prize cow

Bessie is enrolled in a running class, where she is eventually

expected to run a marathon through the downtown area of the city near

Farmer John's farm!



The marathon course consists of N checkpoints (3 <= N <= 100,000) to

be visited in sequence, where checkpoint 1 is the starting location

and checkpoint N is the finish.  Bessie is supposed to visit all of

these checkpoints one by one, but being the lazy cow she is, she

decides that she will skip up to one checkpoint in order to shorten

her total journey.  She cannot skip checkpoints 1 or N, however, since

that would be too noticeable.



Please help Bessie find the minimum distance that she has to run if

she can skip up to one checkpoint.  



Note that since the course is set in a downtown area with a grid of

streets, the distance between two checkpoints at locations (x1, y1)

and (x2, y2) is given by |x1-x2| + |y1-y2|.  This way of measuring

distance -- by the difference in x plus the difference in y -- is

sometimes known as "Manhattan" distance because it reflects the fact

that in a downtown grid, you can travel parallel to the x or y axes,

but you cannot travel along a direct line "as the crow flies".



INPUT: (file marathon.in)



The first line gives the value of N.



The next N lines each contain two space-separated integers, x and y,

representing a checkpoint (-1000 <= x <= 1000, -1000 <= y <= 1000).

The checkpoints are given in the order that they must be visited.

Note that the course might cross over itself several times, with

several checkpoints occurring at the same physical location.  When

Bessie skips such a checkpoint, she only skips one instance of the

checkpoint -- she does not skip every checkpoint occurring at the same

location.



SAMPLE INPUT:



4

0 0

8 3

11 -1

10 0



OUTPUT: (file marathon.out)



Output the minimum distance that Bessie can run by skipping up to one

checkpoint.  Don't forget to end your output with a newline.  In the

sample case shown here, skipping the checkpoint at (8, 3) leads to the

minimum total distance of 14.



SAMPLE OUTPUT:



14




\section*{Input Format}


\section*{Output Format}


\section*{Sample Input}
\begin{verbatim}

\end{verbatim}

\section*{Sample Output}
\begin{verbatim}

\end{verbatim}

\section*{Solution}


\section*{Problem URL}
https://usaco.org/index.php?page=viewproblem2&cpid=487

\section*{Source}
USACO DEC14 Contest, Silver Division, Problem ID: 487

\end{document}
