\documentclass[12pt]{article}
\usepackage{amsmath}
\usepackage{amssymb}
\usepackage{graphicx}
\usepackage{enumitem}
\usepackage{hyperref}
\usepackage{xcolor}

\title{View problem}
\author{USACO Contest: JAN21 Contest - Silver Division}
\date{\today}

\begin{document}

\maketitle

\section*{Problem Statement}

Bessie has recently received a painting set, and she wants to paint the long
fence at one end of her pasture.  The fence consists of $N$ consecutive 1-meter 
segments ($1\le N\le 10^5$).  Bessie has 26 different colors available, which
she labels with the letters 'A' through 'Z' in increasing order of darkness ('A'
is a very light color, and 'Z' is very dark).  She can therefore describe the
desired color she wants to paint each fence segment as a length-$N$ string where
each character is a letter.

Initially, all fence segments are uncolored.  Bessie can color any  contiguous
range of segments with a single color in a single brush stroke as long as she
never paints a lighter color over a darker color (she can only paint darker
colors over lighter colors).  

For example, an initially uncolored segment of length four can be colored as
follows:


.... -> BBB. -> BBLL -> BQQL

Running short on time, Bessie thinks she may need to leave some consecutive
range of fence segments unpainted! Currently, she is considering $Q$  candidate
ranges ($1\le Q\le 10^5$), each described by  by two integers $(a,b)$ with
$1 \leq a \leq b \leq N$ giving the indices of  endpoints of the range
$a \ldots b$ of segments to be left unpainted.

For each candidate range, what is the minimum number of strokes needed to paint
every fence segment outside those in the range with its desired color while 
leaving all fence segments inside the range uncolored?  Note that Bessie does
not actually do any painting during this process, so the answers for each
candidate range are independent. 

INPUT FORMAT (input arrives from the terminal / stdin):
The first line contains $N$ and $Q$.

The next line contains a string of length $N$ characters representing the
desired color  for each fence segment.

The next $Q$ lines each contain two space-separated integers $a$ and $b$
representing a candidate range to possibly leave unpainted.

OUTPUT FORMAT (print output to the terminal / stdout):
For each of the $Q$ candidates, output the answer on a new line.

SAMPLE INPUT:
8 2
ABBAABCB
3 6
1 4
SAMPLE OUTPUT: 
4
3

In this example, excluding the sub-range corresponding to the desired pattern
$\texttt{BAAB}$ requires four strokes to paint while excluding $\texttt{ABBA}$ 
requires only three.


.... -> AA.. -> ABBB -> ABCB

SCORING:
Test cases 1-4 satisfy $N,Q\le 100$.Test cases 5-7 satisfy
$N,Q\le 5000$.Test cases 8-13 satisfy no additional constraints.


Problem credits: Andi Qu and Brian Dean



\section*{Input Format}
The next line contains a string of length $N$ characters representing the
desired color  for each fence segment.The next $Q$ lines each contain two space-separated integers $a$ and $b$
representing a candidate range to possibly leave unpainted.

The next $Q$ lines each contain two space-separated integers $a$ and $b$
representing a candidate range to possibly leave unpainted.

OUTPUT FORMAT (print output to the terminal / stdout):For each of the $Q$ candidates, output the answer on a new line.SAMPLE INPUT:8 2
ABBAABCB
3 6
1 4SAMPLE OUTPUT:4
3In this example, excluding the sub-range corresponding to the desired pattern
$\texttt{BAAB}$ requires four strokes to paint while excluding $\texttt{ABBA}$ 
requires only three..... -> AA.. -> ABBB -> ABCBSCORING:Test cases 1-4 satisfy $N,Q\le 100$.Test cases 5-7 satisfy
$N,Q\le 5000$.Test cases 8-13 satisfy no additional constraints.Problem credits: Andi Qu and Brian Dean

\section*{Output Format}
SAMPLE INPUT:8 2
ABBAABCB
3 6
1 4SAMPLE OUTPUT:4
3In this example, excluding the sub-range corresponding to the desired pattern
$\texttt{BAAB}$ requires four strokes to paint while excluding $\texttt{ABBA}$ 
requires only three..... -> AA.. -> ABBB -> ABCBSCORING:Test cases 1-4 satisfy $N,Q\le 100$.Test cases 5-7 satisfy
$N,Q\le 5000$.Test cases 8-13 satisfy no additional constraints.Problem credits: Andi Qu and Brian Dean

\section*{Sample Input}
\begin{verbatim}
8 2
ABBAABCB
3 6
1 4
\end{verbatim}

\section*{Sample Output}
\begin{verbatim}
4
3

In this example, excluding the sub-range corresponding to the desired pattern
$\texttt{BAAB}$ requires four strokes to paint while excluding $\texttt{ABBA}$ 
requires only three..... -> AA.. -> ABBB -> ABCBSCORING:Test cases 1-4 satisfy $N,Q\le 100$.Test cases 5-7 satisfy
$N,Q\le 5000$.Test cases 8-13 satisfy no additional constraints.Problem credits: Andi Qu and Brian Dean

.... -> AA.. -> ABBB -> ABCBSCORING:Test cases 1-4 satisfy $N,Q\le 100$.Test cases 5-7 satisfy
$N,Q\le 5000$.Test cases 8-13 satisfy no additional constraints.Problem credits: Andi Qu and Brian Dean

.... -> AA.. -> ABBB -> ABCB

SCORING:Test cases 1-4 satisfy $N,Q\le 100$.Test cases 5-7 satisfy
$N,Q\le 5000$.Test cases 8-13 satisfy no additional constraints.Problem credits: Andi Qu and Brian Dean
\end{verbatim}

\section*{Solution}


\section*{Problem URL}
https://usaco.org/index.php?page=viewproblem2&cpid=1087

\section*{Source}
USACO JAN21 Contest, Silver Division, Problem ID: 1087

\end{document}
