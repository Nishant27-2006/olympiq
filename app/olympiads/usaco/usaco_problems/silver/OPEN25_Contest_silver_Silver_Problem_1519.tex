\documentclass[12pt]{article}
\usepackage{amsmath}
\usepackage{amssymb}
\usepackage{graphicx}
\usepackage{enumitem}
\usepackage{hyperref}
\usepackage{xcolor}

\title{View problem}
\author{USACO Contest: OPEN25 Contest - Silver Division}
\date{\today}

\begin{document}

\maketitle

\section*{Problem Statement}


Deep in the countryside, Farmer John’s cows aren’t just ordinary farm
animals—they are part of a clandestine bovine intelligence network. Each cow
carries an ID number, carefully assigned by the elite cow cryptographers.
However, due to Farmer John's rather haphazard tagging system, some cows ended
up with the same ID.

Farmer John noted that there are $N$ ($1\le N\le 2\cdot 10^5$) unique ID
numbers, and for each unique ID $d_i$ ($0\le d_i\le 10^9$), there are $n_i$
($1\le n_i\le 10^9$) cows who shared it.

The cows can only communicate in pairs, and their secret encryption method has
one strict rule: two cows can only exchange information if they are not the same
cow and the sum of their ID numbers equals either $A$ or $B$
($0\le A\le B\le 2\cdot 10^9$). A cow can only engage in one conversation at a
time (i.e., no cow can be part of more than one pair).

Farmer John would like to maximize the number of disjoint communication pairs to
ensure the best information flow. Can you determine the largest number of
conversations that can happen at once?

INPUT FORMAT (input arrives from the terminal / stdin):
The first line contains $N$, $A$, $B$.

The next $N$ lines each contain $n_i$ and $d_i$. No two $d_i$ are the same. 

OUTPUT FORMAT (print output to the terminal / stdout):
The maximum number of disjoint communicating pairs that can be formed at the
same time.
    Note that the large size of integers involved in this problem may require the
use of 64-bit integer data types (e.g., a "long long" in C/C++).

SAMPLE INPUT:
4 4 5
17 2
100 0
10 1
200 4
SAMPLE OUTPUT: 
118

A cow with an ID of $0$ can communicate with another cow with an ID of $4$
because the sum of their IDs is $4$.  Since there are a total of $100$ cows of
ID $0$ and $200$ cows of ID $4$,  there can be up to $100$ communicating pairs
with this combination of IDs.

A cow with an ID of $4$ can also communicate with another cow with an ID of $1$
(sum to $5$). There are $10$ cows of ID $1$ and $100$ remaining unpaired cows of
ID $4$, allowing for another $10$ pairs.

Finally, a cow with an ID of $2$ can communicate with another cow of the same
ID.  Since there are a total of $17$ cows of ID $2$, there can be up to $8$ more
pairs.

In total, there are $100+10+8=118$ communicating pairs. It can be shown that
this is the maximum possible number of pairs.

SAMPLE INPUT:
4 4 5
100 0
10 1
100 3
20 4
SAMPLE OUTPUT: 
30

Pairing IDs $0$ and $4$ makes $20$ pairs, while pairing IDs $1$ and $3$ makes
$10$ pairs. It can be shown that this is the optimal pairing, resulting in a
total of $30$ pairs.

SCORING:
Inputs 3-4: $A=B$ Inputs 5-7: $N\le 1000$ Inputs 8-12: No additional constraints


Problem credits: Benjamin Qi



\section*{Input Format}
The next $N$ lines each contain $n_i$ and $d_i$. No two $d_i$ are the same.

OUTPUT FORMAT (print output to the terminal / stdout):The maximum number of disjoint communicating pairs that can be formed at the
same time.Note that the large size of integers involved in this problem may require the
use of 64-bit integer data types (e.g., a "long long" in C/C++).SAMPLE INPUT:4 4 5
17 2
100 0
10 1
200 4SAMPLE OUTPUT:118A cow with an ID of $0$ can communicate with another cow with an ID of $4$
because the sum of their IDs is $4$.  Since there are a total of $100$ cows of
ID $0$ and $200$ cows of ID $4$,  there can be up to $100$ communicating pairs
with this combination of IDs.A cow with an ID of $4$ can also communicate with another cow with an ID of $1$
(sum to $5$). There are $10$ cows of ID $1$ and $100$ remaining unpaired cows of
ID $4$, allowing for another $10$ pairs.Finally, a cow with an ID of $2$ can communicate with another cow of the same
ID.  Since there are a total of $17$ cows of ID $2$, there can be up to $8$ more
pairs.In total, there are $100+10+8=118$ communicating pairs. It can be shown that
this is the maximum possible number of pairs.SAMPLE INPUT:4 4 5
100 0
10 1
100 3
20 4SAMPLE OUTPUT:30Pairing IDs $0$ and $4$ makes $20$ pairs, while pairing IDs $1$ and $3$ makes
$10$ pairs. It can be shown that this is the optimal pairing, resulting in a
total of $30$ pairs.SCORING:Inputs 3-4: $A=B$Inputs 5-7: $N\le 1000$Inputs 8-12: No additional constraintsProblem credits: Benjamin Qi

\section*{Output Format}
Note that the large size of integers involved in this problem may require the
use of 64-bit integer data types (e.g., a "long long" in C/C++).

SAMPLE INPUT:4 4 5
17 2
100 0
10 1
200 4SAMPLE OUTPUT:118A cow with an ID of $0$ can communicate with another cow with an ID of $4$
because the sum of their IDs is $4$.  Since there are a total of $100$ cows of
ID $0$ and $200$ cows of ID $4$,  there can be up to $100$ communicating pairs
with this combination of IDs.A cow with an ID of $4$ can also communicate with another cow with an ID of $1$
(sum to $5$). There are $10$ cows of ID $1$ and $100$ remaining unpaired cows of
ID $4$, allowing for another $10$ pairs.Finally, a cow with an ID of $2$ can communicate with another cow of the same
ID.  Since there are a total of $17$ cows of ID $2$, there can be up to $8$ more
pairs.In total, there are $100+10+8=118$ communicating pairs. It can be shown that
this is the maximum possible number of pairs.SAMPLE INPUT:4 4 5
100 0
10 1
100 3
20 4SAMPLE OUTPUT:30Pairing IDs $0$ and $4$ makes $20$ pairs, while pairing IDs $1$ and $3$ makes
$10$ pairs. It can be shown that this is the optimal pairing, resulting in a
total of $30$ pairs.SCORING:Inputs 3-4: $A=B$Inputs 5-7: $N\le 1000$Inputs 8-12: No additional constraintsProblem credits: Benjamin Qi

\section*{Sample Input}
\begin{verbatim}
4 4 5
17 2
100 0
10 1
200 4

4 4 5
100 0
10 1
100 3
20 4
\end{verbatim}

\section*{Sample Output}
\begin{verbatim}
118

A cow with an ID of $4$ can also communicate with another cow with an ID of $1$
(sum to $5$). There are $10$ cows of ID $1$ and $100$ remaining unpaired cows of
ID $4$, allowing for another $10$ pairs.Finally, a cow with an ID of $2$ can communicate with another cow of the same
ID.  Since there are a total of $17$ cows of ID $2$, there can be up to $8$ more
pairs.In total, there are $100+10+8=118$ communicating pairs. It can be shown that
this is the maximum possible number of pairs.SAMPLE INPUT:4 4 5
100 0
10 1
100 3
20 4SAMPLE OUTPUT:30Pairing IDs $0$ and $4$ makes $20$ pairs, while pairing IDs $1$ and $3$ makes
$10$ pairs. It can be shown that this is the optimal pairing, resulting in a
total of $30$ pairs.SCORING:Inputs 3-4: $A=B$Inputs 5-7: $N\le 1000$Inputs 8-12: No additional constraintsProblem credits: Benjamin Qi

Finally, a cow with an ID of $2$ can communicate with another cow of the same
ID.  Since there are a total of $17$ cows of ID $2$, there can be up to $8$ more
pairs.In total, there are $100+10+8=118$ communicating pairs. It can be shown that
this is the maximum possible number of pairs.SAMPLE INPUT:4 4 5
100 0
10 1
100 3
20 4SAMPLE OUTPUT:30Pairing IDs $0$ and $4$ makes $20$ pairs, while pairing IDs $1$ and $3$ makes
$10$ pairs. It can be shown that this is the optimal pairing, resulting in a
total of $30$ pairs.SCORING:Inputs 3-4: $A=B$Inputs 5-7: $N\le 1000$Inputs 8-12: No additional constraintsProblem credits: Benjamin Qi

In total, there are $100+10+8=118$ communicating pairs. It can be shown that
this is the maximum possible number of pairs.SAMPLE INPUT:4 4 5
100 0
10 1
100 3
20 4SAMPLE OUTPUT:30Pairing IDs $0$ and $4$ makes $20$ pairs, while pairing IDs $1$ and $3$ makes
$10$ pairs. It can be shown that this is the optimal pairing, resulting in a
total of $30$ pairs.SCORING:Inputs 3-4: $A=B$Inputs 5-7: $N\le 1000$Inputs 8-12: No additional constraintsProblem credits: Benjamin Qi

SAMPLE INPUT:4 4 5
100 0
10 1
100 3
20 4SAMPLE OUTPUT:30Pairing IDs $0$ and $4$ makes $20$ pairs, while pairing IDs $1$ and $3$ makes
$10$ pairs. It can be shown that this is the optimal pairing, resulting in a
total of $30$ pairs.SCORING:Inputs 3-4: $A=B$Inputs 5-7: $N\le 1000$Inputs 8-12: No additional constraintsProblem credits: Benjamin Qi

30

Pairing IDs $0$ and $4$ makes $20$ pairs, while pairing IDs $1$ and $3$ makes
$10$ pairs. It can be shown that this is the optimal pairing, resulting in a
total of $30$ pairs.SCORING:Inputs 3-4: $A=B$Inputs 5-7: $N\le 1000$Inputs 8-12: No additional constraintsProblem credits: Benjamin Qi

SCORING:Inputs 3-4: $A=B$Inputs 5-7: $N\le 1000$Inputs 8-12: No additional constraintsProblem credits: Benjamin Qi
\end{verbatim}

\section*{Solution}


\section*{Problem URL}
https://usaco.org/index.php?page=viewproblem2&cpid=1519

\section*{Source}
USACO OPEN25 Contest, Silver Division, Problem ID: 1519

\end{document}
