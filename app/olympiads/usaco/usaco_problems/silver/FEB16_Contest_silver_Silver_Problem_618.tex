\documentclass[12pt]{article}
\usepackage{amsmath}
\usepackage{amssymb}
\usepackage{graphicx}
\usepackage{enumitem}
\usepackage{hyperref}
\usepackage{xcolor}

\title{View problem}
\author{USACO Contest: FEB16 Contest - Silver Division}
\date{\today}

\begin{document}

\maketitle

\section*{Problem Statement}

Being a fan of contemporary architecture, Farmer John has built a new barn in
the shape of a perfect circle.  Inside, the barn consists of a ring of $n$
rooms,  numbered clockwise from $1 \ldots n$ around the perimeter of the barn
($3 \leq n \leq 1000$).   Each room has doors to its two neighboring rooms, and
also a door opening to the exterior of the barn.

Farmer John owns $n$ cows, and he wants exactly one cow to end up in each room
in the barn.  However, the cows, being slightly confused, line up at haphazard
doors, with possibly multiple cows lining up at the same door.  Precisely $c_i$
cows line up outside the door to room $i$, so $\sum c_i = n$.

To manage the process of herding the cows so that one cow ends up in each room,
Farmer John wants to use the following approach: each cow enters at the door at
which she initially lined up, then walks clockwise through the rooms until she
reaches a suitable destination.  Given that a cow walking through $d$ doors
consumes $d^2$ energy, please determine the minimum amount of energy needed to
distribute the cows so one ends up in each room.  

INPUT FORMAT (file cbarn.in):
The first line of input contains $n$.  Each of the remaining $n$ lines contain
$c_1 \ldots c_n$.

OUTPUT FORMAT (file cbarn.out):
Please write out the minimum amount of energy consumed by the cows.  

SAMPLE INPUT:
10
1
0
0
2
0
0
1
2
2
2
SAMPLE OUTPUT: 
33

Problem credits: Brian Dean



\section*{Input Format}
OUTPUT FORMAT (file cbarn.out):Please write out the minimum amount of energy consumed by the cows.SAMPLE INPUT:10
1
0
0
2
0
0
1
2
2
2SAMPLE OUTPUT:33Problem credits: Brian Dean

\section*{Output Format}
SAMPLE INPUT:10
1
0
0
2
0
0
1
2
2
2SAMPLE OUTPUT:33Problem credits: Brian Dean

\section*{Sample Input}
\begin{verbatim}
10
1
0
0
2
0
0
1
2
2
2
\end{verbatim}

\section*{Sample Output}
\begin{verbatim}
33

Problem credits: Brian Dean
\end{verbatim}

\section*{Solution}


\section*{Problem URL}
https://usaco.org/index.php?page=viewproblem2&cpid=618

\section*{Source}
USACO FEB16 Contest, Silver Division, Problem ID: 618

\end{document}
