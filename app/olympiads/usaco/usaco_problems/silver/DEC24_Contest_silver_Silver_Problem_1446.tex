\documentclass[12pt]{article}
\usepackage{amsmath}
\usepackage{amssymb}
\usepackage{graphicx}
\usepackage{enumitem}
\usepackage{hyperref}
\usepackage{xcolor}

\title{View problem}
\author{USACO Contest: DEC24 Contest - Silver Division}
\date{\today}

\begin{document}

\maketitle

\section*{Problem Statement}


Bessie and Elsie have discovered a row of $N$ cakes
$(2 \leq N \leq 5\cdot 10^5, N \text{ even})$ cakes, with sizes
$a_1,a_2,\dots,a_N$ in that order ($1\le a_i\le 10^9$).

Each cow wants to eat as much as possible.  However, being very civilized cows,
they decided to play a game to split it! The game proceeds with both cows
alternating turns. Each turn consists of one of the following:

Bessie chooses two adjacent cakes and stacks them, creating a new cake with
the sum of the sizes.Elsie chooses either the leftmost or rightmost cake and puts it in her
stash.
When only one cake remains, Bessie eats it, and Elsie eats all cakes in her
stash. If both cows play optimally to maximize the amount of cake they eat and
Bessie plays first, how much cake will each cow eat?

INPUT FORMAT (input arrives from the terminal / stdin):
Each input consists of $T$ ($1\le T\le 10$) independent test cases. It is
guaranteed that the sum of all $N$ within an input does not exceed $10^6$.

Each test case is formatted as follows. The first line contains $N$. The next
line contains $N$ space-separated integers, $a_1,a_2,\ldots,a_N$.

OUTPUT FORMAT (print output to the terminal / stdout):
For each test case, output a line containing $b$ and $e$, representing the
amounts of cake Bessie and Elsie respectively will consume if both cows play
optimally.

SAMPLE INPUT:
2
4
40 30 20 10
4
10 20 30 40
SAMPLE OUTPUT: 
60 40
60 40

For the first test case, under optimal play,

Bessie will stack the middle two cakes. The cakes now have sizes
$[40,50,10]$.Elsie will eat the leftmost cake. The remaining cakes now
have sizes
$[50,10]$.Bessie stacks the remaining two cakes.
Bessie will eat $30+20+10=60$ cake, while Elsie will eat $40$ cake.

The second test case is the reverse of the first, so the answer is the same.

SCORING:
Input 2: All $a_i$ are equal.Input 3: $N\le 10$Inputs 4-7: $N\le 5000$Inputs 8-11: No additional constraints.


Problem credits: Linda Zhao, Agastya Goel, Gavin Ye



\section*{Input Format}
Each test case is formatted as follows. The first line contains $N$. The next
line contains $N$ space-separated integers, $a_1,a_2,\ldots,a_N$.

OUTPUT FORMAT (print output to the terminal / stdout):For each test case, output a line containing $b$ and $e$, representing the
amounts of cake Bessie and Elsie respectively will consume if both cows play
optimally.SAMPLE INPUT:2
4
40 30 20 10
4
10 20 30 40SAMPLE OUTPUT:60 40
60 40For the first test case, under optimal play,Bessie will stack the middle two cakes. The cakes now have sizes
$[40,50,10]$.Elsie will eat the leftmost cake. The remaining cakes now
have sizes
$[50,10]$.Bessie stacks the remaining two cakes.Bessie will eat $30+20+10=60$ cake, while Elsie will eat $40$ cake.The second test case is the reverse of the first, so the answer is the same.SCORING:Input 2: All $a_i$ are equal.Input 3: $N\le 10$Inputs 4-7: $N\le 5000$Inputs 8-11: No additional constraints.Problem credits: Linda Zhao, Agastya Goel, Gavin Ye

\section*{Output Format}
SAMPLE INPUT:2
4
40 30 20 10
4
10 20 30 40SAMPLE OUTPUT:60 40
60 40For the first test case, under optimal play,Bessie will stack the middle two cakes. The cakes now have sizes
$[40,50,10]$.Elsie will eat the leftmost cake. The remaining cakes now
have sizes
$[50,10]$.Bessie stacks the remaining two cakes.Bessie will eat $30+20+10=60$ cake, while Elsie will eat $40$ cake.The second test case is the reverse of the first, so the answer is the same.SCORING:Input 2: All $a_i$ are equal.Input 3: $N\le 10$Inputs 4-7: $N\le 5000$Inputs 8-11: No additional constraints.Problem credits: Linda Zhao, Agastya Goel, Gavin Ye

\section*{Sample Input}
\begin{verbatim}
2
4
40 30 20 10
4
10 20 30 40
\end{verbatim}

\section*{Sample Output}
\begin{verbatim}
60 40
60 40

For the first test case, under optimal play,Bessie will stack the middle two cakes. The cakes now have sizes
$[40,50,10]$.Elsie will eat the leftmost cake. The remaining cakes now
have sizes
$[50,10]$.Bessie stacks the remaining two cakes.Bessie will eat $30+20+10=60$ cake, while Elsie will eat $40$ cake.The second test case is the reverse of the first, so the answer is the same.SCORING:Input 2: All $a_i$ are equal.Input 3: $N\le 10$Inputs 4-7: $N\le 5000$Inputs 8-11: No additional constraints.Problem credits: Linda Zhao, Agastya Goel, Gavin Ye

Bessie will stack the middle two cakes. The cakes now have sizes
$[40,50,10]$.Elsie will eat the leftmost cake. The remaining cakes now
have sizes
$[50,10]$.Bessie stacks the remaining two cakes.Bessie will eat $30+20+10=60$ cake, while Elsie will eat $40$ cake.The second test case is the reverse of the first, so the answer is the same.SCORING:Input 2: All $a_i$ are equal.Input 3: $N\le 10$Inputs 4-7: $N\le 5000$Inputs 8-11: No additional constraints.Problem credits: Linda Zhao, Agastya Goel, Gavin Ye

Bessie will eat $30+20+10=60$ cake, while Elsie will eat $40$ cake.The second test case is the reverse of the first, so the answer is the same.SCORING:Input 2: All $a_i$ are equal.Input 3: $N\le 10$Inputs 4-7: $N\le 5000$Inputs 8-11: No additional constraints.Problem credits: Linda Zhao, Agastya Goel, Gavin Ye

The second test case is the reverse of the first, so the answer is the same.SCORING:Input 2: All $a_i$ are equal.Input 3: $N\le 10$Inputs 4-7: $N\le 5000$Inputs 8-11: No additional constraints.Problem credits: Linda Zhao, Agastya Goel, Gavin Ye

SCORING:Input 2: All $a_i$ are equal.Input 3: $N\le 10$Inputs 4-7: $N\le 5000$Inputs 8-11: No additional constraints.Problem credits: Linda Zhao, Agastya Goel, Gavin Ye
\end{verbatim}

\section*{Solution}


\section*{Problem URL}
https://usaco.org/index.php?page=viewproblem2&cpid=1446

\section*{Source}
USACO DEC24 Contest, Silver Division, Problem ID: 1446

\end{document}
