\documentclass[12pt]{article}
\usepackage{amsmath}
\usepackage{amssymb}
\usepackage{graphicx}
\usepackage{enumitem}
\usepackage{hyperref}
\usepackage{xcolor}

\title{View problem}
\author{USACO Contest: DEC15 Contest - Silver Division}
\date{\today}

\begin{document}

\maketitle

\section*{Problem Statement}

Farmer John's $N$ cows, conveniently numbered $1 \ldots N$, are all standing in
a row (they seem to do so often that it now takes very little prompting from Farmer
John to line them up).  Each cow has a breed ID: 1 for Holsteins, 2 for
Guernseys, and 3 for Jerseys. Farmer John would like your help counting the
number of cows of each breed that lie within certain intervals of the ordering.

INPUT FORMAT (file bcount.in):

The first line of input contains $N$ and $Q$ ($1 \leq N \leq 100,000$,
$1 \leq Q \leq 100,000$).  

The next $N$ lines contain an integer that is either 1, 2, or 3, giving the 
breed ID of a single cow in the ordering.

The next $Q$ lines describe a query in the form of two integers $a, b$
($a \leq b$).


OUTPUT FORMAT (file bcount.out):

For each of the $Q$ queries $(a,b)$, print a line containing three numbers: the
number of cows numbered $a \ldots b$ that are Holsteins (breed 1), Guernseys
(breed 2), and Jerseys (breed 3).


SAMPLE INPUT:
6 3
2
1
1
3
2
1
1 6
3 3
2 4

SAMPLE OUTPUT: 
3 2 1
1 0 0
2 0 1

Problem credits: Nick Wu



\section*{Input Format}
The next $N$ lines contain an integer that is either 1, 2, or 3, giving the 
breed ID of a single cow in the ordering.The next $Q$ lines describe a query in the form of two integers $a, b$
($a \leq b$).

The next $Q$ lines describe a query in the form of two integers $a, b$
($a \leq b$).

OUTPUT FORMAT (file bcount.out):For each of the $Q$ queries $(a,b)$, print a line containing three numbers: the
number of cows numbered $a \ldots b$ that are Holsteins (breed 1), Guernseys
(breed 2), and Jerseys (breed 3).SAMPLE INPUT:6 3
2
1
1
3
2
1
1 6
3 3
2 4SAMPLE OUTPUT:3 2 1
1 0 0
2 0 1Problem credits: Nick Wu

\section*{Output Format}
SAMPLE INPUT:6 3
2
1
1
3
2
1
1 6
3 3
2 4SAMPLE OUTPUT:3 2 1
1 0 0
2 0 1Problem credits: Nick Wu

\section*{Sample Input}
\begin{verbatim}
6 3
2
1
1
3
2
1
1 6
3 3
2 4
\end{verbatim}

\section*{Sample Output}
\begin{verbatim}
3 2 1
1 0 0
2 0 1

Problem credits: Nick Wu
\end{verbatim}

\section*{Solution}


\section*{Problem URL}
https://usaco.org/index.php?page=viewproblem2&cpid=572

\section*{Source}
USACO DEC15 Contest, Silver Division, Problem ID: 572

\end{document}
