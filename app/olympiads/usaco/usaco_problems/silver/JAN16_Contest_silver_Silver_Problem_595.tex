\documentclass[12pt]{article}
\usepackage{amsmath}
\usepackage{amssymb}
\usepackage{graphicx}
\usepackage{enumitem}
\usepackage{hyperref}
\usepackage{xcolor}

\title{View problem}
\author{USACO Contest: JAN16 Contest - Silver Division}
\date{\today}

\begin{document}

\maketitle

\section*{Problem Statement}

Farmer John's $N$ cows are standing in a row, as they have a tendency to do from
time to time.  Each cow is labeled with a distinct integer ID number so FJ can
tell them apart. FJ would like to take a photo of a contiguous group of cows
but, due to a traumatic  childhood incident involving the numbers $1 \ldots 6$,
he only wants to take a picture of a group of cows if their IDs add up to a
multiple of 7.  

Please help FJ determine the size of the largest group he can photograph.  
INPUT FORMAT (file div7.in):
The first line of input contains $N$ ($1 \leq N \leq 50,000$).  The next $N$
lines each contain the $N$ integer IDs of the cows (all are in the range
$0 \ldots 1,000,000$).

OUTPUT FORMAT (file div7.out):
Please output the number of cows in the largest consecutive group whose IDs sum
to a multiple of 7.  If no such group exists, output 0.  

You may want to note that the sum of the IDs of a large group of cows might be
too large to fit into a standard 32-bit integer.  If you are summing up large
groups of IDs, you may therefore want to use a larger integer data type, like a
64-bit "long long" in C/C++.

SAMPLE INPUT:
7
3
5
1
6
2
14
10
SAMPLE OUTPUT: 
5

In this example, 5+1+6+2+14 = 28.

Problem credits: Brian Dean



\section*{Input Format}
OUTPUT FORMAT (file div7.out):Please output the number of cows in the largest consecutive group whose IDs sum
to a multiple of 7.  If no such group exists, output 0.You may want to note that the sum of the IDs of a large group of cows might be
too large to fit into a standard 32-bit integer.  If you are summing up large
groups of IDs, you may therefore want to use a larger integer data type, like a
64-bit "long long" in C/C++.SAMPLE INPUT:7
3
5
1
6
2
14
10SAMPLE OUTPUT:5In this example, 5+1+6+2+14 = 28.Problem credits: Brian Dean

\section*{Output Format}
You may want to note that the sum of the IDs of a large group of cows might be
too large to fit into a standard 32-bit integer.  If you are summing up large
groups of IDs, you may therefore want to use a larger integer data type, like a
64-bit "long long" in C/C++.SAMPLE INPUT:7
3
5
1
6
2
14
10SAMPLE OUTPUT:5In this example, 5+1+6+2+14 = 28.Problem credits: Brian Dean

SAMPLE INPUT:7
3
5
1
6
2
14
10SAMPLE OUTPUT:5In this example, 5+1+6+2+14 = 28.Problem credits: Brian Dean

\section*{Sample Input}
\begin{verbatim}
7
3
5
1
6
2
14
10
\end{verbatim}

\section*{Sample Output}
\begin{verbatim}
5

In this example, 5+1+6+2+14 = 28.Problem credits: Brian Dean

Problem credits: Brian Dean
\end{verbatim}

\section*{Solution}


\section*{Problem URL}
https://usaco.org/index.php?page=viewproblem2&cpid=595

\section*{Source}
USACO JAN16 Contest, Silver Division, Problem ID: 595

\end{document}
