\documentclass[12pt]{article}
\usepackage{amsmath}
\usepackage{amssymb}
\usepackage{graphicx}
\usepackage{enumitem}
\usepackage{hyperref}
\usepackage{xcolor}

\title{View problem}
\author{USACO Contest: FEB17 Contest - Silver Division}
\date{\today}

\begin{document}

\maketitle

\section*{Problem Statement}

Why did the cow cross the road?  Well, one reason is that Farmer John's farm
simply has a lot of roads, making it impossible for his cows to travel around
without crossing many of them.

FJ's farm is arranged as an $N \times N$ square grid of fields
($2 \leq N \leq 100$), Certain pairs of adjacent fields (e.g., north-south or
east-west) are separated by roads, and a tall fence runs around the external 
perimeter of the entire grid, preventing cows from leaving the farm.   Cows can
move freely from any field to any other adjacent field (north, east, south, or
west), although they prefer not to cross roads unless absolutely necessary.

There are $K$ cows ($1 \leq K \leq 100, K \leq N^2$) on FJ's farm, each located
in a  different field.  A pair of cows is said to be "distant" if, in order for
one cow to visit the other, it is necessary to cross at least one road. Please
help FJ count the number of distant pairs of cows.

INPUT FORMAT (file countcross.in):
The first line of input contains $N$, $K$, and $R$.  The next $R$ lines 
describe $R$ roads that exist between pairs of adjacent fields.  Each line is of
the form $r$ $c$ $r'$ $c'$ (integers in the range $1 \ldots N$), indicating a
road between the field in  (row $r$, column $c$) and the adjacent field in (row $r'$,
column $c'$).  The final $K$ lines indicate the locations of the $K$ cows, each
specified  in terms of a row and column.

OUTPUT FORMAT (file countcross.out):
Print the number of pairs of cows that are distant.

SAMPLE INPUT:
3 3 3
2 2 2 3
3 3 3 2
3 3 2 3
3 3
2 2
2 3
SAMPLE OUTPUT: 
2


Problem credits: Brian Dean



\section*{Input Format}
OUTPUT FORMAT (file countcross.out):Print the number of pairs of cows that are distant.SAMPLE INPUT:3 3 3
2 2 2 3
3 3 3 2
3 3 2 3
3 3
2 2
2 3SAMPLE OUTPUT:2Problem credits: Brian Dean

\section*{Output Format}
SAMPLE INPUT:3 3 3
2 2 2 3
3 3 3 2
3 3 2 3
3 3
2 2
2 3SAMPLE OUTPUT:2Problem credits: Brian Dean

\section*{Sample Input}
\begin{verbatim}
3 3 3
2 2 2 3
3 3 3 2
3 3 2 3
3 3
2 2
2 3
\end{verbatim}

\section*{Sample Output}
\begin{verbatim}
2

Problem credits: Brian Dean

Problem credits: Brian Dean
\end{verbatim}

\section*{Solution}


\section*{Problem URL}
https://usaco.org/index.php?page=viewproblem2&cpid=716

\section*{Source}
USACO FEB17 Contest, Silver Division, Problem ID: 716

\end{document}
