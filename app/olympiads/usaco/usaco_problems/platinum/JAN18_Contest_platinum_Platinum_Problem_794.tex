\documentclass[12pt]{article}
\usepackage{amsmath}
\usepackage{amssymb}
\usepackage{graphicx}
\usepackage{enumitem}
\usepackage{hyperref}
\usepackage{xcolor}

\title{View problem}
\author{USACO Contest: JAN18 Contest - Platinum Division}
\date{\today}

\begin{document}

\maketitle

\section*{Problem Statement}

Farmer John has a large field, and he is thinking of planting sweet corn in some
part of it. After surveying his field, FJ found that it forms an
$(N-1) \times (N-1)$ square. The southwest corner is at coordinates $(0,0)$, and
the northeast corner is at $(N-1,N-1)$.

At some integer coordinates there are double-headed sprinklers, each one
sprinkling both water and fertilizer. A double-heading sprinkler at coordinates
$(i,j)$  sprinkles water on the part of the field north and east of it, and
sprinkles fertilizer on the part of the field south and west of it. Formally, it
waters all real coordinates $(x,y)$ for which $N \geq x \geq i$ and
$N \geq y \geq j$, and it fertilizes all real coordinates $(x,y)$ for which
$0 \leq x \leq i$ and $0 \leq y \leq j$.

Farmer John wants to plant sweet corn in some axis-aligned rectangle in his
field with integer-valued corner coordinates. However, for the sweet corn to
grow, all points in the rectangle must be both watered and fertilized by the
double-headed sprinklers. And of course the rectangle must have positive area,
or Farmer John wouldn't be able to grow any corn in it!

Help Farmer John determine the number of rectangles of positive area in which he
could grow sweet corn. Since this number may be large, output the remainder of
this number modulo
$10^9 + 7$.

INPUT FORMAT (file sprinklers.in):
The first line of the input consists of a single integer $N$, the size of the
field ($1 \leq N \leq 10^5$).

The next $N$ lines each contain two space-separated integers. If these integers
are $i$ and $j$, where $0 \leq i,j \leq N-1$, they denote a sprinkler located at
$(i,j)$.

It is guaranteed that there is exactly one sprinkler in each column and exactly
one sprinkler in each row. That is, no two sprinklers have the same
$x$-coordinate, and no two sprinklers have the same $y$-coordinate.


OUTPUT FORMAT (file sprinklers.out):
The output should consist of a single integer: the number of rectangles of
positive area which are fully watered and fully fertilized, modulo $10^9 + 7$.

SAMPLE INPUT:
5
0 4
1 1
2 2
3 0
4 3
SAMPLE OUTPUT: 
21


Problem credits: Dhruv Rohatgi



\section*{Input Format}
The next $N$ lines each contain two space-separated integers. If these integers
are $i$ and $j$, where $0 \leq i,j \leq N-1$, they denote a sprinkler located at
$(i,j)$.It is guaranteed that there is exactly one sprinkler in each column and exactly
one sprinkler in each row. That is, no two sprinklers have the same
$x$-coordinate, and no two sprinklers have the same $y$-coordinate.

It is guaranteed that there is exactly one sprinkler in each column and exactly
one sprinkler in each row. That is, no two sprinklers have the same
$x$-coordinate, and no two sprinklers have the same $y$-coordinate.

OUTPUT FORMAT (file sprinklers.out):The output should consist of a single integer: the number of rectangles of
positive area which are fully watered and fully fertilized, modulo $10^9 + 7$.SAMPLE INPUT:5
0 4
1 1
2 2
3 0
4 3SAMPLE OUTPUT:21Problem credits: Dhruv Rohatgi

\section*{Output Format}
SAMPLE INPUT:5
0 4
1 1
2 2
3 0
4 3SAMPLE OUTPUT:21Problem credits: Dhruv Rohatgi

\section*{Sample Input}
\begin{verbatim}
5
0 4
1 1
2 2
3 0
4 3
\end{verbatim}

\section*{Sample Output}
\begin{verbatim}
21

Problem credits: Dhruv Rohatgi

Problem credits: Dhruv Rohatgi
\end{verbatim}

\section*{Solution}


\section*{Problem URL}
https://usaco.org/index.php?page=viewproblem2&cpid=794

\section*{Source}
USACO JAN18 Contest, Platinum Division, Problem ID: 794

\end{document}
