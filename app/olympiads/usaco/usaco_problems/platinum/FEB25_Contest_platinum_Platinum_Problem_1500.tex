\documentclass[12pt]{article}
\usepackage{amsmath}
\usepackage{amssymb}
\usepackage{graphicx}
\usepackage{enumitem}
\usepackage{hyperref}
\usepackage{xcolor}

\title{View problem}
\author{USACO Contest: FEB25 Contest - Platinum Division}
\date{\today}

\begin{document}

\maketitle

\section*{Problem Statement}


**Note: The time limit for this problem is 3s, 1.5x the default.**
You are given a length-$N$ integer array $a_1,a_2,\dots,a_N$
($2\le N\le 10^6, 1\le a_i\le N$). Output the sum of the answers for the
subproblem below over all $N(N+1)/2$ contiguous subarrays of $a$.

Given a nonempty list of integers, alternate the following operations (starting
with the first operation) until the list has size exactly one.

Replace two consecutive integers in the list with their minimum.Replace two consecutive integers in the list with their maximum.
Determine the maximum possible value of the final remaining integer.

For example,


[4, 10, 3] -> [4, 3] -> [4]
[3, 4, 10] -> [3, 10] -> [10]

In the first array, $(10, 3)$ is replaced by $\min(10, 3)=3$ and $(4, 3)$ is
replaced by $\max(4, 3)=4$.

INPUT FORMAT (input arrives from the terminal / stdin):
The first line contains $N$.

The second line contains $a_1,a_2,\dots,a_N$.

OUTPUT FORMAT (print output to the terminal / stdout):
The sum of the answer to the subproblem over all subarrays.

SAMPLE INPUT:
2
2 1
SAMPLE OUTPUT: 
4

The answer for $[2]$ is $2$, the answer for $[1]$ is $1$, and the answer for
$[2, 1]$ is $1$. 

Thus, our output should be $2+1+1 = 4$. 

SAMPLE INPUT:
3
3 1 3
SAMPLE OUTPUT: 
12

SAMPLE INPUT:
4
2 4 1 3
SAMPLE OUTPUT: 
22

Consider the subarray $[2, 4, 1, 3]$. 

 Applying the first operation on (1, 3), our new array is $[2, 4, 1]$.  Applying the second operation on (4, 1), our new array is $[2, 4]$.  Applying the third operation on (2, 4), our final number is $2$. 
It can be proven that $2$ is the maximum possible value of the final number. 

SCORING:
Inputs 4-5: $N\le 100$Inputs 6-7: $N\le 5000$Inputs 8-9: $\max(a)\le 10$Inputs 10-13: No additional constraints.


Problem credits: Benjamin Qi



\section*{Input Format}
The second line contains $a_1,a_2,\dots,a_N$.

OUTPUT FORMAT (print output to the terminal / stdout):The sum of the answer to the subproblem over all subarrays.SAMPLE INPUT:2
2 1SAMPLE OUTPUT:4The answer for $[2]$ is $2$, the answer for $[1]$ is $1$, and the answer for
$[2, 1]$ is $1$.Thus, our output should be $2+1+1 = 4$.SAMPLE INPUT:3
3 1 3SAMPLE OUTPUT:12SAMPLE INPUT:4
2 4 1 3SAMPLE OUTPUT:22Consider the subarray $[2, 4, 1, 3]$.Applying the first operation on (1, 3), our new array is $[2, 4, 1]$.Applying the second operation on (4, 1), our new array is $[2, 4]$.Applying the third operation on (2, 4), our final number is $2$.It can be proven that $2$ is the maximum possible value of the final number.SCORING:Inputs 4-5: $N\le 100$Inputs 6-7: $N\le 5000$Inputs 8-9: $\max(a)\le 10$Inputs 10-13: No additional constraints.Problem credits: Benjamin Qi

\section*{Output Format}
SAMPLE INPUT:2
2 1SAMPLE OUTPUT:4The answer for $[2]$ is $2$, the answer for $[1]$ is $1$, and the answer for
$[2, 1]$ is $1$.Thus, our output should be $2+1+1 = 4$.SAMPLE INPUT:3
3 1 3SAMPLE OUTPUT:12SAMPLE INPUT:4
2 4 1 3SAMPLE OUTPUT:22Consider the subarray $[2, 4, 1, 3]$.Applying the first operation on (1, 3), our new array is $[2, 4, 1]$.Applying the second operation on (4, 1), our new array is $[2, 4]$.Applying the third operation on (2, 4), our final number is $2$.It can be proven that $2$ is the maximum possible value of the final number.SCORING:Inputs 4-5: $N\le 100$Inputs 6-7: $N\le 5000$Inputs 8-9: $\max(a)\le 10$Inputs 10-13: No additional constraints.Problem credits: Benjamin Qi

\section*{Sample Input}
\begin{verbatim}
2
2 1

3
3 1 3

4
2 4 1 3
\end{verbatim}

\section*{Sample Output}
\begin{verbatim}
4

The answer for $[2]$ is $2$, the answer for $[1]$ is $1$, and the answer for
$[2, 1]$ is $1$.Thus, our output should be $2+1+1 = 4$.SAMPLE INPUT:3
3 1 3SAMPLE OUTPUT:12SAMPLE INPUT:4
2 4 1 3SAMPLE OUTPUT:22Consider the subarray $[2, 4, 1, 3]$.Applying the first operation on (1, 3), our new array is $[2, 4, 1]$.Applying the second operation on (4, 1), our new array is $[2, 4]$.Applying the third operation on (2, 4), our final number is $2$.It can be proven that $2$ is the maximum possible value of the final number.SCORING:Inputs 4-5: $N\le 100$Inputs 6-7: $N\le 5000$Inputs 8-9: $\max(a)\le 10$Inputs 10-13: No additional constraints.Problem credits: Benjamin Qi

Thus, our output should be $2+1+1 = 4$.SAMPLE INPUT:3
3 1 3SAMPLE OUTPUT:12SAMPLE INPUT:4
2 4 1 3SAMPLE OUTPUT:22Consider the subarray $[2, 4, 1, 3]$.Applying the first operation on (1, 3), our new array is $[2, 4, 1]$.Applying the second operation on (4, 1), our new array is $[2, 4]$.Applying the third operation on (2, 4), our final number is $2$.It can be proven that $2$ is the maximum possible value of the final number.SCORING:Inputs 4-5: $N\le 100$Inputs 6-7: $N\le 5000$Inputs 8-9: $\max(a)\le 10$Inputs 10-13: No additional constraints.Problem credits: Benjamin Qi

SAMPLE INPUT:3
3 1 3SAMPLE OUTPUT:12SAMPLE INPUT:4
2 4 1 3SAMPLE OUTPUT:22Consider the subarray $[2, 4, 1, 3]$.Applying the first operation on (1, 3), our new array is $[2, 4, 1]$.Applying the second operation on (4, 1), our new array is $[2, 4]$.Applying the third operation on (2, 4), our final number is $2$.It can be proven that $2$ is the maximum possible value of the final number.SCORING:Inputs 4-5: $N\le 100$Inputs 6-7: $N\le 5000$Inputs 8-9: $\max(a)\le 10$Inputs 10-13: No additional constraints.Problem credits: Benjamin Qi

12

SAMPLE INPUT:4
2 4 1 3SAMPLE OUTPUT:22Consider the subarray $[2, 4, 1, 3]$.Applying the first operation on (1, 3), our new array is $[2, 4, 1]$.Applying the second operation on (4, 1), our new array is $[2, 4]$.Applying the third operation on (2, 4), our final number is $2$.It can be proven that $2$ is the maximum possible value of the final number.SCORING:Inputs 4-5: $N\le 100$Inputs 6-7: $N\le 5000$Inputs 8-9: $\max(a)\le 10$Inputs 10-13: No additional constraints.Problem credits: Benjamin Qi

22

Consider the subarray $[2, 4, 1, 3]$.Applying the first operation on (1, 3), our new array is $[2, 4, 1]$.Applying the second operation on (4, 1), our new array is $[2, 4]$.Applying the third operation on (2, 4), our final number is $2$.It can be proven that $2$ is the maximum possible value of the final number.SCORING:Inputs 4-5: $N\le 100$Inputs 6-7: $N\le 5000$Inputs 8-9: $\max(a)\le 10$Inputs 10-13: No additional constraints.Problem credits: Benjamin Qi

Applying the first operation on (1, 3), our new array is $[2, 4, 1]$.Applying the second operation on (4, 1), our new array is $[2, 4]$.Applying the third operation on (2, 4), our final number is $2$.It can be proven that $2$ is the maximum possible value of the final number.SCORING:Inputs 4-5: $N\le 100$Inputs 6-7: $N\le 5000$Inputs 8-9: $\max(a)\le 10$Inputs 10-13: No additional constraints.Problem credits: Benjamin Qi

It can be proven that $2$ is the maximum possible value of the final number.SCORING:Inputs 4-5: $N\le 100$Inputs 6-7: $N\le 5000$Inputs 8-9: $\max(a)\le 10$Inputs 10-13: No additional constraints.Problem credits: Benjamin Qi

SCORING:Inputs 4-5: $N\le 100$Inputs 6-7: $N\le 5000$Inputs 8-9: $\max(a)\le 10$Inputs 10-13: No additional constraints.Problem credits: Benjamin Qi
\end{verbatim}

\section*{Solution}


\section*{Problem URL}
https://usaco.org/index.php?page=viewproblem2&cpid=1500

\section*{Source}
USACO FEB25 Contest, Platinum Division, Problem ID: 1500

\end{document}
