\documentclass[12pt]{article}
\usepackage{amsmath}
\usepackage{amssymb}
\usepackage{graphicx}
\usepackage{enumitem}
\usepackage{hyperref}
\usepackage{xcolor}

\title{View problem}
\author{USACO Contest: OPEN23 Contest - Platinum Division}
\date{\today}

\begin{document}

\maketitle

\section*{Problem Statement}


There are initially $N-1$ pairs of friends among FJ's $N$
($2\le N\le 2\cdot 10^5$) cows labeled $1\dots N$, forming a tree. The cows are
leaving the farm for vacation one by one. On day $i$, the $i$th cow leaves the
farm, and then all pairs of the $i$th cow's friends still present on the farm
become friends. 

For each $i$ from $1$ to $N$, just before the $i$th cow leaves,  how many
ordered triples of distinct cows $(a,b,c)$ are there such that none of $a,b,c$
are on vacation, $a$ is friends with $b$, and $b$ is friends with $c$?

INPUT FORMAT (input arrives from the terminal / stdin):
The first line contains $N$.

The next $N-1$ lines contain two integers $u_i$ and $v_i$ denoting that cows
$u_i$ and $v_i$ are initially friends ($1\le u_i,v_i\le N$).

OUTPUT FORMAT (print output to the terminal / stdout):
The answers for $i$ from $1$ to $N$ on separate lines.

SAMPLE INPUT:
3
1 2
2 3
SAMPLE OUTPUT: 
2
0
0

$(1,2,3)$ and $(3,2,1)$ are the triples just before cow $1$ leaves.

After cow
$1$ leaves, there are less than $3$ cows left, so no triples are possible.

SAMPLE INPUT:
4
1 2
1 3
1 4
SAMPLE OUTPUT: 
6
6
0
0

At the beginning, cow $1$ is friends with all other cows, and no other pairs of
cows are friends, so the triples are $(a, 1, c)$ where $a, c$ are different cows
from $\{2, 3, 4\}$, which gives $3 \cdot 2 = 6$ triples.

After cow $1$ leaves, the remaining three cows are all friends, so the triples
are just those three cows in any of the $3! = 6$ possible orders.

After cow $2$ leaves, there are less than $3$ cows left, so no triples are
possible.

SAMPLE INPUT:
5
3 5
5 1
1 4
1 2
SAMPLE OUTPUT: 
8
10
2
0
0

SCORING:
Inputs 4-5: $N\le 500$Inputs 6-10: $N\le 5000$Inputs 11-20: No additional constraints.


Problem credits: Aryansh Shrivastava, Benjamin Qi



\section*{Input Format}
The next $N-1$ lines contain two integers $u_i$ and $v_i$ denoting that cows
$u_i$ and $v_i$ are initially friends ($1\le u_i,v_i\le N$).

OUTPUT FORMAT (print output to the terminal / stdout):The answers for $i$ from $1$ to $N$ on separate lines.SAMPLE INPUT:3
1 2
2 3SAMPLE OUTPUT:2
0
0$(1,2,3)$ and $(3,2,1)$ are the triples just before cow $1$ leaves.After cow
$1$ leaves, there are less than $3$ cows left, so no triples are possible.SAMPLE INPUT:4
1 2
1 3
1 4SAMPLE OUTPUT:6
6
0
0At the beginning, cow $1$ is friends with all other cows, and no other pairs of
cows are friends, so the triples are $(a, 1, c)$ where $a, c$ are different cows
from $\{2, 3, 4\}$, which gives $3 \cdot 2 = 6$ triples.After cow $1$ leaves, the remaining three cows are all friends, so the triples
are just those three cows in any of the $3! = 6$ possible orders.After cow $2$ leaves, there are less than $3$ cows left, so no triples are
possible.SAMPLE INPUT:5
3 5
5 1
1 4
1 2SAMPLE OUTPUT:8
10
2
0
0SCORING:Inputs 4-5: $N\le 500$Inputs 6-10: $N\le 5000$Inputs 11-20: No additional constraints.Problem credits: Aryansh Shrivastava, Benjamin Qi

\section*{Output Format}
SAMPLE INPUT:3
1 2
2 3SAMPLE OUTPUT:2
0
0$(1,2,3)$ and $(3,2,1)$ are the triples just before cow $1$ leaves.After cow
$1$ leaves, there are less than $3$ cows left, so no triples are possible.SAMPLE INPUT:4
1 2
1 3
1 4SAMPLE OUTPUT:6
6
0
0At the beginning, cow $1$ is friends with all other cows, and no other pairs of
cows are friends, so the triples are $(a, 1, c)$ where $a, c$ are different cows
from $\{2, 3, 4\}$, which gives $3 \cdot 2 = 6$ triples.After cow $1$ leaves, the remaining three cows are all friends, so the triples
are just those three cows in any of the $3! = 6$ possible orders.After cow $2$ leaves, there are less than $3$ cows left, so no triples are
possible.SAMPLE INPUT:5
3 5
5 1
1 4
1 2SAMPLE OUTPUT:8
10
2
0
0SCORING:Inputs 4-5: $N\le 500$Inputs 6-10: $N\le 5000$Inputs 11-20: No additional constraints.Problem credits: Aryansh Shrivastava, Benjamin Qi

\section*{Sample Input}
\begin{verbatim}
3
1 2
2 3

4
1 2
1 3
1 4

5
3 5
5 1
1 4
1 2
\end{verbatim}

\section*{Sample Output}
\begin{verbatim}
2
0
0

$(1,2,3)$ and $(3,2,1)$ are the triples just before cow $1$ leaves.After cow
$1$ leaves, there are less than $3$ cows left, so no triples are possible.SAMPLE INPUT:4
1 2
1 3
1 4SAMPLE OUTPUT:6
6
0
0At the beginning, cow $1$ is friends with all other cows, and no other pairs of
cows are friends, so the triples are $(a, 1, c)$ where $a, c$ are different cows
from $\{2, 3, 4\}$, which gives $3 \cdot 2 = 6$ triples.After cow $1$ leaves, the remaining three cows are all friends, so the triples
are just those three cows in any of the $3! = 6$ possible orders.After cow $2$ leaves, there are less than $3$ cows left, so no triples are
possible.SAMPLE INPUT:5
3 5
5 1
1 4
1 2SAMPLE OUTPUT:8
10
2
0
0SCORING:Inputs 4-5: $N\le 500$Inputs 6-10: $N\le 5000$Inputs 11-20: No additional constraints.Problem credits: Aryansh Shrivastava, Benjamin Qi

After cow
$1$ leaves, there are less than $3$ cows left, so no triples are possible.SAMPLE INPUT:4
1 2
1 3
1 4SAMPLE OUTPUT:6
6
0
0At the beginning, cow $1$ is friends with all other cows, and no other pairs of
cows are friends, so the triples are $(a, 1, c)$ where $a, c$ are different cows
from $\{2, 3, 4\}$, which gives $3 \cdot 2 = 6$ triples.After cow $1$ leaves, the remaining three cows are all friends, so the triples
are just those three cows in any of the $3! = 6$ possible orders.After cow $2$ leaves, there are less than $3$ cows left, so no triples are
possible.SAMPLE INPUT:5
3 5
5 1
1 4
1 2SAMPLE OUTPUT:8
10
2
0
0SCORING:Inputs 4-5: $N\le 500$Inputs 6-10: $N\le 5000$Inputs 11-20: No additional constraints.Problem credits: Aryansh Shrivastava, Benjamin Qi

SAMPLE INPUT:4
1 2
1 3
1 4SAMPLE OUTPUT:6
6
0
0At the beginning, cow $1$ is friends with all other cows, and no other pairs of
cows are friends, so the triples are $(a, 1, c)$ where $a, c$ are different cows
from $\{2, 3, 4\}$, which gives $3 \cdot 2 = 6$ triples.After cow $1$ leaves, the remaining three cows are all friends, so the triples
are just those three cows in any of the $3! = 6$ possible orders.After cow $2$ leaves, there are less than $3$ cows left, so no triples are
possible.SAMPLE INPUT:5
3 5
5 1
1 4
1 2SAMPLE OUTPUT:8
10
2
0
0SCORING:Inputs 4-5: $N\le 500$Inputs 6-10: $N\le 5000$Inputs 11-20: No additional constraints.Problem credits: Aryansh Shrivastava, Benjamin Qi

6
6
0
0

At the beginning, cow $1$ is friends with all other cows, and no other pairs of
cows are friends, so the triples are $(a, 1, c)$ where $a, c$ are different cows
from $\{2, 3, 4\}$, which gives $3 \cdot 2 = 6$ triples.After cow $1$ leaves, the remaining three cows are all friends, so the triples
are just those three cows in any of the $3! = 6$ possible orders.After cow $2$ leaves, there are less than $3$ cows left, so no triples are
possible.SAMPLE INPUT:5
3 5
5 1
1 4
1 2SAMPLE OUTPUT:8
10
2
0
0SCORING:Inputs 4-5: $N\le 500$Inputs 6-10: $N\le 5000$Inputs 11-20: No additional constraints.Problem credits: Aryansh Shrivastava, Benjamin Qi

After cow $1$ leaves, the remaining three cows are all friends, so the triples
are just those three cows in any of the $3! = 6$ possible orders.After cow $2$ leaves, there are less than $3$ cows left, so no triples are
possible.SAMPLE INPUT:5
3 5
5 1
1 4
1 2SAMPLE OUTPUT:8
10
2
0
0SCORING:Inputs 4-5: $N\le 500$Inputs 6-10: $N\le 5000$Inputs 11-20: No additional constraints.Problem credits: Aryansh Shrivastava, Benjamin Qi

After cow $2$ leaves, there are less than $3$ cows left, so no triples are
possible.SAMPLE INPUT:5
3 5
5 1
1 4
1 2SAMPLE OUTPUT:8
10
2
0
0SCORING:Inputs 4-5: $N\le 500$Inputs 6-10: $N\le 5000$Inputs 11-20: No additional constraints.Problem credits: Aryansh Shrivastava, Benjamin Qi

SAMPLE INPUT:5
3 5
5 1
1 4
1 2SAMPLE OUTPUT:8
10
2
0
0SCORING:Inputs 4-5: $N\le 500$Inputs 6-10: $N\le 5000$Inputs 11-20: No additional constraints.Problem credits: Aryansh Shrivastava, Benjamin Qi

8
10
2
0
0

SCORING:Inputs 4-5: $N\le 500$Inputs 6-10: $N\le 5000$Inputs 11-20: No additional constraints.Problem credits: Aryansh Shrivastava, Benjamin Qi
\end{verbatim}

\section*{Solution}


\section*{Problem URL}
https://usaco.org/index.php?page=viewproblem2&cpid=1334

\section*{Source}
USACO OPEN23 Contest, Platinum Division, Problem ID: 1334

\end{document}
