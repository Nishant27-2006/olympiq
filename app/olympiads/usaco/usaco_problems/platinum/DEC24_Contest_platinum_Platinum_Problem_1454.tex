\documentclass[12pt]{article}
\usepackage{amsmath}
\usepackage{amssymb}
\usepackage{graphicx}
\usepackage{enumitem}
\usepackage{hyperref}
\usepackage{xcolor}

\title{View problem}
\author{USACO Contest: DEC24 Contest - Platinum Division}
\date{\today}

\begin{document}

\maketitle

\section*{Problem Statement}


**Note: The time limit for this problem is 4s, twice the default.**
Moo! You are given an integer $N$ ($2\le N\le 2000$). Consider all permutations
$p=[p_0,p_1,\dots, p_{N-1}]$ of $[0,1,2\dots, N-1]$.  Let
$f(p)=\min_{i=0}^{N-2}|p_i-p_{i+1}|$ denote the minimum absolute difference
between any two consecutive elements of $p$. and let $S$ denote the set of all
$p$ that achieve the maximum possible value of $f(p)$.

You are additionally given $K$ ($0\le K\le N$) constraints of the form $p_i=j$
($0\le i,j<N$). Count the number of permutations in $S$ satisfying all
constraints, modulo $10^9+7$.

INPUT FORMAT (input arrives from the terminal / stdin):
The first line contains $T$ ($1\le TN\le 2\cdot 10^4$) and $N$, meaning that you
will need to solve $T$ independent test cases, each specified by a different set
of constraints.

Each test case starts with $K$, followed by $K$ lines each containing $i$ and
$j$. It is guaranteed that 
The same $i$ does not appear more than once within the same test case.The same $j$ does not appear more than once within the same test case.

OUTPUT FORMAT (print output to the terminal / stdout):
For each test case, the answer modulo $10^9+7$ on a separate line.

SAMPLE INPUT:
3 4
0
1
1 1
2
0 2
2 3
SAMPLE OUTPUT: 
2
0
1

The maximum possible value of $f(p)$ is $2$, and $S=\{[2,0,3,1], [1,3,0,2]\}$.

SAMPLE INPUT:
9 11
2
0 5
6 9
3
0 5
6 9
1 0
4
0 5
6 9
1 0
4 7
5
0 5
6 9
1 0
4 7
2 6
6
0 5
6 9
1 0
4 7
2 6
9 3
7
0 5
6 9
1 0
4 7
2 6
9 3
5 2
8
0 5
6 9
1 0
4 7
2 6
9 3
5 2
7 4
9
0 5
6 9
1 0
4 7
2 6
9 3
5 2
7 4
3 1
10
0 5
6 9
1 0
4 7
2 6
9 3
5 2
7 4
3 1
8 10
SAMPLE OUTPUT: 
6
6
1
1
1
1
1
1
1

$p=[5, 0, 6, 1, 7, 2, 9, 4, 10, 3, 8]$ should be counted for all test cases.

SAMPLE INPUT:
10 11
0
1
3 8
2
3 8
5 7
3
3 8
5 7
4 2
4
3 8
5 7
4 2
10 6
5
3 8
5 7
4 2
10 6
8 10
6
3 8
5 7
4 2
10 6
8 10
1 9
7
3 8
5 7
4 2
10 6
8 10
1 9
7 5
8
3 8
5 7
4 2
10 6
8 10
1 9
7 5
2 3
9
3 8
5 7
4 2
10 6
8 10
1 9
7 5
2 3
6 0
SAMPLE OUTPUT: 
160
20
8
7
2
1
1
1
1
1

$p=[4, 9, 3, 8, 2, 7, 0, 5, 10, 1, 6]$ should be counted for all test cases.

SAMPLE INPUT:
5 987
3
654 321
543 210
432 106
2
654 321
543 210
1
654 321
1
0 493
0
SAMPLE OUTPUT: 
0
538184948
693625420
932738155
251798971

Make sure to output the answer modulo $10^9+7$.

SCORING:
Input 5: $N=15$Input 6: $N=2000$Inputs 7-9: For all test cases, the constraint $p_0=\lfloor N/2\rfloor$
appears.Inputs 10-13: For all test cases, there exists some constraint $p_i = j$
with $j$ equal to
$\lfloor N/2\rfloor$.Inputs 14-20: No additional constraints.


Problem credits: Benjamin Qi



\section*{Input Format}
Each test case starts with $K$, followed by $K$ lines each containing $i$ and
$j$. It is guaranteed thatThe same $i$ does not appear more than once within the same test case.The same $j$ does not appear more than once within the same test case.

OUTPUT FORMAT (print output to the terminal / stdout):For each test case, the answer modulo $10^9+7$ on a separate line.SAMPLE INPUT:3 4
0
1
1 1
2
0 2
2 3SAMPLE OUTPUT:2
0
1The maximum possible value of $f(p)$ is $2$, and $S=\{[2,0,3,1], [1,3,0,2]\}$.SAMPLE INPUT:9 11
2
0 5
6 9
3
0 5
6 9
1 0
4
0 5
6 9
1 0
4 7
5
0 5
6 9
1 0
4 7
2 6
6
0 5
6 9
1 0
4 7
2 6
9 3
7
0 5
6 9
1 0
4 7
2 6
9 3
5 2
8
0 5
6 9
1 0
4 7
2 6
9 3
5 2
7 4
9
0 5
6 9
1 0
4 7
2 6
9 3
5 2
7 4
3 1
10
0 5
6 9
1 0
4 7
2 6
9 3
5 2
7 4
3 1
8 10SAMPLE OUTPUT:6
6
1
1
1
1
1
1
1$p=[5, 0, 6, 1, 7, 2, 9, 4, 10, 3, 8]$ should be counted for all test cases.SAMPLE INPUT:10 11
0
1
3 8
2
3 8
5 7
3
3 8
5 7
4 2
4
3 8
5 7
4 2
10 6
5
3 8
5 7
4 2
10 6
8 10
6
3 8
5 7
4 2
10 6
8 10
1 9
7
3 8
5 7
4 2
10 6
8 10
1 9
7 5
8
3 8
5 7
4 2
10 6
8 10
1 9
7 5
2 3
9
3 8
5 7
4 2
10 6
8 10
1 9
7 5
2 3
6 0SAMPLE OUTPUT:160
20
8
7
2
1
1
1
1
1$p=[4, 9, 3, 8, 2, 7, 0, 5, 10, 1, 6]$ should be counted for all test cases.SAMPLE INPUT:5 987
3
654 321
543 210
432 106
2
654 321
543 210
1
654 321
1
0 493
0SAMPLE OUTPUT:0
538184948
693625420
932738155
251798971Make sure to output the answer modulo $10^9+7$.SCORING:Input 5: $N=15$Input 6: $N=2000$Inputs 7-9: For all test cases, the constraint $p_0=\lfloor N/2\rfloor$
appears.Inputs 10-13: For all test cases, there exists some constraint $p_i = j$
with $j$ equal to
$\lfloor N/2\rfloor$.Inputs 14-20: No additional constraints.Problem credits: Benjamin Qi

\section*{Output Format}
SAMPLE INPUT:3 4
0
1
1 1
2
0 2
2 3SAMPLE OUTPUT:2
0
1The maximum possible value of $f(p)$ is $2$, and $S=\{[2,0,3,1], [1,3,0,2]\}$.SAMPLE INPUT:9 11
2
0 5
6 9
3
0 5
6 9
1 0
4
0 5
6 9
1 0
4 7
5
0 5
6 9
1 0
4 7
2 6
6
0 5
6 9
1 0
4 7
2 6
9 3
7
0 5
6 9
1 0
4 7
2 6
9 3
5 2
8
0 5
6 9
1 0
4 7
2 6
9 3
5 2
7 4
9
0 5
6 9
1 0
4 7
2 6
9 3
5 2
7 4
3 1
10
0 5
6 9
1 0
4 7
2 6
9 3
5 2
7 4
3 1
8 10SAMPLE OUTPUT:6
6
1
1
1
1
1
1
1$p=[5, 0, 6, 1, 7, 2, 9, 4, 10, 3, 8]$ should be counted for all test cases.SAMPLE INPUT:10 11
0
1
3 8
2
3 8
5 7
3
3 8
5 7
4 2
4
3 8
5 7
4 2
10 6
5
3 8
5 7
4 2
10 6
8 10
6
3 8
5 7
4 2
10 6
8 10
1 9
7
3 8
5 7
4 2
10 6
8 10
1 9
7 5
8
3 8
5 7
4 2
10 6
8 10
1 9
7 5
2 3
9
3 8
5 7
4 2
10 6
8 10
1 9
7 5
2 3
6 0SAMPLE OUTPUT:160
20
8
7
2
1
1
1
1
1$p=[4, 9, 3, 8, 2, 7, 0, 5, 10, 1, 6]$ should be counted for all test cases.SAMPLE INPUT:5 987
3
654 321
543 210
432 106
2
654 321
543 210
1
654 321
1
0 493
0SAMPLE OUTPUT:0
538184948
693625420
932738155
251798971Make sure to output the answer modulo $10^9+7$.SCORING:Input 5: $N=15$Input 6: $N=2000$Inputs 7-9: For all test cases, the constraint $p_0=\lfloor N/2\rfloor$
appears.Inputs 10-13: For all test cases, there exists some constraint $p_i = j$
with $j$ equal to
$\lfloor N/2\rfloor$.Inputs 14-20: No additional constraints.Problem credits: Benjamin Qi

\section*{Sample Input}
\begin{verbatim}
3 4
0
1
1 1
2
0 2
2 3

9 11
2
0 5
6 9
3
0 5
6 9
1 0
4
0 5
6 9
1 0
4 7
5
0 5
6 9
1 0
4 7
2 6
6
0 5
6 9
1 0
4 7
2 6
9 3
7
0 5
6 9
1 0
4 7
2 6
9 3
5 2
8
0 5
6 9
1 0
4 7
2 6
9 3
5 2
7 4
9
0 5
6 9
1 0
4 7
2 6
9 3
5 2
7 4
3 1
10
0 5
6 9
1 0
4 7
2 6
9 3
5 2
7 4
3 1
8 10

10 11
0
1
3 8
2
3 8
5 7
3
3 8
5 7
4 2
4
3 8
5 7
4 2
10 6
5
3 8
5 7
4 2
10 6
8 10
6
3 8
5 7
4 2
10 6
8 10
1 9
7
3 8
5 7
4 2
10 6
8 10
1 9
7 5
8
3 8
5 7
4 2
10 6
8 10
1 9
7 5
2 3
9
3 8
5 7
4 2
10 6
8 10
1 9
7 5
2 3
6 0

5 987
3
654 321
543 210
432 106
2
654 321
543 210
1
654 321
1
0 493
0
\end{verbatim}

\section*{Sample Output}
\begin{verbatim}
2
0
1

The maximum possible value of $f(p)$ is $2$, and $S=\{[2,0,3,1], [1,3,0,2]\}$.SAMPLE INPUT:9 11
2
0 5
6 9
3
0 5
6 9
1 0
4
0 5
6 9
1 0
4 7
5
0 5
6 9
1 0
4 7
2 6
6
0 5
6 9
1 0
4 7
2 6
9 3
7
0 5
6 9
1 0
4 7
2 6
9 3
5 2
8
0 5
6 9
1 0
4 7
2 6
9 3
5 2
7 4
9
0 5
6 9
1 0
4 7
2 6
9 3
5 2
7 4
3 1
10
0 5
6 9
1 0
4 7
2 6
9 3
5 2
7 4
3 1
8 10SAMPLE OUTPUT:6
6
1
1
1
1
1
1
1$p=[5, 0, 6, 1, 7, 2, 9, 4, 10, 3, 8]$ should be counted for all test cases.SAMPLE INPUT:10 11
0
1
3 8
2
3 8
5 7
3
3 8
5 7
4 2
4
3 8
5 7
4 2
10 6
5
3 8
5 7
4 2
10 6
8 10
6
3 8
5 7
4 2
10 6
8 10
1 9
7
3 8
5 7
4 2
10 6
8 10
1 9
7 5
8
3 8
5 7
4 2
10 6
8 10
1 9
7 5
2 3
9
3 8
5 7
4 2
10 6
8 10
1 9
7 5
2 3
6 0SAMPLE OUTPUT:160
20
8
7
2
1
1
1
1
1$p=[4, 9, 3, 8, 2, 7, 0, 5, 10, 1, 6]$ should be counted for all test cases.SAMPLE INPUT:5 987
3
654 321
543 210
432 106
2
654 321
543 210
1
654 321
1
0 493
0SAMPLE OUTPUT:0
538184948
693625420
932738155
251798971Make sure to output the answer modulo $10^9+7$.SCORING:Input 5: $N=15$Input 6: $N=2000$Inputs 7-9: For all test cases, the constraint $p_0=\lfloor N/2\rfloor$
appears.Inputs 10-13: For all test cases, there exists some constraint $p_i = j$
with $j$ equal to
$\lfloor N/2\rfloor$.Inputs 14-20: No additional constraints.Problem credits: Benjamin Qi

SAMPLE INPUT:9 11
2
0 5
6 9
3
0 5
6 9
1 0
4
0 5
6 9
1 0
4 7
5
0 5
6 9
1 0
4 7
2 6
6
0 5
6 9
1 0
4 7
2 6
9 3
7
0 5
6 9
1 0
4 7
2 6
9 3
5 2
8
0 5
6 9
1 0
4 7
2 6
9 3
5 2
7 4
9
0 5
6 9
1 0
4 7
2 6
9 3
5 2
7 4
3 1
10
0 5
6 9
1 0
4 7
2 6
9 3
5 2
7 4
3 1
8 10SAMPLE OUTPUT:6
6
1
1
1
1
1
1
1$p=[5, 0, 6, 1, 7, 2, 9, 4, 10, 3, 8]$ should be counted for all test cases.SAMPLE INPUT:10 11
0
1
3 8
2
3 8
5 7
3
3 8
5 7
4 2
4
3 8
5 7
4 2
10 6
5
3 8
5 7
4 2
10 6
8 10
6
3 8
5 7
4 2
10 6
8 10
1 9
7
3 8
5 7
4 2
10 6
8 10
1 9
7 5
8
3 8
5 7
4 2
10 6
8 10
1 9
7 5
2 3
9
3 8
5 7
4 2
10 6
8 10
1 9
7 5
2 3
6 0SAMPLE OUTPUT:160
20
8
7
2
1
1
1
1
1$p=[4, 9, 3, 8, 2, 7, 0, 5, 10, 1, 6]$ should be counted for all test cases.SAMPLE INPUT:5 987
3
654 321
543 210
432 106
2
654 321
543 210
1
654 321
1
0 493
0SAMPLE OUTPUT:0
538184948
693625420
932738155
251798971Make sure to output the answer modulo $10^9+7$.SCORING:Input 5: $N=15$Input 6: $N=2000$Inputs 7-9: For all test cases, the constraint $p_0=\lfloor N/2\rfloor$
appears.Inputs 10-13: For all test cases, there exists some constraint $p_i = j$
with $j$ equal to
$\lfloor N/2\rfloor$.Inputs 14-20: No additional constraints.Problem credits: Benjamin Qi

6
6
1
1
1
1
1
1
1

$p=[5, 0, 6, 1, 7, 2, 9, 4, 10, 3, 8]$ should be counted for all test cases.SAMPLE INPUT:10 11
0
1
3 8
2
3 8
5 7
3
3 8
5 7
4 2
4
3 8
5 7
4 2
10 6
5
3 8
5 7
4 2
10 6
8 10
6
3 8
5 7
4 2
10 6
8 10
1 9
7
3 8
5 7
4 2
10 6
8 10
1 9
7 5
8
3 8
5 7
4 2
10 6
8 10
1 9
7 5
2 3
9
3 8
5 7
4 2
10 6
8 10
1 9
7 5
2 3
6 0SAMPLE OUTPUT:160
20
8
7
2
1
1
1
1
1$p=[4, 9, 3, 8, 2, 7, 0, 5, 10, 1, 6]$ should be counted for all test cases.SAMPLE INPUT:5 987
3
654 321
543 210
432 106
2
654 321
543 210
1
654 321
1
0 493
0SAMPLE OUTPUT:0
538184948
693625420
932738155
251798971Make sure to output the answer modulo $10^9+7$.SCORING:Input 5: $N=15$Input 6: $N=2000$Inputs 7-9: For all test cases, the constraint $p_0=\lfloor N/2\rfloor$
appears.Inputs 10-13: For all test cases, there exists some constraint $p_i = j$
with $j$ equal to
$\lfloor N/2\rfloor$.Inputs 14-20: No additional constraints.Problem credits: Benjamin Qi

SAMPLE INPUT:10 11
0
1
3 8
2
3 8
5 7
3
3 8
5 7
4 2
4
3 8
5 7
4 2
10 6
5
3 8
5 7
4 2
10 6
8 10
6
3 8
5 7
4 2
10 6
8 10
1 9
7
3 8
5 7
4 2
10 6
8 10
1 9
7 5
8
3 8
5 7
4 2
10 6
8 10
1 9
7 5
2 3
9
3 8
5 7
4 2
10 6
8 10
1 9
7 5
2 3
6 0SAMPLE OUTPUT:160
20
8
7
2
1
1
1
1
1$p=[4, 9, 3, 8, 2, 7, 0, 5, 10, 1, 6]$ should be counted for all test cases.SAMPLE INPUT:5 987
3
654 321
543 210
432 106
2
654 321
543 210
1
654 321
1
0 493
0SAMPLE OUTPUT:0
538184948
693625420
932738155
251798971Make sure to output the answer modulo $10^9+7$.SCORING:Input 5: $N=15$Input 6: $N=2000$Inputs 7-9: For all test cases, the constraint $p_0=\lfloor N/2\rfloor$
appears.Inputs 10-13: For all test cases, there exists some constraint $p_i = j$
with $j$ equal to
$\lfloor N/2\rfloor$.Inputs 14-20: No additional constraints.Problem credits: Benjamin Qi

160
20
8
7
2
1
1
1
1
1

$p=[4, 9, 3, 8, 2, 7, 0, 5, 10, 1, 6]$ should be counted for all test cases.SAMPLE INPUT:5 987
3
654 321
543 210
432 106
2
654 321
543 210
1
654 321
1
0 493
0SAMPLE OUTPUT:0
538184948
693625420
932738155
251798971Make sure to output the answer modulo $10^9+7$.SCORING:Input 5: $N=15$Input 6: $N=2000$Inputs 7-9: For all test cases, the constraint $p_0=\lfloor N/2\rfloor$
appears.Inputs 10-13: For all test cases, there exists some constraint $p_i = j$
with $j$ equal to
$\lfloor N/2\rfloor$.Inputs 14-20: No additional constraints.Problem credits: Benjamin Qi

SAMPLE INPUT:5 987
3
654 321
543 210
432 106
2
654 321
543 210
1
654 321
1
0 493
0SAMPLE OUTPUT:0
538184948
693625420
932738155
251798971Make sure to output the answer modulo $10^9+7$.SCORING:Input 5: $N=15$Input 6: $N=2000$Inputs 7-9: For all test cases, the constraint $p_0=\lfloor N/2\rfloor$
appears.Inputs 10-13: For all test cases, there exists some constraint $p_i = j$
with $j$ equal to
$\lfloor N/2\rfloor$.Inputs 14-20: No additional constraints.Problem credits: Benjamin Qi

0
538184948
693625420
932738155
251798971

Make sure to output the answer modulo $10^9+7$.SCORING:Input 5: $N=15$Input 6: $N=2000$Inputs 7-9: For all test cases, the constraint $p_0=\lfloor N/2\rfloor$
appears.Inputs 10-13: For all test cases, there exists some constraint $p_i = j$
with $j$ equal to
$\lfloor N/2\rfloor$.Inputs 14-20: No additional constraints.Problem credits: Benjamin Qi

SCORING:Input 5: $N=15$Input 6: $N=2000$Inputs 7-9: For all test cases, the constraint $p_0=\lfloor N/2\rfloor$
appears.Inputs 10-13: For all test cases, there exists some constraint $p_i = j$
with $j$ equal to
$\lfloor N/2\rfloor$.Inputs 14-20: No additional constraints.Problem credits: Benjamin Qi
\end{verbatim}

\section*{Solution}


\section*{Problem URL}
https://usaco.org/index.php?page=viewproblem2&cpid=1454

\section*{Source}
USACO DEC24 Contest, Platinum Division, Problem ID: 1454

\end{document}
