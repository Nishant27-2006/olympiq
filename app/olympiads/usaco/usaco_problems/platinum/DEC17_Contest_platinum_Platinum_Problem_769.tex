\documentclass[12pt]{article}
\usepackage{amsmath}
\usepackage{amssymb}
\usepackage{graphicx}
\usepackage{enumitem}
\usepackage{hyperref}
\usepackage{xcolor}

\title{View problem}
\author{USACO Contest: DEC17 Contest - Platinum Division}
\date{\today}

\begin{document}

\maketitle

\section*{Problem Statement}

Bessie and her friends have invented a new game. The game is named accurately,
but not particularly creatively. They call it the "Push A Box Around The Barn To
Get It In The Right Spot And Don't Move The Hay" game (if you think that's
excessive, you should see some of the variable names the cows use when they
write code...)

The barn can be modeled as an $N \times M$ rectangular grid. Some of the grid
cells have hay in them. Bessie occupies one cell in this grid, and a large
wooden box occupies another cell.  Bessie and the box are not able to fit in the
same  cell at the same time, and neither can fit into a cell containing hay.

Bessie can move in the 4 orthogonal directions (north, east, south, west) as
long as she does not walk into hay. If she attempts to walk to the space with
the box, then the box will be pushed one space in that direction, as long as
there is an empty cell on the other side. If there is no empty cell, then Bessie
will not be able to make that move.

A certain grid cell is designated as the goal. Bessie's goal is to get the box
into that location.  

Given the layout of the barn, including the starting positions of the box and
the cow, and the target position of the box, determine if it possible to win the
game.

Note: This problem allows 512MB of memory usage, up from the default limit of 256MB.

INPUT FORMAT (file pushabox.in):
The first line has three numbers, $N$, $M$, and $Q$, where $N$ is the number of
rows in the grid and $M$ is the number of columns.


$1 \le N,M \le 1500$.

$1 \le Q \le 50,000$.

On the next $N$ lines is a representation of the grid, where characters
represent empty cells (.), hay (#), Bessie's starting position (A), and the
box's initial location (B).

This is followed by $Q$ lines, each with a pair of integers $(R, C)$. For each
pair, you should determine if it is possible to get the box to that cell at row
$R$, column $C$, starting from the initial state of the barn. The top row is row
1, and the left column is column 1.


OUTPUT FORMAT (file pushabox.out):
$Q$ lines, each with either the string "YES" or "NO".

SAMPLE INPUT:
5 5 4
##.##
##.##
A.B..
##.##
##.##
3 2
3 5
1 3
5 3
SAMPLE OUTPUT: 
NO
YES
NO
NO

To push the box to the position (3, 5), the cow just needs to move 3 spaces to
the right.

None of the other three positions are attainable.


Problem credits: Nathan Pinsker



\section*{Input Format}
The first line has three numbers, $N$, $M$, and $Q$, where $N$ is the number of
rows in the grid and $M$ is the number of columns.$1 \le N,M \le 1500$.$1 \le Q \le 50,000$.On the next $N$ lines is a representation of the grid, where characters
represent empty cells (.), hay (#), Bessie's starting position (A), and the
box's initial location (B).This is followed by $Q$ lines, each with a pair of integers $(R, C)$. For each
pair, you should determine if it is possible to get the box to that cell at row
$R$, column $C$, starting from the initial state of the barn. The top row is row
1, and the left column is column 1.

$1 \le N,M \le 1500$.$1 \le Q \le 50,000$.On the next $N$ lines is a representation of the grid, where characters
represent empty cells (.), hay (#), Bessie's starting position (A), and the
box's initial location (B).This is followed by $Q$ lines, each with a pair of integers $(R, C)$. For each
pair, you should determine if it is possible to get the box to that cell at row
$R$, column $C$, starting from the initial state of the barn. The top row is row
1, and the left column is column 1.

On the next $N$ lines is a representation of the grid, where characters
represent empty cells (.), hay (#), Bessie's starting position (A), and the
box's initial location (B).This is followed by $Q$ lines, each with a pair of integers $(R, C)$. For each
pair, you should determine if it is possible to get the box to that cell at row
$R$, column $C$, starting from the initial state of the barn. The top row is row
1, and the left column is column 1.

This is followed by $Q$ lines, each with a pair of integers $(R, C)$. For each
pair, you should determine if it is possible to get the box to that cell at row
$R$, column $C$, starting from the initial state of the barn. The top row is row
1, and the left column is column 1.

OUTPUT FORMAT (file pushabox.out):$Q$ lines, each with either the string "YES" or "NO".SAMPLE INPUT:5 5 4
##.##
##.##
A.B..
##.##
##.##
3 2
3 5
1 3
5 3SAMPLE OUTPUT:NO
YES
NO
NOTo push the box to the position (3, 5), the cow just needs to move 3 spaces to
the right.None of the other three positions are attainable.Problem credits: Nathan Pinsker

\section*{Output Format}
SAMPLE INPUT:5 5 4
##.##
##.##
A.B..
##.##
##.##
3 2
3 5
1 3
5 3SAMPLE OUTPUT:NO
YES
NO
NOTo push the box to the position (3, 5), the cow just needs to move 3 spaces to
the right.None of the other three positions are attainable.Problem credits: Nathan Pinsker

\section*{Sample Input}
\begin{verbatim}
5 5 4
##.##
##.##
A.B..
##.##
##.##
3 2
3 5
1 3
5 3
\end{verbatim}

\section*{Sample Output}
\begin{verbatim}
NO
YES
NO
NO

To push the box to the position (3, 5), the cow just needs to move 3 spaces to
the right.None of the other three positions are attainable.Problem credits: Nathan Pinsker

None of the other three positions are attainable.Problem credits: Nathan Pinsker

Problem credits: Nathan Pinsker

Problem credits: Nathan Pinsker
\end{verbatim}

\section*{Solution}


\section*{Problem URL}
https://usaco.org/index.php?page=viewproblem2&cpid=769

\section*{Source}
USACO DEC17 Contest, Platinum Division, Problem ID: 769

\end{document}
