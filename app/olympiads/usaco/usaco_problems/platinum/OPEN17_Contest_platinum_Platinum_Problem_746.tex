\documentclass[12pt]{article}
\usepackage{amsmath}
\usepackage{amssymb}
\usepackage{graphicx}
\usepackage{enumitem}
\usepackage{hyperref}
\usepackage{xcolor}

\title{View problem}
\author{USACO Contest: OPEN17 Contest - Platinum Division}
\date{\today}

\begin{document}

\maketitle

\section*{Problem Statement}

Bessie has invented a new programming language, but since there is no compiler
yet, she needs your help to actually run her programs. 

COWBASIC is a simple, elegant language. It has two key features: addition and
MOO loops. Bessie has devised a clever solution to overflow: all addition is
done modulo $10^9+7$. But Bessie's real achievement is the MOO loop, which runs
a block of code a fixed number of times. MOO loops and addition can, of course,
be nested.

Given a COWBASIC program, please help Bessie determine what number it returns.

INPUT FORMAT (file cowbasic.in):
You are given a COWBASIC program at most 100 lines long, with each line being at
most 350 characters long. A COWBASIC program is a list of statements.

There are three types of statements:

<variable> = <expression>

<literal> MOO {
  <list of statements>
}

RETURN <variable>

There are three types of expressions:

<literal>

<variable>

( <expression> ) + ( <expression> )

A literal is a positive integer at most 100,000.

A variable is a string of at most 10 lowercase English letters.

It is guaranteed that no variable will be used or RETURNed before it is defined.
It is guaranteed that RETURN will happen exactly once, on the last line of the
program.

OUTPUT FORMAT (file cowbasic.out):
Output a single positive integer, giving the value of the RETURNed variable.

Scoring

In 20 percent of all test cases - MOO loops are not nested.

In another 20 percent of all test cases - The program only has 1 variable. MOO
loops can be nested.

In the remaining test cases, there are no further restrictions.


SAMPLE INPUT:
x = 1
10 MOO {
  x = ( x ) + ( x )
}
RETURN x
SAMPLE OUTPUT: 
1024

This COWBASIC program computes $2^{10}$.
SAMPLE INPUT:
n = 1
nsq = 1
100000 MOO {
  100000 MOO {
    nsq = ( nsq ) + ( ( n ) + ( ( n ) + ( 1 ) ) )
    n = ( n ) + ( 1 )
  }
}
RETURN nsq
SAMPLE OUTPUT: 
4761

This COWBASIC program computes $(10^5*10^5+1)^2$ (modulo $10^9 + 7$).

Problem credits: Jonathan Paulson



\section*{Input Format}
There are three types of statements:<variable> = <expression>

<literal> MOO {
  <list of statements>
}

RETURN <variable>There are three types of expressions:<literal>

<variable>

( <expression> ) + ( <expression> )A literal is a positive integer at most 100,000.A variable is a string of at most 10 lowercase English letters.It is guaranteed that no variable will be used or RETURNed before it is defined.
It is guaranteed that RETURN will happen exactly once, on the last line of the
program.

<variable> = <expression>

<literal> MOO {
  <list of statements>
}

RETURN <variable>

There are three types of expressions:<literal>

<variable>

( <expression> ) + ( <expression> )A literal is a positive integer at most 100,000.A variable is a string of at most 10 lowercase English letters.It is guaranteed that no variable will be used or RETURNed before it is defined.
It is guaranteed that RETURN will happen exactly once, on the last line of the
program.

<literal>

<variable>

( <expression> ) + ( <expression> )

A literal is a positive integer at most 100,000.A variable is a string of at most 10 lowercase English letters.It is guaranteed that no variable will be used or RETURNed before it is defined.
It is guaranteed that RETURN will happen exactly once, on the last line of the
program.

A variable is a string of at most 10 lowercase English letters.It is guaranteed that no variable will be used or RETURNed before it is defined.
It is guaranteed that RETURN will happen exactly once, on the last line of the
program.

It is guaranteed that no variable will be used or RETURNed before it is defined.
It is guaranteed that RETURN will happen exactly once, on the last line of the
program.

OUTPUT FORMAT (file cowbasic.out):Output a single positive integer, giving the value of the RETURNed variable.ScoringIn 20 percent of all test cases - MOO loops are not nested.In another 20 percent of all test cases - The program only has 1 variable. MOO
loops can be nested.In the remaining test cases, there are no further restrictions.SAMPLE INPUT:x = 1
10 MOO {
  x = ( x ) + ( x )
}
RETURN xSAMPLE OUTPUT:1024This COWBASIC program computes $2^{10}$.SAMPLE INPUT:n = 1
nsq = 1
100000 MOO {
  100000 MOO {
    nsq = ( nsq ) + ( ( n ) + ( ( n ) + ( 1 ) ) )
    n = ( n ) + ( 1 )
  }
}
RETURN nsqSAMPLE OUTPUT:4761This COWBASIC program computes $(10^5*10^5+1)^2$ (modulo $10^9 + 7$).Problem credits: Jonathan Paulson

\section*{Output Format}
ScoringIn 20 percent of all test cases - MOO loops are not nested.In another 20 percent of all test cases - The program only has 1 variable. MOO
loops can be nested.In the remaining test cases, there are no further restrictions.SAMPLE INPUT:x = 1
10 MOO {
  x = ( x ) + ( x )
}
RETURN xSAMPLE OUTPUT:1024This COWBASIC program computes $2^{10}$.SAMPLE INPUT:n = 1
nsq = 1
100000 MOO {
  100000 MOO {
    nsq = ( nsq ) + ( ( n ) + ( ( n ) + ( 1 ) ) )
    n = ( n ) + ( 1 )
  }
}
RETURN nsqSAMPLE OUTPUT:4761This COWBASIC program computes $(10^5*10^5+1)^2$ (modulo $10^9 + 7$).Problem credits: Jonathan Paulson

\section*{Sample Input}
\begin{verbatim}
x = 1
10 MOO {
  x = ( x ) + ( x )
}
RETURN x

n = 1
nsq = 1
100000 MOO {
  100000 MOO {
    nsq = ( nsq ) + ( ( n ) + ( ( n ) + ( 1 ) ) )
    n = ( n ) + ( 1 )
  }
}
RETURN nsq
\end{verbatim}

\section*{Sample Output}
\begin{verbatim}
1024

4761

Problem credits: Jonathan Paulson
\end{verbatim}

\section*{Solution}


\section*{Problem URL}
https://usaco.org/index.php?page=viewproblem2&cpid=746

\section*{Source}
USACO OPEN17 Contest, Platinum Division, Problem ID: 746

\end{document}
