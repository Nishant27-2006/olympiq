\documentclass[12pt]{article}
\usepackage{amsmath}
\usepackage{amssymb}
\usepackage{graphicx}
\usepackage{enumitem}
\usepackage{hyperref}
\usepackage{xcolor}

\title{View problem}
\author{USACO Contest: DEC20 Contest - Platinum Division}
\date{\today}

\begin{document}

\maketitle

\section*{Problem Statement}

Bessie the cow has been abducted by aliens and is now trapped inside an alien
spaceship! The spaceship has $N$ $(1\le N\le 60)$ rooms labeled $1\ldots N$, with
one-way doors connecting between some pairs of rooms (due to the strange alien
technology at play, it is even possible for a door to lead from a room back to
itself!). However, no two doors share the same starting and end room.
Additionally, Bessie has a remote with buttons numbered $1\ldots K$
$(1 \le K \le 60)$.

The aliens will release Bessie if she can complete a strange task. First, they
will choose two rooms, $s$ and $t$ $(1 \le s, t \le N)$, and two numbers, $b_s$
and $b_t$ $(1 \le b_s, b_t \le K)$. They will start Bessie in room $s$ and
immediately have her press button $b_s$. Bessie will then proceed to navigate
the ship while pressing buttons. There are a few rules for what Bessie can do:

In each room, after pressing exactly one button, she must choose to either
exit through a door to another (possibly the same) room or stop.Once
Bessie presses a button, it is invalid for her to press the same button again
unless, in the time between uses, she has pressed a button with a higher number.
In other words, pressing button number $x$ will make it unavailable for use, 
while all buttons with numbers $<x$ will be reset and again available for
use.If Bessie presses an invalid button, she automatically fails and the aliens
will keep her.

Bessie is released only if she stops in room $t$, the last button she pressed
was $b_t$, and no invalid buttons were ever pressed.

Bessie is worried that she may not be able to complete the task. For $Q$
$(1\le Q\le 60)$ queries, each consisting of what Bessie considers a likely
choice of $s, t, b_s$, and $b_t$, Bessie wants to know the number of sequences
of rooms and button presses that would lead to her release. Report your answers
modulo $10^9 + 7$ as they may be very large.

INPUT FORMAT (input arrives from the terminal / stdin):
The first line contains $N,K,Q$.

The next $N$ lines each contain $N$ bits (each 0 or 1). The $j$-th entry of the
$i$-th line is 1 if there exists a door from room $i$ to room $j$, and 0 if no
such door exists.

This is followed by $Q$ lines, each containing four integers $b_s$, $s$, $b_t$,
$t$, denoting the starting button, starting room, final button, and final room
respectively.

OUTPUT FORMAT (print output to the terminal / stdout):
The number of sequences for each of the $Q$ queries modulo $10^9+7$ on separate
lines.

SAMPLE INPUT:
6 3 8
010000
001000
000100
000010
000000
000001
1 1 1 1
3 3 1 1
1 1 3 3
1 1 1 5
2 1 1 5
1 1 2 5
3 1 3 5
2 6 2 6
SAMPLE OUTPUT: 
1
0
1
3
2
2
0
5

The doors connect rooms $1\to 2$, $2 \to 3$, $3\to 4$, $4\to 5$, and $6\to 6$.

For the first query, Bessie must stop immediately after pressing the first
button.

For the second query, the answer is clearly zero because there is no way to get
to room 1 from room 3.

For the third query, Bessie's only option is to move from room 1 to room 2 to
room 3 while pressing buttons 1, 2, and 3.

For the fourth query, Bessie's pattern of movement is fixed, and she has three
possible sequences of button presses:

$(1,2,3,2,1)$$(1,2,1,3,1)$$(1,3,1,2,1)$
For the last query, Bessie has five possible sequences of button presses:

$(2)$$(2,3,2)$$(2,3,1,2)$$(2,1,3,2)$$(2,1,3,1,2)$
SAMPLE INPUT:
6 4 6
001100
001110
101101
010111
110111
000111
3 2 4 3
3 1 4 4
3 4 4 1
3 3 4 3
3 6 4 3
3 1 4 2
SAMPLE OUTPUT: 
26
49
29
27
18
22

This test case satisfies the constraints for all subtasks aside from the first.

SAMPLE INPUT:
6 10 5
110101
011001
001111
101111
111010
000001
2 5 2 5
6 1 5 2
3 4 8 3
9 3 3 5
5 1 3 4
SAMPLE OUTPUT: 
713313311
716721076
782223918
335511486
539247783

Make sure to output the answers modulo $10^9+7$.

SCORING:
In test cases 4-7, $K\le 5$ and $(b_s,s)$ is the same for all queries.In test cases 8-11, $b_s=K-1$ and $b_t=K$ for each query.In test cases 12-15, $N,K,Q\le 20$.In test cases 16-23, there are no additional constraints.


Problem credits: Benjamin Qi



\section*{Input Format}
The next $N$ lines each contain $N$ bits (each 0 or 1). The $j$-th entry of the
$i$-th line is 1 if there exists a door from room $i$ to room $j$, and 0 if no
such door exists.This is followed by $Q$ lines, each containing four integers $b_s$, $s$, $b_t$,
$t$, denoting the starting button, starting room, final button, and final room
respectively.

This is followed by $Q$ lines, each containing four integers $b_s$, $s$, $b_t$,
$t$, denoting the starting button, starting room, final button, and final room
respectively.

OUTPUT FORMAT (print output to the terminal / stdout):The number of sequences for each of the $Q$ queries modulo $10^9+7$ on separate
lines.SAMPLE INPUT:6 3 8
010000
001000
000100
000010
000000
000001
1 1 1 1
3 3 1 1
1 1 3 3
1 1 1 5
2 1 1 5
1 1 2 5
3 1 3 5
2 6 2 6SAMPLE OUTPUT:1
0
1
3
2
2
0
5The doors connect rooms $1\to 2$, $2 \to 3$, $3\to 4$, $4\to 5$, and $6\to 6$.For the first query, Bessie must stop immediately after pressing the first
button.For the second query, the answer is clearly zero because there is no way to get
to room 1 from room 3.For the third query, Bessie's only option is to move from room 1 to room 2 to
room 3 while pressing buttons 1, 2, and 3.For the fourth query, Bessie's pattern of movement is fixed, and she has three
possible sequences of button presses:$(1,2,3,2,1)$$(1,2,1,3,1)$$(1,3,1,2,1)$For the last query, Bessie has five possible sequences of button presses:$(2)$$(2,3,2)$$(2,3,1,2)$$(2,1,3,2)$$(2,1,3,1,2)$SAMPLE INPUT:6 4 6
001100
001110
101101
010111
110111
000111
3 2 4 3
3 1 4 4
3 4 4 1
3 3 4 3
3 6 4 3
3 1 4 2SAMPLE OUTPUT:26
49
29
27
18
22This test case satisfies the constraints for all subtasks aside from the first.SAMPLE INPUT:6 10 5
110101
011001
001111
101111
111010
000001
2 5 2 5
6 1 5 2
3 4 8 3
9 3 3 5
5 1 3 4SAMPLE OUTPUT:713313311
716721076
782223918
335511486
539247783Make sure to output the answers modulo $10^9+7$.SCORING:In test cases 4-7, $K\le 5$ and $(b_s,s)$ is the same for all queries.In test cases 8-11, $b_s=K-1$ and $b_t=K$ for each query.In test cases 12-15, $N,K,Q\le 20$.In test cases 16-23, there are no additional constraints.Problem credits: Benjamin Qi

\section*{Output Format}
SAMPLE INPUT:6 3 8
010000
001000
000100
000010
000000
000001
1 1 1 1
3 3 1 1
1 1 3 3
1 1 1 5
2 1 1 5
1 1 2 5
3 1 3 5
2 6 2 6SAMPLE OUTPUT:1
0
1
3
2
2
0
5The doors connect rooms $1\to 2$, $2 \to 3$, $3\to 4$, $4\to 5$, and $6\to 6$.For the first query, Bessie must stop immediately after pressing the first
button.For the second query, the answer is clearly zero because there is no way to get
to room 1 from room 3.For the third query, Bessie's only option is to move from room 1 to room 2 to
room 3 while pressing buttons 1, 2, and 3.For the fourth query, Bessie's pattern of movement is fixed, and she has three
possible sequences of button presses:$(1,2,3,2,1)$$(1,2,1,3,1)$$(1,3,1,2,1)$For the last query, Bessie has five possible sequences of button presses:$(2)$$(2,3,2)$$(2,3,1,2)$$(2,1,3,2)$$(2,1,3,1,2)$SAMPLE INPUT:6 4 6
001100
001110
101101
010111
110111
000111
3 2 4 3
3 1 4 4
3 4 4 1
3 3 4 3
3 6 4 3
3 1 4 2SAMPLE OUTPUT:26
49
29
27
18
22This test case satisfies the constraints for all subtasks aside from the first.SAMPLE INPUT:6 10 5
110101
011001
001111
101111
111010
000001
2 5 2 5
6 1 5 2
3 4 8 3
9 3 3 5
5 1 3 4SAMPLE OUTPUT:713313311
716721076
782223918
335511486
539247783Make sure to output the answers modulo $10^9+7$.SCORING:In test cases 4-7, $K\le 5$ and $(b_s,s)$ is the same for all queries.In test cases 8-11, $b_s=K-1$ and $b_t=K$ for each query.In test cases 12-15, $N,K,Q\le 20$.In test cases 16-23, there are no additional constraints.Problem credits: Benjamin Qi

\section*{Sample Input}
\begin{verbatim}
6 3 8
010000
001000
000100
000010
000000
000001
1 1 1 1
3 3 1 1
1 1 3 3
1 1 1 5
2 1 1 5
1 1 2 5
3 1 3 5
2 6 2 6

6 4 6
001100
001110
101101
010111
110111
000111
3 2 4 3
3 1 4 4
3 4 4 1
3 3 4 3
3 6 4 3
3 1 4 2

6 10 5
110101
011001
001111
101111
111010
000001
2 5 2 5
6 1 5 2
3 4 8 3
9 3 3 5
5 1 3 4
\end{verbatim}

\section*{Sample Output}
\begin{verbatim}
1
0
1
3
2
2
0
5

The doors connect rooms $1\to 2$, $2 \to 3$, $3\to 4$, $4\to 5$, and $6\to 6$.For the first query, Bessie must stop immediately after pressing the first
button.For the second query, the answer is clearly zero because there is no way to get
to room 1 from room 3.For the third query, Bessie's only option is to move from room 1 to room 2 to
room 3 while pressing buttons 1, 2, and 3.For the fourth query, Bessie's pattern of movement is fixed, and she has three
possible sequences of button presses:$(1,2,3,2,1)$$(1,2,1,3,1)$$(1,3,1,2,1)$For the last query, Bessie has five possible sequences of button presses:$(2)$$(2,3,2)$$(2,3,1,2)$$(2,1,3,2)$$(2,1,3,1,2)$SAMPLE INPUT:6 4 6
001100
001110
101101
010111
110111
000111
3 2 4 3
3 1 4 4
3 4 4 1
3 3 4 3
3 6 4 3
3 1 4 2SAMPLE OUTPUT:26
49
29
27
18
22This test case satisfies the constraints for all subtasks aside from the first.SAMPLE INPUT:6 10 5
110101
011001
001111
101111
111010
000001
2 5 2 5
6 1 5 2
3 4 8 3
9 3 3 5
5 1 3 4SAMPLE OUTPUT:713313311
716721076
782223918
335511486
539247783Make sure to output the answers modulo $10^9+7$.SCORING:In test cases 4-7, $K\le 5$ and $(b_s,s)$ is the same for all queries.In test cases 8-11, $b_s=K-1$ and $b_t=K$ for each query.In test cases 12-15, $N,K,Q\le 20$.In test cases 16-23, there are no additional constraints.Problem credits: Benjamin Qi

For the first query, Bessie must stop immediately after pressing the first
button.For the second query, the answer is clearly zero because there is no way to get
to room 1 from room 3.For the third query, Bessie's only option is to move from room 1 to room 2 to
room 3 while pressing buttons 1, 2, and 3.For the fourth query, Bessie's pattern of movement is fixed, and she has three
possible sequences of button presses:$(1,2,3,2,1)$$(1,2,1,3,1)$$(1,3,1,2,1)$For the last query, Bessie has five possible sequences of button presses:$(2)$$(2,3,2)$$(2,3,1,2)$$(2,1,3,2)$$(2,1,3,1,2)$SAMPLE INPUT:6 4 6
001100
001110
101101
010111
110111
000111
3 2 4 3
3 1 4 4
3 4 4 1
3 3 4 3
3 6 4 3
3 1 4 2SAMPLE OUTPUT:26
49
29
27
18
22This test case satisfies the constraints for all subtasks aside from the first.SAMPLE INPUT:6 10 5
110101
011001
001111
101111
111010
000001
2 5 2 5
6 1 5 2
3 4 8 3
9 3 3 5
5 1 3 4SAMPLE OUTPUT:713313311
716721076
782223918
335511486
539247783Make sure to output the answers modulo $10^9+7$.SCORING:In test cases 4-7, $K\le 5$ and $(b_s,s)$ is the same for all queries.In test cases 8-11, $b_s=K-1$ and $b_t=K$ for each query.In test cases 12-15, $N,K,Q\le 20$.In test cases 16-23, there are no additional constraints.Problem credits: Benjamin Qi

For the second query, the answer is clearly zero because there is no way to get
to room 1 from room 3.For the third query, Bessie's only option is to move from room 1 to room 2 to
room 3 while pressing buttons 1, 2, and 3.For the fourth query, Bessie's pattern of movement is fixed, and she has three
possible sequences of button presses:$(1,2,3,2,1)$$(1,2,1,3,1)$$(1,3,1,2,1)$For the last query, Bessie has five possible sequences of button presses:$(2)$$(2,3,2)$$(2,3,1,2)$$(2,1,3,2)$$(2,1,3,1,2)$SAMPLE INPUT:6 4 6
001100
001110
101101
010111
110111
000111
3 2 4 3
3 1 4 4
3 4 4 1
3 3 4 3
3 6 4 3
3 1 4 2SAMPLE OUTPUT:26
49
29
27
18
22This test case satisfies the constraints for all subtasks aside from the first.SAMPLE INPUT:6 10 5
110101
011001
001111
101111
111010
000001
2 5 2 5
6 1 5 2
3 4 8 3
9 3 3 5
5 1 3 4SAMPLE OUTPUT:713313311
716721076
782223918
335511486
539247783Make sure to output the answers modulo $10^9+7$.SCORING:In test cases 4-7, $K\le 5$ and $(b_s,s)$ is the same for all queries.In test cases 8-11, $b_s=K-1$ and $b_t=K$ for each query.In test cases 12-15, $N,K,Q\le 20$.In test cases 16-23, there are no additional constraints.Problem credits: Benjamin Qi

For the third query, Bessie's only option is to move from room 1 to room 2 to
room 3 while pressing buttons 1, 2, and 3.For the fourth query, Bessie's pattern of movement is fixed, and she has three
possible sequences of button presses:$(1,2,3,2,1)$$(1,2,1,3,1)$$(1,3,1,2,1)$For the last query, Bessie has five possible sequences of button presses:$(2)$$(2,3,2)$$(2,3,1,2)$$(2,1,3,2)$$(2,1,3,1,2)$SAMPLE INPUT:6 4 6
001100
001110
101101
010111
110111
000111
3 2 4 3
3 1 4 4
3 4 4 1
3 3 4 3
3 6 4 3
3 1 4 2SAMPLE OUTPUT:26
49
29
27
18
22This test case satisfies the constraints for all subtasks aside from the first.SAMPLE INPUT:6 10 5
110101
011001
001111
101111
111010
000001
2 5 2 5
6 1 5 2
3 4 8 3
9 3 3 5
5 1 3 4SAMPLE OUTPUT:713313311
716721076
782223918
335511486
539247783Make sure to output the answers modulo $10^9+7$.SCORING:In test cases 4-7, $K\le 5$ and $(b_s,s)$ is the same for all queries.In test cases 8-11, $b_s=K-1$ and $b_t=K$ for each query.In test cases 12-15, $N,K,Q\le 20$.In test cases 16-23, there are no additional constraints.Problem credits: Benjamin Qi

For the fourth query, Bessie's pattern of movement is fixed, and she has three
possible sequences of button presses:$(1,2,3,2,1)$$(1,2,1,3,1)$$(1,3,1,2,1)$For the last query, Bessie has five possible sequences of button presses:$(2)$$(2,3,2)$$(2,3,1,2)$$(2,1,3,2)$$(2,1,3,1,2)$SAMPLE INPUT:6 4 6
001100
001110
101101
010111
110111
000111
3 2 4 3
3 1 4 4
3 4 4 1
3 3 4 3
3 6 4 3
3 1 4 2SAMPLE OUTPUT:26
49
29
27
18
22This test case satisfies the constraints for all subtasks aside from the first.SAMPLE INPUT:6 10 5
110101
011001
001111
101111
111010
000001
2 5 2 5
6 1 5 2
3 4 8 3
9 3 3 5
5 1 3 4SAMPLE OUTPUT:713313311
716721076
782223918
335511486
539247783Make sure to output the answers modulo $10^9+7$.SCORING:In test cases 4-7, $K\le 5$ and $(b_s,s)$ is the same for all queries.In test cases 8-11, $b_s=K-1$ and $b_t=K$ for each query.In test cases 12-15, $N,K,Q\le 20$.In test cases 16-23, there are no additional constraints.Problem credits: Benjamin Qi

$(1,2,3,2,1)$$(1,2,1,3,1)$$(1,3,1,2,1)$For the last query, Bessie has five possible sequences of button presses:$(2)$$(2,3,2)$$(2,3,1,2)$$(2,1,3,2)$$(2,1,3,1,2)$SAMPLE INPUT:6 4 6
001100
001110
101101
010111
110111
000111
3 2 4 3
3 1 4 4
3 4 4 1
3 3 4 3
3 6 4 3
3 1 4 2SAMPLE OUTPUT:26
49
29
27
18
22This test case satisfies the constraints for all subtasks aside from the first.SAMPLE INPUT:6 10 5
110101
011001
001111
101111
111010
000001
2 5 2 5
6 1 5 2
3 4 8 3
9 3 3 5
5 1 3 4SAMPLE OUTPUT:713313311
716721076
782223918
335511486
539247783Make sure to output the answers modulo $10^9+7$.SCORING:In test cases 4-7, $K\le 5$ and $(b_s,s)$ is the same for all queries.In test cases 8-11, $b_s=K-1$ and $b_t=K$ for each query.In test cases 12-15, $N,K,Q\le 20$.In test cases 16-23, there are no additional constraints.Problem credits: Benjamin Qi

For the last query, Bessie has five possible sequences of button presses:$(2)$$(2,3,2)$$(2,3,1,2)$$(2,1,3,2)$$(2,1,3,1,2)$SAMPLE INPUT:6 4 6
001100
001110
101101
010111
110111
000111
3 2 4 3
3 1 4 4
3 4 4 1
3 3 4 3
3 6 4 3
3 1 4 2SAMPLE OUTPUT:26
49
29
27
18
22This test case satisfies the constraints for all subtasks aside from the first.SAMPLE INPUT:6 10 5
110101
011001
001111
101111
111010
000001
2 5 2 5
6 1 5 2
3 4 8 3
9 3 3 5
5 1 3 4SAMPLE OUTPUT:713313311
716721076
782223918
335511486
539247783Make sure to output the answers modulo $10^9+7$.SCORING:In test cases 4-7, $K\le 5$ and $(b_s,s)$ is the same for all queries.In test cases 8-11, $b_s=K-1$ and $b_t=K$ for each query.In test cases 12-15, $N,K,Q\le 20$.In test cases 16-23, there are no additional constraints.Problem credits: Benjamin Qi

$(2)$$(2,3,2)$$(2,3,1,2)$$(2,1,3,2)$$(2,1,3,1,2)$SAMPLE INPUT:6 4 6
001100
001110
101101
010111
110111
000111
3 2 4 3
3 1 4 4
3 4 4 1
3 3 4 3
3 6 4 3
3 1 4 2SAMPLE OUTPUT:26
49
29
27
18
22This test case satisfies the constraints for all subtasks aside from the first.SAMPLE INPUT:6 10 5
110101
011001
001111
101111
111010
000001
2 5 2 5
6 1 5 2
3 4 8 3
9 3 3 5
5 1 3 4SAMPLE OUTPUT:713313311
716721076
782223918
335511486
539247783Make sure to output the answers modulo $10^9+7$.SCORING:In test cases 4-7, $K\le 5$ and $(b_s,s)$ is the same for all queries.In test cases 8-11, $b_s=K-1$ and $b_t=K$ for each query.In test cases 12-15, $N,K,Q\le 20$.In test cases 16-23, there are no additional constraints.Problem credits: Benjamin Qi

SAMPLE INPUT:6 4 6
001100
001110
101101
010111
110111
000111
3 2 4 3
3 1 4 4
3 4 4 1
3 3 4 3
3 6 4 3
3 1 4 2SAMPLE OUTPUT:26
49
29
27
18
22This test case satisfies the constraints for all subtasks aside from the first.SAMPLE INPUT:6 10 5
110101
011001
001111
101111
111010
000001
2 5 2 5
6 1 5 2
3 4 8 3
9 3 3 5
5 1 3 4SAMPLE OUTPUT:713313311
716721076
782223918
335511486
539247783Make sure to output the answers modulo $10^9+7$.SCORING:In test cases 4-7, $K\le 5$ and $(b_s,s)$ is the same for all queries.In test cases 8-11, $b_s=K-1$ and $b_t=K$ for each query.In test cases 12-15, $N,K,Q\le 20$.In test cases 16-23, there are no additional constraints.Problem credits: Benjamin Qi

26
49
29
27
18
22

This test case satisfies the constraints for all subtasks aside from the first.SAMPLE INPUT:6 10 5
110101
011001
001111
101111
111010
000001
2 5 2 5
6 1 5 2
3 4 8 3
9 3 3 5
5 1 3 4SAMPLE OUTPUT:713313311
716721076
782223918
335511486
539247783Make sure to output the answers modulo $10^9+7$.SCORING:In test cases 4-7, $K\le 5$ and $(b_s,s)$ is the same for all queries.In test cases 8-11, $b_s=K-1$ and $b_t=K$ for each query.In test cases 12-15, $N,K,Q\le 20$.In test cases 16-23, there are no additional constraints.Problem credits: Benjamin Qi

SAMPLE INPUT:6 10 5
110101
011001
001111
101111
111010
000001
2 5 2 5
6 1 5 2
3 4 8 3
9 3 3 5
5 1 3 4SAMPLE OUTPUT:713313311
716721076
782223918
335511486
539247783Make sure to output the answers modulo $10^9+7$.SCORING:In test cases 4-7, $K\le 5$ and $(b_s,s)$ is the same for all queries.In test cases 8-11, $b_s=K-1$ and $b_t=K$ for each query.In test cases 12-15, $N,K,Q\le 20$.In test cases 16-23, there are no additional constraints.Problem credits: Benjamin Qi

713313311
716721076
782223918
335511486
539247783

Make sure to output the answers modulo $10^9+7$.SCORING:In test cases 4-7, $K\le 5$ and $(b_s,s)$ is the same for all queries.In test cases 8-11, $b_s=K-1$ and $b_t=K$ for each query.In test cases 12-15, $N,K,Q\le 20$.In test cases 16-23, there are no additional constraints.Problem credits: Benjamin Qi

SCORING:In test cases 4-7, $K\le 5$ and $(b_s,s)$ is the same for all queries.In test cases 8-11, $b_s=K-1$ and $b_t=K$ for each query.In test cases 12-15, $N,K,Q\le 20$.In test cases 16-23, there are no additional constraints.Problem credits: Benjamin Qi
\end{verbatim}

\section*{Solution}


\section*{Problem URL}
https://usaco.org/index.php?page=viewproblem2&cpid=1069

\section*{Source}
USACO DEC20 Contest, Platinum Division, Problem ID: 1069

\end{document}
