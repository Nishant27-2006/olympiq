\documentclass[12pt]{article}
\usepackage{amsmath}
\usepackage{amssymb}
\usepackage{graphicx}
\usepackage{enumitem}
\usepackage{hyperref}
\usepackage{xcolor}

\title{View problem}
\author{USACO Contest: JAN17 Contest - Platinum Division}
\date{\today}

\begin{document}

\maketitle

\section*{Problem Statement}

Farmer John is arranging his $N$ cows in a line to take a photo
($1 \leq N \leq 50$).  The height of the $i$th cow in sequence is $a(i)$, and
Farmer John thinks it would make for an aesthetically pleasing photo if the cow
lineup has a large increasing subsequence of cows by height.

To recall, a subsequence is a subset $a(i_1), a(i_2), \ldots, a(i_k)$ of
elements from the cow sequence, found at some series of indices
$i_1 < i_2 < \ldots < i_k$.  We say the subsequence is increasing if
$a(i_1) \leq a(i_2) \leq \ldots \leq a(i_k)$.  

FJ would like there to be a long increasing subsequence within his ordering of
the cows.   In order to ensure this, he allows himself initially to choose any
subsequence and reverse its elements.

For example, if we had the list

1 6 2 3 4 3 5 3 4

We can reverse the chosen elements

1 6 2 3 4 3 5 3 4
  ^         ^ ^ ^

to get

1 4 2 3 4 3 3 5 6
  ^         ^ ^ ^

Observe how the subsequence being reversed ends up using the same indices as it
initially occupied, leaving the other elements unchanged.  

Please find the maximum possible length of an increasing subsequence, given that
you can choose to reverse an arbitrary subsequence once.

INPUT FORMAT (file subrev.in):
The first line of input contains $N$. The remaining $N$ lines contain
$a(1) \ldots a(N)$, each an integer in the range $1 \ldots 50$.

OUTPUT FORMAT (file subrev.out):
Output the number of elements that can possibly form a longest increasing
subsequence after reversing the contents of at most one subsequence.

SAMPLE INPUT:
9
1
2
3
9
5
6
8
7
4
SAMPLE OUTPUT: 
9


Problem credits: Lewin Gan



\section*{Input Format}
OUTPUT FORMAT (file subrev.out):Output the number of elements that can possibly form a longest increasing
subsequence after reversing the contents of at most one subsequence.SAMPLE INPUT:9
1
2
3
9
5
6
8
7
4SAMPLE OUTPUT:9Problem credits: Lewin Gan

\section*{Output Format}
SAMPLE INPUT:9
1
2
3
9
5
6
8
7
4SAMPLE OUTPUT:9Problem credits: Lewin Gan

\section*{Sample Input}
\begin{verbatim}
9
1
2
3
9
5
6
8
7
4
\end{verbatim}

\section*{Sample Output}
\begin{verbatim}
9

Problem credits: Lewin Gan

Problem credits: Lewin Gan
\end{verbatim}

\section*{Solution}


\section*{Problem URL}
https://usaco.org/index.php?page=viewproblem2&cpid=698

\section*{Source}
USACO JAN17 Contest, Platinum Division, Problem ID: 698

\end{document}
