\documentclass[12pt]{article}
\usepackage{amsmath}
\usepackage{amssymb}
\usepackage{graphicx}
\usepackage{enumitem}
\usepackage{hyperref}
\usepackage{xcolor}

\title{View problem}
\author{USACO Contest: JAN23 Contest - Platinum Division}
\date{\today}

\begin{document}

\maketitle

\section*{Problem Statement}


For the New Year celebration, Bessie and her friends have constructed a giant
tree with many glowing ornaments. Bessie has the ability to turn the ornaments
on and off through remote control. Before the sun rises, she wants to toggle
some of the ornaments in some order (possibly toggling an ornament more than
once) such that the tree starts and ends with no ornaments on. Bessie thinks the
tree looks cool if the set of activated ornaments is exactly a subtree rooted at
some vertex. She wants the order of ornaments she toggles to satisfy the
property that, for every subtree, at some point in time it was exactly the set
of all turned on ornaments. Additionally, it takes energy to switch on and off
ornaments, and Bessie does not want to waste energy, so she wants to find the
minimum number of toggles she can perform.

Formally, you have a tree with vertices labeled $1\dots N$
($2\le N\le 2\cdot 10^5$) rooted at $1$.  Each vertex is initially inactive. In
one operation, you can toggle the state of a single vertex from inactive to
active or vice versa. Output the minimum possible length of a sequence of
operations satisfying both of the following conditions:

Define the subtree rooted at a vertex $r$ to consist of all vertices $v$
such that $r$ lies on the path from $1$ to $v$ inclusive. For every one of the
$N$ subtrees of the tree, there is a moment in time when the set of active
vertices is precisely those in that subtree. Every vertex is inactive after the entire sequence of operations.
INPUT FORMAT (input arrives from the terminal / stdin):
The first line contains $N$.

The second line contains $p_2 \dots p_N$ ($1\le p_i<i$), where $p_i$ denotes the
parent of vertex $i$ in the tree.

OUTPUT FORMAT (print output to the terminal / stdout):
Output the minimum possible length.

SAMPLE INPUT:
3
1 1
SAMPLE OUTPUT: 
6

There are three subtrees, corresponding to $\{1,2,3\}$, $\{2\}$, and $\{3\}$.
Here is one sequence of operations of the minimum possible length:


activate 2
(activated vertices form the subtree rooted at 2)
activate 1
activate 3
(activated vertices form the subtree rooted at 1)
deactivate 1
deactivate 2
(activated vertices form the subtree rooted at 3)
deactivate 3

SCORING:
Inputs 2-3: $N \le 8$Inputs 4-9: $N \le 40$Inputs 10-15: $N \le 5000$Inputs 16-21: No  additional constraints.


Problem credits: Benjamin Qi



\section*{Input Format}
The second line contains $p_2 \dots p_N$ ($1\le p_i<i$), where $p_i$ denotes the
parent of vertex $i$ in the tree.

OUTPUT FORMAT (print output to the terminal / stdout):Output the minimum possible length.SAMPLE INPUT:3
1 1SAMPLE OUTPUT:6There are three subtrees, corresponding to $\{1,2,3\}$, $\{2\}$, and $\{3\}$.
Here is one sequence of operations of the minimum possible length:activate 2
(activated vertices form the subtree rooted at 2)
activate 1
activate 3
(activated vertices form the subtree rooted at 1)
deactivate 1
deactivate 2
(activated vertices form the subtree rooted at 3)
deactivate 3SCORING:Inputs 2-3: $N \le 8$Inputs 4-9: $N \le 40$Inputs 10-15: $N \le 5000$Inputs 16-21: No  additional constraints.Problem credits: Benjamin Qi

\section*{Output Format}
SAMPLE INPUT:3
1 1SAMPLE OUTPUT:6There are three subtrees, corresponding to $\{1,2,3\}$, $\{2\}$, and $\{3\}$.
Here is one sequence of operations of the minimum possible length:activate 2
(activated vertices form the subtree rooted at 2)
activate 1
activate 3
(activated vertices form the subtree rooted at 1)
deactivate 1
deactivate 2
(activated vertices form the subtree rooted at 3)
deactivate 3SCORING:Inputs 2-3: $N \le 8$Inputs 4-9: $N \le 40$Inputs 10-15: $N \le 5000$Inputs 16-21: No  additional constraints.Problem credits: Benjamin Qi

\section*{Sample Input}
\begin{verbatim}
3
1 1
\end{verbatim}

\section*{Sample Output}
\begin{verbatim}
6

There are three subtrees, corresponding to $\{1,2,3\}$, $\{2\}$, and $\{3\}$.
Here is one sequence of operations of the minimum possible length:activate 2
(activated vertices form the subtree rooted at 2)
activate 1
activate 3
(activated vertices form the subtree rooted at 1)
deactivate 1
deactivate 2
(activated vertices form the subtree rooted at 3)
deactivate 3SCORING:Inputs 2-3: $N \le 8$Inputs 4-9: $N \le 40$Inputs 10-15: $N \le 5000$Inputs 16-21: No  additional constraints.Problem credits: Benjamin Qi

activate 2
(activated vertices form the subtree rooted at 2)
activate 1
activate 3
(activated vertices form the subtree rooted at 1)
deactivate 1
deactivate 2
(activated vertices form the subtree rooted at 3)
deactivate 3SCORING:Inputs 2-3: $N \le 8$Inputs 4-9: $N \le 40$Inputs 10-15: $N \le 5000$Inputs 16-21: No  additional constraints.Problem credits: Benjamin Qi

activate 2
(activated vertices form the subtree rooted at 2)
activate 1
activate 3
(activated vertices form the subtree rooted at 1)
deactivate 1
deactivate 2
(activated vertices form the subtree rooted at 3)
deactivate 3

SCORING:Inputs 2-3: $N \le 8$Inputs 4-9: $N \le 40$Inputs 10-15: $N \le 5000$Inputs 16-21: No  additional constraints.Problem credits: Benjamin Qi
\end{verbatim}

\section*{Solution}


\section*{Problem URL}
https://usaco.org/index.php?page=viewproblem2&cpid=1286

\section*{Source}
USACO JAN23 Contest, Platinum Division, Problem ID: 1286

\end{document}
