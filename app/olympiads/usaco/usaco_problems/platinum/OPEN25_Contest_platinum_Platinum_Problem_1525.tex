\documentclass[12pt]{article}
\usepackage{amsmath}
\usepackage{amssymb}
\usepackage{graphicx}
\usepackage{enumitem}
\usepackage{hyperref}
\usepackage{xcolor}

\title{View problem}
\author{USACO Contest: OPEN25 Contest - Platinum Division}
\date{\today}

\begin{document}

\maketitle

\section*{Problem Statement}


Farmer John has $N$ cows ($2 \leq N \leq 5\cdot 10^6$) and is attempting to get
them to sort a non-negative integer array $A$ of length $N$ by  relying on their
laziness. He has a lot of heavy boxes so he lines the cows up one behind 
another, where cow $i+1$ is behind cow $i$, and gives $a_i$ boxes to cow $i$
($0\le a_i$). 

Cows are inherently lazy so they always look to pass their work off to someone
else. From cow $1$ to $N-1$ in order, each cow looks to the cow behind them. If
cow $i$ has strictly more boxes than cow $i+1$, cow $i$ thinks this is "unfair"
and gives one of its boxes to cow $i+1$. This process repeats until every cow is
satisfied.

Farmer John will then note the number of boxes $b_i$ that each cow $i$ is
holding and create an array $B$ out of these values. If $B = sorted(A)$, then
Farmer John will be happy. Unfortunately, Farmer John forgot all but $Q$ values
($2 \leq Q \leq \min(N, 100)$) in $A$. Luckily, those values include the number
of boxes he was going to give to the first and last cow. Each value that FJ
remembers is given in the form $c_i \; v_i$ representing that $a_{c_i}=v_i$
($1 \leq c_i \leq N$, $1\le v_i\le 10^9$). Determine the number of different
ways the missing values can be filled in so that he will be happy mod $10^9+7$. 

INPUT FORMAT (input arrives from the terminal / stdin):
The first line contains two space-separated integers $N$ and $Q$ representing
the number of cows and queries respectively.

The next $Q$ lines contain two space separated integers $c_i \; v_i$
representing that cow $c_i$ initially holds $v_i$ boxes. It is guaranteed that
$c_1 = 1$, $c_Q = N$, and $c_i < c_{i+1}$ (the order of the cows is strictly
increasing).

OUTPUT FORMAT (print output to the terminal / stdout):
Print the number of different ways modulo $10^9+7$ that values $a_i$ can be
assigned such that Farmer John will be happy after the cows perform the lazy
sort. It is guaranteed that there will be at least one valid assignment. 

SAMPLE INPUT:
3 2
1 3
3 2
SAMPLE OUTPUT: 
2

In this example, FJ remembers the values at the ends of the array. The arrays
$[3,2,2]$ and $[3,3,2]$ are the valid arrays that will make FJ happy at the end
of the lazy sorting.
SAMPLE INPUT:
6 3
1 1
3 3
6 5
SAMPLE OUTPUT: 
89

SCORING:
Inputs 3-4: $N,v_i\le 100$Inputs 5-6: $N\le 100$ and $v_i \leq 10^6$Inputs 7-9: $N \leq 2\cdot 10^5$ and $v_i \le 10^6$Inputs 10-12: $N \leq 2\cdot 10^5$Inputs 13-15: No additional constraints.


Problem credits: Suhas Nagar



\section*{Input Format}
The next $Q$ lines contain two space separated integers $c_i \; v_i$
representing that cow $c_i$ initially holds $v_i$ boxes. It is guaranteed that
$c_1 = 1$, $c_Q = N$, and $c_i < c_{i+1}$ (the order of the cows is strictly
increasing).

OUTPUT FORMAT (print output to the terminal / stdout):Print the number of different ways modulo $10^9+7$ that values $a_i$ can be
assigned such that Farmer John will be happy after the cows perform the lazy
sort. It is guaranteed that there will be at least one valid assignment.SAMPLE INPUT:3 2
1 3
3 2SAMPLE OUTPUT:2In this example, FJ remembers the values at the ends of the array. The arrays
$[3,2,2]$ and $[3,3,2]$ are the valid arrays that will make FJ happy at the end
of the lazy sorting.SAMPLE INPUT:6 3
1 1
3 3
6 5SAMPLE OUTPUT:89SCORING:Inputs 3-4: $N,v_i\le 100$Inputs 5-6: $N\le 100$ and $v_i \leq 10^6$Inputs 7-9: $N \leq 2\cdot 10^5$ and $v_i \le 10^6$Inputs 10-12: $N \leq 2\cdot 10^5$Inputs 13-15: No additional constraints.Problem credits: Suhas Nagar

\section*{Output Format}
SAMPLE INPUT:3 2
1 3
3 2SAMPLE OUTPUT:2In this example, FJ remembers the values at the ends of the array. The arrays
$[3,2,2]$ and $[3,3,2]$ are the valid arrays that will make FJ happy at the end
of the lazy sorting.SAMPLE INPUT:6 3
1 1
3 3
6 5SAMPLE OUTPUT:89SCORING:Inputs 3-4: $N,v_i\le 100$Inputs 5-6: $N\le 100$ and $v_i \leq 10^6$Inputs 7-9: $N \leq 2\cdot 10^5$ and $v_i \le 10^6$Inputs 10-12: $N \leq 2\cdot 10^5$Inputs 13-15: No additional constraints.Problem credits: Suhas Nagar

\section*{Sample Input}
\begin{verbatim}
3 2
1 3
3 2

6 3
1 1
3 3
6 5
\end{verbatim}

\section*{Sample Output}
\begin{verbatim}
2

89

SCORING:Inputs 3-4: $N,v_i\le 100$Inputs 5-6: $N\le 100$ and $v_i \leq 10^6$Inputs 7-9: $N \leq 2\cdot 10^5$ and $v_i \le 10^6$Inputs 10-12: $N \leq 2\cdot 10^5$Inputs 13-15: No additional constraints.Problem credits: Suhas Nagar
\end{verbatim}

\section*{Solution}


\section*{Problem URL}
https://usaco.org/index.php?page=viewproblem2&cpid=1525

\section*{Source}
USACO OPEN25 Contest, Platinum Division, Problem ID: 1525

\end{document}
