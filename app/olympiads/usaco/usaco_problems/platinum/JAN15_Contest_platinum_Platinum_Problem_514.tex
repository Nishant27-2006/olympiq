\documentclass[12pt]{article}
\usepackage{amsmath}
\usepackage{amssymb}
\usepackage{graphicx}
\usepackage{enumitem}
\usepackage{hyperref}
\usepackage{xcolor}

\title{View problem}
\author{USACO Contest: JAN15 Contest - Platinum Division}
\date{\today}

\begin{document}

\maketitle

\section*{Problem Statement}

Problem 1: Cow Rectangles [Nick Wu, 2015]

The locations of Farmer John's N cows (1 <= N <= 500) are described by
distinct points in the 2D plane.  The cows belong to two different
breeds: Holsteins and Guernseys.  Farmer John wants to build a
rectangular fence with sides parallel to the coordinate axes enclosing
only Holsteins, with no Guernseys (a cow counts as enclosed even if it
is on the boundary of the fence).  Among all such fences, Farmer John
wants to build a fence enclosing the maximum number of Holsteins.  And
among all these fences, Farmer John wants to build a fence of minimum
possible area.  Please determine this area.  A fence of zero width or
height is allowable.

INPUT: (file cowrect.in)

The first line of input contains N.  Each of the next N lines
describes a cow, and contains two integers and a character. The
integers indicate a point (x,y) (0 <= x, y <= 1000) at which the cow
is located. The character is H or G, indicating the cow's breed.  No
two cows are located at the same point, and there is always at least
one Holstein.

SAMPLE INPUT:

5
1 1 H
2 2 H
3 3 G
4 4 H
6 6 H

OUTPUT: (file cowrect.out)

Print two integers. The first line should contain the maximum number
of Holsteins that can be enclosed by a fence containing no Guernseys,
and second line should contain the minimum area enclosed by such a
fence.

SAMPLE OUTPUT:

2
1



\section*{Input Format}


\section*{Output Format}


\section*{Sample Input}
\begin{verbatim}

\end{verbatim}

\section*{Sample Output}
\begin{verbatim}

\end{verbatim}

\section*{Solution}


\section*{Problem URL}
https://usaco.org/index.php?page=viewproblem2&cpid=514

\section*{Source}
USACO JAN15 Contest, Platinum Division, Problem ID: 514

\end{document}
