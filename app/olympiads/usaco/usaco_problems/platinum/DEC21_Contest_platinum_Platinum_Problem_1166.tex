\documentclass[12pt]{article}
\usepackage{amsmath}
\usepackage{amssymb}
\usepackage{graphicx}
\usepackage{enumitem}
\usepackage{hyperref}
\usepackage{xcolor}

\title{View problem}
\author{USACO Contest: DEC21 Contest - Platinum Division}
\date{\today}

\begin{document}

\maketitle

\section*{Problem Statement}

Bessie knows a number $x+0.5$ where $x$ is some integer between $0$ to $N,$
inclusive ($1\le N\le 5000$).  

Elsie is trying to guess this number. She can ask questions of the form  "is $i$
high or low?" for some integer $i$ between $1$ and $N,$ inclusive.  Bessie
responds by saying "HI!" if $i$ is greater than $x+0.5$, or "LO!" if $i$ is less
than $x+0.5$.

Elsie comes up with the following strategy for guessing Bessie's number. Before
making any guesses, she creates a list of $N$ numbers, where every number from
$1$ to $N$ occurs exactly once (in other words, the list is a permutation of
size $N$.) Then, she goes through the list, guessing numbers that appear in the
list in order. However, Elsie skips any unnecessary guesses. That is, if Elsie
is about to guess some number $i$ and Elsie previously guessed some $j < i$ such
that Bessie responded with "HI!," Elsie will not guess $i$ and will move on to
the next number in the list. Similarly, if she is about to guess some number $i$
and she previously guessed some $j > i$ such that Bessie responded with "LO!,"
Elsie will not guess $i$ and will move on to the next number in the list. It can
be proven that using this strategy, Elsie always uniquely determines $x$
regardless of the permutation she creates. 

If we concatenate all of Bessie's responses of either "HI" or "LO" into a single
string $S,$ the number of times Bessie says "HILO" is the number of length $4$
substrings of $S$ that are equal to "HILO."

Bessie knows that Elsie will use this strategy and has already chosen the value
of $x$, but she does not know what permutation Elsie will use. Your goal is to
compute the sum of the number of times Bessie says "HILO" over all permutations
that Elsie could possibly choose, modulo $10^9+7$.

INPUT FORMAT (input arrives from the terminal / stdin):
The only line of input contains $N$ and $x$.

OUTPUT FORMAT (print output to the terminal / stdout):
The total number of HILOs modulo $10^9+7$.

SAMPLE INPUT:
4 2
SAMPLE OUTPUT: 
17

In this test case, Bessie's number is $2.5$.

For example, if Elsie's permutation is $(4,1,3,2)$, then Bessie will say
"HILOHILO,"  for a total of two "HILO"s. As another example, if Elsie's
permutation is $(3,1,2,4)$, then Bessie will say "HILOLO," for a total of one
"HILO."

SAMPLE INPUT:
60 10
SAMPLE OUTPUT: 
508859913

Make sure to output the sum modulo $10^9+7$.

SCORING:
Test cases 3-10 satisfy $N\le 50$.Test cases 11-18 satisfy $N\le 500$.Test cases 19-26 satisfy no additional constraints.


Problem credits: Richard Qi



\section*{Input Format}
OUTPUT FORMAT (print output to the terminal / stdout):The total number of HILOs modulo $10^9+7$.SAMPLE INPUT:4 2SAMPLE OUTPUT:17In this test case, Bessie's number is $2.5$.For example, if Elsie's permutation is $(4,1,3,2)$, then Bessie will say
"HILOHILO,"  for a total of two "HILO"s. As another example, if Elsie's
permutation is $(3,1,2,4)$, then Bessie will say "HILOLO," for a total of one
"HILO."SAMPLE INPUT:60 10SAMPLE OUTPUT:508859913Make sure to output the sum modulo $10^9+7$.SCORING:Test cases 3-10 satisfy $N\le 50$.Test cases 11-18 satisfy $N\le 500$.Test cases 19-26 satisfy no additional constraints.Problem credits: Richard Qi

\section*{Output Format}
SAMPLE INPUT:4 2SAMPLE OUTPUT:17In this test case, Bessie's number is $2.5$.For example, if Elsie's permutation is $(4,1,3,2)$, then Bessie will say
"HILOHILO,"  for a total of two "HILO"s. As another example, if Elsie's
permutation is $(3,1,2,4)$, then Bessie will say "HILOLO," for a total of one
"HILO."SAMPLE INPUT:60 10SAMPLE OUTPUT:508859913Make sure to output the sum modulo $10^9+7$.SCORING:Test cases 3-10 satisfy $N\le 50$.Test cases 11-18 satisfy $N\le 500$.Test cases 19-26 satisfy no additional constraints.Problem credits: Richard Qi

\section*{Sample Input}
\begin{verbatim}
4 2

60 10
\end{verbatim}

\section*{Sample Output}
\begin{verbatim}
17

In this test case, Bessie's number is $2.5$.For example, if Elsie's permutation is $(4,1,3,2)$, then Bessie will say
"HILOHILO,"  for a total of two "HILO"s. As another example, if Elsie's
permutation is $(3,1,2,4)$, then Bessie will say "HILOLO," for a total of one
"HILO."SAMPLE INPUT:60 10SAMPLE OUTPUT:508859913Make sure to output the sum modulo $10^9+7$.SCORING:Test cases 3-10 satisfy $N\le 50$.Test cases 11-18 satisfy $N\le 500$.Test cases 19-26 satisfy no additional constraints.Problem credits: Richard Qi

For example, if Elsie's permutation is $(4,1,3,2)$, then Bessie will say
"HILOHILO,"  for a total of two "HILO"s. As another example, if Elsie's
permutation is $(3,1,2,4)$, then Bessie will say "HILOLO," for a total of one
"HILO."SAMPLE INPUT:60 10SAMPLE OUTPUT:508859913Make sure to output the sum modulo $10^9+7$.SCORING:Test cases 3-10 satisfy $N\le 50$.Test cases 11-18 satisfy $N\le 500$.Test cases 19-26 satisfy no additional constraints.Problem credits: Richard Qi

SAMPLE INPUT:60 10SAMPLE OUTPUT:508859913Make sure to output the sum modulo $10^9+7$.SCORING:Test cases 3-10 satisfy $N\le 50$.Test cases 11-18 satisfy $N\le 500$.Test cases 19-26 satisfy no additional constraints.Problem credits: Richard Qi

508859913

Make sure to output the sum modulo $10^9+7$.SCORING:Test cases 3-10 satisfy $N\le 50$.Test cases 11-18 satisfy $N\le 500$.Test cases 19-26 satisfy no additional constraints.Problem credits: Richard Qi

SCORING:Test cases 3-10 satisfy $N\le 50$.Test cases 11-18 satisfy $N\le 500$.Test cases 19-26 satisfy no additional constraints.Problem credits: Richard Qi
\end{verbatim}

\section*{Solution}


\section*{Problem URL}
https://usaco.org/index.php?page=viewproblem2&cpid=1166

\section*{Source}
USACO DEC21 Contest, Platinum Division, Problem ID: 1166

\end{document}
