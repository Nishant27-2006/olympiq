\documentclass[12pt]{article}
\usepackage{amsmath}
\usepackage{amssymb}
\usepackage{graphicx}
\usepackage{enumitem}
\usepackage{hyperref}
\usepackage{xcolor}

\title{View problem}
\author{USACO Contest: FEB20 Contest - Platinum Division}
\date{\today}

\begin{document}

\maketitle

\section*{Problem Statement}

Farmer John's farm consists of $N$ pastures ($2 \leq N \leq 10^5$) connected by
$N-1$ roads, so that any pasture is reachable from any other pasture. That is,
the farm is a tree. But after 28 years of dealing with the tricky algorithmic
problems that inevitably arise from trees, FJ has decided that a farm in the
shape of a tree is just too complex. He believes that algorithmic problems are
simpler on paths.

Thus, his plan is to partition the set of roads into several paths and delegate
responsibility for these path among his worthy farm hands. He doesn't care about
the number of paths. However, he wants to make sure that these paths are all as
large as possible, so that no farm hand can get away with asymptotically
inefficient algorithms!

Help Farmer John determine the largest positive integer $K$ such that the roads
can be partitioned into paths of length at least $K$.

SCORING:
In test cases 2-4 the tree forms a star; at most one vertex has degree
greater than two.Test cases 5-8 satisfy $N\le 10^3$.Test cases 9-15 satisfy no additional constraints..

INPUT FORMAT (file deleg.in):
The first line contains a single integer $N$. 

The next $N-1$ lines each contain two space-separated integers $a$ and $b$
describing an edge between vertices $a$ and $b$.  Both $a$ and $b$ are in the
range $1 \ldots N$. 

OUTPUT FORMAT (file deleg.out):
Print $K$.

SAMPLE INPUT:
8
1 2
1 3
1 4
4 5
1 6
6 7
7 8
SAMPLE OUTPUT: 
3

One possible set of paths is as follows:
$$2-1-6-7-8, 3-1-4-5$$

Problem credits: Mark Gordon and Dhruv Rohatgi



\section*{Input Format}
The next $N-1$ lines each contain two space-separated integers $a$ and $b$
describing an edge between vertices $a$ and $b$.  Both $a$ and $b$ are in the
range $1 \ldots N$.

OUTPUT FORMAT (file deleg.out):Print $K$.SAMPLE INPUT:8
1 2
1 3
1 4
4 5
1 6
6 7
7 8SAMPLE OUTPUT:3One possible set of paths is as follows:$$2-1-6-7-8, 3-1-4-5$$Problem credits: Mark Gordon and Dhruv Rohatgi

\section*{Output Format}
SAMPLE INPUT:8
1 2
1 3
1 4
4 5
1 6
6 7
7 8SAMPLE OUTPUT:3One possible set of paths is as follows:$$2-1-6-7-8, 3-1-4-5$$Problem credits: Mark Gordon and Dhruv Rohatgi

\section*{Sample Input}
\begin{verbatim}
8
1 2
1 3
1 4
4 5
1 6
6 7
7 8
\end{verbatim}

\section*{Sample Output}
\begin{verbatim}
3

One possible set of paths is as follows:$$2-1-6-7-8, 3-1-4-5$$Problem credits: Mark Gordon and Dhruv Rohatgi

Problem credits: Mark Gordon and Dhruv Rohatgi

Problem credits: Mark Gordon and Dhruv Rohatgi
\end{verbatim}

\section*{Solution}


\section*{Problem URL}
https://usaco.org/index.php?page=viewproblem2&cpid=1020

\section*{Source}
USACO FEB20 Contest, Platinum Division, Problem ID: 1020

\end{document}
