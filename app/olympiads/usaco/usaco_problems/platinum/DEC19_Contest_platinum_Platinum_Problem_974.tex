\documentclass[12pt]{article}
\usepackage{amsmath}
\usepackage{amssymb}
\usepackage{graphicx}
\usepackage{enumitem}
\usepackage{hyperref}
\usepackage{xcolor}

\title{View problem}
\author{USACO Contest: DEC19 Contest - Platinum Division}
\date{\today}

\begin{document}

\maketitle

\section*{Problem Statement}

For the new year, Farmer John decided to give his cows a festive binary search tree
(BST)! 

To generate the BST, FJ starts with a permutation $a=\{a_1,a_2,\ldots,a_N\}$
of the integers $1\ldots N$, where $N\le 300$.  He then runs the following
pseudocode with arguments $1$ and $N.$


generate(l,r):
  if l > r, return empty subtree;
  x = argmin_{l <= i <= r} a_i; // index of min a_i in {a_l,...,a_r}
  return a BST with x as the root, 
    generate(l,x-1) as the left subtree,
    generate(x+1,r) as the right subtree;

For example, the permutation $\{3,2,5,1,4\}$ generates the following BST:


    4
   / \
  2   5
 / \ 
1   3

Let $d_i(a)$ denote the depth of node $i$ in the tree corresponding to $a,$ 
meaning the number of nodes on the path from $a_i$ to the root. In the above
example, $d_4(a)=1, d_2(a)=d_5(a)=2,$ and $d_1(a)=d_3(a)=3.$

The number of inversions of $a$ is equal to the number of pairs of integers
$(i,j)$ such that $1\le i<j\le N$ and $a_i>a_j.$ The cows know that the $a$ that
FJ will use to generate the BST has exactly $K$ inversions
$(0\le K\le \frac{N(N-1)}{2})$.  Over all $a$ satisfying this condition, compute
the remainder when $\sum_ad_i(a)$ is divided by $M$ for each $1\le i\le N.$

INPUT FORMAT (file treedepth.in):
The only line of input consists of three space-separated integers $N, K,$ and
$M$, followed by a new line. $M$ will be a prime number in the range
$[10^8,10^9+9].$

OUTPUT FORMAT (file treedepth.out):
Print $N$ space-separated integers denoting $\sum_ad_i(a)\pmod{M}$ for each 
$1\le i\le N.$

BATCHING:
Test cases 3-4 satisfy $N\le 8.$ Test cases 5-7 satisfy $N\le 20.$ Test cases 8-10 satisfy $N\le 50.$ 

SAMPLE INPUT:
3 0 192603497
SAMPLE OUTPUT: 
1 2 3 
Here, the only permutation is $a=\{1,2,3\}.$ 

SAMPLE INPUT:
3 1 144408983
SAMPLE OUTPUT: 
3 4 4 
Here, the two permutations are $a=\{1,3,2\}$ and $a=\{2,1,3\}.$


Problem credits: Yinzhan Xu



\section*{Input Format}
OUTPUT FORMAT (file treedepth.out):Print $N$ space-separated integers denoting $\sum_ad_i(a)\pmod{M}$ for each 
$1\le i\le N.$BATCHING:Test cases 3-4 satisfy $N\le 8.$Test cases 5-7 satisfy $N\le 20.$Test cases 8-10 satisfy $N\le 50.$SAMPLE INPUT:3 0 192603497SAMPLE OUTPUT:1 2 3Here, the only permutation is $a=\{1,2,3\}.$SAMPLE INPUT:3 1 144408983SAMPLE OUTPUT:3 4 4Here, the two permutations are $a=\{1,3,2\}$ and $a=\{2,1,3\}.$Problem credits: Yinzhan Xu

\section*{Output Format}
BATCHING:Test cases 3-4 satisfy $N\le 8.$Test cases 5-7 satisfy $N\le 20.$Test cases 8-10 satisfy $N\le 50.$SAMPLE INPUT:3 0 192603497SAMPLE OUTPUT:1 2 3Here, the only permutation is $a=\{1,2,3\}.$SAMPLE INPUT:3 1 144408983SAMPLE OUTPUT:3 4 4Here, the two permutations are $a=\{1,3,2\}$ and $a=\{2,1,3\}.$Problem credits: Yinzhan Xu

\section*{Sample Input}
\begin{verbatim}
3 0 192603497

3 1 144408983
\end{verbatim}

\section*{Sample Output}
\begin{verbatim}
1 2 3

Here, the only permutation is $a=\{1,2,3\}.$SAMPLE INPUT:3 1 144408983SAMPLE OUTPUT:3 4 4Here, the two permutations are $a=\{1,3,2\}$ and $a=\{2,1,3\}.$Problem credits: Yinzhan Xu

SAMPLE INPUT:3 1 144408983SAMPLE OUTPUT:3 4 4Here, the two permutations are $a=\{1,3,2\}$ and $a=\{2,1,3\}.$Problem credits: Yinzhan Xu

3 4 4

Here, the two permutations are $a=\{1,3,2\}$ and $a=\{2,1,3\}.$Problem credits: Yinzhan Xu

Problem credits: Yinzhan Xu

Problem credits: Yinzhan Xu
\end{verbatim}

\section*{Solution}


\section*{Problem URL}
https://usaco.org/index.php?page=viewproblem2&cpid=974

\section*{Source}
USACO DEC19 Contest, Platinum Division, Problem ID: 974

\end{document}
