\documentclass[12pt]{article}
\usepackage{amsmath}
\usepackage{amssymb}
\usepackage{graphicx}
\usepackage{enumitem}
\usepackage{hyperref}
\usepackage{xcolor}

\title{View problem}
\author{USACO Contest: FEB18 Contest - Platinum Division}
\date{\today}

\begin{document}

\maketitle

\section*{Problem Statement}

Farmer John notices that his cows tend to get into arguments if they are packed
too closely together, so he wants to open a series of new barns to help spread
them out.

Whenever FJ constructs a new barn, he connects it with at most one bidirectional
pathway to  an existing barn.  In order to make sure his cows are spread
sufficiently far apart, he sometimes wants to determine the distance from a
certain barn to the farthest possible barn reachable from it (the distance
between two barns is the number of paths one must traverse to go from one barn
to the other).

FJ will give a total of $Q$ ($1 \leq Q \leq 10^5$) queries, each either of the
form "build" or "distance". For a build query, FJ builds a barn and links it
with at most one previously built barn. For a distance query, FJ asks you the
distance from a certain barn to the farthest barn reachable from it via a series
of pathways.  It is guaranteed that the queried barn has already been built.
Please help FJ answer all of these queries.

INPUT FORMAT (file newbarn.in):
The first line contains the integer $Q$. Each of the next $Q$ lines contains a
query. Each query is of the form "B p" or "Q k", respectively telling you to
build a barn and connect it with barn $p$, or give the farthest distance, as
defined, from barn $k$. If $p = -1$, then the new barn will be connected to no
other barn. Otherwise, $p$ is the index of a barn that has already been built.
The barn indices start from $1$, so the first barn built is barn $1$, the second
is barn $2$, and so on.

OUTPUT FORMAT (file newbarn.out):
Please write one line of output for each distance query.  Note that a barn
which is connected to no other barns has farthest distance $0$.

SAMPLE INPUT:
7

B -1

Q 1

B 1

B 2

Q 3

B 2

Q 2

SAMPLE OUTPUT: 
0

2

1


The example input corresponds to this network of barns:


  (1) 
    \   
     (2)---(4)
    /
  (3)

In query 1, we build barn number 1. In query 2, we ask for the distance of 1 to
the farthest connected barn. Since barn 1 is connected to no other barns, the
answer is 0. In query 3, we build barn number 2 and connect it to barn 1. In
query 4, we build barn number 3 and connect it to barn 2. In query 5, we ask for
the distance of 3 to the farthest connected barn. In this case, the farthest is
barn 1, which is 2 units away. In query 6, we build barn number 4 and connect it
to barn 2. In query 7, we ask for the distance of 2 to the farthest connected
barn. All three barns 1, 3, 4 are the same distance away, which is 1, so this is
our answer. 


Problem credits: Anson Hu



\section*{Input Format}
OUTPUT FORMAT (file newbarn.out):Please write one line of output for each distance query.  Note that a barn
which is connected to no other barns has farthest distance $0$.SAMPLE INPUT:7

B -1

Q 1

B 1

B 2

Q 3

B 2

Q 2SAMPLE OUTPUT:0

2

1The example input corresponds to this network of barns:(1) 
    \   
     (2)---(4)
    /
  (3)In query 1, we build barn number 1. In query 2, we ask for the distance of 1 to
the farthest connected barn. Since barn 1 is connected to no other barns, the
answer is 0. In query 3, we build barn number 2 and connect it to barn 1. In
query 4, we build barn number 3 and connect it to barn 2. In query 5, we ask for
the distance of 3 to the farthest connected barn. In this case, the farthest is
barn 1, which is 2 units away. In query 6, we build barn number 4 and connect it
to barn 2. In query 7, we ask for the distance of 2 to the farthest connected
barn. All three barns 1, 3, 4 are the same distance away, which is 1, so this is
our answer.Problem credits: Anson Hu

\section*{Output Format}
SAMPLE INPUT:7

B -1

Q 1

B 1

B 2

Q 3

B 2

Q 2SAMPLE OUTPUT:0

2

1The example input corresponds to this network of barns:(1) 
    \   
     (2)---(4)
    /
  (3)In query 1, we build barn number 1. In query 2, we ask for the distance of 1 to
the farthest connected barn. Since barn 1 is connected to no other barns, the
answer is 0. In query 3, we build barn number 2 and connect it to barn 1. In
query 4, we build barn number 3 and connect it to barn 2. In query 5, we ask for
the distance of 3 to the farthest connected barn. In this case, the farthest is
barn 1, which is 2 units away. In query 6, we build barn number 4 and connect it
to barn 2. In query 7, we ask for the distance of 2 to the farthest connected
barn. All three barns 1, 3, 4 are the same distance away, which is 1, so this is
our answer.Problem credits: Anson Hu

\section*{Sample Input}
\begin{verbatim}
7

B -1

Q 1

B 1

B 2

Q 3

B 2

Q 2
\end{verbatim}

\section*{Sample Output}
\begin{verbatim}
0

2

1

The example input corresponds to this network of barns:(1) 
    \   
     (2)---(4)
    /
  (3)In query 1, we build barn number 1. In query 2, we ask for the distance of 1 to
the farthest connected barn. Since barn 1 is connected to no other barns, the
answer is 0. In query 3, we build barn number 2 and connect it to barn 1. In
query 4, we build barn number 3 and connect it to barn 2. In query 5, we ask for
the distance of 3 to the farthest connected barn. In this case, the farthest is
barn 1, which is 2 units away. In query 6, we build barn number 4 and connect it
to barn 2. In query 7, we ask for the distance of 2 to the farthest connected
barn. All three barns 1, 3, 4 are the same distance away, which is 1, so this is
our answer.Problem credits: Anson Hu

(1) 
    \   
     (2)---(4)
    /
  (3)In query 1, we build barn number 1. In query 2, we ask for the distance of 1 to
the farthest connected barn. Since barn 1 is connected to no other barns, the
answer is 0. In query 3, we build barn number 2 and connect it to barn 1. In
query 4, we build barn number 3 and connect it to barn 2. In query 5, we ask for
the distance of 3 to the farthest connected barn. In this case, the farthest is
barn 1, which is 2 units away. In query 6, we build barn number 4 and connect it
to barn 2. In query 7, we ask for the distance of 2 to the farthest connected
barn. All three barns 1, 3, 4 are the same distance away, which is 1, so this is
our answer.Problem credits: Anson Hu

(1) 
    \   
     (2)---(4)
    /
  (3)

In query 1, we build barn number 1. In query 2, we ask for the distance of 1 to
the farthest connected barn. Since barn 1 is connected to no other barns, the
answer is 0. In query 3, we build barn number 2 and connect it to barn 1. In
query 4, we build barn number 3 and connect it to barn 2. In query 5, we ask for
the distance of 3 to the farthest connected barn. In this case, the farthest is
barn 1, which is 2 units away. In query 6, we build barn number 4 and connect it
to barn 2. In query 7, we ask for the distance of 2 to the farthest connected
barn. All three barns 1, 3, 4 are the same distance away, which is 1, so this is
our answer.Problem credits: Anson Hu

Problem credits: Anson Hu

Problem credits: Anson Hu
\end{verbatim}

\section*{Solution}


\section*{Problem URL}
https://usaco.org/index.php?page=viewproblem2&cpid=817

\section*{Source}
USACO FEB18 Contest, Platinum Division, Problem ID: 817

\end{document}
