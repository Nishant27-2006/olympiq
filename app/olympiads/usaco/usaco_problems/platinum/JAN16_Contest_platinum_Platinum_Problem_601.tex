\documentclass[12pt]{article}
\usepackage{amsmath}
\usepackage{amssymb}
\usepackage{graphicx}
\usepackage{enumitem}
\usepackage{hyperref}
\usepackage{xcolor}

\title{View problem}
\author{USACO Contest: JAN16 Contest - Platinum Division}
\date{\today}

\begin{document}

\maketitle

\section*{Problem Statement}

Farmer John is quite reliable in all aspects of managing his farm, except one:
he is terrible at mowing the grass in a timely fashion.   He only manages to
move the mowing machine once per day, in fact. On day 1, he starts at position
$(x_1, y_1)$ and on day $d$ he mows along a straight segment to the position
$(x_d, y_d)$, moving either horizontally or vertically on the 2D map of his
farm; that is, either  $x_d = x_{d-1}$, or $y_d = y_{d-1}$.  FJ alternates
between horizontal and vertical moves on successive days.

So slow is FJ's progress that some of the grass he mows might grow back before
he is finished with all his mowing. Any section of grass that is cut in day $d$
will reappear on day $d + T$, so if FJ's mowing path crosses a path he cut at
least $T$ days earlier, he will end up cutting grass at the same point again. 
In an effort to try and reform his poor mowing strategy, FJ would like to count
the number of times this happens.  

Please count the number of times FJ's mowing path crosses over an earlier
segment on which grass has already grown back.  You should only count
"perpendicular" crossings, defined as a point in common between a horizontal and
a vertical segment that is an endpoint of neither.

INPUT FORMAT (file mowing.in):
The first line of input contains $N$ ($2 \leq N \leq 100,000$) and $T$
($1 \leq T \leq N$, $T$ even).  

The next $N$ lines describe the position of the mower on days $1 \ldots N$.  The
$i$th of these lines contains integers $x_i$ and $y_i$ (nonnegative integers
each at most 1,000,000,000).

OUTPUT FORMAT (file mowing.out):
Please output a count of the number of crossing points described above, where FJ
re-cuts a point of grass that had grown back after being cut earlier.

SAMPLE INPUT:
7 4
0 10
10 10
10 5
3 5
3 12
6 12
6 3
SAMPLE OUTPUT: 
1

Here, FJ crosses on day 7 a segment of grass he cut on day 2, which counts. The
other intersections do not count.

Note: This problem has expanded limits: 5 seconds per test case (10 for Python and Java), and 512 MB of memory.

Problem credits: Chad Waters and Brian Dean



\section*{Input Format}
The next $N$ lines describe the position of the mower on days $1 \ldots N$.  The
$i$th of these lines contains integers $x_i$ and $y_i$ (nonnegative integers
each at most 1,000,000,000).

OUTPUT FORMAT (file mowing.out):Please output a count of the number of crossing points described above, where FJ
re-cuts a point of grass that had grown back after being cut earlier.SAMPLE INPUT:7 4
0 10
10 10
10 5
3 5
3 12
6 12
6 3SAMPLE OUTPUT:1Here, FJ crosses on day 7 a segment of grass he cut on day 2, which counts. The
other intersections do not count.Note: This problem has expanded limits: 5 seconds per test case (10 for Python and Java), and 512 MB of memory.Problem credits: Chad Waters and Brian Dean

\section*{Output Format}
SAMPLE INPUT:7 4
0 10
10 10
10 5
3 5
3 12
6 12
6 3SAMPLE OUTPUT:1Here, FJ crosses on day 7 a segment of grass he cut on day 2, which counts. The
other intersections do not count.Note: This problem has expanded limits: 5 seconds per test case (10 for Python and Java), and 512 MB of memory.Problem credits: Chad Waters and Brian Dean

\section*{Sample Input}
\begin{verbatim}
7 4
0 10
10 10
10 5
3 5
3 12
6 12
6 3
\end{verbatim}

\section*{Sample Output}
\begin{verbatim}
1

Here, FJ crosses on day 7 a segment of grass he cut on day 2, which counts. The
other intersections do not count.Note: This problem has expanded limits: 5 seconds per test case (10 for Python and Java), and 512 MB of memory.Problem credits: Chad Waters and Brian Dean

Note: This problem has expanded limits: 5 seconds per test case (10 for Python and Java), and 512 MB of memory.Problem credits: Chad Waters and Brian Dean

Problem credits: Chad Waters and Brian Dean
\end{verbatim}

\section*{Solution}


\section*{Problem URL}
https://usaco.org/index.php?page=viewproblem2&cpid=601

\section*{Source}
USACO JAN16 Contest, Platinum Division, Problem ID: 601

\end{document}
