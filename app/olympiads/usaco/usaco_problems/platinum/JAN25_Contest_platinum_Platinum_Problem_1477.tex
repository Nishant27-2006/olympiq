\documentclass[12pt]{article}
\usepackage{amsmath}
\usepackage{amssymb}
\usepackage{graphicx}
\usepackage{enumitem}
\usepackage{hyperref}
\usepackage{xcolor}

\title{View problem}
\author{USACO Contest: JAN25 Contest - Platinum Division}
\date{\today}

\begin{document}

\maketitle

\section*{Problem Statement}


Bessie is experimenting with a powerful hoof implant that has the ability to
create massive shock waves. She has $N$ ($2 \leq N \leq 10^5$) tiles lined up in
front of her, which require powers of at least $p_0,p_1,\dots,p_{N-1}$ to break,
respectively
($0 \leq p_i \leq 10^{18}$).

Bessie can apply power by punching a specific tile but due to the strange nature
of her implant, it will not apply any power to the tile she punches. Instead, if
she chooses to punch tile $x$ once, where $x$ is an integer in $[0,N-1]$, 
it applies $|i-x|$ power to tile $i$ for all integers $i$ in the range
$[0,N-1]$. This power is also cumulative, so applying $2$ power twice to a tile
will apply a total of $4$ power to the tile. 

Please determine the fewest number of punches required to break all the tiles.

INPUT FORMAT (input arrives from the terminal / stdin):
The first line contains $T$ ($1 \leq T \leq 100$) representing the number of
test cases. 

Line $2t$ contains a single integer $N$, the number of tiles in test case $t$.

Line $2t+1$ contains $N$ space separated numbers $p_0,p_1, \ldots, p_{N-1}$ representing
that tile $i$ takes $p_i$ power to be broken.

It is guaranteed that the sum of all $N$ in a single input does not exceed
$5\cdot 10^5$.

OUTPUT FORMAT (print output to the terminal / stdout):
$T$ lines, the $i$th line representing the answer to the $i$th test case.

SAMPLE INPUT:
6
5
0 2 4 5 8
5
6 5 4 5 6
5
1 1 1 1 1
5
12 10 8 6 4
7
6 1 2 3 5 8 13
2
1000000000000000000 1000000000000000000
SAMPLE OUTPUT: 
2
3
2
4
4
2000000000000000000

For the first test, the only way for Bessie to break all the tiles with two
punches is to punch $0$ twice, applying total powers of $[0,2,4,6,8]$
respectively.

For the second test, one way for Bessie to break all the tiles with three
punches is to punch $0$, $2$, and $4$ one time each, applying total powers of
$[6,5,4,5,6]$ respectively.

For the third test, one way for Bessie to break all the tiles with two punches
is to punch $0$ and $1$ one time each, applying total powers of $[1,1,3,5,7]$
respectively.

SCORING:
Input 2: All $p_i$ are equal.Inputs 3-6: $N\le 100$Inputs 7-14: No additional constraints.


Problem credits: Suhas Nagar



\section*{Input Format}
Line $2t$ contains a single integer $N$, the number of tiles in test case $t$.Line $2t+1$ contains $N$ space separated numbers $p_0,p_1, \ldots, p_{N-1}$ representing
that tile $i$ takes $p_i$ power to be broken.It is guaranteed that the sum of all $N$ in a single input does not exceed
$5\cdot 10^5$.

Line $2t+1$ contains $N$ space separated numbers $p_0,p_1, \ldots, p_{N-1}$ representing
that tile $i$ takes $p_i$ power to be broken.It is guaranteed that the sum of all $N$ in a single input does not exceed
$5\cdot 10^5$.

It is guaranteed that the sum of all $N$ in a single input does not exceed
$5\cdot 10^5$.

OUTPUT FORMAT (print output to the terminal / stdout):$T$ lines, the $i$th line representing the answer to the $i$th test case.SAMPLE INPUT:6
5
0 2 4 5 8
5
6 5 4 5 6
5
1 1 1 1 1
5
12 10 8 6 4
7
6 1 2 3 5 8 13
2
1000000000000000000 1000000000000000000SAMPLE OUTPUT:2
3
2
4
4
2000000000000000000For the first test, the only way for Bessie to break all the tiles with two
punches is to punch $0$ twice, applying total powers of $[0,2,4,6,8]$
respectively.For the second test, one way for Bessie to break all the tiles with three
punches is to punch $0$, $2$, and $4$ one time each, applying total powers of
$[6,5,4,5,6]$ respectively.For the third test, one way for Bessie to break all the tiles with two punches
is to punch $0$ and $1$ one time each, applying total powers of $[1,1,3,5,7]$
respectively.SCORING:Input 2: All $p_i$ are equal.Inputs 3-6: $N\le 100$Inputs 7-14: No additional constraints.Problem credits: Suhas Nagar

\section*{Output Format}
SAMPLE INPUT:6
5
0 2 4 5 8
5
6 5 4 5 6
5
1 1 1 1 1
5
12 10 8 6 4
7
6 1 2 3 5 8 13
2
1000000000000000000 1000000000000000000SAMPLE OUTPUT:2
3
2
4
4
2000000000000000000For the first test, the only way for Bessie to break all the tiles with two
punches is to punch $0$ twice, applying total powers of $[0,2,4,6,8]$
respectively.For the second test, one way for Bessie to break all the tiles with three
punches is to punch $0$, $2$, and $4$ one time each, applying total powers of
$[6,5,4,5,6]$ respectively.For the third test, one way for Bessie to break all the tiles with two punches
is to punch $0$ and $1$ one time each, applying total powers of $[1,1,3,5,7]$
respectively.SCORING:Input 2: All $p_i$ are equal.Inputs 3-6: $N\le 100$Inputs 7-14: No additional constraints.Problem credits: Suhas Nagar

\section*{Sample Input}
\begin{verbatim}
6
5
0 2 4 5 8
5
6 5 4 5 6
5
1 1 1 1 1
5
12 10 8 6 4
7
6 1 2 3 5 8 13
2
1000000000000000000 1000000000000000000
\end{verbatim}

\section*{Sample Output}
\begin{verbatim}
2
3
2
4
4
2000000000000000000

For the first test, the only way for Bessie to break all the tiles with two
punches is to punch $0$ twice, applying total powers of $[0,2,4,6,8]$
respectively.For the second test, one way for Bessie to break all the tiles with three
punches is to punch $0$, $2$, and $4$ one time each, applying total powers of
$[6,5,4,5,6]$ respectively.For the third test, one way for Bessie to break all the tiles with two punches
is to punch $0$ and $1$ one time each, applying total powers of $[1,1,3,5,7]$
respectively.SCORING:Input 2: All $p_i$ are equal.Inputs 3-6: $N\le 100$Inputs 7-14: No additional constraints.Problem credits: Suhas Nagar

For the second test, one way for Bessie to break all the tiles with three
punches is to punch $0$, $2$, and $4$ one time each, applying total powers of
$[6,5,4,5,6]$ respectively.For the third test, one way for Bessie to break all the tiles with two punches
is to punch $0$ and $1$ one time each, applying total powers of $[1,1,3,5,7]$
respectively.SCORING:Input 2: All $p_i$ are equal.Inputs 3-6: $N\le 100$Inputs 7-14: No additional constraints.Problem credits: Suhas Nagar

For the third test, one way for Bessie to break all the tiles with two punches
is to punch $0$ and $1$ one time each, applying total powers of $[1,1,3,5,7]$
respectively.SCORING:Input 2: All $p_i$ are equal.Inputs 3-6: $N\le 100$Inputs 7-14: No additional constraints.Problem credits: Suhas Nagar

SCORING:Input 2: All $p_i$ are equal.Inputs 3-6: $N\le 100$Inputs 7-14: No additional constraints.Problem credits: Suhas Nagar
\end{verbatim}

\section*{Solution}


\section*{Problem URL}
https://usaco.org/index.php?page=viewproblem2&cpid=1477

\section*{Source}
USACO JAN25 Contest, Platinum Division, Problem ID: 1477

\end{document}
