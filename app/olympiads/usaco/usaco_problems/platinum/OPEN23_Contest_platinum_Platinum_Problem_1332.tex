\documentclass[12pt]{article}
\usepackage{amsmath}
\usepackage{amssymb}
\usepackage{graphicx}
\usepackage{enumitem}
\usepackage{hyperref}
\usepackage{xcolor}

\title{View problem}
\author{USACO Contest: OPEN23 Contest - Platinum Division}
\date{\today}

\begin{document}

\maketitle

\section*{Problem Statement}


**Note: The time limit for this problem is 4s, twice the default. The memory
limit for this problem is 512MB, twice the default.**
Pareidolia is the phenomenon where your eyes tend to see familiar patterns in
images where none really exist -- for example seeing a face in a cloud.  As you
might imagine, with Farmer John's constant proximity to cows, he often sees
cow-related patterns in everyday objects.  For example, if he looks at the
string "bqessiyexbesszieb", Farmer John's  eyes ignore some of the letters and
all he sees is "bessiebessie".  

Given a string $s$, let $B(s)$ represent the maximum number of repeated copies
of  "bessie" one can form by deleting zero or more of the characters from $s$. 
In the example above, $B($"bqessiyexbesszieb"$) = 2$.  Furthermore, given a 
string $t$, let $A(t)$ represent the sum of $B(s)$ over all contiguous 
substrings $s$ of $t$.

Farmer John has a string $t$ of length at most $2\cdot 10^5$ consisting only of
characters a-z.  Please compute $A(t)$, and how $A(t)$ would change after $U$
($1\le U\le 2\cdot 10^5$) updates, each changing a character of $t$.  Updates
are cumulative.

INPUT FORMAT (input arrives from the terminal / stdin):
The first line of input contains $t$.

The next line contains $U$, followed by $U$ lines each containing a position $p$
($1\le p\le N$) and a character $c$ in the range a-z, meaning that the $p$th
character of $t$ is changed to $c$.

OUTPUT FORMAT (print output to the terminal / stdout):
Output $U+1$ lines, the total number of bessies that can be made across all
substrings of $t$ before any updates and after each update.

SAMPLE INPUT:
bessiebessie
3
3 l
7 s
3 s
SAMPLE OUTPUT: 
14
7
1
7

Before any updates, twelve substrings contain exactly 1 "bessie" and 1 string
contains exactly 2 "bessie"s, so the total number of bessies is
$12\cdot 1 + 1 \cdot 2 = 14$.

After one update, $t$ is "belsiebessie." Seven substrings contain exactly one
"bessie."

After two updates, $t$ is "belsiesessie." Only the entire string contains
"bessie."

SCORING:
Input 2: $|t|, U\le 300$Inputs 3-5: $U\le 10$Inputs 6-13: $|t|, U\le 10^5$Inputs 14-21: No additional constraints.


Problem credits: Brandon Wang and Benjamin Qi



\section*{Input Format}
The next line contains $U$, followed by $U$ lines each containing a position $p$
($1\le p\le N$) and a character $c$ in the range a-z, meaning that the $p$th
character of $t$ is changed to $c$.

OUTPUT FORMAT (print output to the terminal / stdout):Output $U+1$ lines, the total number of bessies that can be made across all
substrings of $t$ before any updates and after each update.SAMPLE INPUT:bessiebessie
3
3 l
7 s
3 sSAMPLE OUTPUT:14
7
1
7Before any updates, twelve substrings contain exactly 1 "bessie" and 1 string
contains exactly 2 "bessie"s, so the total number of bessies is
$12\cdot 1 + 1 \cdot 2 = 14$.After one update, $t$ is "belsiebessie." Seven substrings contain exactly one
"bessie."After two updates, $t$ is "belsiesessie." Only the entire string contains
"bessie."SCORING:Input 2: $|t|, U\le 300$Inputs 3-5: $U\le 10$Inputs 6-13: $|t|, U\le 10^5$Inputs 14-21: No additional constraints.Problem credits: Brandon Wang and Benjamin Qi

\section*{Output Format}
SAMPLE INPUT:bessiebessie
3
3 l
7 s
3 sSAMPLE OUTPUT:14
7
1
7Before any updates, twelve substrings contain exactly 1 "bessie" and 1 string
contains exactly 2 "bessie"s, so the total number of bessies is
$12\cdot 1 + 1 \cdot 2 = 14$.After one update, $t$ is "belsiebessie." Seven substrings contain exactly one
"bessie."After two updates, $t$ is "belsiesessie." Only the entire string contains
"bessie."SCORING:Input 2: $|t|, U\le 300$Inputs 3-5: $U\le 10$Inputs 6-13: $|t|, U\le 10^5$Inputs 14-21: No additional constraints.Problem credits: Brandon Wang and Benjamin Qi

\section*{Sample Input}
\begin{verbatim}
bessiebessie
3
3 l
7 s
3 s
\end{verbatim}

\section*{Sample Output}
\begin{verbatim}
14
7
1
7

Before any updates, twelve substrings contain exactly 1 "bessie" and 1 string
contains exactly 2 "bessie"s, so the total number of bessies is
$12\cdot 1 + 1 \cdot 2 = 14$.After one update, $t$ is "belsiebessie." Seven substrings contain exactly one
"bessie."After two updates, $t$ is "belsiesessie." Only the entire string contains
"bessie."SCORING:Input 2: $|t|, U\le 300$Inputs 3-5: $U\le 10$Inputs 6-13: $|t|, U\le 10^5$Inputs 14-21: No additional constraints.Problem credits: Brandon Wang and Benjamin Qi

After one update, $t$ is "belsiebessie." Seven substrings contain exactly one
"bessie."After two updates, $t$ is "belsiesessie." Only the entire string contains
"bessie."SCORING:Input 2: $|t|, U\le 300$Inputs 3-5: $U\le 10$Inputs 6-13: $|t|, U\le 10^5$Inputs 14-21: No additional constraints.Problem credits: Brandon Wang and Benjamin Qi

After two updates, $t$ is "belsiesessie." Only the entire string contains
"bessie."SCORING:Input 2: $|t|, U\le 300$Inputs 3-5: $U\le 10$Inputs 6-13: $|t|, U\le 10^5$Inputs 14-21: No additional constraints.Problem credits: Brandon Wang and Benjamin Qi

SCORING:Input 2: $|t|, U\le 300$Inputs 3-5: $U\le 10$Inputs 6-13: $|t|, U\le 10^5$Inputs 14-21: No additional constraints.Problem credits: Brandon Wang and Benjamin Qi
\end{verbatim}

\section*{Solution}


\section*{Problem URL}
https://usaco.org/index.php?page=viewproblem2&cpid=1332

\section*{Source}
USACO OPEN23 Contest, Platinum Division, Problem ID: 1332

\end{document}
