\documentclass[12pt]{article}
\usepackage{amsmath}
\usepackage{amssymb}
\usepackage{graphicx}
\usepackage{enumitem}
\usepackage{hyperref}
\usepackage{xcolor}

\title{View problem}
\author{USACO Contest: OPEN15 Contest - Platinum Division}
\date{\today}

\begin{document}

\maketitle

\section*{Problem Statement}

Farmer John has received a shipment of N large hay bales ($1 \le N \le 100,000$),
and placed them at various locations along the road leading to his barn.
Unfortunately, he completely forgets that Bessie the cow is out grazing along
the road, and she may now be trapped within the bales!
Each bale $j$ has a size $S_j$ and a position $P_j$ giving its location along
the one-dimensional road.  Bessie the cow can move around freely along the road,
even up to the position at which a bale is located, but she cannot cross through
this position.  As an exception, if she runs in the same direction for $D$ units
of distance, she builds up enough speed to break through and permanently
eliminate any hay bale of size strictly less than $D$.  Of course, after
doing this, she might open up more space to allow her to make a run at other hay
bales, eliminating them as well.  
Bessie can escape to freedom if she can eventually break through either the 
leftmost or rightmost hay bale.  Please compute the total area of the road
consisting of real-valued starting positions from which Bessie cannot escape.
INPUT FORMAT (file trapped.in):
The first line of input contains $N$.  Each of the next $N$ lines describes a
bale, and contains two integers giving its size and position, each in the range
$1\ldots 10^9$. All positions are distinct.

OUTPUT FORMAT (file trapped.out):
Print a single integer, giving the area of the road from which Bessie cannot
escape.

SAMPLE INPUT:5
8 1
1 4
8 8
7 15
4 20
SAMPLE OUTPUT: 14

[Problem credits: Brian Dean, 2015]



\section*{Input Format}
OUTPUT FORMAT (file trapped.out):Print a single integer, giving the area of the road from which Bessie cannot
escape.SAMPLE INPUT:5
8 1
1 4
8 8
7 15
4 20SAMPLE OUTPUT:14[Problem credits: Brian Dean, 2015]

\section*{Output Format}
SAMPLE INPUT:5
8 1
1 4
8 8
7 15
4 20SAMPLE OUTPUT:14[Problem credits: Brian Dean, 2015]

\section*{Sample Input}
\begin{verbatim}
5
8 1
1 4
8 8
7 15
4 20
\end{verbatim}

\section*{Sample Output}
\begin{verbatim}
14

[Problem credits: Brian Dean, 2015]
\end{verbatim}

\section*{Solution}


\section*{Problem URL}
https://usaco.org/index.php?page=viewproblem2&cpid=554

\section*{Source}
USACO OPEN15 Contest, Platinum Division, Problem ID: 554

\end{document}
