\documentclass[12pt]{article}
\usepackage{amsmath}
\usepackage{amssymb}
\usepackage{graphicx}
\usepackage{enumitem}
\usepackage{hyperref}
\usepackage{xcolor}

\title{View problem}
\author{USACO Contest: FEB15 Contest - Platinum Division}
\date{\today}

\begin{document}

\maketitle

\section*{Problem Statement}

Farmer John needs your help deciding where to build a fence in the shape of a straight line to help restrict the movement of his cows.  He has considered several possible fence locations and needs your help to determine which of these are usable, where a fence is considered usable if all of the cows are on the same side of the fence.  A fence is not usable if there is a cow that lies directly on it.  FJ will be asking you a number of queries regarding possible fence locations; a query should be answered "YES" if it corresponds to a usable fence location, "NO" otherwise.

Additionally, FJ may occasionally bring new cows into the herd.  When a new cow joins the herd, all fence queries from that point onward will require her to be on the same side of a fence as the rest of the herd for the fence to be usable.
INPUT FORMAT: (file fencing.in)
The first line of input contains N (1 <= N <= 100,000) and Q (1 <= Q <= 100,000) separated by a space.  These give the number of cows initially in the herd and the number of queries, respectively.

The following N lines describe the initial state of the herd. Each line will contain two space separated integers x and y giving the position of a cow.

The remaining Q lines contain queries either adding a new cow to the herd or testing a fence for usability.  A line of the form "1 x y" means that a new cow has been added to the herd at position (x, y).  A line of the form "2 A B C" indicates that FJ would like to test a fence described by the line Ax + By = C.

All cow positions will be unique over the whole data set and will satisfy (-10^9 <= x, y <= 10^9).  Additionally the fence queries will satisfy -10^9 <= A, B <= 10^9 and -10^18 <= C <= 10^18.  No fence query will have A = B = 0.

OUTPUT FORMAT: (file fencing.out)
For each fence query, output "YES" if the fence is usable.  Otherwise output "NO".

SAMPLE INPUT:3 4
0 0
0 1
1 0
2 2 2 3
1 1 1
2 2 2 3
2 0 1 1
SAMPLE OUTPUT:YES
NO
NO


The line 2x + 2y = 3 leaves the initial 3 cows on the same side.  However the cow (1, 1) is on the other side of this fence making it no longer usable after she joins the herd.  The line Y = 1 cannot be used because the cows (0, 1) and (1, 1) lie directly on it.

Warning: The I/O for this problem is fairly large.  C++ users may consider using scanf or the line "ios_base::sync_with_stdio(false)" to read input faster.  Java users should avoid using java.util.Scanner.  Do not flush output (e.g. using std::endl) after each query.
[Problem credits: Richard Peng, 2015]



\section*{Input Format}
The first line of input contains N (1 <= N <= 100,000) and Q (1 <= Q <= 100,000) separated by a space.  These give the number of cows initially in the herd and the number of queries, respectively.

The following N lines describe the initial state of the herd. Each line will contain two space separated integers x and y giving the position of a cow.

The remaining Q lines contain queries either adding a new cow to the herd or testing a fence for usability.  A line of the form "1 x y" means that a new cow has been added to the herd at position (x, y).  A line of the form "2 A B C" indicates that FJ would like to test a fence described by the line Ax + By = C.

All cow positions will be unique over the whole data set and will satisfy (-10^9 <= x, y <= 10^9).  Additionally the fence queries will satisfy -10^9 <= A, B <= 10^9 and -10^18 <= C <= 10^18.  No fence query will have A = B = 0.

\section*{Output Format}
For each fence query, output "YES" if the fence is usable.  Otherwise output "NO".

\section*{Sample Input}
\begin{verbatim}
3 4
0 0
0 1
1 0
2 2 2 3
1 1 1
2 2 2 3
2 0 1 1
\end{verbatim}

\section*{Sample Output}
\begin{verbatim}
YES
NO
NO

The line 2x + 2y = 3 leaves the initial 3 cows on the same side.  However the cow (1, 1) is on the other side of this fence making it no longer usable after she joins the herd.  The line Y = 1 cannot be used because the cows (0, 1) and (1, 1) lie directly on it.

Warning: The I/O for this problem is fairly large.  C++ users may consider using scanf or the line "ios_base::sync_with_stdio(false)" to read input faster.  Java users should avoid using java.util.Scanner.  Do not flush output (e.g. using std::endl) after each query.

[Problem credits: Richard Peng, 2015]
\end{verbatim}

\section*{Solution}


\section*{Problem URL}
https://usaco.org/index.php?page=viewproblem2&cpid=534

\section*{Source}
USACO FEB15 Contest, Platinum Division, Problem ID: 534

\end{document}
