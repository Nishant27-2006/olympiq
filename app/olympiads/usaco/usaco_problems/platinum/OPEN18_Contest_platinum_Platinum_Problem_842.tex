\documentclass[12pt]{article}
\usepackage{amsmath}
\usepackage{amssymb}
\usepackage{graphicx}
\usepackage{enumitem}
\usepackage{hyperref}
\usepackage{xcolor}

\title{View problem}
\author{USACO Contest: OPEN18 Contest - Platinum Division}
\date{\today}

\begin{document}

\maketitle

\section*{Problem Statement}

Farmer John prides himself on running a well-connected farm.  The farm is a 
collection of $N$ pastures ($2 \leq N \leq 50,000$), pairs of which are 
connected with $N-1$ bi-directional pathways, each having unit length. Farmer
John notices that using an appropriate series of these pathways,  it is possible
to travel from any pasture to any other pasture.

Although FJ's farm is connected, he worries what might happen if one of  the
pathways gets blocked, as this would effectively partition his farm into two
disjoint sets of pastures, where the cows could travel within each set but not
between the sets.  FJ therefore builds a set of $M$ additional  bi-directional
pathways ($1 \leq M \leq 50,000$), each with a positive integer length at most
$10^9$.  The cows still only use the original pathways  for transit, unless one
of these becomes blocked.  

If one of the original pathways becomes blocked, the farm becomes partitioned
into two disjoint pieces, and FJ will select from among his extra pathways a 
single replacement pathway that re-establishes connectivity between these two
pieces, so the cows can once again travel from any pasture to any other pasture.

For each of the original pathways on the farm, help FJ select the shortest
suitable replacement pathway.

INPUT FORMAT (file disrupt.in):
The first line of input contains $N$ and $M$.  Each of the next $N-1$ lines
describes an original pathway using integers $p$, $q$, where $p \neq q$ are the
pastures connected by the pathway (in the range $1 \ldots N$).  The remaining
$M$ lines each describe an extra pathway in terms of three integers: $p$, $q$,
and $r$, where $r$ is the length of the pathway. At most one pathway runs
between any pair of pastures.

OUTPUT FORMAT (file disrupt.out):
For each of the $N-1$ original pathways in the order they appear in the input,
output the length of the shortest suitable replacement pathway which would 
re-connect the farm if that original pathway were to be blocked.  If no suitable
replacement exists, output -1.

SAMPLE INPUT:
6 3
1 2
1 3
4 1
4 5
6 5
2 3 7
3 6 8
6 4 5
SAMPLE OUTPUT: 
7
7
8
5
5


Problem credits: Brian Dean



\section*{Input Format}
OUTPUT FORMAT (file disrupt.out):For each of the $N-1$ original pathways in the order they appear in the input,
output the length of the shortest suitable replacement pathway which would 
re-connect the farm if that original pathway were to be blocked.  If no suitable
replacement exists, output -1.SAMPLE INPUT:6 3
1 2
1 3
4 1
4 5
6 5
2 3 7
3 6 8
6 4 5SAMPLE OUTPUT:7
7
8
5
5Problem credits: Brian Dean

\section*{Output Format}
SAMPLE INPUT:6 3
1 2
1 3
4 1
4 5
6 5
2 3 7
3 6 8
6 4 5SAMPLE OUTPUT:7
7
8
5
5Problem credits: Brian Dean

\section*{Sample Input}
\begin{verbatim}
6 3
1 2
1 3
4 1
4 5
6 5
2 3 7
3 6 8
6 4 5
\end{verbatim}

\section*{Sample Output}
\begin{verbatim}
7
7
8
5
5

Problem credits: Brian Dean

Problem credits: Brian Dean
\end{verbatim}

\section*{Solution}


\section*{Problem URL}
https://usaco.org/index.php?page=viewproblem2&cpid=842

\section*{Source}
USACO OPEN18 Contest, Platinum Division, Problem ID: 842

\end{document}
