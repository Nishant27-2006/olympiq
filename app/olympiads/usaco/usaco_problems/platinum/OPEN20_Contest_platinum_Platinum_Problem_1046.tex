\documentclass[12pt]{article}
\usepackage{amsmath}
\usepackage{amssymb}
\usepackage{graphicx}
\usepackage{enumitem}
\usepackage{hyperref}
\usepackage{xcolor}

\title{View problem}
\author{USACO Contest: OPEN20 Contest - Platinum Division}
\date{\today}

\begin{document}

\maketitle

\section*{Problem Statement}

The $N$ cows of Farmer John's Circus ($1 \leq N \leq 10^5$) are preparing their
upcoming acts. The acts all take place on a tree with vertices labeled
$1\ldots N$. The "starting state" of an act is defined by a number
$1 \leq K \leq N$ and an assignment of cows $1\dots K$ to the vertices of the
tree, so that no two cows are located at the same vertex.

In an act, the cows make an arbitrarily large number of "moves." In a move, a
single cow moves from her current vertex to an unoccupied adjacent vertex. Two
starting states are said to be equivalent if one may be reached from the other
by some sequence of moves.

For each $1 \leq K \leq N$, help the cows determine the number of equivalence
classes of starting states: that is, the maximum number of starting states they
can pick such that no two are equivalent. Since these numbers may be very large,
output their remainders modulo $10^9 + 7$.

INPUT FORMAT (file circus.in):
Line $1$ contains $N$.

Lines $2\le i\le N$ each contain two integers $a_i$ and $b_i$ denoting an edge
between $a_i$ and $b_i$ in the tree.

OUTPUT FORMAT (file circus.out):
For each $1\le i\le N,$ the $i$-th line of output should contain the answer for
$K=i$ modulo $10^9+7$.

SAMPLE INPUT:
5
1 2
2 3
3 4
3 5
SAMPLE OUTPUT: 
1
1
3
24
120

For $K=1$ and $K=2,$ any two states can be transformed into one another.

Now consider $K=3$, and let $c_i$ denote the location of cow $i$. The state
$(c_1,c_2,c_3)=(1,2,3)$ is equivalent to the states $(1,2,5)$ and $(1,3,2).$
However, it is not equivalent to the state $(2,1,3).$

SAMPLE INPUT:
8
1 3
2 3
3 4
4 5
5 6
6 7
6 8
SAMPLE OUTPUT: 
1
1
1
6
30
180
5040
40320

SCORING:
Test cases 3-4 satisfy $N\le 8.$Test cases 5-7 satisfy $N\le 16.$Test cases 8-10 satisfy $N\le 100$ and the tree forms a "star;" at most one
vertex has degree greater than two.Test cases 11-15 satisfy $N\le 100$.Test cases 16-20 satisfy no additional constraints.


Problem credits: Dhruv Rohatgi



\section*{Input Format}
Lines $2\le i\le N$ each contain two integers $a_i$ and $b_i$ denoting an edge
between $a_i$ and $b_i$ in the tree.

OUTPUT FORMAT (file circus.out):For each $1\le i\le N,$ the $i$-th line of output should contain the answer for
$K=i$ modulo $10^9+7$.SAMPLE INPUT:5
1 2
2 3
3 4
3 5SAMPLE OUTPUT:1
1
3
24
120For $K=1$ and $K=2,$ any two states can be transformed into one another.Now consider $K=3$, and let $c_i$ denote the location of cow $i$. The state
$(c_1,c_2,c_3)=(1,2,3)$ is equivalent to the states $(1,2,5)$ and $(1,3,2).$
However, it is not equivalent to the state $(2,1,3).$SAMPLE INPUT:8
1 3
2 3
3 4
4 5
5 6
6 7
6 8SAMPLE OUTPUT:1
1
1
6
30
180
5040
40320SCORING:Test cases 3-4 satisfy $N\le 8.$Test cases 5-7 satisfy $N\le 16.$Test cases 8-10 satisfy $N\le 100$ and the tree forms a "star;" at most one
vertex has degree greater than two.Test cases 11-15 satisfy $N\le 100$.Test cases 16-20 satisfy no additional constraints.Problem credits: Dhruv Rohatgi

\section*{Output Format}
SAMPLE INPUT:5
1 2
2 3
3 4
3 5SAMPLE OUTPUT:1
1
3
24
120For $K=1$ and $K=2,$ any two states can be transformed into one another.Now consider $K=3$, and let $c_i$ denote the location of cow $i$. The state
$(c_1,c_2,c_3)=(1,2,3)$ is equivalent to the states $(1,2,5)$ and $(1,3,2).$
However, it is not equivalent to the state $(2,1,3).$SAMPLE INPUT:8
1 3
2 3
3 4
4 5
5 6
6 7
6 8SAMPLE OUTPUT:1
1
1
6
30
180
5040
40320SCORING:Test cases 3-4 satisfy $N\le 8.$Test cases 5-7 satisfy $N\le 16.$Test cases 8-10 satisfy $N\le 100$ and the tree forms a "star;" at most one
vertex has degree greater than two.Test cases 11-15 satisfy $N\le 100$.Test cases 16-20 satisfy no additional constraints.Problem credits: Dhruv Rohatgi

\section*{Sample Input}
\begin{verbatim}
5
1 2
2 3
3 4
3 5

8
1 3
2 3
3 4
4 5
5 6
6 7
6 8
\end{verbatim}

\section*{Sample Output}
\begin{verbatim}
1
1
3
24
120

Now consider $K=3$, and let $c_i$ denote the location of cow $i$. The state
$(c_1,c_2,c_3)=(1,2,3)$ is equivalent to the states $(1,2,5)$ and $(1,3,2).$
However, it is not equivalent to the state $(2,1,3).$SAMPLE INPUT:8
1 3
2 3
3 4
4 5
5 6
6 7
6 8SAMPLE OUTPUT:1
1
1
6
30
180
5040
40320SCORING:Test cases 3-4 satisfy $N\le 8.$Test cases 5-7 satisfy $N\le 16.$Test cases 8-10 satisfy $N\le 100$ and the tree forms a "star;" at most one
vertex has degree greater than two.Test cases 11-15 satisfy $N\le 100$.Test cases 16-20 satisfy no additional constraints.Problem credits: Dhruv Rohatgi

SAMPLE INPUT:8
1 3
2 3
3 4
4 5
5 6
6 7
6 8SAMPLE OUTPUT:1
1
1
6
30
180
5040
40320SCORING:Test cases 3-4 satisfy $N\le 8.$Test cases 5-7 satisfy $N\le 16.$Test cases 8-10 satisfy $N\le 100$ and the tree forms a "star;" at most one
vertex has degree greater than two.Test cases 11-15 satisfy $N\le 100$.Test cases 16-20 satisfy no additional constraints.Problem credits: Dhruv Rohatgi

1
1
1
6
30
180
5040
40320

SCORING:Test cases 3-4 satisfy $N\le 8.$Test cases 5-7 satisfy $N\le 16.$Test cases 8-10 satisfy $N\le 100$ and the tree forms a "star;" at most one
vertex has degree greater than two.Test cases 11-15 satisfy $N\le 100$.Test cases 16-20 satisfy no additional constraints.Problem credits: Dhruv Rohatgi
\end{verbatim}

\section*{Solution}


\section*{Problem URL}
https://usaco.org/index.php?page=viewproblem2&cpid=1046

\section*{Source}
USACO OPEN20 Contest, Platinum Division, Problem ID: 1046

\end{document}
