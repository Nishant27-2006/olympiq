\documentclass[12pt]{article}
\usepackage{amsmath}
\usepackage{amssymb}
\usepackage{graphicx}
\usepackage{enumitem}
\usepackage{hyperref}
\usepackage{xcolor}

\title{View problem}
\author{USACO Contest: JAN20 Contest - Platinum Division}
\date{\today}

\begin{document}

\maketitle

\section*{Problem Statement}

Bessie has become an artist and is creating paintings of caves! Her current work
in progress is a grid of height $N$ such that each row of the grid contains
exactly $M$ squares ($1\le N,M\le 1000$). Each square is either empty, filled
with rock, or filled with water. Bessie has already painted the squares
containing rock, including the entire border of the painting. She now wants to
fill some empty squares with water such that if the painting were real, there
would be no net motion of water. Define the height of a square in the $i$-th row
from the top to be $N+1-i$. Bessie wants her painting to satisfy the following
constraint: 

Suppose that square $a$ is filled with water. Then if there exists a path from
$a$ to square $b$ using only empty or water squares that are not higher than $a$
such that every two adjacent squares on the path share a side, then $b$ is also
filled with water. 

Find the number of different paintings Bessie can make modulo $10^9+7$.  Bessie
may fill any number of empty squares with water, including none or all. 

SCORING:
Test cases 1-5 satisfy $N,M\le 10.$Test cases 6-15 satisfy no
additional constraints.

INPUT FORMAT (file cave.in):
The first line contains two space-separated integers $N$ and $M$.

The next $N$ lines of input each contain $M$ characters. Each character is
either '.' or '#', representing an empty square and a square filled with rock,
respectively. The first and last row and first and last column only contain '#'.

OUTPUT FORMAT (file cave.out):
A single integer: the number of paintings satisfying the constraint modulo
$10^9+7$.

SAMPLE INPUT:
4 9
#########
#...#...#
#.#...#.#
#########
SAMPLE OUTPUT: 
9

If a square in the second row is filled with water, then all empty squares must
be filled with water. Otherwise, assume that no such squares are filled with
water. Then Bessie can choose to fill any subset of the three horizontally
contiguous regions of empty squares in the third row. Thus, the number of
paintings is equal to $1+2^3=9.$


Problem credits: Daniel Zhang



\section*{Input Format}
The next $N$ lines of input each contain $M$ characters. Each character is
either '.' or '#', representing an empty square and a square filled with rock,
respectively. The first and last row and first and last column only contain '#'.

OUTPUT FORMAT (file cave.out):A single integer: the number of paintings satisfying the constraint modulo
$10^9+7$.SAMPLE INPUT:4 9
#########
#...#...#
#.#...#.#
#########SAMPLE OUTPUT:9If a square in the second row is filled with water, then all empty squares must
be filled with water. Otherwise, assume that no such squares are filled with
water. Then Bessie can choose to fill any subset of the three horizontally
contiguous regions of empty squares in the third row. Thus, the number of
paintings is equal to $1+2^3=9.$Problem credits: Daniel Zhang

\section*{Output Format}
SAMPLE INPUT:4 9
#########
#...#...#
#.#...#.#
#########SAMPLE OUTPUT:9If a square in the second row is filled with water, then all empty squares must
be filled with water. Otherwise, assume that no such squares are filled with
water. Then Bessie can choose to fill any subset of the three horizontally
contiguous regions of empty squares in the third row. Thus, the number of
paintings is equal to $1+2^3=9.$Problem credits: Daniel Zhang

\section*{Sample Input}
\begin{verbatim}
4 9
#########
#...#...#
#.#...#.#
#########
\end{verbatim}

\section*{Sample Output}
\begin{verbatim}
9

If a square in the second row is filled with water, then all empty squares must
be filled with water. Otherwise, assume that no such squares are filled with
water. Then Bessie can choose to fill any subset of the three horizontally
contiguous regions of empty squares in the third row. Thus, the number of
paintings is equal to $1+2^3=9.$Problem credits: Daniel Zhang

Problem credits: Daniel Zhang

Problem credits: Daniel Zhang
\end{verbatim}

\section*{Solution}


\section*{Problem URL}
https://usaco.org/index.php?page=viewproblem2&cpid=996

\section*{Source}
USACO JAN20 Contest, Platinum Division, Problem ID: 996

\end{document}
