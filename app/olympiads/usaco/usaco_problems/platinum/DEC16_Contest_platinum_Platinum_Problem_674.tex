\documentclass[12pt]{article}
\usepackage{amsmath}
\usepackage{amssymb}
\usepackage{graphicx}
\usepackage{enumitem}
\usepackage{hyperref}
\usepackage{xcolor}

\title{View problem}
\author{USACO Contest: DEC16 Contest - Platinum Division}
\date{\today}

\begin{document}

\maketitle

\section*{Problem Statement}

Bessie is hoping to fool Farmer John by building a herd of $K$ realistic robotic
cows ($1 \leq K \leq 100,000$).

It turns out that building a robotic cow is somewhat complicated.
There are $N$ ($1 \leq n \leq 100,000$) individual locations on the
robot into which microcontrollers must be connected (so a single
microcontroller must be connected at each location).  For each of
these locations, Bessie can select from a number of different models
of microcontroller, each varying in cost.  

For the herd of robotic cows to look convincing to Farmer John, no two robots
should behave identically.  Therefore, no two robots should have exactly the
same set of microcontrollers. For any pair of robots, there should be at least
one location at which the two robots use a  different microcontroller model.  It
is guaranteed that there will always be enough different microcontroller models
to satisfy this constraint.

Bessie wants to make her robotic herd as cheaply as possible.  Help her
determine the  minimum possible cost to do this! 

INPUT FORMAT (file roboherd.in):
The first line of input contains $N$ and $K$ separated by a space.

The following $N$ lines contain a description of the different microcontroller
models available for each location.  The $i$th such line starts with $M_i$
($1 \leq M_i \leq 10$), giving the number of models available for location $i$. 
This is followed by $M_i$ space separated integers $P_{i,j}$ giving the costs of
these different models ($1 \le P_{i,j} \le 100,000,000$).

OUTPUT FORMAT (file roboherd.out):
Output a single line, giving the minimum cost to construct $K$ robots.

SAMPLE INPUT:
3 10
4 1 5 3 10
3 2 3 3
5 1 3 4 6 6
SAMPLE OUTPUT: 
61


Problem credits: Richard Peng and Nathan Pinsker



\section*{Input Format}
The following $N$ lines contain a description of the different microcontroller
models available for each location.  The $i$th such line starts with $M_i$
($1 \leq M_i \leq 10$), giving the number of models available for location $i$. 
This is followed by $M_i$ space separated integers $P_{i,j}$ giving the costs of
these different models ($1 \le P_{i,j} \le 100,000,000$).

OUTPUT FORMAT (file roboherd.out):Output a single line, giving the minimum cost to construct $K$ robots.SAMPLE INPUT:3 10
4 1 5 3 10
3 2 3 3
5 1 3 4 6 6SAMPLE OUTPUT:61Problem credits: Richard Peng and Nathan Pinsker

\section*{Output Format}
SAMPLE INPUT:3 10
4 1 5 3 10
3 2 3 3
5 1 3 4 6 6SAMPLE OUTPUT:61Problem credits: Richard Peng and Nathan Pinsker

\section*{Sample Input}
\begin{verbatim}
3 10
4 1 5 3 10
3 2 3 3
5 1 3 4 6 6
\end{verbatim}

\section*{Sample Output}
\begin{verbatim}
61

Problem credits: Richard Peng and Nathan Pinsker

Problem credits: Richard Peng and Nathan Pinsker
\end{verbatim}

\section*{Solution}


\section*{Problem URL}
https://usaco.org/index.php?page=viewproblem2&cpid=674

\section*{Source}
USACO DEC16 Contest, Platinum Division, Problem ID: 674

\end{document}
