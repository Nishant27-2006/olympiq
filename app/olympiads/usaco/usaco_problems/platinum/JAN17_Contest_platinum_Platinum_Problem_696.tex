\documentclass[12pt]{article}
\usepackage{amsmath}
\usepackage{amssymb}
\usepackage{graphicx}
\usepackage{enumitem}
\usepackage{hyperref}
\usepackage{xcolor}

\title{View problem}
\author{USACO Contest: JAN17 Contest - Platinum Division}
\date{\today}

\begin{document}

\maketitle

\section*{Problem Statement}

The cows have once again tried to form a startup company, failing to remember
from past experience that cows make terrible managers!

The cows, conveniently numbered $1 \ldots N$ ($1 \leq N \leq 100,000$),
organize the company as a tree, with cow 1 as the president (the root of the
tree).  Each cow except the president has a single manager (its "parent" in the
tree).  Each cow $i$ has a distinct  proficiency rating, $p(i)$, which describes
how good she is at her job.  If cow $i$ is an ancestor (e.g., a manager of a
manager of a manager) of cow $j$, then we say $j$ is a subordinate of $i$.

Unfortunately, the cows find that it is often the case that a manager has less
proficiency than several of her subordinates, in which case the manager should
consider promoting some of her subordinates.  Your task is to help the cows
figure out when this is happening.  For each cow $i$ in the company, please
count the number of subordinates $j$ where $p(j) > p(i)$.

INPUT FORMAT (file promote.in):
The first line of input contains $N$.

The next $N$ lines of input contain the proficiency ratings $p(1) \ldots p(N)$
for the cows.  Each is a distinct integer in the range $1 \ldots 1,000,000,000$.

The next $N-1$ lines describe the manager (parent) for cows $2 \ldots N$. 
Recall that cow 1 has no manager, being the president.

OUTPUT FORMAT (file promote.out):
Please print $N$ lines of output.  The $i$th line of output should tell the
number of subordinates of cow $i$ with higher proficiency than cow $i$.

SAMPLE INPUT:
5
804289384
846930887
681692778
714636916
957747794
1
1
2
3
SAMPLE OUTPUT: 
2
0
1
0
0


Problem credits: Karthik Nair



\section*{Input Format}
The next $N$ lines of input contain the proficiency ratings $p(1) \ldots p(N)$
for the cows.  Each is a distinct integer in the range $1 \ldots 1,000,000,000$.The next $N-1$ lines describe the manager (parent) for cows $2 \ldots N$. 
Recall that cow 1 has no manager, being the president.

The next $N-1$ lines describe the manager (parent) for cows $2 \ldots N$. 
Recall that cow 1 has no manager, being the president.

OUTPUT FORMAT (file promote.out):Please print $N$ lines of output.  The $i$th line of output should tell the
number of subordinates of cow $i$ with higher proficiency than cow $i$.SAMPLE INPUT:5
804289384
846930887
681692778
714636916
957747794
1
1
2
3SAMPLE OUTPUT:2
0
1
0
0Problem credits: Karthik Nair

\section*{Output Format}
SAMPLE INPUT:5
804289384
846930887
681692778
714636916
957747794
1
1
2
3SAMPLE OUTPUT:2
0
1
0
0Problem credits: Karthik Nair

\section*{Sample Input}
\begin{verbatim}
5
804289384
846930887
681692778
714636916
957747794
1
1
2
3
\end{verbatim}

\section*{Sample Output}
\begin{verbatim}
2
0
1
0
0

Problem credits: Karthik Nair

Problem credits: Karthik Nair
\end{verbatim}

\section*{Solution}


\section*{Problem URL}
https://usaco.org/index.php?page=viewproblem2&cpid=696

\section*{Source}
USACO JAN17 Contest, Platinum Division, Problem ID: 696

\end{document}
