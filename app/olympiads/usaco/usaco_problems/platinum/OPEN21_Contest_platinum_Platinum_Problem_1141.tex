\documentclass[12pt]{article}
\usepackage{amsmath}
\usepackage{amssymb}
\usepackage{graphicx}
\usepackage{enumitem}
\usepackage{hyperref}
\usepackage{xcolor}

\title{View problem}
\author{USACO Contest: OPEN21 Contest - Platinum Division}
\date{\today}

\begin{document}

\maketitle

\section*{Problem Statement}

Consider a network of $N$ ($2\le N\le 100$) nodes labeled $1\ldots N$. Each node
is designated as a sender, a receiver, or neither. The number of senders, $S$,
is equal to the number of receivers ($S\ge 1$). 

The connections between the nodes in this network can be described by a list of
directed edges each of the form $i\to j$, meaning that node $i$ may route to
node $j$. Interestingly, all of these edges satisfy the property that $i<j$,
aside from $K$ that satisfy $i>j$ ($0\le K\le 2$). There are no self-loops
(edges of the form $i\to i$).

The description of a "routing scheme" consists of a set of $S$ directed paths
from senders to receivers such that no two of these paths share an endpoint.
That is, the paths connect distinct senders to distinct receivers. A path from a
sender $s$ to a receiver $r$ can be described as a sequence of nodes
$$s=v_0\to v_1 \to v_2\to \cdots \to v_e=r$$
such that the directed edges $v_i\to v_{i+1}$ exist for all $0\le i<e$. A node
may appear more than once within the same path.

Count the number of distinct routing schemes such that every directed edge is
traversed exactly once. Since the answer may be very large, report it modulo
$10^9+7$. It is guaranteed that there is at least
one routing scheme satisfying these constraints.

Each input contains $T$ ($1\le T\le 20$) test cases that should be solved
independently. It is guaranteed that the sum of $N^2$ over all test cases does
not exceed $2\cdot 10^4$.

INPUT FORMAT (input arrives from the terminal / stdin):
The first line of the input contains $T$, the number of test cases.

The first line of each test case contains the integers $N$ and $K$. Note that
$S$  is not explicitly given within the input.

The second line of each test case contains a string of length $N$. The $i$-th
character of the string  is equal to S if the $i$-th node is a sender, R if the
$i$-th node is a receiver, or . if the $i$-th node is neither. The number of Rs
in this string is equal to the number of Ss, and there is at least one S.

The next $N$ lines of each test case each contain a bit string of $N$ zeros and
ones. The $j$-th bit of the $i$-th line is equal to $1$ if there exists a
directed edge from node $i$  to node $j$, and $0$ otherwise. As there are no
self-loops, the main diagonal of the matrix consists solely of zeros.
Furthermore, there are exactly $K$ ones below the main diagonal. 

Consecutive test cases are separated by newlines for readability.

OUTPUT FORMAT (print output to the terminal / stdout):
For each test case, the number of routing schemes such that every edge is
traversed exactly once, modulo $10^9+7$. It is guaranteed that there is at least
one valid routing scheme for each test case.

SAMPLE INPUT:
2

8 0
SS....RR
00100000
00100000
00011000
00000100
00000100
00000011
00000000
00000000

13 0
SSS.RRRSS.RR.
0001000000000
0001000000000
0001000000000
0000111000000
0000000000000
0000000000000
0000000000000
0000000001000
0000000001000
0000000000110
0000000000000
0000000000000
0000000000000
SAMPLE OUTPUT: 
4
12

For the first test case, the edges are
$1\to 3, 2\to 3, 3\to 4, 3\to 5, 4\to 6, 5\to 6, 6\to 7, 6\to 8$.

There are four possible routing schemes:

$1\to 3\to 4\to 6\to 7, 2\to 3\to 5\to 6\to 8$$1\to 3\to 5\to 6\to 7, 2\to 3\to 4\to 6\to 8$$1\to 3\to 4\to 6\to 8, 2\to 3\to 5\to 6\to 7$$1\to 3\to 5\to 6\to 8, 2\to 3\to 4\to 6\to 7$
For the second test case, the edges are
$1\to 4, 2\to 4, 3\to 4, 4\to 5,4\to 6,4\to 7, 8\to 10, 9\to 10, 10\to 11, 10\to 12$.

One possible routing scheme consists of the following paths:

$1\to 4\to 5$$2\to 4\to 7$$3\to 4\to 6$$8\to 10\to 12$$9\to 10\to 11$
In general, senders $\{1,2,3\}$ can route to some permutation of receivers
$\{5,6,7\}$ and senders $\{8,9\}$ can route to some permutation of receivers
$\{11,12\}$, giving an answer of $6\cdot 2=12$.

SAMPLE INPUT:
2

5 1
SS.RR
00101
00100
10010
00000
00000

6 2
S....R
001000
000100
010001
000010
001000
000000
SAMPLE OUTPUT: 
3
1

For the first test case, the edges are $1\to 3, 1\to 5, 2\to 3, 3\to 1, 3\to 4$.

There are three possible routing schemes:

$1\to 3\to 1\to 5$, $2\to 3\to 4$$1\to 3\to 4$,
$2\to 3\to 1\to 5$$1\to 5$, $2\to 3\to 1\to 3\to 4$
For the second test case, the edges are
$1\to 3, 2\to 4, 3\to 2,3\to 6, 4\to 5, 5\to 3$.

There is only one possible routing scheme:
$1\to 3\to 2\to 4\to 5\to 3\to 6$.

SAMPLE INPUT:
5

3 2
RS.
010
101
100

4 2
.R.S
0100
0010
1000
0100

4 2
.SR.
0000
0011
0100
0010

5 2
.SSRR
01000
10101
01010
00000
00000

6 2
SS..RR
001010
000010
000010
000010
100101
000000
SAMPLE OUTPUT: 
2
1
2
6
24

Some additional small test cases.

SCORING:
Test cases 4-5 satisfy $N\le 6$.Test cases 6-7 satisfy $K=0$.Test cases 8-12 satisfy $K=1$.Test cases 13-24 satisfy $K=2$.


Problem credits: Benjamin Qi



\section*{Input Format}
The first line of each test case contains the integers $N$ and $K$. Note that
$S$  is not explicitly given within the input.The second line of each test case contains a string of length $N$. The $i$-th
character of the string  is equal to S if the $i$-th node is a sender, R if the
$i$-th node is a receiver, or . if the $i$-th node is neither. The number of Rs
in this string is equal to the number of Ss, and there is at least one S.The next $N$ lines of each test case each contain a bit string of $N$ zeros and
ones. The $j$-th bit of the $i$-th line is equal to $1$ if there exists a
directed edge from node $i$  to node $j$, and $0$ otherwise. As there are no
self-loops, the main diagonal of the matrix consists solely of zeros.
Furthermore, there are exactly $K$ ones below the main diagonal.Consecutive test cases are separated by newlines for readability.

The second line of each test case contains a string of length $N$. The $i$-th
character of the string  is equal to S if the $i$-th node is a sender, R if the
$i$-th node is a receiver, or . if the $i$-th node is neither. The number of Rs
in this string is equal to the number of Ss, and there is at least one S.The next $N$ lines of each test case each contain a bit string of $N$ zeros and
ones. The $j$-th bit of the $i$-th line is equal to $1$ if there exists a
directed edge from node $i$  to node $j$, and $0$ otherwise. As there are no
self-loops, the main diagonal of the matrix consists solely of zeros.
Furthermore, there are exactly $K$ ones below the main diagonal.Consecutive test cases are separated by newlines for readability.

The next $N$ lines of each test case each contain a bit string of $N$ zeros and
ones. The $j$-th bit of the $i$-th line is equal to $1$ if there exists a
directed edge from node $i$  to node $j$, and $0$ otherwise. As there are no
self-loops, the main diagonal of the matrix consists solely of zeros.
Furthermore, there are exactly $K$ ones below the main diagonal.Consecutive test cases are separated by newlines for readability.

Consecutive test cases are separated by newlines for readability.

OUTPUT FORMAT (print output to the terminal / stdout):For each test case, the number of routing schemes such that every edge is
traversed exactly once, modulo $10^9+7$. It is guaranteed that there is at least
one valid routing scheme for each test case.SAMPLE INPUT:2

8 0
SS....RR
00100000
00100000
00011000
00000100
00000100
00000011
00000000
00000000

13 0
SSS.RRRSS.RR.
0001000000000
0001000000000
0001000000000
0000111000000
0000000000000
0000000000000
0000000000000
0000000001000
0000000001000
0000000000110
0000000000000
0000000000000
0000000000000SAMPLE OUTPUT:4
12For the first test case, the edges are
$1\to 3, 2\to 3, 3\to 4, 3\to 5, 4\to 6, 5\to 6, 6\to 7, 6\to 8$.There are four possible routing schemes:$1\to 3\to 4\to 6\to 7, 2\to 3\to 5\to 6\to 8$$1\to 3\to 5\to 6\to 7, 2\to 3\to 4\to 6\to 8$$1\to 3\to 4\to 6\to 8, 2\to 3\to 5\to 6\to 7$$1\to 3\to 5\to 6\to 8, 2\to 3\to 4\to 6\to 7$For the second test case, the edges are
$1\to 4, 2\to 4, 3\to 4, 4\to 5,4\to 6,4\to 7, 8\to 10, 9\to 10, 10\to 11, 10\to 12$.One possible routing scheme consists of the following paths:$1\to 4\to 5$$2\to 4\to 7$$3\to 4\to 6$$8\to 10\to 12$$9\to 10\to 11$In general, senders $\{1,2,3\}$ can route to some permutation of receivers
$\{5,6,7\}$ and senders $\{8,9\}$ can route to some permutation of receivers
$\{11,12\}$, giving an answer of $6\cdot 2=12$.SAMPLE INPUT:2

5 1
SS.RR
00101
00100
10010
00000
00000

6 2
S....R
001000
000100
010001
000010
001000
000000SAMPLE OUTPUT:3
1For the first test case, the edges are $1\to 3, 1\to 5, 2\to 3, 3\to 1, 3\to 4$.There are three possible routing schemes:$1\to 3\to 1\to 5$, $2\to 3\to 4$$1\to 3\to 4$,
$2\to 3\to 1\to 5$$1\to 5$, $2\to 3\to 1\to 3\to 4$For the second test case, the edges are
$1\to 3, 2\to 4, 3\to 2,3\to 6, 4\to 5, 5\to 3$.There is only one possible routing scheme:
$1\to 3\to 2\to 4\to 5\to 3\to 6$.SAMPLE INPUT:5

3 2
RS.
010
101
100

4 2
.R.S
0100
0010
1000
0100

4 2
.SR.
0000
0011
0100
0010

5 2
.SSRR
01000
10101
01010
00000
00000

6 2
SS..RR
001010
000010
000010
000010
100101
000000SAMPLE OUTPUT:2
1
2
6
24Some additional small test cases.SCORING:Test cases 4-5 satisfy $N\le 6$.Test cases 6-7 satisfy $K=0$.Test cases 8-12 satisfy $K=1$.Test cases 13-24 satisfy $K=2$.Problem credits: Benjamin Qi

\section*{Output Format}
SAMPLE INPUT:2

8 0
SS....RR
00100000
00100000
00011000
00000100
00000100
00000011
00000000
00000000

13 0
SSS.RRRSS.RR.
0001000000000
0001000000000
0001000000000
0000111000000
0000000000000
0000000000000
0000000000000
0000000001000
0000000001000
0000000000110
0000000000000
0000000000000
0000000000000SAMPLE OUTPUT:4
12For the first test case, the edges are
$1\to 3, 2\to 3, 3\to 4, 3\to 5, 4\to 6, 5\to 6, 6\to 7, 6\to 8$.There are four possible routing schemes:$1\to 3\to 4\to 6\to 7, 2\to 3\to 5\to 6\to 8$$1\to 3\to 5\to 6\to 7, 2\to 3\to 4\to 6\to 8$$1\to 3\to 4\to 6\to 8, 2\to 3\to 5\to 6\to 7$$1\to 3\to 5\to 6\to 8, 2\to 3\to 4\to 6\to 7$For the second test case, the edges are
$1\to 4, 2\to 4, 3\to 4, 4\to 5,4\to 6,4\to 7, 8\to 10, 9\to 10, 10\to 11, 10\to 12$.One possible routing scheme consists of the following paths:$1\to 4\to 5$$2\to 4\to 7$$3\to 4\to 6$$8\to 10\to 12$$9\to 10\to 11$In general, senders $\{1,2,3\}$ can route to some permutation of receivers
$\{5,6,7\}$ and senders $\{8,9\}$ can route to some permutation of receivers
$\{11,12\}$, giving an answer of $6\cdot 2=12$.SAMPLE INPUT:2

5 1
SS.RR
00101
00100
10010
00000
00000

6 2
S....R
001000
000100
010001
000010
001000
000000SAMPLE OUTPUT:3
1For the first test case, the edges are $1\to 3, 1\to 5, 2\to 3, 3\to 1, 3\to 4$.There are three possible routing schemes:$1\to 3\to 1\to 5$, $2\to 3\to 4$$1\to 3\to 4$,
$2\to 3\to 1\to 5$$1\to 5$, $2\to 3\to 1\to 3\to 4$For the second test case, the edges are
$1\to 3, 2\to 4, 3\to 2,3\to 6, 4\to 5, 5\to 3$.There is only one possible routing scheme:
$1\to 3\to 2\to 4\to 5\to 3\to 6$.SAMPLE INPUT:5

3 2
RS.
010
101
100

4 2
.R.S
0100
0010
1000
0100

4 2
.SR.
0000
0011
0100
0010

5 2
.SSRR
01000
10101
01010
00000
00000

6 2
SS..RR
001010
000010
000010
000010
100101
000000SAMPLE OUTPUT:2
1
2
6
24Some additional small test cases.SCORING:Test cases 4-5 satisfy $N\le 6$.Test cases 6-7 satisfy $K=0$.Test cases 8-12 satisfy $K=1$.Test cases 13-24 satisfy $K=2$.Problem credits: Benjamin Qi

\section*{Sample Input}
\begin{verbatim}
2

8 0
SS....RR
00100000
00100000
00011000
00000100
00000100
00000011
00000000
00000000

13 0
SSS.RRRSS.RR.
0001000000000
0001000000000
0001000000000
0000111000000
0000000000000
0000000000000
0000000000000
0000000001000
0000000001000
0000000000110
0000000000000
0000000000000
0000000000000

2

5 1
SS.RR
00101
00100
10010
00000
00000

6 2
S....R
001000
000100
010001
000010
001000
000000

5

3 2
RS.
010
101
100

4 2
.R.S
0100
0010
1000
0100

4 2
.SR.
0000
0011
0100
0010

5 2
.SSRR
01000
10101
01010
00000
00000

6 2
SS..RR
001010
000010
000010
000010
100101
000000
\end{verbatim}

\section*{Sample Output}
\begin{verbatim}
4
12

For the first test case, the edges are
$1\to 3, 2\to 3, 3\to 4, 3\to 5, 4\to 6, 5\to 6, 6\to 7, 6\to 8$.There are four possible routing schemes:$1\to 3\to 4\to 6\to 7, 2\to 3\to 5\to 6\to 8$$1\to 3\to 5\to 6\to 7, 2\to 3\to 4\to 6\to 8$$1\to 3\to 4\to 6\to 8, 2\to 3\to 5\to 6\to 7$$1\to 3\to 5\to 6\to 8, 2\to 3\to 4\to 6\to 7$For the second test case, the edges are
$1\to 4, 2\to 4, 3\to 4, 4\to 5,4\to 6,4\to 7, 8\to 10, 9\to 10, 10\to 11, 10\to 12$.One possible routing scheme consists of the following paths:$1\to 4\to 5$$2\to 4\to 7$$3\to 4\to 6$$8\to 10\to 12$$9\to 10\to 11$In general, senders $\{1,2,3\}$ can route to some permutation of receivers
$\{5,6,7\}$ and senders $\{8,9\}$ can route to some permutation of receivers
$\{11,12\}$, giving an answer of $6\cdot 2=12$.SAMPLE INPUT:2

5 1
SS.RR
00101
00100
10010
00000
00000

6 2
S....R
001000
000100
010001
000010
001000
000000SAMPLE OUTPUT:3
1For the first test case, the edges are $1\to 3, 1\to 5, 2\to 3, 3\to 1, 3\to 4$.There are three possible routing schemes:$1\to 3\to 1\to 5$, $2\to 3\to 4$$1\to 3\to 4$,
$2\to 3\to 1\to 5$$1\to 5$, $2\to 3\to 1\to 3\to 4$For the second test case, the edges are
$1\to 3, 2\to 4, 3\to 2,3\to 6, 4\to 5, 5\to 3$.There is only one possible routing scheme:
$1\to 3\to 2\to 4\to 5\to 3\to 6$.SAMPLE INPUT:5

3 2
RS.
010
101
100

4 2
.R.S
0100
0010
1000
0100

4 2
.SR.
0000
0011
0100
0010

5 2
.SSRR
01000
10101
01010
00000
00000

6 2
SS..RR
001010
000010
000010
000010
100101
000000SAMPLE OUTPUT:2
1
2
6
24Some additional small test cases.SCORING:Test cases 4-5 satisfy $N\le 6$.Test cases 6-7 satisfy $K=0$.Test cases 8-12 satisfy $K=1$.Test cases 13-24 satisfy $K=2$.Problem credits: Benjamin Qi

There are four possible routing schemes:$1\to 3\to 4\to 6\to 7, 2\to 3\to 5\to 6\to 8$$1\to 3\to 5\to 6\to 7, 2\to 3\to 4\to 6\to 8$$1\to 3\to 4\to 6\to 8, 2\to 3\to 5\to 6\to 7$$1\to 3\to 5\to 6\to 8, 2\to 3\to 4\to 6\to 7$For the second test case, the edges are
$1\to 4, 2\to 4, 3\to 4, 4\to 5,4\to 6,4\to 7, 8\to 10, 9\to 10, 10\to 11, 10\to 12$.One possible routing scheme consists of the following paths:$1\to 4\to 5$$2\to 4\to 7$$3\to 4\to 6$$8\to 10\to 12$$9\to 10\to 11$In general, senders $\{1,2,3\}$ can route to some permutation of receivers
$\{5,6,7\}$ and senders $\{8,9\}$ can route to some permutation of receivers
$\{11,12\}$, giving an answer of $6\cdot 2=12$.SAMPLE INPUT:2

5 1
SS.RR
00101
00100
10010
00000
00000

6 2
S....R
001000
000100
010001
000010
001000
000000SAMPLE OUTPUT:3
1For the first test case, the edges are $1\to 3, 1\to 5, 2\to 3, 3\to 1, 3\to 4$.There are three possible routing schemes:$1\to 3\to 1\to 5$, $2\to 3\to 4$$1\to 3\to 4$,
$2\to 3\to 1\to 5$$1\to 5$, $2\to 3\to 1\to 3\to 4$For the second test case, the edges are
$1\to 3, 2\to 4, 3\to 2,3\to 6, 4\to 5, 5\to 3$.There is only one possible routing scheme:
$1\to 3\to 2\to 4\to 5\to 3\to 6$.SAMPLE INPUT:5

3 2
RS.
010
101
100

4 2
.R.S
0100
0010
1000
0100

4 2
.SR.
0000
0011
0100
0010

5 2
.SSRR
01000
10101
01010
00000
00000

6 2
SS..RR
001010
000010
000010
000010
100101
000000SAMPLE OUTPUT:2
1
2
6
24Some additional small test cases.SCORING:Test cases 4-5 satisfy $N\le 6$.Test cases 6-7 satisfy $K=0$.Test cases 8-12 satisfy $K=1$.Test cases 13-24 satisfy $K=2$.Problem credits: Benjamin Qi

$1\to 3\to 4\to 6\to 7, 2\to 3\to 5\to 6\to 8$$1\to 3\to 5\to 6\to 7, 2\to 3\to 4\to 6\to 8$$1\to 3\to 4\to 6\to 8, 2\to 3\to 5\to 6\to 7$$1\to 3\to 5\to 6\to 8, 2\to 3\to 4\to 6\to 7$For the second test case, the edges are
$1\to 4, 2\to 4, 3\to 4, 4\to 5,4\to 6,4\to 7, 8\to 10, 9\to 10, 10\to 11, 10\to 12$.One possible routing scheme consists of the following paths:$1\to 4\to 5$$2\to 4\to 7$$3\to 4\to 6$$8\to 10\to 12$$9\to 10\to 11$In general, senders $\{1,2,3\}$ can route to some permutation of receivers
$\{5,6,7\}$ and senders $\{8,9\}$ can route to some permutation of receivers
$\{11,12\}$, giving an answer of $6\cdot 2=12$.SAMPLE INPUT:2

5 1
SS.RR
00101
00100
10010
00000
00000

6 2
S....R
001000
000100
010001
000010
001000
000000SAMPLE OUTPUT:3
1For the first test case, the edges are $1\to 3, 1\to 5, 2\to 3, 3\to 1, 3\to 4$.There are three possible routing schemes:$1\to 3\to 1\to 5$, $2\to 3\to 4$$1\to 3\to 4$,
$2\to 3\to 1\to 5$$1\to 5$, $2\to 3\to 1\to 3\to 4$For the second test case, the edges are
$1\to 3, 2\to 4, 3\to 2,3\to 6, 4\to 5, 5\to 3$.There is only one possible routing scheme:
$1\to 3\to 2\to 4\to 5\to 3\to 6$.SAMPLE INPUT:5

3 2
RS.
010
101
100

4 2
.R.S
0100
0010
1000
0100

4 2
.SR.
0000
0011
0100
0010

5 2
.SSRR
01000
10101
01010
00000
00000

6 2
SS..RR
001010
000010
000010
000010
100101
000000SAMPLE OUTPUT:2
1
2
6
24Some additional small test cases.SCORING:Test cases 4-5 satisfy $N\le 6$.Test cases 6-7 satisfy $K=0$.Test cases 8-12 satisfy $K=1$.Test cases 13-24 satisfy $K=2$.Problem credits: Benjamin Qi

For the second test case, the edges are
$1\to 4, 2\to 4, 3\to 4, 4\to 5,4\to 6,4\to 7, 8\to 10, 9\to 10, 10\to 11, 10\to 12$.One possible routing scheme consists of the following paths:$1\to 4\to 5$$2\to 4\to 7$$3\to 4\to 6$$8\to 10\to 12$$9\to 10\to 11$In general, senders $\{1,2,3\}$ can route to some permutation of receivers
$\{5,6,7\}$ and senders $\{8,9\}$ can route to some permutation of receivers
$\{11,12\}$, giving an answer of $6\cdot 2=12$.SAMPLE INPUT:2

5 1
SS.RR
00101
00100
10010
00000
00000

6 2
S....R
001000
000100
010001
000010
001000
000000SAMPLE OUTPUT:3
1For the first test case, the edges are $1\to 3, 1\to 5, 2\to 3, 3\to 1, 3\to 4$.There are three possible routing schemes:$1\to 3\to 1\to 5$, $2\to 3\to 4$$1\to 3\to 4$,
$2\to 3\to 1\to 5$$1\to 5$, $2\to 3\to 1\to 3\to 4$For the second test case, the edges are
$1\to 3, 2\to 4, 3\to 2,3\to 6, 4\to 5, 5\to 3$.There is only one possible routing scheme:
$1\to 3\to 2\to 4\to 5\to 3\to 6$.SAMPLE INPUT:5

3 2
RS.
010
101
100

4 2
.R.S
0100
0010
1000
0100

4 2
.SR.
0000
0011
0100
0010

5 2
.SSRR
01000
10101
01010
00000
00000

6 2
SS..RR
001010
000010
000010
000010
100101
000000SAMPLE OUTPUT:2
1
2
6
24Some additional small test cases.SCORING:Test cases 4-5 satisfy $N\le 6$.Test cases 6-7 satisfy $K=0$.Test cases 8-12 satisfy $K=1$.Test cases 13-24 satisfy $K=2$.Problem credits: Benjamin Qi

One possible routing scheme consists of the following paths:$1\to 4\to 5$$2\to 4\to 7$$3\to 4\to 6$$8\to 10\to 12$$9\to 10\to 11$In general, senders $\{1,2,3\}$ can route to some permutation of receivers
$\{5,6,7\}$ and senders $\{8,9\}$ can route to some permutation of receivers
$\{11,12\}$, giving an answer of $6\cdot 2=12$.SAMPLE INPUT:2

5 1
SS.RR
00101
00100
10010
00000
00000

6 2
S....R
001000
000100
010001
000010
001000
000000SAMPLE OUTPUT:3
1For the first test case, the edges are $1\to 3, 1\to 5, 2\to 3, 3\to 1, 3\to 4$.There are three possible routing schemes:$1\to 3\to 1\to 5$, $2\to 3\to 4$$1\to 3\to 4$,
$2\to 3\to 1\to 5$$1\to 5$, $2\to 3\to 1\to 3\to 4$For the second test case, the edges are
$1\to 3, 2\to 4, 3\to 2,3\to 6, 4\to 5, 5\to 3$.There is only one possible routing scheme:
$1\to 3\to 2\to 4\to 5\to 3\to 6$.SAMPLE INPUT:5

3 2
RS.
010
101
100

4 2
.R.S
0100
0010
1000
0100

4 2
.SR.
0000
0011
0100
0010

5 2
.SSRR
01000
10101
01010
00000
00000

6 2
SS..RR
001010
000010
000010
000010
100101
000000SAMPLE OUTPUT:2
1
2
6
24Some additional small test cases.SCORING:Test cases 4-5 satisfy $N\le 6$.Test cases 6-7 satisfy $K=0$.Test cases 8-12 satisfy $K=1$.Test cases 13-24 satisfy $K=2$.Problem credits: Benjamin Qi

$1\to 4\to 5$$2\to 4\to 7$$3\to 4\to 6$$8\to 10\to 12$$9\to 10\to 11$In general, senders $\{1,2,3\}$ can route to some permutation of receivers
$\{5,6,7\}$ and senders $\{8,9\}$ can route to some permutation of receivers
$\{11,12\}$, giving an answer of $6\cdot 2=12$.SAMPLE INPUT:2

5 1
SS.RR
00101
00100
10010
00000
00000

6 2
S....R
001000
000100
010001
000010
001000
000000SAMPLE OUTPUT:3
1For the first test case, the edges are $1\to 3, 1\to 5, 2\to 3, 3\to 1, 3\to 4$.There are three possible routing schemes:$1\to 3\to 1\to 5$, $2\to 3\to 4$$1\to 3\to 4$,
$2\to 3\to 1\to 5$$1\to 5$, $2\to 3\to 1\to 3\to 4$For the second test case, the edges are
$1\to 3, 2\to 4, 3\to 2,3\to 6, 4\to 5, 5\to 3$.There is only one possible routing scheme:
$1\to 3\to 2\to 4\to 5\to 3\to 6$.SAMPLE INPUT:5

3 2
RS.
010
101
100

4 2
.R.S
0100
0010
1000
0100

4 2
.SR.
0000
0011
0100
0010

5 2
.SSRR
01000
10101
01010
00000
00000

6 2
SS..RR
001010
000010
000010
000010
100101
000000SAMPLE OUTPUT:2
1
2
6
24Some additional small test cases.SCORING:Test cases 4-5 satisfy $N\le 6$.Test cases 6-7 satisfy $K=0$.Test cases 8-12 satisfy $K=1$.Test cases 13-24 satisfy $K=2$.Problem credits: Benjamin Qi

In general, senders $\{1,2,3\}$ can route to some permutation of receivers
$\{5,6,7\}$ and senders $\{8,9\}$ can route to some permutation of receivers
$\{11,12\}$, giving an answer of $6\cdot 2=12$.SAMPLE INPUT:2

5 1
SS.RR
00101
00100
10010
00000
00000

6 2
S....R
001000
000100
010001
000010
001000
000000SAMPLE OUTPUT:3
1For the first test case, the edges are $1\to 3, 1\to 5, 2\to 3, 3\to 1, 3\to 4$.There are three possible routing schemes:$1\to 3\to 1\to 5$, $2\to 3\to 4$$1\to 3\to 4$,
$2\to 3\to 1\to 5$$1\to 5$, $2\to 3\to 1\to 3\to 4$For the second test case, the edges are
$1\to 3, 2\to 4, 3\to 2,3\to 6, 4\to 5, 5\to 3$.There is only one possible routing scheme:
$1\to 3\to 2\to 4\to 5\to 3\to 6$.SAMPLE INPUT:5

3 2
RS.
010
101
100

4 2
.R.S
0100
0010
1000
0100

4 2
.SR.
0000
0011
0100
0010

5 2
.SSRR
01000
10101
01010
00000
00000

6 2
SS..RR
001010
000010
000010
000010
100101
000000SAMPLE OUTPUT:2
1
2
6
24Some additional small test cases.SCORING:Test cases 4-5 satisfy $N\le 6$.Test cases 6-7 satisfy $K=0$.Test cases 8-12 satisfy $K=1$.Test cases 13-24 satisfy $K=2$.Problem credits: Benjamin Qi

SAMPLE INPUT:2

5 1
SS.RR
00101
00100
10010
00000
00000

6 2
S....R
001000
000100
010001
000010
001000
000000SAMPLE OUTPUT:3
1For the first test case, the edges are $1\to 3, 1\to 5, 2\to 3, 3\to 1, 3\to 4$.There are three possible routing schemes:$1\to 3\to 1\to 5$, $2\to 3\to 4$$1\to 3\to 4$,
$2\to 3\to 1\to 5$$1\to 5$, $2\to 3\to 1\to 3\to 4$For the second test case, the edges are
$1\to 3, 2\to 4, 3\to 2,3\to 6, 4\to 5, 5\to 3$.There is only one possible routing scheme:
$1\to 3\to 2\to 4\to 5\to 3\to 6$.SAMPLE INPUT:5

3 2
RS.
010
101
100

4 2
.R.S
0100
0010
1000
0100

4 2
.SR.
0000
0011
0100
0010

5 2
.SSRR
01000
10101
01010
00000
00000

6 2
SS..RR
001010
000010
000010
000010
100101
000000SAMPLE OUTPUT:2
1
2
6
24Some additional small test cases.SCORING:Test cases 4-5 satisfy $N\le 6$.Test cases 6-7 satisfy $K=0$.Test cases 8-12 satisfy $K=1$.Test cases 13-24 satisfy $K=2$.Problem credits: Benjamin Qi

3
1

For the first test case, the edges are $1\to 3, 1\to 5, 2\to 3, 3\to 1, 3\to 4$.There are three possible routing schemes:$1\to 3\to 1\to 5$, $2\to 3\to 4$$1\to 3\to 4$,
$2\to 3\to 1\to 5$$1\to 5$, $2\to 3\to 1\to 3\to 4$For the second test case, the edges are
$1\to 3, 2\to 4, 3\to 2,3\to 6, 4\to 5, 5\to 3$.There is only one possible routing scheme:
$1\to 3\to 2\to 4\to 5\to 3\to 6$.SAMPLE INPUT:5

3 2
RS.
010
101
100

4 2
.R.S
0100
0010
1000
0100

4 2
.SR.
0000
0011
0100
0010

5 2
.SSRR
01000
10101
01010
00000
00000

6 2
SS..RR
001010
000010
000010
000010
100101
000000SAMPLE OUTPUT:2
1
2
6
24Some additional small test cases.SCORING:Test cases 4-5 satisfy $N\le 6$.Test cases 6-7 satisfy $K=0$.Test cases 8-12 satisfy $K=1$.Test cases 13-24 satisfy $K=2$.Problem credits: Benjamin Qi

There are three possible routing schemes:$1\to 3\to 1\to 5$, $2\to 3\to 4$$1\to 3\to 4$,
$2\to 3\to 1\to 5$$1\to 5$, $2\to 3\to 1\to 3\to 4$For the second test case, the edges are
$1\to 3, 2\to 4, 3\to 2,3\to 6, 4\to 5, 5\to 3$.There is only one possible routing scheme:
$1\to 3\to 2\to 4\to 5\to 3\to 6$.SAMPLE INPUT:5

3 2
RS.
010
101
100

4 2
.R.S
0100
0010
1000
0100

4 2
.SR.
0000
0011
0100
0010

5 2
.SSRR
01000
10101
01010
00000
00000

6 2
SS..RR
001010
000010
000010
000010
100101
000000SAMPLE OUTPUT:2
1
2
6
24Some additional small test cases.SCORING:Test cases 4-5 satisfy $N\le 6$.Test cases 6-7 satisfy $K=0$.Test cases 8-12 satisfy $K=1$.Test cases 13-24 satisfy $K=2$.Problem credits: Benjamin Qi

$1\to 3\to 1\to 5$, $2\to 3\to 4$$1\to 3\to 4$,
$2\to 3\to 1\to 5$$1\to 5$, $2\to 3\to 1\to 3\to 4$For the second test case, the edges are
$1\to 3, 2\to 4, 3\to 2,3\to 6, 4\to 5, 5\to 3$.There is only one possible routing scheme:
$1\to 3\to 2\to 4\to 5\to 3\to 6$.SAMPLE INPUT:5

3 2
RS.
010
101
100

4 2
.R.S
0100
0010
1000
0100

4 2
.SR.
0000
0011
0100
0010

5 2
.SSRR
01000
10101
01010
00000
00000

6 2
SS..RR
001010
000010
000010
000010
100101
000000SAMPLE OUTPUT:2
1
2
6
24Some additional small test cases.SCORING:Test cases 4-5 satisfy $N\le 6$.Test cases 6-7 satisfy $K=0$.Test cases 8-12 satisfy $K=1$.Test cases 13-24 satisfy $K=2$.Problem credits: Benjamin Qi

For the second test case, the edges are
$1\to 3, 2\to 4, 3\to 2,3\to 6, 4\to 5, 5\to 3$.There is only one possible routing scheme:
$1\to 3\to 2\to 4\to 5\to 3\to 6$.SAMPLE INPUT:5

3 2
RS.
010
101
100

4 2
.R.S
0100
0010
1000
0100

4 2
.SR.
0000
0011
0100
0010

5 2
.SSRR
01000
10101
01010
00000
00000

6 2
SS..RR
001010
000010
000010
000010
100101
000000SAMPLE OUTPUT:2
1
2
6
24Some additional small test cases.SCORING:Test cases 4-5 satisfy $N\le 6$.Test cases 6-7 satisfy $K=0$.Test cases 8-12 satisfy $K=1$.Test cases 13-24 satisfy $K=2$.Problem credits: Benjamin Qi

There is only one possible routing scheme:
$1\to 3\to 2\to 4\to 5\to 3\to 6$.SAMPLE INPUT:5

3 2
RS.
010
101
100

4 2
.R.S
0100
0010
1000
0100

4 2
.SR.
0000
0011
0100
0010

5 2
.SSRR
01000
10101
01010
00000
00000

6 2
SS..RR
001010
000010
000010
000010
100101
000000SAMPLE OUTPUT:2
1
2
6
24Some additional small test cases.SCORING:Test cases 4-5 satisfy $N\le 6$.Test cases 6-7 satisfy $K=0$.Test cases 8-12 satisfy $K=1$.Test cases 13-24 satisfy $K=2$.Problem credits: Benjamin Qi

SAMPLE INPUT:5

3 2
RS.
010
101
100

4 2
.R.S
0100
0010
1000
0100

4 2
.SR.
0000
0011
0100
0010

5 2
.SSRR
01000
10101
01010
00000
00000

6 2
SS..RR
001010
000010
000010
000010
100101
000000SAMPLE OUTPUT:2
1
2
6
24Some additional small test cases.SCORING:Test cases 4-5 satisfy $N\le 6$.Test cases 6-7 satisfy $K=0$.Test cases 8-12 satisfy $K=1$.Test cases 13-24 satisfy $K=2$.Problem credits: Benjamin Qi

2
1
2
6
24

Some additional small test cases.SCORING:Test cases 4-5 satisfy $N\le 6$.Test cases 6-7 satisfy $K=0$.Test cases 8-12 satisfy $K=1$.Test cases 13-24 satisfy $K=2$.Problem credits: Benjamin Qi

SCORING:Test cases 4-5 satisfy $N\le 6$.Test cases 6-7 satisfy $K=0$.Test cases 8-12 satisfy $K=1$.Test cases 13-24 satisfy $K=2$.Problem credits: Benjamin Qi
\end{verbatim}

\section*{Solution}


\section*{Problem URL}
https://usaco.org/index.php?page=viewproblem2&cpid=1141

\section*{Source}
USACO OPEN21 Contest, Platinum Division, Problem ID: 1141

\end{document}
