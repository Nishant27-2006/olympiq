\documentclass[12pt]{article}
\usepackage{amsmath}
\usepackage{amssymb}
\usepackage{graphicx}
\usepackage{enumitem}
\usepackage{hyperref}
\usepackage{xcolor}

\title{View problem}
\author{USACO Contest: DEC19 Contest - Platinum Division}
\date{\today}

\begin{document}

\maketitle

\section*{Problem Statement}

Snow has arrived on the farm, and as she does at the beginning of every winter,
Bessie is building a snow-cow! Most of the time, Bessie strives to make her
sculpture look as much like a real cow as possible.  However, feeling
artistically inspired, this year she decides to pursue a more abstract route and
build a sculpture in the shape of a tree, consisting of $N$ snowballs
$(1\le N\le 10^5)$ connected by  $N-1$ branches, each connecting a pair of
snowballs such that there is a  unique path between every pair of snowballs. 

Bessie has added a nose to one of the snowballs, so it represents the head of
the abstract snow cow.  She designates it as snowball number 1.  To add more
visual interest, she plans to dye some of the snowballs different colors in an
artistic fashion by filling old milk pails with colored dye and splashing them
onto the sculpture.  Colors are identified by integers in the range
$1 \ldots 10^5$, and  Bessie has an unlimited supply of buckets filled with dyes
of every possible color.

When Bessie splashes a snowball with a bucket of dye, all the snowballs in its 
subtree are also splashed with the same dye (snowball $y$ is in the subtree of
snowball $x$ if $x$ lies on the path from $y$ to the head snowball). By
splashing each color with great care, Bessie makes sure that all colors a 
snowball has been splashed with will remain visible. For example, if a snowball
had colors $[1,2,3]$ and Bessie splashes it with color $4$, the snowball will
then have colors $[1,2,3,4]$. 

After splashing the snowballs some number of times, Bessie may also want to know
how colorful a part of her snow-cow is.  The "colorfulness" of a snowball $x$ is
equal to the number of distinct colors $c$ such that snowball $x$ is colored
$c$. If Bessie asks you about snowball $x$, you should reply with the sum of the
colorfulness values of all snowballs in the subtree of $x.$

Please help Bessie find the colorfulness of her snow-cow at certain points in
time.

SCORING:
$Q$ is defined below.

Test cases 2-3 satisfy $N\le 10^2, Q\le 2\cdot 10^2.$ Test cases 4-6 satisfy $N\le 10^3, Q\le 2\cdot 10^3.$ 

INPUT FORMAT (file snowcow.in):
The first line contains $N,$ and the number of queries $Q$ ($1\le Q\le 10^5$). 

The next $N-1$ lines each contain two space-separated integers $a$ and $b,$
describing a branch connecting snowballs $a$ and $b$ ($1 \le a, b \le N$).

Finally, the last $Q$ lines each contain a query.  A query of the form

1 x c
indicates that Bessie splashed a bucket of juice of color $c$ on snowball $x,$
coloring all snowballs in the subtree of $x$. A line of the form 

2 x
is a query for the sum of the colorfulness values of all snowballs in the
subtree of $x$. Of course, $1\le x\le N$ and $1\le c\le 10^5.$ 
OUTPUT FORMAT (file snowcow.out):
For each query of type 2, print the sum of colorfulness values within 
the corresponding subtree. 
Note that you should use 64-bit integers to avoid overflow. 

SAMPLE INPUT:
5 18
1 2
1 3
3 4
3 5
1 4 1
2 1
2 2
2 3
2 4
2 5
1 5 1
2 1
2 2
2 3
2 4
2 5
1 1 1
2 1
2 2
2 3
2 4
2 5
SAMPLE OUTPUT: 
1
0
1
1
0
2
0
2
1
1
5
1
3
1
1

After the first query of type 1, snowball 4 is dyed with color 1.

After the second query of type 1, snowballs 4 and 5 are dyed with color 1.

After the third query of type 1, all snowballs are dyed with color 1.


Problem credits: Michael Cao and Benjamin Qi



\section*{Input Format}
The next $N-1$ lines each contain two space-separated integers $a$ and $b,$
describing a branch connecting snowballs $a$ and $b$ ($1 \le a, b \le N$).Finally, the last $Q$ lines each contain a query.  A query of the form1 x cindicates that Bessie splashed a bucket of juice of color $c$ on snowball $x,$
coloring all snowballs in the subtree of $x$. A line of the form2 xis a query for the sum of the colorfulness values of all snowballs in the
subtree of $x$. Of course, $1\le x\le N$ and $1\le c\le 10^5.$

Finally, the last $Q$ lines each contain a query.  A query of the form1 x cindicates that Bessie splashed a bucket of juice of color $c$ on snowball $x,$
coloring all snowballs in the subtree of $x$. A line of the form2 xis a query for the sum of the colorfulness values of all snowballs in the
subtree of $x$. Of course, $1\le x\le N$ and $1\le c\le 10^5.$

1 x cindicates that Bessie splashed a bucket of juice of color $c$ on snowball $x,$
coloring all snowballs in the subtree of $x$. A line of the form2 xis a query for the sum of the colorfulness values of all snowballs in the
subtree of $x$. Of course, $1\le x\le N$ and $1\le c\le 10^5.$

1 x c

indicates that Bessie splashed a bucket of juice of color $c$ on snowball $x,$
coloring all snowballs in the subtree of $x$. A line of the form2 xis a query for the sum of the colorfulness values of all snowballs in the
subtree of $x$. Of course, $1\le x\le N$ and $1\le c\le 10^5.$

2 xis a query for the sum of the colorfulness values of all snowballs in the
subtree of $x$. Of course, $1\le x\le N$ and $1\le c\le 10^5.$

2 x

is a query for the sum of the colorfulness values of all snowballs in the
subtree of $x$. Of course, $1\le x\le N$ and $1\le c\le 10^5.$

OUTPUT FORMAT (file snowcow.out):For each query of type 2, print the sum of colorfulness values within 
the corresponding subtree.Note that you should use 64-bit integers to avoid overflow.SAMPLE INPUT:5 18
1 2
1 3
3 4
3 5
1 4 1
2 1
2 2
2 3
2 4
2 5
1 5 1
2 1
2 2
2 3
2 4
2 5
1 1 1
2 1
2 2
2 3
2 4
2 5SAMPLE OUTPUT:1
0
1
1
0
2
0
2
1
1
5
1
3
1
1After the first query of type 1, snowball 4 is dyed with color 1.After the second query of type 1, snowballs 4 and 5 are dyed with color 1.After the third query of type 1, all snowballs are dyed with color 1.Problem credits: Michael Cao and Benjamin Qi

\section*{Output Format}
SAMPLE INPUT:5 18
1 2
1 3
3 4
3 5
1 4 1
2 1
2 2
2 3
2 4
2 5
1 5 1
2 1
2 2
2 3
2 4
2 5
1 1 1
2 1
2 2
2 3
2 4
2 5SAMPLE OUTPUT:1
0
1
1
0
2
0
2
1
1
5
1
3
1
1After the first query of type 1, snowball 4 is dyed with color 1.After the second query of type 1, snowballs 4 and 5 are dyed with color 1.After the third query of type 1, all snowballs are dyed with color 1.Problem credits: Michael Cao and Benjamin Qi

\section*{Sample Input}
\begin{verbatim}
5 18
1 2
1 3
3 4
3 5
1 4 1
2 1
2 2
2 3
2 4
2 5
1 5 1
2 1
2 2
2 3
2 4
2 5
1 1 1
2 1
2 2
2 3
2 4
2 5
\end{verbatim}

\section*{Sample Output}
\begin{verbatim}
1
0
1
1
0
2
0
2
1
1
5
1
3
1
1

After the first query of type 1, snowball 4 is dyed with color 1.After the second query of type 1, snowballs 4 and 5 are dyed with color 1.After the third query of type 1, all snowballs are dyed with color 1.Problem credits: Michael Cao and Benjamin Qi

After the second query of type 1, snowballs 4 and 5 are dyed with color 1.After the third query of type 1, all snowballs are dyed with color 1.Problem credits: Michael Cao and Benjamin Qi

After the third query of type 1, all snowballs are dyed with color 1.Problem credits: Michael Cao and Benjamin Qi

Problem credits: Michael Cao and Benjamin Qi

Problem credits: Michael Cao and Benjamin Qi
\end{verbatim}

\section*{Solution}


\section*{Problem URL}
https://usaco.org/index.php?page=viewproblem2&cpid=973

\section*{Source}
USACO DEC19 Contest, Platinum Division, Problem ID: 973

\end{document}
