\documentclass[12pt]{article}
\usepackage{amsmath}
\usepackage{amssymb}
\usepackage{graphicx}
\usepackage{enumitem}
\usepackage{hyperref}
\usepackage{xcolor}

\title{View problem}
\author{USACO Contest: DEC23 Contest - Platinum Division}
\date{\today}

\begin{document}

\maketitle

\section*{Problem Statement}


To improve her mathematical knowledge, Bessie has been taking a graph theory
course and finds herself stumped by the following problem.  Please help her!

You are given a connected, undirected graph with vertices labeled $1\dots N$ and
edges labeled $1\dots M$ ($2\le N\le 2\cdot 10^5$, $N-1\le M\le 4\cdot 10^5$).
For each vertex $v$ in the graph, the following process is conducted:
​
Let $S=\{v\}$ and $h=0$.While $|S|<N$, 
​
Out of all edges with exactly one endpoint in $S$, let $e$ be the edge with
the minimum label.Add the endpoint of $e$ not in $S$ to $S$.Set $h=10h+e$.
​
Return $h\pmod{10^9+7}$.
​
Determine all the return values of this process.
​
INPUT FORMAT (input arrives from the terminal / stdin):
The first line contains $N$ and $M$.
​
Then follow $M$ lines, the $e$th containing the endpoints $(a_e,b_e)$ of the 
$e$th edge ($1\le a_e<b_e\le N$). It is guaranteed that these edges form a
connected graph, and at most one edge connects each pair of vertices.

​
OUTPUT FORMAT (print output to the terminal / stdout):
Output $N$ lines, where the $i$th line should contain the return value of the
process starting at vertex $i$.

SAMPLE INPUT:
3 2
1 2
2 3
SAMPLE OUTPUT: 
12
12
21

SAMPLE INPUT:
5 6
1 2
3 4
2 4
2 3
2 5
1 5
SAMPLE OUTPUT: 
1325
1325
2315
2315
5132

Consider starting at $i=3$. First, we choose edge $2$, after which
$S = \{3, 4\}$ and $h = 2$. Second, we choose edge $3$, after which
$S = \{2, 3, 4\}$ and $h = 23$. Third, we choose edge $1$, after which
$S = \{1, 2, 3, 4\}$ and $h = 231$. Finally, we choose edge $5$, after which
$S = \{1, 2, 3, 4, 5\}$ and $h = 2315$. The answer for $i=3$ is therefore
$2315$.

SAMPLE INPUT:
15 14
1 2
2 3
3 4
4 5
5 6
6 7
7 8
8 9
9 10
10 11
11 12
12 13
13 14
14 15
SAMPLE OUTPUT: 
678925929
678925929
678862929
678787329
678709839
678632097
178554320
218476543
321398766
431520989
542453212
653475435
764507558
875540761
986574081

Make sure to output the answers modulo $10^9+7$.

SCORING:
Input 4: $N,M\le 2000$Inputs 5-6: $N\le 2000$Inputs 7-10: $N\le 10000$Inputs 11-14: $a_e+1=b_e$ for all $e$Inputs 15-23: No additional constraints.


Problem credits: Benjamin Qi



\section*{Input Format}


\section*{Output Format}
SAMPLE INPUT:3 2
1 2
2 3SAMPLE OUTPUT:12
12
21SAMPLE INPUT:5 6
1 2
3 4
2 4
2 3
2 5
1 5SAMPLE OUTPUT:1325
1325
2315
2315
5132Consider starting at $i=3$. First, we choose edge $2$, after which
$S = \{3, 4\}$ and $h = 2$. Second, we choose edge $3$, after which
$S = \{2, 3, 4\}$ and $h = 23$. Third, we choose edge $1$, after which
$S = \{1, 2, 3, 4\}$ and $h = 231$. Finally, we choose edge $5$, after which
$S = \{1, 2, 3, 4, 5\}$ and $h = 2315$. The answer for $i=3$ is therefore
$2315$.SAMPLE INPUT:15 14
1 2
2 3
3 4
4 5
5 6
6 7
7 8
8 9
9 10
10 11
11 12
12 13
13 14
14 15SAMPLE OUTPUT:678925929
678925929
678862929
678787329
678709839
678632097
178554320
218476543
321398766
431520989
542453212
653475435
764507558
875540761
986574081Make sure to output the answers modulo $10^9+7$.SCORING:Input 4: $N,M\le 2000$Inputs 5-6: $N\le 2000$Inputs 7-10: $N\le 10000$Inputs 11-14: $a_e+1=b_e$ for all $e$Inputs 15-23: No additional constraints.Problem credits: Benjamin Qi

\section*{Sample Input}
\begin{verbatim}
3 2
1 2
2 3

5 6
1 2
3 4
2 4
2 3
2 5
1 5

15 14
1 2
2 3
3 4
4 5
5 6
6 7
7 8
8 9
9 10
10 11
11 12
12 13
13 14
14 15
\end{verbatim}

\section*{Sample Output}
\begin{verbatim}
12
12
21

SAMPLE INPUT:5 6
1 2
3 4
2 4
2 3
2 5
1 5SAMPLE OUTPUT:1325
1325
2315
2315
5132Consider starting at $i=3$. First, we choose edge $2$, after which
$S = \{3, 4\}$ and $h = 2$. Second, we choose edge $3$, after which
$S = \{2, 3, 4\}$ and $h = 23$. Third, we choose edge $1$, after which
$S = \{1, 2, 3, 4\}$ and $h = 231$. Finally, we choose edge $5$, after which
$S = \{1, 2, 3, 4, 5\}$ and $h = 2315$. The answer for $i=3$ is therefore
$2315$.SAMPLE INPUT:15 14
1 2
2 3
3 4
4 5
5 6
6 7
7 8
8 9
9 10
10 11
11 12
12 13
13 14
14 15SAMPLE OUTPUT:678925929
678925929
678862929
678787329
678709839
678632097
178554320
218476543
321398766
431520989
542453212
653475435
764507558
875540761
986574081Make sure to output the answers modulo $10^9+7$.SCORING:Input 4: $N,M\le 2000$Inputs 5-6: $N\le 2000$Inputs 7-10: $N\le 10000$Inputs 11-14: $a_e+1=b_e$ for all $e$Inputs 15-23: No additional constraints.Problem credits: Benjamin Qi

1325
1325
2315
2315
5132

Consider starting at $i=3$. First, we choose edge $2$, after which
$S = \{3, 4\}$ and $h = 2$. Second, we choose edge $3$, after which
$S = \{2, 3, 4\}$ and $h = 23$. Third, we choose edge $1$, after which
$S = \{1, 2, 3, 4\}$ and $h = 231$. Finally, we choose edge $5$, after which
$S = \{1, 2, 3, 4, 5\}$ and $h = 2315$. The answer for $i=3$ is therefore
$2315$.SAMPLE INPUT:15 14
1 2
2 3
3 4
4 5
5 6
6 7
7 8
8 9
9 10
10 11
11 12
12 13
13 14
14 15SAMPLE OUTPUT:678925929
678925929
678862929
678787329
678709839
678632097
178554320
218476543
321398766
431520989
542453212
653475435
764507558
875540761
986574081Make sure to output the answers modulo $10^9+7$.SCORING:Input 4: $N,M\le 2000$Inputs 5-6: $N\le 2000$Inputs 7-10: $N\le 10000$Inputs 11-14: $a_e+1=b_e$ for all $e$Inputs 15-23: No additional constraints.Problem credits: Benjamin Qi

SAMPLE INPUT:15 14
1 2
2 3
3 4
4 5
5 6
6 7
7 8
8 9
9 10
10 11
11 12
12 13
13 14
14 15SAMPLE OUTPUT:678925929
678925929
678862929
678787329
678709839
678632097
178554320
218476543
321398766
431520989
542453212
653475435
764507558
875540761
986574081Make sure to output the answers modulo $10^9+7$.SCORING:Input 4: $N,M\le 2000$Inputs 5-6: $N\le 2000$Inputs 7-10: $N\le 10000$Inputs 11-14: $a_e+1=b_e$ for all $e$Inputs 15-23: No additional constraints.Problem credits: Benjamin Qi

678925929
678925929
678862929
678787329
678709839
678632097
178554320
218476543
321398766
431520989
542453212
653475435
764507558
875540761
986574081

Make sure to output the answers modulo $10^9+7$.SCORING:Input 4: $N,M\le 2000$Inputs 5-6: $N\le 2000$Inputs 7-10: $N\le 10000$Inputs 11-14: $a_e+1=b_e$ for all $e$Inputs 15-23: No additional constraints.Problem credits: Benjamin Qi

SCORING:Input 4: $N,M\le 2000$Inputs 5-6: $N\le 2000$Inputs 7-10: $N\le 10000$Inputs 11-14: $a_e+1=b_e$ for all $e$Inputs 15-23: No additional constraints.Problem credits: Benjamin Qi
\end{verbatim}

\section*{Solution}


\section*{Problem URL}
https://usaco.org/index.php?page=viewproblem2&cpid=1357

\section*{Source}
USACO DEC23 Contest, Platinum Division, Problem ID: 1357

\end{document}
