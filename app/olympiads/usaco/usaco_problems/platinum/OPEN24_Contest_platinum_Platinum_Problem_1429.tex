\documentclass[12pt]{article}
\usepackage{amsmath}
\usepackage{amssymb}
\usepackage{graphicx}
\usepackage{enumitem}
\usepackage{hyperref}
\usepackage{xcolor}

\title{View problem}
\author{USACO Contest: OPEN24 Contest - Platinum Division}
\date{\today}

\begin{document}

\maketitle

\section*{Problem Statement}


Farmer John wants to fairly split haybales between his two favorite cows Bessie
and Elsie. He has $N$ ( $1\le N\le 2\cdot 10^5$) haybales sorted in
non-increasing order,  where the $i$-th haybale has $a_i$ units of hay
($2\cdot 10^5\ge a_1\ge a_2 \ge \dots \ge a_N \ge 1$).

Farmer John is considering splitting a contiguous range of haybales
$a_l, \dots, a_r$ between Bessie and Elsie. He has decided to process the
haybales in order from $l$ to $r$, and when processing the $i$-th haybale he
will give it to the cow who currently has less hay (if it is a tie, he will give
it to Bessie).

You are given $Q$ ($1\le Q\le 2\cdot 10^5$) queries, each with three integers
$l,r,x$ ($1\le l\le r\le N$, $|x|\le 10^9$). For each query, output how many
more units of hay Bessie will have than Elsie after processing haybales $l$ to
$r$, if Bessie starts with $x$ more units than Elsie. Note that this value is
negative if Elsie ends up with more haybales than Bessie. 

INPUT FORMAT (input arrives from the terminal / stdin):
First line contains $N$.

Second line contains $a_1\dots a_N$.

Third line contains $Q$.

Next $Q$ lines contain $l, r, x$.

OUTPUT FORMAT (print output to the terminal / stdout):
Output $Q$ lines, containing the answer for each query.

SAMPLE INPUT:
2
3 1
15
1 1 -2
1 1 -1
1 1 0
1 1 1
1 1 2
1 2 -2
1 2 -1
1 2 0
1 2 1
1 2 2
2 2 -2
2 2 -1
2 2 0
2 2 1
2 2 2
SAMPLE OUTPUT: 
1
2
3
-2
-1
0
1
2
-1
0
-1
0
1
0
1

For the 1st query, Elsie starts with $2$ more hay than Bessie. Then, after
processing haybale $1$, Bessie will receive $3$ hay, and Bessie will have $1$
more hay than Elsie. 

For the 3rd query, Elsie and Bessie start with the same number of hay. After
processing haybale $1$, Bessie will receive $3$ hay, and Bessie will have $3$
more hay than Elsie.

For the 9th query, Bessie starts with $1$ more hay than Elsie, then after
processing the 1st haybale, has $2$ less hay than Elsie, and after processing
the 2nd haybale, has $1$ less hay than Elsie.
SAMPLE INPUT:
5
4 4 3 1 1
7
1 1 20
1 2 20
1 5 20
1 1 0
1 5 0
1 4 0
3 5 2
SAMPLE OUTPUT: 
16
12
7
4
1
2
1

In the 5th query, there are $5$ haybales to process. Bessie receives $4$ hay, then Elsie
receives $4$ hay, then Bessie receives $3$ hay, then Elsie receives $1$ hay,
then Elsie receives $1$ hay.

SCORING:
Input 3: $Q\le 100$Inputs 4-6: At most $100$ distinct $a_i$Inputs 7-22: No additional constraints.


Problem credits: Benjamin Qi



\section*{Input Format}
Second line contains $a_1\dots a_N$.Third line contains $Q$.Next $Q$ lines contain $l, r, x$.

Third line contains $Q$.Next $Q$ lines contain $l, r, x$.

Next $Q$ lines contain $l, r, x$.

OUTPUT FORMAT (print output to the terminal / stdout):Output $Q$ lines, containing the answer for each query.SAMPLE INPUT:2
3 1
15
1 1 -2
1 1 -1
1 1 0
1 1 1
1 1 2
1 2 -2
1 2 -1
1 2 0
1 2 1
1 2 2
2 2 -2
2 2 -1
2 2 0
2 2 1
2 2 2SAMPLE OUTPUT:1
2
3
-2
-1
0
1
2
-1
0
-1
0
1
0
1For the 1st query, Elsie starts with $2$ more hay than Bessie. Then, after
processing haybale $1$, Bessie will receive $3$ hay, and Bessie will have $1$
more hay than Elsie.For the 3rd query, Elsie and Bessie start with the same number of hay. After
processing haybale $1$, Bessie will receive $3$ hay, and Bessie will have $3$
more hay than Elsie.For the 9th query, Bessie starts with $1$ more hay than Elsie, then after
processing the 1st haybale, has $2$ less hay than Elsie, and after processing
the 2nd haybale, has $1$ less hay than Elsie.SAMPLE INPUT:5
4 4 3 1 1
7
1 1 20
1 2 20
1 5 20
1 1 0
1 5 0
1 4 0
3 5 2SAMPLE OUTPUT:16
12
7
4
1
2
1In the 5th query, there are $5$ haybales to process. Bessie receives $4$ hay, then Elsie
receives $4$ hay, then Bessie receives $3$ hay, then Elsie receives $1$ hay,
then Elsie receives $1$ hay.SCORING:Input 3: $Q\le 100$Inputs 4-6: At most $100$ distinct $a_i$Inputs 7-22: No additional constraints.Problem credits: Benjamin Qi

\section*{Output Format}
SAMPLE INPUT:2
3 1
15
1 1 -2
1 1 -1
1 1 0
1 1 1
1 1 2
1 2 -2
1 2 -1
1 2 0
1 2 1
1 2 2
2 2 -2
2 2 -1
2 2 0
2 2 1
2 2 2SAMPLE OUTPUT:1
2
3
-2
-1
0
1
2
-1
0
-1
0
1
0
1For the 1st query, Elsie starts with $2$ more hay than Bessie. Then, after
processing haybale $1$, Bessie will receive $3$ hay, and Bessie will have $1$
more hay than Elsie.For the 3rd query, Elsie and Bessie start with the same number of hay. After
processing haybale $1$, Bessie will receive $3$ hay, and Bessie will have $3$
more hay than Elsie.For the 9th query, Bessie starts with $1$ more hay than Elsie, then after
processing the 1st haybale, has $2$ less hay than Elsie, and after processing
the 2nd haybale, has $1$ less hay than Elsie.SAMPLE INPUT:5
4 4 3 1 1
7
1 1 20
1 2 20
1 5 20
1 1 0
1 5 0
1 4 0
3 5 2SAMPLE OUTPUT:16
12
7
4
1
2
1In the 5th query, there are $5$ haybales to process. Bessie receives $4$ hay, then Elsie
receives $4$ hay, then Bessie receives $3$ hay, then Elsie receives $1$ hay,
then Elsie receives $1$ hay.SCORING:Input 3: $Q\le 100$Inputs 4-6: At most $100$ distinct $a_i$Inputs 7-22: No additional constraints.Problem credits: Benjamin Qi

\section*{Sample Input}
\begin{verbatim}
2
3 1
15
1 1 -2
1 1 -1
1 1 0
1 1 1
1 1 2
1 2 -2
1 2 -1
1 2 0
1 2 1
1 2 2
2 2 -2
2 2 -1
2 2 0
2 2 1
2 2 2

5
4 4 3 1 1
7
1 1 20
1 2 20
1 5 20
1 1 0
1 5 0
1 4 0
3 5 2
\end{verbatim}

\section*{Sample Output}
\begin{verbatim}
1
2
3
-2
-1
0
1
2
-1
0
-1
0
1
0
1

For the 3rd query, Elsie and Bessie start with the same number of hay. After
processing haybale $1$, Bessie will receive $3$ hay, and Bessie will have $3$
more hay than Elsie.For the 9th query, Bessie starts with $1$ more hay than Elsie, then after
processing the 1st haybale, has $2$ less hay than Elsie, and after processing
the 2nd haybale, has $1$ less hay than Elsie.SAMPLE INPUT:5
4 4 3 1 1
7
1 1 20
1 2 20
1 5 20
1 1 0
1 5 0
1 4 0
3 5 2SAMPLE OUTPUT:16
12
7
4
1
2
1In the 5th query, there are $5$ haybales to process. Bessie receives $4$ hay, then Elsie
receives $4$ hay, then Bessie receives $3$ hay, then Elsie receives $1$ hay,
then Elsie receives $1$ hay.SCORING:Input 3: $Q\le 100$Inputs 4-6: At most $100$ distinct $a_i$Inputs 7-22: No additional constraints.Problem credits: Benjamin Qi

For the 9th query, Bessie starts with $1$ more hay than Elsie, then after
processing the 1st haybale, has $2$ less hay than Elsie, and after processing
the 2nd haybale, has $1$ less hay than Elsie.SAMPLE INPUT:5
4 4 3 1 1
7
1 1 20
1 2 20
1 5 20
1 1 0
1 5 0
1 4 0
3 5 2SAMPLE OUTPUT:16
12
7
4
1
2
1In the 5th query, there are $5$ haybales to process. Bessie receives $4$ hay, then Elsie
receives $4$ hay, then Bessie receives $3$ hay, then Elsie receives $1$ hay,
then Elsie receives $1$ hay.SCORING:Input 3: $Q\le 100$Inputs 4-6: At most $100$ distinct $a_i$Inputs 7-22: No additional constraints.Problem credits: Benjamin Qi

16
12
7
4
1
2
1

In the 5th query, there are $5$ haybales to process. Bessie receives $4$ hay, then Elsie
receives $4$ hay, then Bessie receives $3$ hay, then Elsie receives $1$ hay,
then Elsie receives $1$ hay.SCORING:Input 3: $Q\le 100$Inputs 4-6: At most $100$ distinct $a_i$Inputs 7-22: No additional constraints.Problem credits: Benjamin Qi

SCORING:Input 3: $Q\le 100$Inputs 4-6: At most $100$ distinct $a_i$Inputs 7-22: No additional constraints.Problem credits: Benjamin Qi
\end{verbatim}

\section*{Solution}


\section*{Problem URL}
https://usaco.org/index.php?page=viewproblem2&cpid=1429

\section*{Source}
USACO OPEN24 Contest, Platinum Division, Problem ID: 1429

\end{document}
