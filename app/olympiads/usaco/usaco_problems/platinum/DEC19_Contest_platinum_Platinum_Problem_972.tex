\documentclass[12pt]{article}
\usepackage{amsmath}
\usepackage{amssymb}
\usepackage{graphicx}
\usepackage{enumitem}
\usepackage{hyperref}
\usepackage{xcolor}

\title{View problem}
\author{USACO Contest: DEC19 Contest - Platinum Division}
\date{\today}

\begin{document}

\maketitle

\section*{Problem Statement}

Farmer John has $M$ cows, conveniently labeled $1 \ldots M$, who enjoy the occasional change of pace
from eating grass.  As a treat for the cows, Farmer John has baked $N$ pies ($1 \leq N \leq 300$), labeled
$1 \ldots N$.  Cow $i$ enjoys pies with labels in the range $[l_i, r_i]$ (from $l_i$ to $r_i$ inclusive),
and no two cows enjoy the exact same range of pies.  Cow $i$ also has a weight, $w_i$, which 
is an integer in the range $1 \ldots 10^6$.

Farmer John may choose a sequence of cows $c_1,c_2,\ldots, c_K,$ after which the
selected cows will take turns eating in that order. Unfortunately, the cows 
don't know how to share! When it is cow $c_i$'s turn to eat, she will consume
all of the  pies that she enjoys --- that is, all remaining pies in the interval
$[l_{c_i},r_{c_i}]$.  Farmer John would like to avoid the awkward situation
occurring when it is a cows turn to eat but all of the pies she enjoys have already been
consumed. Therefore, he wants you to compute the largest possible total weight
($w_{c_1}+w_{c_2}+\ldots+w_{c_K}$) of a sequence $c_1,c_2,\ldots, c_K$ for which each cow in the
sequence eats at least one pie.

SCORING:
Test cases 2-5 satisfy $N\le 50$ and $M\le 20$.  Test cases 6-9 satisfy $N\le 50.$ 

INPUT FORMAT (file pieaters.in):
The first line contains two integers $N$ and $M$
$\left(1\le M\le \frac{N(N+1)}{2}\right)$. 

The next $M$ lines each describe a cow in terms of the integers $w_i, l_i$, and $r_i$.

OUTPUT FORMAT (file pieaters.out):
Print the maximum possible total weight of a valid sequence.

SAMPLE INPUT:
2 2
100 1 2
100 1 1
SAMPLE OUTPUT: 
200

In this example, if cow 1 eats first, then there will be nothing left for cow 2 to eat. However,
if cow 2 eats first, then cow 1 will be satisfied by eating the second pie only.

Problem credits: Benjamin Qi



\section*{Input Format}
The next $M$ lines each describe a cow in terms of the integers $w_i, l_i$, and $r_i$.

OUTPUT FORMAT (file pieaters.out):Print the maximum possible total weight of a valid sequence.SAMPLE INPUT:2 2
100 1 2
100 1 1SAMPLE OUTPUT:200In this example, if cow 1 eats first, then there will be nothing left for cow 2 to eat. However,
if cow 2 eats first, then cow 1 will be satisfied by eating the second pie only.Problem credits: Benjamin Qi

\section*{Output Format}
SAMPLE INPUT:2 2
100 1 2
100 1 1SAMPLE OUTPUT:200In this example, if cow 1 eats first, then there will be nothing left for cow 2 to eat. However,
if cow 2 eats first, then cow 1 will be satisfied by eating the second pie only.Problem credits: Benjamin Qi

\section*{Sample Input}
\begin{verbatim}
2 2
100 1 2
100 1 1
\end{verbatim}

\section*{Sample Output}
\begin{verbatim}
200

In this example, if cow 1 eats first, then there will be nothing left for cow 2 to eat. However,
if cow 2 eats first, then cow 1 will be satisfied by eating the second pie only.Problem credits: Benjamin Qi

Problem credits: Benjamin Qi
\end{verbatim}

\section*{Solution}


\section*{Problem URL}
https://usaco.org/index.php?page=viewproblem2&cpid=972

\section*{Source}
USACO DEC19 Contest, Platinum Division, Problem ID: 972

\end{document}
