\documentclass[12pt]{article}
\usepackage{amsmath}
\usepackage{amssymb}
\usepackage{graphicx}
\usepackage{enumitem}
\usepackage{hyperref}
\usepackage{xcolor}

\title{View problem}
\author{USACO Contest: JAN16 Contest - Platinum Division}
\date{\today}

\begin{document}

\maketitle

\section*{Problem Statement}

Bessie is building a fort with her friend Elsie. Like any good fort, this one
needs to start with a sturdy frame. Bessie wants to build a frame in the shape
of a one-meter-wide rectangular outline, atop which she will build the fort.

Bessie has already chosen a site on which to build the fort -- a piece of land
measuring $N$ meters by $M$ meters ($1 \leq N, M \leq 200$). Unfortunately, the 
site has some swampy areas that cannot be used to support the frame.  Please
help Bessie determine the largest area she can cover with her fort (the area of
the rectangle supported by the frame), such that the frame avoids sitting on any
of the swampy areas.

INPUT FORMAT (file fortmoo.in):
Line 1 contains integers $N$ and $M$.  

The next $N$ lines each contain $M$ characters, forming a grid describing the 
site.  A character of '.' represents normal grass, while 'X' represents a swampy
spot.

OUTPUT FORMAT (file fortmoo.out):
A single integer representing the maximum area that Bessie can cover with her
fort.

SAMPLE INPUT:
5 6
......
..X..X
X..X..
......
..X...
SAMPLE OUTPUT: 
16

In the example, the placement of the optimal frame is indicated by 'f's below:

.ffff.
.fX.fX
Xf.Xf.
.ffff.
..X...

Problem credits: Nathan Pinsker



\section*{Input Format}
The next $N$ lines each contain $M$ characters, forming a grid describing the 
site.  A character of '.' represents normal grass, while 'X' represents a swampy
spot.

OUTPUT FORMAT (file fortmoo.out):A single integer representing the maximum area that Bessie can cover with her
fort.SAMPLE INPUT:5 6
......
..X..X
X..X..
......
..X...SAMPLE OUTPUT:16In the example, the placement of the optimal frame is indicated by 'f's below:.ffff.
.fX.fX
Xf.Xf.
.ffff.
..X...Problem credits: Nathan Pinsker

\section*{Output Format}
SAMPLE INPUT:5 6
......
..X..X
X..X..
......
..X...SAMPLE OUTPUT:16In the example, the placement of the optimal frame is indicated by 'f's below:.ffff.
.fX.fX
Xf.Xf.
.ffff.
..X...Problem credits: Nathan Pinsker

\section*{Sample Input}
\begin{verbatim}
5 6
......
..X..X
X..X..
......
..X...
\end{verbatim}

\section*{Sample Output}
\begin{verbatim}
16

In the example, the placement of the optimal frame is indicated by 'f's below:.ffff.
.fX.fX
Xf.Xf.
.ffff.
..X...Problem credits: Nathan Pinsker

.ffff.
.fX.fX
Xf.Xf.
.ffff.
..X...

Problem credits: Nathan Pinsker
\end{verbatim}

\section*{Solution}


\section*{Problem URL}
https://usaco.org/index.php?page=viewproblem2&cpid=600

\section*{Source}
USACO JAN16 Contest, Platinum Division, Problem ID: 600

\end{document}
