\documentclass[12pt]{article}
\usepackage{amsmath}
\usepackage{amssymb}
\usepackage{graphicx}
\usepackage{enumitem}
\usepackage{hyperref}
\usepackage{xcolor}

\title{View problem}
\author{USACO Contest: DEC14 Contest - Platinum Division}
\date{\today}

\begin{document}

\maketitle

\section*{Problem Statement}

Problem 2: Marathon [Nick Wu, 2014]



An avid marathon runner herself, Bessie enjoys creating marathon

courses for her fellow cows to run.  She has most recently designed a

route specified by N checkpoints (1 <= N <= 100,000) that must be

visited in sequence.



Unfortunately, Bessie realizes that the other cows may not have the

stamina to run the full route. She therefore wants to know how long

certain sub-routes take to run, where a sub-route is a contiguous

subsequence of the checkpoints from the full route.  To make matters

more complicated, Bessie knows that the other cows, being lazy, might

choose to skip one checkpoint any time they run a sub-route --

whichever one results in a minimum total travel time.  They are not

allowed to skip the first or last checkpoints in a sub-route, however.



In order to build the best possible marathon course, Bessie wants to

investigate the ramifications of making changes to the checkpoint

locations in her current course.  Please help her determine how

certain changes to checkpoint locations will affect the time required

to run different sub-routes (taking into account that the cows might

choose to omit one checkpoint any time they run a sub-route).



Since the course is set in a downtown area with a grid of streets, the

distance between two checkpoints at locations (x1, y1) and (x2, y2) is

given by |x1-x2| + |y1-y2|.



INPUT: (file marathon.in)



The first line gives N and Q (1 <= Q <= 100,000).



The next N lines contain (x, y) locations of N checkpoints in the

sequence they must be visited along the route.  All coordinates

are in the range -1000 .. 1000.



The next Q lines consist of updates and queries, one per line, to be

processed in the order they are given. Lines will either take the form

"U I X Y" or "Q I J".  A line of the form "U I X Y" indicates that the

location of checkpoint I (1 <= I <= N) is to be changed to location

(X, Y).  A line of the form "Q I J" asks for the minimum travel time

of the sub-route from checkpoint I to checkpoint J (I <= J), given

that the cows choose to skip one checkpoint along this sub-route (but

not the endpoints I and J).



SAMPLE INPUT:



5 5

-4 4

-5 -3

-1 5

-3 4

0 5

Q 1 5

U 4 0 1

U 4 -1 1

Q 2 4

Q 1 4



OUTPUT: (file marathon.out)



For each sub-route length request, output on a single line the desired

length.



SAMPLE OUTPUT:



11

8

8




\section*{Input Format}


\section*{Output Format}


\section*{Sample Input}
\begin{verbatim}

\end{verbatim}

\section*{Sample Output}
\begin{verbatim}

\end{verbatim}

\section*{Solution}


\section*{Problem URL}
https://usaco.org/index.php?page=viewproblem2&cpid=495

\section*{Source}
USACO DEC14 Contest, Platinum Division, Problem ID: 495

\end{document}
