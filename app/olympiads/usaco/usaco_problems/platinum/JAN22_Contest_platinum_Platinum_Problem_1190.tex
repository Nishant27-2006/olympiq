\documentclass[12pt]{article}
\usepackage{amsmath}
\usepackage{amssymb}
\usepackage{graphicx}
\usepackage{enumitem}
\usepackage{hyperref}
\usepackage{xcolor}

\title{View problem}
\author{USACO Contest: JAN22 Contest - Platinum Division}
\date{\today}

\begin{document}

\maketitle

\section*{Problem Statement}

The cows are taking a multiple choice test. But instead of a standard test where your
selected choices are scored for each question individually and then summed, in
this test your selected choices are summed before being scored.

Specifically, you are given $N$ ($2\le N\le 10^5$) groups of integer vectors  on
the 2D plane, where each vector is denoted by an ordered pair $(x,y)$. Choose
one vector from each group such that the sum of the vectors is as far away from
the origin as possible.

It is guaranteed that the total number of vectors is at most $2\cdot 10^5$. 
Each group has size at least $2$, and within a group, all vectors are distinct. 
It is also guaranteed that every $x$ and $y$ coordinate has absolute value at
most $\frac{10^9}{N}$.

INPUT FORMAT (input arrives from the terminal / stdin):
The first line contains $N$, the number of groups.

Each group starts with $G$, the number of vectors in the group, followed by $G$
lines containing the vectors in that group. Consecutive groups are separated by
newlines.

OUTPUT FORMAT (print output to the terminal / stdout):
The maximum possible squared Euclidean distance.

SAMPLE INPUT:
3

2
-2 0
1 0

2
0 -2
0 1

3
-5 -5
5 1
10 10
SAMPLE OUTPUT: 
242

It is optimal to select $(1,0)$ from the first group, $(0,1)$ from the second
group, and $(10,10)$ from the third group. The sum of these vectors is
$(11,11)$, which is squared distance $11^2+11^2=242$ from the origin.

SCORING:
In test cases 1-5, the total number of vectors is at most $10^3$.In test cases 6-9, every group has size exactly two.Test cases 10-17 satisfy no additional constraints.


Problem credits: Benjamin Qi



\section*{Input Format}
Each group starts with $G$, the number of vectors in the group, followed by $G$
lines containing the vectors in that group. Consecutive groups are separated by
newlines.

OUTPUT FORMAT (print output to the terminal / stdout):The maximum possible squared Euclidean distance.SAMPLE INPUT:3

2
-2 0
1 0

2
0 -2
0 1

3
-5 -5
5 1
10 10SAMPLE OUTPUT:242It is optimal to select $(1,0)$ from the first group, $(0,1)$ from the second
group, and $(10,10)$ from the third group. The sum of these vectors is
$(11,11)$, which is squared distance $11^2+11^2=242$ from the origin.SCORING:In test cases 1-5, the total number of vectors is at most $10^3$.In test cases 6-9, every group has size exactly two.Test cases 10-17 satisfy no additional constraints.Problem credits: Benjamin Qi

\section*{Output Format}
SAMPLE INPUT:3

2
-2 0
1 0

2
0 -2
0 1

3
-5 -5
5 1
10 10SAMPLE OUTPUT:242It is optimal to select $(1,0)$ from the first group, $(0,1)$ from the second
group, and $(10,10)$ from the third group. The sum of these vectors is
$(11,11)$, which is squared distance $11^2+11^2=242$ from the origin.SCORING:In test cases 1-5, the total number of vectors is at most $10^3$.In test cases 6-9, every group has size exactly two.Test cases 10-17 satisfy no additional constraints.Problem credits: Benjamin Qi

\section*{Sample Input}
\begin{verbatim}
3

2
-2 0
1 0

2
0 -2
0 1

3
-5 -5
5 1
10 10
\end{verbatim}

\section*{Sample Output}
\begin{verbatim}
242

It is optimal to select $(1,0)$ from the first group, $(0,1)$ from the second
group, and $(10,10)$ from the third group. The sum of these vectors is
$(11,11)$, which is squared distance $11^2+11^2=242$ from the origin.SCORING:In test cases 1-5, the total number of vectors is at most $10^3$.In test cases 6-9, every group has size exactly two.Test cases 10-17 satisfy no additional constraints.Problem credits: Benjamin Qi

SCORING:In test cases 1-5, the total number of vectors is at most $10^3$.In test cases 6-9, every group has size exactly two.Test cases 10-17 satisfy no additional constraints.Problem credits: Benjamin Qi
\end{verbatim}

\section*{Solution}


\section*{Problem URL}
https://usaco.org/index.php?page=viewproblem2&cpid=1190

\section*{Source}
USACO JAN22 Contest, Platinum Division, Problem ID: 1190

\end{document}
