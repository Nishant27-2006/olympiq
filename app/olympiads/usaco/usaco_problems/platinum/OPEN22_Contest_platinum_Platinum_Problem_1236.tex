\documentclass[12pt]{article}
\usepackage{amsmath}
\usepackage{amssymb}
\usepackage{graphicx}
\usepackage{enumitem}
\usepackage{hyperref}
\usepackage{xcolor}

\title{View problem}
\author{USACO Contest: OPEN22 Contest - Platinum Division}
\date{\today}

\begin{document}

\maketitle

\section*{Problem Statement}

Bessie likes downloading games to play on her cell phone, even though she does
find the small touch screen rather cumbersome to use with her large hooves.

She is particularly intrigued by the current game she is playing. The game
starts with a sequence of $N$ positive integers $a_1,a_2,\ldots,a_N$
($2\le N\le 262,144$),  each in the range $1\ldots 10^6$. In one move, Bessie
can take two adjacent numbers  and replace them with a single number equal to one
greater than the maximum of the two (e.g., she  might replace an adjacent pair
$(5,7)$ with an $8$). The game ends after $N-1$  moves, at which point only a
single number remains. The goal is to minimize this final number.

Bessie knows that this game is too easy for you. So your job is not just to play
the game optimally on $a$, but for every contiguous subsequence of $a$. 

Output the sum of the minimum possible final numbers over all $\frac{N(N+1)}{2}$
contiguous subsequences of $a$.

INPUT FORMAT (input arrives from the terminal / stdin):
First line contains $N$.

The next line contains $N$ space-separated integers denoting the input sequence.

OUTPUT FORMAT (print output to the terminal / stdout):
A single line containing the sum.

SAMPLE INPUT:
6
1 3 1 2 1 10
SAMPLE OUTPUT: 
115

There are $\frac{6\cdot 7}{2}=21$ contiguous subsequences in total. For example,
the minimum possible final number for the contiguous subsequence $[1,3,1,2,1]$
is $5$, which can be obtained via the following sequence of operations:


original    -> [1,3,1,2,1]
combine 1&3 -> [4,1,2,1]
combine 2&1 -> [4,1,3]
combine 1&3 -> [4,4]
combine 4&4 -> [5]

Here are the minimum possible final numbers for each contiguous subsequence:


final(1:1) = 1
final(1:2) = 4
final(1:3) = 5
final(1:4) = 5
final(1:5) = 5
final(1:6) = 11
final(2:2) = 3
final(2:3) = 4
final(2:4) = 4
final(2:5) = 5
final(2:6) = 11
final(3:3) = 1
final(3:4) = 3
final(3:5) = 4
final(3:6) = 11
final(4:4) = 2
final(4:5) = 3
final(4:6) = 11
final(5:5) = 1
final(5:6) = 11
final(6:6) = 10

SCORING:
Test cases 2-3 satisfy $N\le 300$.Test cases 4-5 satisfy $N\le 3000$.In test cases 6-8, all values are at most $40$.In test cases 9-11, the input sequence is non-decreasing.Test cases 12-23 satisfy no additional constraints.


Problem credits: Benjamin Qi



\section*{Input Format}
The next line contains $N$ space-separated integers denoting the input sequence.

OUTPUT FORMAT (print output to the terminal / stdout):A single line containing the sum.SAMPLE INPUT:6
1 3 1 2 1 10SAMPLE OUTPUT:115There are $\frac{6\cdot 7}{2}=21$ contiguous subsequences in total. For example,
the minimum possible final number for the contiguous subsequence $[1,3,1,2,1]$
is $5$, which can be obtained via the following sequence of operations:original    -> [1,3,1,2,1]
combine 1&3 -> [4,1,2,1]
combine 2&1 -> [4,1,3]
combine 1&3 -> [4,4]
combine 4&4 -> [5]Here are the minimum possible final numbers for each contiguous subsequence:final(1:1) = 1
final(1:2) = 4
final(1:3) = 5
final(1:4) = 5
final(1:5) = 5
final(1:6) = 11
final(2:2) = 3
final(2:3) = 4
final(2:4) = 4
final(2:5) = 5
final(2:6) = 11
final(3:3) = 1
final(3:4) = 3
final(3:5) = 4
final(3:6) = 11
final(4:4) = 2
final(4:5) = 3
final(4:6) = 11
final(5:5) = 1
final(5:6) = 11
final(6:6) = 10SCORING:Test cases 2-3 satisfy $N\le 300$.Test cases 4-5 satisfy $N\le 3000$.In test cases 6-8, all values are at most $40$.In test cases 9-11, the input sequence is non-decreasing.Test cases 12-23 satisfy no additional constraints.Problem credits: Benjamin Qi

\section*{Output Format}
SAMPLE INPUT:6
1 3 1 2 1 10SAMPLE OUTPUT:115There are $\frac{6\cdot 7}{2}=21$ contiguous subsequences in total. For example,
the minimum possible final number for the contiguous subsequence $[1,3,1,2,1]$
is $5$, which can be obtained via the following sequence of operations:original    -> [1,3,1,2,1]
combine 1&3 -> [4,1,2,1]
combine 2&1 -> [4,1,3]
combine 1&3 -> [4,4]
combine 4&4 -> [5]Here are the minimum possible final numbers for each contiguous subsequence:final(1:1) = 1
final(1:2) = 4
final(1:3) = 5
final(1:4) = 5
final(1:5) = 5
final(1:6) = 11
final(2:2) = 3
final(2:3) = 4
final(2:4) = 4
final(2:5) = 5
final(2:6) = 11
final(3:3) = 1
final(3:4) = 3
final(3:5) = 4
final(3:6) = 11
final(4:4) = 2
final(4:5) = 3
final(4:6) = 11
final(5:5) = 1
final(5:6) = 11
final(6:6) = 10SCORING:Test cases 2-3 satisfy $N\le 300$.Test cases 4-5 satisfy $N\le 3000$.In test cases 6-8, all values are at most $40$.In test cases 9-11, the input sequence is non-decreasing.Test cases 12-23 satisfy no additional constraints.Problem credits: Benjamin Qi

\section*{Sample Input}
\begin{verbatim}
6
1 3 1 2 1 10
\end{verbatim}

\section*{Sample Output}
\begin{verbatim}
115

There are $\frac{6\cdot 7}{2}=21$ contiguous subsequences in total. For example,
the minimum possible final number for the contiguous subsequence $[1,3,1,2,1]$
is $5$, which can be obtained via the following sequence of operations:original    -> [1,3,1,2,1]
combine 1&3 -> [4,1,2,1]
combine 2&1 -> [4,1,3]
combine 1&3 -> [4,4]
combine 4&4 -> [5]Here are the minimum possible final numbers for each contiguous subsequence:final(1:1) = 1
final(1:2) = 4
final(1:3) = 5
final(1:4) = 5
final(1:5) = 5
final(1:6) = 11
final(2:2) = 3
final(2:3) = 4
final(2:4) = 4
final(2:5) = 5
final(2:6) = 11
final(3:3) = 1
final(3:4) = 3
final(3:5) = 4
final(3:6) = 11
final(4:4) = 2
final(4:5) = 3
final(4:6) = 11
final(5:5) = 1
final(5:6) = 11
final(6:6) = 10SCORING:Test cases 2-3 satisfy $N\le 300$.Test cases 4-5 satisfy $N\le 3000$.In test cases 6-8, all values are at most $40$.In test cases 9-11, the input sequence is non-decreasing.Test cases 12-23 satisfy no additional constraints.Problem credits: Benjamin Qi

original    -> [1,3,1,2,1]
combine 1&3 -> [4,1,2,1]
combine 2&1 -> [4,1,3]
combine 1&3 -> [4,4]
combine 4&4 -> [5]Here are the minimum possible final numbers for each contiguous subsequence:final(1:1) = 1
final(1:2) = 4
final(1:3) = 5
final(1:4) = 5
final(1:5) = 5
final(1:6) = 11
final(2:2) = 3
final(2:3) = 4
final(2:4) = 4
final(2:5) = 5
final(2:6) = 11
final(3:3) = 1
final(3:4) = 3
final(3:5) = 4
final(3:6) = 11
final(4:4) = 2
final(4:5) = 3
final(4:6) = 11
final(5:5) = 1
final(5:6) = 11
final(6:6) = 10SCORING:Test cases 2-3 satisfy $N\le 300$.Test cases 4-5 satisfy $N\le 3000$.In test cases 6-8, all values are at most $40$.In test cases 9-11, the input sequence is non-decreasing.Test cases 12-23 satisfy no additional constraints.Problem credits: Benjamin Qi

original    -> [1,3,1,2,1]
combine 1&3 -> [4,1,2,1]
combine 2&1 -> [4,1,3]
combine 1&3 -> [4,4]
combine 4&4 -> [5]

Here are the minimum possible final numbers for each contiguous subsequence:final(1:1) = 1
final(1:2) = 4
final(1:3) = 5
final(1:4) = 5
final(1:5) = 5
final(1:6) = 11
final(2:2) = 3
final(2:3) = 4
final(2:4) = 4
final(2:5) = 5
final(2:6) = 11
final(3:3) = 1
final(3:4) = 3
final(3:5) = 4
final(3:6) = 11
final(4:4) = 2
final(4:5) = 3
final(4:6) = 11
final(5:5) = 1
final(5:6) = 11
final(6:6) = 10SCORING:Test cases 2-3 satisfy $N\le 300$.Test cases 4-5 satisfy $N\le 3000$.In test cases 6-8, all values are at most $40$.In test cases 9-11, the input sequence is non-decreasing.Test cases 12-23 satisfy no additional constraints.Problem credits: Benjamin Qi

final(1:1) = 1
final(1:2) = 4
final(1:3) = 5
final(1:4) = 5
final(1:5) = 5
final(1:6) = 11
final(2:2) = 3
final(2:3) = 4
final(2:4) = 4
final(2:5) = 5
final(2:6) = 11
final(3:3) = 1
final(3:4) = 3
final(3:5) = 4
final(3:6) = 11
final(4:4) = 2
final(4:5) = 3
final(4:6) = 11
final(5:5) = 1
final(5:6) = 11
final(6:6) = 10SCORING:Test cases 2-3 satisfy $N\le 300$.Test cases 4-5 satisfy $N\le 3000$.In test cases 6-8, all values are at most $40$.In test cases 9-11, the input sequence is non-decreasing.Test cases 12-23 satisfy no additional constraints.Problem credits: Benjamin Qi

final(1:1) = 1
final(1:2) = 4
final(1:3) = 5
final(1:4) = 5
final(1:5) = 5
final(1:6) = 11
final(2:2) = 3
final(2:3) = 4
final(2:4) = 4
final(2:5) = 5
final(2:6) = 11
final(3:3) = 1
final(3:4) = 3
final(3:5) = 4
final(3:6) = 11
final(4:4) = 2
final(4:5) = 3
final(4:6) = 11
final(5:5) = 1
final(5:6) = 11
final(6:6) = 10

SCORING:Test cases 2-3 satisfy $N\le 300$.Test cases 4-5 satisfy $N\le 3000$.In test cases 6-8, all values are at most $40$.In test cases 9-11, the input sequence is non-decreasing.Test cases 12-23 satisfy no additional constraints.Problem credits: Benjamin Qi
\end{verbatim}

\section*{Solution}


\section*{Problem URL}
https://usaco.org/index.php?page=viewproblem2&cpid=1236

\section*{Source}
USACO OPEN22 Contest, Platinum Division, Problem ID: 1236

\end{document}
