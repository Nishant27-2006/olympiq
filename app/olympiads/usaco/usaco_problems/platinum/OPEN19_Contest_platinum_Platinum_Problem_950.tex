\documentclass[12pt]{article}
\usepackage{amsmath}
\usepackage{amssymb}
\usepackage{graphicx}
\usepackage{enumitem}
\usepackage{hyperref}
\usepackage{xcolor}

\title{View problem}
\author{USACO Contest: OPEN19 Contest - Platinum Division}
\date{\today}

\begin{document}

\maketitle

\section*{Problem Statement}

Bessie likes sightseeing, and today she is looking for scenic valleys.

Of interest is an $N \times N$ grid of cells, where each cell has a height.
Every cell outside this square grid can be considered to have infinite height.

A valley is a region of this grid which is contiguous, has no holes, and is such
that every cell immediately surrounding it is higher than all cells in the
region.

More formally:
 A set of cells is called "edgewise-contiguous" if one can reach any cell of
the set from any other by a sequence of moves up, down, left, or right. A set of cells is called "pointwise-contiguous" if one can reach any cell
of the set from any other by a sequence of moves up, down, left, right, or
diagonally. A "region" is a non-empty edgewise-contiguous set of
cells.  A region is called "holey" if the complement of the region
(which includes the infinite cells outside the $N \times N$ grid) is not
pointwise-contiguous.  The "border" of a region is the set of cells
orthogonally adjacent (up, down,   left, or right) to some cell  in the region,
but which is not in the region itself.  A "valley" is any non-holey
region such that every cell in the region has height lower than every cell on
the region's border.
Bessie's goal is to determine the sum of the sizes of all valleys.

Examples
This is a region:

oo.
ooo
..o
This is not a region (the middle cell and the lower-right cell are not
edgewise-contiguous):

oo.
oo.
..o
This is a non-holey region:

ooo
o..
o..
This is a holey region (the single cell within the "donut" shape is not
pointwise-contiguous with the "outside" of the region):

ooo
o.o
ooo
This is another non-holey region (the single cell in the enter is
pointwise-contiguous with the cell in the lower-right corner):

ooo
o.o
oo.

INPUT FORMAT (file valleys.in):
First line contains integer $N$, where $1 \le N \le 750$.

Next $N$ lines each contain $N$ integers, the heights of the cells of the grid.
Each height $h$ will satisfy $1 \le h \le 10^6$. Every height will be a distinct
integer.

In at least 19% of the test cases, it is further guaranteed that $N \leq 100$.

OUTPUT FORMAT (file valleys.out):
Output a single integer, the sum of the sizes of all valleys.

SAMPLE INPUT:
3
1 10 2
20 100 30
3 11 50
SAMPLE OUTPUT: 
30

In this example, there are three valleys of size 1:

o.o
...
o..
One valley of size 2:

...
...
oo.
One valley of size 3:

ooo
...
...
One valley of size 6:

ooo
o..
oo.
One valley of size 7:

ooo
o.o
oo.
And one valley of size 9:

ooo
ooo
ooo
Thus, the answer is 1 + 1 + 1 + 2 + 3 + 6 + 7 + 9 = 30.


Problem credits: Travis Hance



\section*{Input Format}
Next $N$ lines each contain $N$ integers, the heights of the cells of the grid.
Each height $h$ will satisfy $1 \le h \le 10^6$. Every height will be a distinct
integer.In at least 19% of the test cases, it is further guaranteed that $N \leq 100$.

In at least 19% of the test cases, it is further guaranteed that $N \leq 100$.

OUTPUT FORMAT (file valleys.out):Output a single integer, the sum of the sizes of all valleys.SAMPLE INPUT:3
1 10 2
20 100 30
3 11 50SAMPLE OUTPUT:30In this example, there are three valleys of size 1:o.o
...
o..One valley of size 2:...
...
oo.One valley of size 3:ooo
...
...One valley of size 6:ooo
o..
oo.One valley of size 7:ooo
o.o
oo.And one valley of size 9:ooo
ooo
oooThus, the answer is 1 + 1 + 1 + 2 + 3 + 6 + 7 + 9 = 30.Problem credits: Travis Hance

\section*{Output Format}
SAMPLE INPUT:3
1 10 2
20 100 30
3 11 50SAMPLE OUTPUT:30In this example, there are three valleys of size 1:o.o
...
o..One valley of size 2:...
...
oo.One valley of size 3:ooo
...
...One valley of size 6:ooo
o..
oo.One valley of size 7:ooo
o.o
oo.And one valley of size 9:ooo
ooo
oooThus, the answer is 1 + 1 + 1 + 2 + 3 + 6 + 7 + 9 = 30.Problem credits: Travis Hance

\section*{Sample Input}
\begin{verbatim}
3
1 10 2
20 100 30
3 11 50
\end{verbatim}

\section*{Sample Output}
\begin{verbatim}
30

In this example, there are three valleys of size 1:o.o
...
o..One valley of size 2:...
...
oo.One valley of size 3:ooo
...
...One valley of size 6:ooo
o..
oo.One valley of size 7:ooo
o.o
oo.And one valley of size 9:ooo
ooo
oooThus, the answer is 1 + 1 + 1 + 2 + 3 + 6 + 7 + 9 = 30.Problem credits: Travis Hance

o.o
...
o..One valley of size 2:...
...
oo.One valley of size 3:ooo
...
...One valley of size 6:ooo
o..
oo.One valley of size 7:ooo
o.o
oo.And one valley of size 9:ooo
ooo
oooThus, the answer is 1 + 1 + 1 + 2 + 3 + 6 + 7 + 9 = 30.Problem credits: Travis Hance

o.o
...
o..

One valley of size 2:...
...
oo.One valley of size 3:ooo
...
...One valley of size 6:ooo
o..
oo.One valley of size 7:ooo
o.o
oo.And one valley of size 9:ooo
ooo
oooThus, the answer is 1 + 1 + 1 + 2 + 3 + 6 + 7 + 9 = 30.Problem credits: Travis Hance

...
...
oo.One valley of size 3:ooo
...
...One valley of size 6:ooo
o..
oo.One valley of size 7:ooo
o.o
oo.And one valley of size 9:ooo
ooo
oooThus, the answer is 1 + 1 + 1 + 2 + 3 + 6 + 7 + 9 = 30.Problem credits: Travis Hance

...
...
oo.

One valley of size 3:ooo
...
...One valley of size 6:ooo
o..
oo.One valley of size 7:ooo
o.o
oo.And one valley of size 9:ooo
ooo
oooThus, the answer is 1 + 1 + 1 + 2 + 3 + 6 + 7 + 9 = 30.Problem credits: Travis Hance

ooo
...
...One valley of size 6:ooo
o..
oo.One valley of size 7:ooo
o.o
oo.And one valley of size 9:ooo
ooo
oooThus, the answer is 1 + 1 + 1 + 2 + 3 + 6 + 7 + 9 = 30.Problem credits: Travis Hance

ooo
...
...

One valley of size 6:ooo
o..
oo.One valley of size 7:ooo
o.o
oo.And one valley of size 9:ooo
ooo
oooThus, the answer is 1 + 1 + 1 + 2 + 3 + 6 + 7 + 9 = 30.Problem credits: Travis Hance

ooo
o..
oo.One valley of size 7:ooo
o.o
oo.And one valley of size 9:ooo
ooo
oooThus, the answer is 1 + 1 + 1 + 2 + 3 + 6 + 7 + 9 = 30.Problem credits: Travis Hance

ooo
o..
oo.

One valley of size 7:ooo
o.o
oo.And one valley of size 9:ooo
ooo
oooThus, the answer is 1 + 1 + 1 + 2 + 3 + 6 + 7 + 9 = 30.Problem credits: Travis Hance

ooo
o.o
oo.And one valley of size 9:ooo
ooo
oooThus, the answer is 1 + 1 + 1 + 2 + 3 + 6 + 7 + 9 = 30.Problem credits: Travis Hance

ooo
o.o
oo.

And one valley of size 9:ooo
ooo
oooThus, the answer is 1 + 1 + 1 + 2 + 3 + 6 + 7 + 9 = 30.Problem credits: Travis Hance

ooo
ooo
oooThus, the answer is 1 + 1 + 1 + 2 + 3 + 6 + 7 + 9 = 30.Problem credits: Travis Hance

ooo
ooo
ooo

Thus, the answer is 1 + 1 + 1 + 2 + 3 + 6 + 7 + 9 = 30.Problem credits: Travis Hance

Problem credits: Travis Hance

Problem credits: Travis Hance
\end{verbatim}

\section*{Solution}


\section*{Problem URL}
https://usaco.org/index.php?page=viewproblem2&cpid=950

\section*{Source}
USACO OPEN19 Contest, Platinum Division, Problem ID: 950

\end{document}
