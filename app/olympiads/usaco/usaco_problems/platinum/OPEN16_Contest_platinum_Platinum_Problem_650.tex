\documentclass[12pt]{article}
\usepackage{amsmath}
\usepackage{amssymb}
\usepackage{graphicx}
\usepackage{enumitem}
\usepackage{hyperref}
\usepackage{xcolor}

\title{View problem}
\author{USACO Contest: OPEN16 Contest - Platinum Division}
\date{\today}

\begin{document}

\maketitle

\section*{Problem Statement}

Farmer John is building a nicely-landscaped garden, and needs to move a large
amount of dirt in the process.

The garden consists of a sequence of $N$ flowerbeds ($1 \leq N \leq 100,000$),
where flowerbed $i$ initially contains $A_i$ units of dirt.  Farmer John would
like to re-landscape the garden so that each flowerbed $i$ instead contains
$B_i$ units of dirt.  The $A_i$'s and $B_i$'s are all integers in the range
$0 \ldots 10$.

To landscape the garden, Farmer John has several options: he can purchase one
unit of dirt and place it in a flowerbed of his choice for $X$ units of money. 
He can remove one unit of dirt from a flowerbed of his choice and have it 
shipped away for $Y$ units of money.  He can also transport one unit of dirt
from  flowerbed $i$ to flowerbed $j$ at a cost of $Z$ times $|i-j|$.  Please
compute the minimum  total cost for Farmer John to complete his landscaping
project.

INPUT FORMAT (file landscape.in):
The first line of input contains $N$, $X$, $Y$, and $Z$
($0 \leq X, Y \le 10^8; 0 \le Z \leq 1000$).  Line $i+1$ contains the integers $A_i$ and $B_i$.

OUTPUT FORMAT (file landscape.out):
Please print the minimum total cost FJ needs to spend on landscaping.

SAMPLE INPUT:
4 100 200 1
1 4
2 3
3 2
4 0
SAMPLE OUTPUT: 
210

Note that this problem has been asked in a previous USACO contest,
at the silver level; however, the limits in the present version have
been raised considerably, so one should not expect many points from
the solution to the previous, easier version. 


Problem credits: Brian Dean



\section*{Input Format}
OUTPUT FORMAT (file landscape.out):Please print the minimum total cost FJ needs to spend on landscaping.SAMPLE INPUT:4 100 200 1
1 4
2 3
3 2
4 0SAMPLE OUTPUT:210Note that this problem has been asked in a previous USACO contest,
at the silver level; however, the limits in the present version have
been raised considerably, so one should not expect many points from
the solution to the previous, easier version.Problem credits: Brian Dean

\section*{Output Format}
SAMPLE INPUT:4 100 200 1
1 4
2 3
3 2
4 0SAMPLE OUTPUT:210Note that this problem has been asked in a previous USACO contest,
at the silver level; however, the limits in the present version have
been raised considerably, so one should not expect many points from
the solution to the previous, easier version.Problem credits: Brian Dean

\section*{Sample Input}
\begin{verbatim}
4 100 200 1
1 4
2 3
3 2
4 0
\end{verbatim}

\section*{Sample Output}
\begin{verbatim}
210

Note that this problem has been asked in a previous USACO contest,
at the silver level; however, the limits in the present version have
been raised considerably, so one should not expect many points from
the solution to the previous, easier version.Problem credits: Brian Dean

Problem credits: Brian Dean

Problem credits: Brian Dean
\end{verbatim}

\section*{Solution}


\section*{Problem URL}
https://usaco.org/index.php?page=viewproblem2&cpid=650

\section*{Source}
USACO OPEN16 Contest, Platinum Division, Problem ID: 650

\end{document}
