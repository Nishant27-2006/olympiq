\documentclass[12pt]{article}
\usepackage{amsmath}
\usepackage{amssymb}
\usepackage{graphicx}
\usepackage{enumitem}
\usepackage{hyperref}
\usepackage{xcolor}

\title{View problem}
\author{USACO Contest: DEC18 Contest - Platinum Division}
\date{\today}

\begin{document}

\maketitle

\section*{Problem Statement}

In order to save money for a new stall in her barn, Bessie the cow 
has started performing in the local circus, demonstrating her remarkable
sense of balance as she carefully walks back and forth on an elevated
balance beam!  

The amount of money Bessie earns in her performance is
related to where she manages to ultimately jump off the beam.  
The beam has positions labeled $0, 1, \ldots, N+1$ from left to right.  
If Bessie ever reaches $0$ or $N+1$ she falls off one of the ends
of the beam and sadly gets no payment.  

If Bessie is at a given position $k$, she can do either of the following: 

1. Flip a coin.  If she sees tails, she goes to position $k-1$, and if she
sees heads, she goes to position $k + 1$ (i.e. $\frac{1}{2}$ probability of either occurrence).

2. Jump off the beam and receive payment of $f(k)$ $(0 \leq f(k) \leq 10^9)$.

Bessie realizes that she may not be able to guarantee any particular 
payment outcome, since her movement is governed by random coin flips. 
However, based on the location where she starts, she wants to determine what her 
expected payment will be if she makes an optimal sequence of decisions ("optimal"
meaning that the decisions lead to the highest possible expected payment).
For example, if her strategy earns her payment of $10$ with probability $1/2$, 
$8$ with probability $1/4$, or $0$ with probability $1/4$, then her expected 
payment will be the weighted average $10(1/2) + 8(1/4) + 0(1/4) = 7$.

INPUT FORMAT (file balance.in):
The first line of input contains $N$ ($2 \leq N \leq 10^5$).  Each of the
remaining $N$ lines contain $f(1) \ldots f(N)$.

OUTPUT FORMAT (file balance.out):
Output $N$ lines.  On line $i$, print out $10^5$ times the expected value of
payment if Bessie starts at position $i$ and plays optimally, rounded down to
the nearest integer.

SAMPLE INPUT:
2
1
3
SAMPLE OUTPUT: 
150000
300000


Problem credits: Franklyn Wang and Spencer Compton



\section*{Input Format}
OUTPUT FORMAT (file balance.out):Output $N$ lines.  On line $i$, print out $10^5$ times the expected value of
payment if Bessie starts at position $i$ and plays optimally, rounded down to
the nearest integer.SAMPLE INPUT:2
1
3SAMPLE OUTPUT:150000
300000Problem credits: Franklyn Wang and Spencer Compton

\section*{Output Format}
SAMPLE INPUT:2
1
3SAMPLE OUTPUT:150000
300000Problem credits: Franklyn Wang and Spencer Compton

\section*{Sample Input}
\begin{verbatim}
2
1
3
\end{verbatim}

\section*{Sample Output}
\begin{verbatim}
150000
300000

Problem credits: Franklyn Wang and Spencer Compton

Problem credits: Franklyn Wang and Spencer Compton
\end{verbatim}

\section*{Solution}


\section*{Problem URL}
https://usaco.org/index.php?page=viewproblem2&cpid=864

\section*{Source}
USACO DEC18 Contest, Platinum Division, Problem ID: 864

\end{document}
