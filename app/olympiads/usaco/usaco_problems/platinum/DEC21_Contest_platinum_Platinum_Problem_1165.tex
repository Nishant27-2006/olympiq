\documentclass[12pt]{article}
\usepackage{amsmath}
\usepackage{amssymb}
\usepackage{graphicx}
\usepackage{enumitem}
\usepackage{hyperref}
\usepackage{xcolor}

\title{View problem}
\author{USACO Contest: DEC21 Contest - Platinum Division}
\date{\today}

\begin{document}

\maketitle

\section*{Problem Statement}

There are a total of $N$ ($1\le N\le 5000$) cows on the number line, each of
which is a  Holstein or a Guernsey. The breed of the $i$-th cow is given by
$b_i\in \{H,G\}$, the location of the $i$-th cow is given by $x_i$
($0 \leq x_i \leq 10^9$), and the weight of the  $i$-th cow is given by
$y_i$ ($1 \leq y_i \leq 10^5$).

At Farmer John's signal, some of the cows will form pairs such that 

Every pair consists of a Holstein $h$ and a Guernsey $g$ whose locations are
within $K$ of each other ($1\le K\le 10^9$); that is, $|x_h-x_g|\le K$.Every cow is either part of a single pair or not part of a pair.The pairing is maximal; that is, no two unpaired cows can form a
pair.
It's up to you to determine the range of possible sums of weights of the
unpaired cows. Specifically,

If $T=1$, compute the minimum possible sum of weights of the unpaired
cows.If $T=2$, compute the maximum possible sum of weights of the
unpaired cows.
INPUT FORMAT (input arrives from the terminal / stdin):
The first input line contains $T$, $N$, and $K$.

Following this are $N$ lines, the $i$-th of which contains $b_i,x_i,y_i$. It is
guaranteed that $0\le x_1< x_2< \cdots< x_N\le 10^9$.

OUTPUT FORMAT (print output to the terminal / stdout):
The minimum or maximum possible sum of weights of the unpaired cows.

SAMPLE INPUT:
2 5 4
G 1 1
H 3 4
G 4 2
H 6 6
H 8 9
SAMPLE OUTPUT: 
16

Cows $2$ and $3$ can pair up because they are at distance $1$, which is at most
$K = 4$. This pairing is maximal, because cow $1$, the only remaining Guernsey,
is at distance $5$ from cow $4$ and distance $7$ from cow $5$, which are more
than $K = 4$. The sum of weights of unpaired cows is
$1 + 6 + 9 = 16$.

SAMPLE INPUT:
1 5 4
G 1 1
H 3 4
G 4 2
H 6 6
H 8 9
SAMPLE OUTPUT: 
6

Cows $1$ and $2$ can pair up because they are at distance $2 \leq K = 4$, and
cows $3$ and $5$ can pair up because they are at distance $4 \leq K = 4$. This
pairing is maximal because only cow $4$ remains. The sum of weights of
unpaired cows is the weight of the only unpaired cow, which is simply $6$.

SAMPLE INPUT:
2 10 76
H 1 18
H 18 465
H 25 278
H 30 291
H 36 202
G 45 96
G 60 375
G 93 941
G 96 870
G 98 540
SAMPLE OUTPUT: 
1893

The answer to this example is $18+465+870+540=1893$.

SCORING:
Test cases 4-7 satisfy $T=1$.Test cases 8-14 satisfy $T=2$ and $N\le 300$.Test cases 15-22 satisfy $T=2$.

**Note: the memory limit for this problem is 512MB, twice the default.**

Problem credits: Benjamin Qi



\section*{Input Format}
Following this are $N$ lines, the $i$-th of which contains $b_i,x_i,y_i$. It is
guaranteed that $0\le x_1< x_2< \cdots< x_N\le 10^9$.

OUTPUT FORMAT (print output to the terminal / stdout):The minimum or maximum possible sum of weights of the unpaired cows.SAMPLE INPUT:2 5 4
G 1 1
H 3 4
G 4 2
H 6 6
H 8 9SAMPLE OUTPUT:16Cows $2$ and $3$ can pair up because they are at distance $1$, which is at most
$K = 4$. This pairing is maximal, because cow $1$, the only remaining Guernsey,
is at distance $5$ from cow $4$ and distance $7$ from cow $5$, which are more
than $K = 4$. The sum of weights of unpaired cows is
$1 + 6 + 9 = 16$.SAMPLE INPUT:1 5 4
G 1 1
H 3 4
G 4 2
H 6 6
H 8 9SAMPLE OUTPUT:6Cows $1$ and $2$ can pair up because they are at distance $2 \leq K = 4$, and
cows $3$ and $5$ can pair up because they are at distance $4 \leq K = 4$. This
pairing is maximal because only cow $4$ remains. The sum of weights of
unpaired cows is the weight of the only unpaired cow, which is simply $6$.SAMPLE INPUT:2 10 76
H 1 18
H 18 465
H 25 278
H 30 291
H 36 202
G 45 96
G 60 375
G 93 941
G 96 870
G 98 540SAMPLE OUTPUT:1893The answer to this example is $18+465+870+540=1893$.SCORING:Test cases 4-7 satisfy $T=1$.Test cases 8-14 satisfy $T=2$ and $N\le 300$.Test cases 15-22 satisfy $T=2$.**Note: the memory limit for this problem is 512MB, twice the default.**Problem credits: Benjamin Qi

\section*{Output Format}
SAMPLE INPUT:2 5 4
G 1 1
H 3 4
G 4 2
H 6 6
H 8 9SAMPLE OUTPUT:16Cows $2$ and $3$ can pair up because they are at distance $1$, which is at most
$K = 4$. This pairing is maximal, because cow $1$, the only remaining Guernsey,
is at distance $5$ from cow $4$ and distance $7$ from cow $5$, which are more
than $K = 4$. The sum of weights of unpaired cows is
$1 + 6 + 9 = 16$.SAMPLE INPUT:1 5 4
G 1 1
H 3 4
G 4 2
H 6 6
H 8 9SAMPLE OUTPUT:6Cows $1$ and $2$ can pair up because they are at distance $2 \leq K = 4$, and
cows $3$ and $5$ can pair up because they are at distance $4 \leq K = 4$. This
pairing is maximal because only cow $4$ remains. The sum of weights of
unpaired cows is the weight of the only unpaired cow, which is simply $6$.SAMPLE INPUT:2 10 76
H 1 18
H 18 465
H 25 278
H 30 291
H 36 202
G 45 96
G 60 375
G 93 941
G 96 870
G 98 540SAMPLE OUTPUT:1893The answer to this example is $18+465+870+540=1893$.SCORING:Test cases 4-7 satisfy $T=1$.Test cases 8-14 satisfy $T=2$ and $N\le 300$.Test cases 15-22 satisfy $T=2$.**Note: the memory limit for this problem is 512MB, twice the default.**Problem credits: Benjamin Qi

\section*{Sample Input}
\begin{verbatim}
2 5 4
G 1 1
H 3 4
G 4 2
H 6 6
H 8 9

1 5 4
G 1 1
H 3 4
G 4 2
H 6 6
H 8 9

2 10 76
H 1 18
H 18 465
H 25 278
H 30 291
H 36 202
G 45 96
G 60 375
G 93 941
G 96 870
G 98 540
\end{verbatim}

\section*{Sample Output}
\begin{verbatim}
16

Cows $2$ and $3$ can pair up because they are at distance $1$, which is at most
$K = 4$. This pairing is maximal, because cow $1$, the only remaining Guernsey,
is at distance $5$ from cow $4$ and distance $7$ from cow $5$, which are more
than $K = 4$. The sum of weights of unpaired cows is
$1 + 6 + 9 = 16$.SAMPLE INPUT:1 5 4
G 1 1
H 3 4
G 4 2
H 6 6
H 8 9SAMPLE OUTPUT:6Cows $1$ and $2$ can pair up because they are at distance $2 \leq K = 4$, and
cows $3$ and $5$ can pair up because they are at distance $4 \leq K = 4$. This
pairing is maximal because only cow $4$ remains. The sum of weights of
unpaired cows is the weight of the only unpaired cow, which is simply $6$.SAMPLE INPUT:2 10 76
H 1 18
H 18 465
H 25 278
H 30 291
H 36 202
G 45 96
G 60 375
G 93 941
G 96 870
G 98 540SAMPLE OUTPUT:1893The answer to this example is $18+465+870+540=1893$.SCORING:Test cases 4-7 satisfy $T=1$.Test cases 8-14 satisfy $T=2$ and $N\le 300$.Test cases 15-22 satisfy $T=2$.**Note: the memory limit for this problem is 512MB, twice the default.**Problem credits: Benjamin Qi

SAMPLE INPUT:1 5 4
G 1 1
H 3 4
G 4 2
H 6 6
H 8 9SAMPLE OUTPUT:6Cows $1$ and $2$ can pair up because they are at distance $2 \leq K = 4$, and
cows $3$ and $5$ can pair up because they are at distance $4 \leq K = 4$. This
pairing is maximal because only cow $4$ remains. The sum of weights of
unpaired cows is the weight of the only unpaired cow, which is simply $6$.SAMPLE INPUT:2 10 76
H 1 18
H 18 465
H 25 278
H 30 291
H 36 202
G 45 96
G 60 375
G 93 941
G 96 870
G 98 540SAMPLE OUTPUT:1893The answer to this example is $18+465+870+540=1893$.SCORING:Test cases 4-7 satisfy $T=1$.Test cases 8-14 satisfy $T=2$ and $N\le 300$.Test cases 15-22 satisfy $T=2$.**Note: the memory limit for this problem is 512MB, twice the default.**Problem credits: Benjamin Qi

6

Cows $1$ and $2$ can pair up because they are at distance $2 \leq K = 4$, and
cows $3$ and $5$ can pair up because they are at distance $4 \leq K = 4$. This
pairing is maximal because only cow $4$ remains. The sum of weights of
unpaired cows is the weight of the only unpaired cow, which is simply $6$.SAMPLE INPUT:2 10 76
H 1 18
H 18 465
H 25 278
H 30 291
H 36 202
G 45 96
G 60 375
G 93 941
G 96 870
G 98 540SAMPLE OUTPUT:1893The answer to this example is $18+465+870+540=1893$.SCORING:Test cases 4-7 satisfy $T=1$.Test cases 8-14 satisfy $T=2$ and $N\le 300$.Test cases 15-22 satisfy $T=2$.**Note: the memory limit for this problem is 512MB, twice the default.**Problem credits: Benjamin Qi

SAMPLE INPUT:2 10 76
H 1 18
H 18 465
H 25 278
H 30 291
H 36 202
G 45 96
G 60 375
G 93 941
G 96 870
G 98 540SAMPLE OUTPUT:1893The answer to this example is $18+465+870+540=1893$.SCORING:Test cases 4-7 satisfy $T=1$.Test cases 8-14 satisfy $T=2$ and $N\le 300$.Test cases 15-22 satisfy $T=2$.**Note: the memory limit for this problem is 512MB, twice the default.**Problem credits: Benjamin Qi

1893

The answer to this example is $18+465+870+540=1893$.SCORING:Test cases 4-7 satisfy $T=1$.Test cases 8-14 satisfy $T=2$ and $N\le 300$.Test cases 15-22 satisfy $T=2$.**Note: the memory limit for this problem is 512MB, twice the default.**Problem credits: Benjamin Qi

SCORING:Test cases 4-7 satisfy $T=1$.Test cases 8-14 satisfy $T=2$ and $N\le 300$.Test cases 15-22 satisfy $T=2$.**Note: the memory limit for this problem is 512MB, twice the default.**Problem credits: Benjamin Qi
\end{verbatim}

\section*{Solution}


\section*{Problem URL}
https://usaco.org/index.php?page=viewproblem2&cpid=1165

\section*{Source}
USACO DEC21 Contest, Platinum Division, Problem ID: 1165

\end{document}
