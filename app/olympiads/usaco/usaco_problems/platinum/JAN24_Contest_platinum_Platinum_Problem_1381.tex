\documentclass[12pt]{article}
\usepackage{amsmath}
\usepackage{amssymb}
\usepackage{graphicx}
\usepackage{enumitem}
\usepackage{hyperref}
\usepackage{xcolor}

\title{View problem}
\author{USACO Contest: JAN24 Contest - Platinum Division}
\date{\today}

\begin{document}

\maketitle

\section*{Problem Statement}


**Note: The memory limit for this problem is 512MB, twice the default.**
Bessie is having fun playing a famous online game, where there are a bunch of
cells of different labels and sizes. Cells get eaten by other cells until only
one winner remains.

There are $N$ ($2\le N\le 5000$) cells in a row labeled $1\dots N$ from left to
right, with initial sizes $s_1,s_2,\dots,s_N$ ($1\le s_i\le 10^5$). While there
is more than one cell, a pair of adjacent cells is selected uniformly at random
and merged into a single new cell according to the following rule: 

If a cell with label $a$ and current size $c_a$ is merged with a cell with label
$b$ and current size $c_b$, the resulting cell has size $c_a+c_b$ and label
equal to that of the larger cell, breaking ties by larger label. Formally, the
label of the resulting cell is
$\begin{cases}
a & c_a > c_b \\
b & c_a < c_b \\
\max(a,b) & c_a = c_b
\end{cases}.$

For each label $i$ in the range $1\dots N$, the probability that the final cell
has label $i$ can be expressed in the form $\frac{a_i}{b_i}$ where
$b_i\not\equiv 0\pmod{10^9+7}$. Output $a_ib_i^{-1}\pmod{10^9+7}$.

INPUT FORMAT (input arrives from the terminal / stdin):
The first line contains $N$.

The next line contains $s_1,s_2,\dots, s_N$.

OUTPUT FORMAT (print output to the terminal / stdout):
The probability of the final cell having label $i$ modulo $10^9+7$ for each $i$ in $1\dots N$ on
separate lines.

SAMPLE INPUT:
3
1 1 1
SAMPLE OUTPUT: 
0
500000004
500000004

There are two possibilities, where $(a,b)\to c$ means that the cells with labels
$a$ and $b$ merge into a new cell with label $c$.


(1, 2) -> 2, (2, 3) -> 2
(2, 3) -> 3, (1, 3) -> 3

So with probability $1/2$ the final cell has label 2 or 3.

SAMPLE INPUT:
4
3 1 1 1
SAMPLE OUTPUT: 
666666672
0
166666668
166666668

The six possibilities are as follows:


(1, 2) -> 1, (1, 3) -> 1, (1, 4) -> 1
(1, 2) -> 1, (3, 4) -> 4, (1, 4) -> 1
(2, 3) -> 3, (1, 3) -> 1, (1, 4) -> 1
(2, 3) -> 3, (3, 4) -> 3, (1, 3) -> 3
(3, 4) -> 4, (2, 4) -> 4, (1, 4) -> 4
(3, 4) -> 4, (1, 2) -> 1, (1, 4) -> 1

So with probability $2/3$ the final cell has label 1, and with probability $1/6$
the final cell has label 3 or 4.

SCORING:
Input 3: $N\le 8$Inputs 4-8: $N\le 100$Inputs 9-14: $N\le 500$Inputs 15-22: No additional constraints.


Problem credits: Benjamin Qi



\section*{Input Format}
The next line contains $s_1,s_2,\dots, s_N$.

OUTPUT FORMAT (print output to the terminal / stdout):The probability of the final cell having label $i$ modulo $10^9+7$ for each $i$ in $1\dots N$ on
separate lines.SAMPLE INPUT:3
1 1 1SAMPLE OUTPUT:0
500000004
500000004There are two possibilities, where $(a,b)\to c$ means that the cells with labels
$a$ and $b$ merge into a new cell with label $c$.(1, 2) -> 2, (2, 3) -> 2
(2, 3) -> 3, (1, 3) -> 3So with probability $1/2$ the final cell has label 2 or 3.SAMPLE INPUT:4
3 1 1 1SAMPLE OUTPUT:666666672
0
166666668
166666668The six possibilities are as follows:(1, 2) -> 1, (1, 3) -> 1, (1, 4) -> 1
(1, 2) -> 1, (3, 4) -> 4, (1, 4) -> 1
(2, 3) -> 3, (1, 3) -> 1, (1, 4) -> 1
(2, 3) -> 3, (3, 4) -> 3, (1, 3) -> 3
(3, 4) -> 4, (2, 4) -> 4, (1, 4) -> 4
(3, 4) -> 4, (1, 2) -> 1, (1, 4) -> 1So with probability $2/3$ the final cell has label 1, and with probability $1/6$
the final cell has label 3 or 4.SCORING:Input 3: $N\le 8$Inputs 4-8: $N\le 100$Inputs 9-14: $N\le 500$Inputs 15-22: No additional constraints.Problem credits: Benjamin Qi

\section*{Output Format}
SAMPLE INPUT:3
1 1 1SAMPLE OUTPUT:0
500000004
500000004There are two possibilities, where $(a,b)\to c$ means that the cells with labels
$a$ and $b$ merge into a new cell with label $c$.(1, 2) -> 2, (2, 3) -> 2
(2, 3) -> 3, (1, 3) -> 3So with probability $1/2$ the final cell has label 2 or 3.SAMPLE INPUT:4
3 1 1 1SAMPLE OUTPUT:666666672
0
166666668
166666668The six possibilities are as follows:(1, 2) -> 1, (1, 3) -> 1, (1, 4) -> 1
(1, 2) -> 1, (3, 4) -> 4, (1, 4) -> 1
(2, 3) -> 3, (1, 3) -> 1, (1, 4) -> 1
(2, 3) -> 3, (3, 4) -> 3, (1, 3) -> 3
(3, 4) -> 4, (2, 4) -> 4, (1, 4) -> 4
(3, 4) -> 4, (1, 2) -> 1, (1, 4) -> 1So with probability $2/3$ the final cell has label 1, and with probability $1/6$
the final cell has label 3 or 4.SCORING:Input 3: $N\le 8$Inputs 4-8: $N\le 100$Inputs 9-14: $N\le 500$Inputs 15-22: No additional constraints.Problem credits: Benjamin Qi

\section*{Sample Input}
\begin{verbatim}
3
1 1 1

4
3 1 1 1
\end{verbatim}

\section*{Sample Output}
\begin{verbatim}
0
500000004
500000004

There are two possibilities, where $(a,b)\to c$ means that the cells with labels
$a$ and $b$ merge into a new cell with label $c$.(1, 2) -> 2, (2, 3) -> 2
(2, 3) -> 3, (1, 3) -> 3So with probability $1/2$ the final cell has label 2 or 3.SAMPLE INPUT:4
3 1 1 1SAMPLE OUTPUT:666666672
0
166666668
166666668The six possibilities are as follows:(1, 2) -> 1, (1, 3) -> 1, (1, 4) -> 1
(1, 2) -> 1, (3, 4) -> 4, (1, 4) -> 1
(2, 3) -> 3, (1, 3) -> 1, (1, 4) -> 1
(2, 3) -> 3, (3, 4) -> 3, (1, 3) -> 3
(3, 4) -> 4, (2, 4) -> 4, (1, 4) -> 4
(3, 4) -> 4, (1, 2) -> 1, (1, 4) -> 1So with probability $2/3$ the final cell has label 1, and with probability $1/6$
the final cell has label 3 or 4.SCORING:Input 3: $N\le 8$Inputs 4-8: $N\le 100$Inputs 9-14: $N\le 500$Inputs 15-22: No additional constraints.Problem credits: Benjamin Qi

(1, 2) -> 2, (2, 3) -> 2
(2, 3) -> 3, (1, 3) -> 3So with probability $1/2$ the final cell has label 2 or 3.SAMPLE INPUT:4
3 1 1 1SAMPLE OUTPUT:666666672
0
166666668
166666668The six possibilities are as follows:(1, 2) -> 1, (1, 3) -> 1, (1, 4) -> 1
(1, 2) -> 1, (3, 4) -> 4, (1, 4) -> 1
(2, 3) -> 3, (1, 3) -> 1, (1, 4) -> 1
(2, 3) -> 3, (3, 4) -> 3, (1, 3) -> 3
(3, 4) -> 4, (2, 4) -> 4, (1, 4) -> 4
(3, 4) -> 4, (1, 2) -> 1, (1, 4) -> 1So with probability $2/3$ the final cell has label 1, and with probability $1/6$
the final cell has label 3 or 4.SCORING:Input 3: $N\le 8$Inputs 4-8: $N\le 100$Inputs 9-14: $N\le 500$Inputs 15-22: No additional constraints.Problem credits: Benjamin Qi

(1, 2) -> 2, (2, 3) -> 2
(2, 3) -> 3, (1, 3) -> 3

So with probability $1/2$ the final cell has label 2 or 3.SAMPLE INPUT:4
3 1 1 1SAMPLE OUTPUT:666666672
0
166666668
166666668The six possibilities are as follows:(1, 2) -> 1, (1, 3) -> 1, (1, 4) -> 1
(1, 2) -> 1, (3, 4) -> 4, (1, 4) -> 1
(2, 3) -> 3, (1, 3) -> 1, (1, 4) -> 1
(2, 3) -> 3, (3, 4) -> 3, (1, 3) -> 3
(3, 4) -> 4, (2, 4) -> 4, (1, 4) -> 4
(3, 4) -> 4, (1, 2) -> 1, (1, 4) -> 1So with probability $2/3$ the final cell has label 1, and with probability $1/6$
the final cell has label 3 or 4.SCORING:Input 3: $N\le 8$Inputs 4-8: $N\le 100$Inputs 9-14: $N\le 500$Inputs 15-22: No additional constraints.Problem credits: Benjamin Qi

SAMPLE INPUT:4
3 1 1 1SAMPLE OUTPUT:666666672
0
166666668
166666668The six possibilities are as follows:(1, 2) -> 1, (1, 3) -> 1, (1, 4) -> 1
(1, 2) -> 1, (3, 4) -> 4, (1, 4) -> 1
(2, 3) -> 3, (1, 3) -> 1, (1, 4) -> 1
(2, 3) -> 3, (3, 4) -> 3, (1, 3) -> 3
(3, 4) -> 4, (2, 4) -> 4, (1, 4) -> 4
(3, 4) -> 4, (1, 2) -> 1, (1, 4) -> 1So with probability $2/3$ the final cell has label 1, and with probability $1/6$
the final cell has label 3 or 4.SCORING:Input 3: $N\le 8$Inputs 4-8: $N\le 100$Inputs 9-14: $N\le 500$Inputs 15-22: No additional constraints.Problem credits: Benjamin Qi

666666672
0
166666668
166666668

The six possibilities are as follows:(1, 2) -> 1, (1, 3) -> 1, (1, 4) -> 1
(1, 2) -> 1, (3, 4) -> 4, (1, 4) -> 1
(2, 3) -> 3, (1, 3) -> 1, (1, 4) -> 1
(2, 3) -> 3, (3, 4) -> 3, (1, 3) -> 3
(3, 4) -> 4, (2, 4) -> 4, (1, 4) -> 4
(3, 4) -> 4, (1, 2) -> 1, (1, 4) -> 1So with probability $2/3$ the final cell has label 1, and with probability $1/6$
the final cell has label 3 or 4.SCORING:Input 3: $N\le 8$Inputs 4-8: $N\le 100$Inputs 9-14: $N\le 500$Inputs 15-22: No additional constraints.Problem credits: Benjamin Qi

(1, 2) -> 1, (1, 3) -> 1, (1, 4) -> 1
(1, 2) -> 1, (3, 4) -> 4, (1, 4) -> 1
(2, 3) -> 3, (1, 3) -> 1, (1, 4) -> 1
(2, 3) -> 3, (3, 4) -> 3, (1, 3) -> 3
(3, 4) -> 4, (2, 4) -> 4, (1, 4) -> 4
(3, 4) -> 4, (1, 2) -> 1, (1, 4) -> 1So with probability $2/3$ the final cell has label 1, and with probability $1/6$
the final cell has label 3 or 4.SCORING:Input 3: $N\le 8$Inputs 4-8: $N\le 100$Inputs 9-14: $N\le 500$Inputs 15-22: No additional constraints.Problem credits: Benjamin Qi

(1, 2) -> 1, (1, 3) -> 1, (1, 4) -> 1
(1, 2) -> 1, (3, 4) -> 4, (1, 4) -> 1
(2, 3) -> 3, (1, 3) -> 1, (1, 4) -> 1
(2, 3) -> 3, (3, 4) -> 3, (1, 3) -> 3
(3, 4) -> 4, (2, 4) -> 4, (1, 4) -> 4
(3, 4) -> 4, (1, 2) -> 1, (1, 4) -> 1

So with probability $2/3$ the final cell has label 1, and with probability $1/6$
the final cell has label 3 or 4.SCORING:Input 3: $N\le 8$Inputs 4-8: $N\le 100$Inputs 9-14: $N\le 500$Inputs 15-22: No additional constraints.Problem credits: Benjamin Qi

SCORING:Input 3: $N\le 8$Inputs 4-8: $N\le 100$Inputs 9-14: $N\le 500$Inputs 15-22: No additional constraints.Problem credits: Benjamin Qi
\end{verbatim}

\section*{Solution}


\section*{Problem URL}
https://usaco.org/index.php?page=viewproblem2&cpid=1381

\section*{Source}
USACO JAN24 Contest, Platinum Division, Problem ID: 1381

\end{document}
