\documentclass[12pt]{article}
\usepackage{amsmath}
\usepackage{amssymb}
\usepackage{graphicx}
\usepackage{enumitem}
\usepackage{hyperref}
\usepackage{xcolor}

\title{View problem}
\author{USACO Contest: FEB23 Contest - Platinum Division}
\date{\today}

\begin{document}

\maketitle

\section*{Problem Statement}


**Note: The memory limit for this problem is 512MB, twice the default.**
Farmer John created $N$ ($1\le N\le 10^5$) problems. He then recruited $M$
($1\le M\le 20$) test-solvers, each of which rated every problem as "easy" or
"hard." 

His goal is now to create a problemset arranged in increasing order of
difficulty, consisting of some subset of his $N$ problems arranged in some
order. There must exist no pair of problems such that some test-solver thinks
the problem later in the order is easy but the problem earlier in the order is
hard. 

Count the number of distinct nonempty problemsets he can form, modulo $10^9+7$.

INPUT FORMAT (input arrives from the terminal / stdin):
The first line contains $N$ and $M$.

The next $M$ lines each contain a string of length $N$. The $i$th character of
this  string is E if the test-solver thinks the $i$th problem is easy, or H
otherwise.

OUTPUT FORMAT (print output to the terminal / stdout):
The number of distinct problemsets FJ can form, modulo $10^9+7$.


SAMPLE INPUT:
3 1
EHE
SAMPLE OUTPUT: 
9

The nine possible problemsets are as follows:


[1]
[1,2]
[1,3]
[1,3,2]
[2]
[3]
[3,1]
[3,2]
[3,1,2]

Note that the order of the problems within the problemset matters.

SAMPLE INPUT:
10 6
EHEEEHHEEH
EHHHEEHHHE
EHEHEHEEHH
HEHEEEHEEE
HHEEHEEEHE
EHHEEEEEHE
SAMPLE OUTPUT: 
33

SCORING:
Inputs 3-4: $M=1$Inputs 5-14: $M\le 16$Inputs 15-22: No additional constraints.


Problem credits: Benjamin Qi



\section*{Input Format}
The next $M$ lines each contain a string of length $N$. The $i$th character of
this  string is E if the test-solver thinks the $i$th problem is easy, or H
otherwise.

OUTPUT FORMAT (print output to the terminal / stdout):The number of distinct problemsets FJ can form, modulo $10^9+7$.SAMPLE INPUT:3 1
EHESAMPLE OUTPUT:9The nine possible problemsets are as follows:[1]
[1,2]
[1,3]
[1,3,2]
[2]
[3]
[3,1]
[3,2]
[3,1,2]Note that the order of the problems within the problemset matters.SAMPLE INPUT:10 6
EHEEEHHEEH
EHHHEEHHHE
EHEHEHEEHH
HEHEEEHEEE
HHEEHEEEHE
EHHEEEEEHESAMPLE OUTPUT:33SCORING:Inputs 3-4: $M=1$Inputs 5-14: $M\le 16$Inputs 15-22: No additional constraints.Problem credits: Benjamin Qi

\section*{Output Format}
SAMPLE INPUT:3 1
EHESAMPLE OUTPUT:9The nine possible problemsets are as follows:[1]
[1,2]
[1,3]
[1,3,2]
[2]
[3]
[3,1]
[3,2]
[3,1,2]Note that the order of the problems within the problemset matters.SAMPLE INPUT:10 6
EHEEEHHEEH
EHHHEEHHHE
EHEHEHEEHH
HEHEEEHEEE
HHEEHEEEHE
EHHEEEEEHESAMPLE OUTPUT:33SCORING:Inputs 3-4: $M=1$Inputs 5-14: $M\le 16$Inputs 15-22: No additional constraints.Problem credits: Benjamin Qi

\section*{Sample Input}
\begin{verbatim}
3 1
EHE

10 6
EHEEEHHEEH
EHHHEEHHHE
EHEHEHEEHH
HEHEEEHEEE
HHEEHEEEHE
EHHEEEEEHE
\end{verbatim}

\section*{Sample Output}
\begin{verbatim}
9

The nine possible problemsets are as follows:[1]
[1,2]
[1,3]
[1,3,2]
[2]
[3]
[3,1]
[3,2]
[3,1,2]Note that the order of the problems within the problemset matters.SAMPLE INPUT:10 6
EHEEEHHEEH
EHHHEEHHHE
EHEHEHEEHH
HEHEEEHEEE
HHEEHEEEHE
EHHEEEEEHESAMPLE OUTPUT:33SCORING:Inputs 3-4: $M=1$Inputs 5-14: $M\le 16$Inputs 15-22: No additional constraints.Problem credits: Benjamin Qi

[1]
[1,2]
[1,3]
[1,3,2]
[2]
[3]
[3,1]
[3,2]
[3,1,2]Note that the order of the problems within the problemset matters.SAMPLE INPUT:10 6
EHEEEHHEEH
EHHHEEHHHE
EHEHEHEEHH
HEHEEEHEEE
HHEEHEEEHE
EHHEEEEEHESAMPLE OUTPUT:33SCORING:Inputs 3-4: $M=1$Inputs 5-14: $M\le 16$Inputs 15-22: No additional constraints.Problem credits: Benjamin Qi

[1]
[1,2]
[1,3]
[1,3,2]
[2]
[3]
[3,1]
[3,2]
[3,1,2]

Note that the order of the problems within the problemset matters.SAMPLE INPUT:10 6
EHEEEHHEEH
EHHHEEHHHE
EHEHEHEEHH
HEHEEEHEEE
HHEEHEEEHE
EHHEEEEEHESAMPLE OUTPUT:33SCORING:Inputs 3-4: $M=1$Inputs 5-14: $M\le 16$Inputs 15-22: No additional constraints.Problem credits: Benjamin Qi

SAMPLE INPUT:10 6
EHEEEHHEEH
EHHHEEHHHE
EHEHEHEEHH
HEHEEEHEEE
HHEEHEEEHE
EHHEEEEEHESAMPLE OUTPUT:33SCORING:Inputs 3-4: $M=1$Inputs 5-14: $M\le 16$Inputs 15-22: No additional constraints.Problem credits: Benjamin Qi

33

SCORING:Inputs 3-4: $M=1$Inputs 5-14: $M\le 16$Inputs 15-22: No additional constraints.Problem credits: Benjamin Qi
\end{verbatim}

\section*{Solution}


\section*{Problem URL}
https://usaco.org/index.php?page=viewproblem2&cpid=1309

\section*{Source}
USACO FEB23 Contest, Platinum Division, Problem ID: 1309

\end{document}
