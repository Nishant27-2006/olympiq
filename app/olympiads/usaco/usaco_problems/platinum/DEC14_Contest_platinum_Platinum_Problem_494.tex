\documentclass[12pt]{article}
\usepackage{amsmath}
\usepackage{amssymb}
\usepackage{graphicx}
\usepackage{enumitem}
\usepackage{hyperref}
\usepackage{xcolor}

\title{View problem}
\author{USACO Contest: DEC14 Contest - Platinum Division}
\date{\today}

\begin{document}

\maketitle

\section*{Problem Statement}

Problem 1: Guard Mark [Bill Cooperman, 2014]



Farmer John and his herd are playing frisbee.  Bessie throws the

frisbee down the field, but it's going straight to Mark the field hand

on the other team!  Mark has height H (1 <= H <= 1,000,000,000), but

there are N cows on Bessie's team gathered around Mark (2 <= N <= 20).

They can only catch the frisbee if they can stack up to be at least as

high as Mark.  Each of the N cows has a height, weight, and strength.

A cow's strength indicates the maximum amount of total weight of the

cows that can be stacked above her.  



Given these constraints, Bessie wants to know if it is possible for

her team to build a tall enough stack to catch the frisbee, and if so,

what is the maximum safety factor of such a stack.  The safety factor

of a stack is the amount of weight that can be added to the top of the

stack without exceeding any cow's strength.



INPUT: (file guard.in)



The first line of input contains N and H.



The next N lines of input each describe a cow, giving its height,

weight, and strength.  All are positive integers at most 1 billion.



SAMPLE INPUT:



4 10

9 4 1

3 3 5

5 5 10

4 4 5



OUTPUT: (file guard.out)



If Bessie's team can build a stack tall enough to catch the frisbee,

please output the maximum achievable safety factor for such a stack.

Otherwise output "Mark is too tall" (without the quotes).



SAMPLE OUTPUT:



2




\section*{Input Format}


\section*{Output Format}


\section*{Sample Input}
\begin{verbatim}

\end{verbatim}

\section*{Sample Output}
\begin{verbatim}

\end{verbatim}

\section*{Solution}


\section*{Problem URL}
https://usaco.org/index.php?page=viewproblem2&cpid=494

\section*{Source}
USACO DEC14 Contest, Platinum Division, Problem ID: 494

\end{document}
