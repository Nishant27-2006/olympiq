\documentclass[12pt]{article}
\usepackage{amsmath}
\usepackage{amssymb}
\usepackage{graphicx}
\usepackage{enumitem}
\usepackage{hyperref}
\usepackage{xcolor}

\title{View problem}
\author{USACO Contest: FEB18 Contest - Platinum Division}
\date{\today}

\begin{document}

\maketitle

\section*{Problem Statement}

One of the farming chores Farmer John dislikes the most is hauling around lots
of cow manure.  In order to streamline this process, he comes up with an
intriguing idea: instead of hauling manure between two points in a cart behind
his tractor, why not shoot it through the air with a giant manure slingshot? 
(indeed, what could possibly go wrong...)

Farmer John's farm is built along a single long straight road, so any location
on his farm can be described simply using its position along this road
(effectively a point on the number line).  FJ builds $N$ slingshots
($1 \leq N \leq 10^5$), where the $i$th slingshot is described by three integers
$x_i$, $y_i$, and $t_i$, specifying that this slingshot can shoot manure from
position $x_i$ to position $y_i$ in only $t_i$ total units of time.  

FJ has $M$ piles of manure to transport ($1 \leq M \leq 10^5$).  The $j$th such
pile needs to be moved from position $a_j$ to position $b_j$.  Hauling manure
with the tractor for a distance of $d$ takes $d$ units of time.  FJ is hoping to
reduce this by allowing up to one use of any slingshot for transporting each
pile of manure.  Time FJ spends moving his tractor without manure in it does
not count.

For each of the $M$ manure piles, please help FJ determine the minimum possible
transportation time, given that FJ can use up to one slingshot during the
process.

INPUT FORMAT (file slingshot.in):
The first line of input contains $N$ and $M$.  The next $N$ lines each describe
a single slingshot in terms of integers $x_i$, $y_i$, and $t_i$
($0 \leq x_i, y_i, t_i \leq 10^9$). The final $M$ lines describe piles of manure
that need to be moved, in terms of integers $a_j$ and $b_j$.

OUTPUT FORMAT (file slingshot.out):
Print $M$ lines of output, one for each manure pile, indicating the minimum time
needed to transport it.

SAMPLE INPUT:
2 3
0 10 1
13 8 2
1 12
5 2
20 7
SAMPLE OUTPUT: 
4
3
10

Here, the first pile of manure needs to move from position 1 to position 12. 
Without using an slingshot, this would take 11 units of time.  However, using
the first slingshot, it takes 1 unit of time to move to position 0 (the
slingshot source), 1 unit of time to fling the manure through the air to land at
position 10 (the slingshot destination), and then 2 units of time to move the
manure to position 12.  The second pile of manure is best moved without any
slingshot, and the third pile of manure should be moved using the second
slingshot.


Problem credits: Brian Dean



\section*{Input Format}
OUTPUT FORMAT (file slingshot.out):Print $M$ lines of output, one for each manure pile, indicating the minimum time
needed to transport it.SAMPLE INPUT:2 3
0 10 1
13 8 2
1 12
5 2
20 7SAMPLE OUTPUT:4
3
10Here, the first pile of manure needs to move from position 1 to position 12. 
Without using an slingshot, this would take 11 units of time.  However, using
the first slingshot, it takes 1 unit of time to move to position 0 (the
slingshot source), 1 unit of time to fling the manure through the air to land at
position 10 (the slingshot destination), and then 2 units of time to move the
manure to position 12.  The second pile of manure is best moved without any
slingshot, and the third pile of manure should be moved using the second
slingshot.Problem credits: Brian Dean

\section*{Output Format}
SAMPLE INPUT:2 3
0 10 1
13 8 2
1 12
5 2
20 7SAMPLE OUTPUT:4
3
10Here, the first pile of manure needs to move from position 1 to position 12. 
Without using an slingshot, this would take 11 units of time.  However, using
the first slingshot, it takes 1 unit of time to move to position 0 (the
slingshot source), 1 unit of time to fling the manure through the air to land at
position 10 (the slingshot destination), and then 2 units of time to move the
manure to position 12.  The second pile of manure is best moved without any
slingshot, and the third pile of manure should be moved using the second
slingshot.Problem credits: Brian Dean

\section*{Sample Input}
\begin{verbatim}
2 3
0 10 1
13 8 2
1 12
5 2
20 7
\end{verbatim}

\section*{Sample Output}
\begin{verbatim}
4
3
10

Here, the first pile of manure needs to move from position 1 to position 12. 
Without using an slingshot, this would take 11 units of time.  However, using
the first slingshot, it takes 1 unit of time to move to position 0 (the
slingshot source), 1 unit of time to fling the manure through the air to land at
position 10 (the slingshot destination), and then 2 units of time to move the
manure to position 12.  The second pile of manure is best moved without any
slingshot, and the third pile of manure should be moved using the second
slingshot.Problem credits: Brian Dean

Problem credits: Brian Dean

Problem credits: Brian Dean
\end{verbatim}

\section*{Solution}


\section*{Problem URL}
https://usaco.org/index.php?page=viewproblem2&cpid=816

\section*{Source}
USACO FEB18 Contest, Platinum Division, Problem ID: 816

\end{document}
