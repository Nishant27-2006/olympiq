\documentclass[12pt]{article}
\usepackage{amsmath}
\usepackage{amssymb}
\usepackage{graphicx}
\usepackage{enumitem}
\usepackage{hyperref}
\usepackage{xcolor}

\title{View problem}
\author{USACO Contest: FEB17 Contest - Platinum Division}
\date{\today}

\begin{document}

\maketitle

\section*{Problem Statement}

Why did the cow cross the road?  We may never know the full reason, but it is
certain that Farmer John's cows do end up crossing the road quite frequently. 
In fact, they end up crossing the road so often that they often bump into
each-other when their paths cross, a situation Farmer John would like to remedy.

Farmer John raises $N$ breeds of cows ($1 \leq N \leq 100,000$), and each of his
fields is dedicated to grazing for one specific breed; for example, a field
dedicated to breed 12 can only be used for cows of breed 12 and not of any other
breed. A long road runs through his farm.  There is a sequence of $N$ fields on
one side of the road (one for each breed), and a sequence of $N$ fields on the
other side of the road (also one for each breed). When a cow crosses the road,
she therefore crosses between the two fields designated for her specific breed.

Had Farmer John planned more carefully, he would have ordered the fields by
breed the same way on both sides of the road, so the two fields for each breed
would be directly across the road from each-other.  This would have  allowed
cows to cross the road without any cows from different breeds bumping into
one-another.  Alas, the orderings on both sides of the road might be different,
so Farmer John observes that there might be pairs of breeds that cross.  A pair
of different breeds $(a,b)$ is "crossing" if any path across the  road for breed
$a$ must intersect any path across the road for breed $b$.  

Farmer John would like to minimize the number of crossing pairs of breeds.  For
logistical reasons, he figures he can move cows around on one side of the road
so the fields on that side undergo a "cyclic shift". That is, for some
$0 \leq k < N$, every cow re-locates to the field $k$ fields ahead of it, with
the cows in the last $k$ fields moving so they now populate the first $k$
fields.  For example, if the fields on one side of the road start out ordered by
breed as 3, 7, 1, 2, 5, 4, 6 and undergo a cyclic shift by $k=2$, the new order
will be 4, 6, 3, 7, 1, 2, 5.  Please determine the minimum possible number of
crossing pairs of breeds that can exist after an appropriate cyclic shift of the
fields on one side of the road.

INPUT FORMAT (file mincross.in):
The first line of input contains $N$.  The next $N$ lines describe the order, by
breed ID, of fields on one side of the road; each breed ID is an integer in the
range $1 \ldots N$.  The last $N$ lines describe the order, by  breed ID, of the
fields on the other side of the road.

OUTPUT FORMAT (file mincross.out):
Please output the minimum number of crossing pairs of breeds after a cyclic
shift of the fields on one side of the road (either side can be shifted).

SAMPLE INPUT:
5
5
4
1
3
2
1
3
2
5
4
SAMPLE OUTPUT: 
0


Problem credits: Brian Dean



\section*{Input Format}
OUTPUT FORMAT (file mincross.out):Please output the minimum number of crossing pairs of breeds after a cyclic
shift of the fields on one side of the road (either side can be shifted).SAMPLE INPUT:5
5
4
1
3
2
1
3
2
5
4SAMPLE OUTPUT:0Problem credits: Brian Dean

\section*{Output Format}
SAMPLE INPUT:5
5
4
1
3
2
1
3
2
5
4SAMPLE OUTPUT:0Problem credits: Brian Dean

\section*{Sample Input}
\begin{verbatim}
5
5
4
1
3
2
1
3
2
5
4
\end{verbatim}

\section*{Sample Output}
\begin{verbatim}
0

Problem credits: Brian Dean

Problem credits: Brian Dean
\end{verbatim}

\section*{Solution}


\section*{Problem URL}
https://usaco.org/index.php?page=viewproblem2&cpid=720

\section*{Source}
USACO FEB17 Contest, Platinum Division, Problem ID: 720

\end{document}
