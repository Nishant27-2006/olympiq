\documentclass[12pt]{article}
\usepackage{amsmath}
\usepackage{amssymb}
\usepackage{graphicx}
\usepackage{enumitem}
\usepackage{hyperref}
\usepackage{xcolor}

\title{View problem}
\author{USACO Contest: DEC16 Contest - Platinum Division}
\date{\today}

\begin{document}

\maketitle

\section*{Problem Statement}

Farmer John is thinking of selling some of his land to earn a bit of extra
income.  His property contains $n$ trees ($3 \leq N \leq 300$), each described
by a point in the 2D plane, no three of which are collinear.  FJ is thinking
about selling triangular lots of land defined by having trees at their vertices;
there are of course $L = \binom{N}{3}$ such lots he can consider, based on all
possible triples of trees on his property.  

A triangular lot has value $v$ if it contains exactly $v$ trees in its interior
(the trees on the corners do not count, and note that there are no trees on the
boundaries since no three trees are collinear).  For every $v = 0 \ldots N-3$,
please help FJ determine how many of his $L$ potential lots have value $v$.

INPUT FORMAT (file triangles.in):
The first line of input contains $N$.

The following $N$ lines contain the $x$ and $y$ coordinates of a single tree;
these are both integers in the range $0 \ldots 1,000,000$.

OUTPUT FORMAT (file triangles.out):
Output $N-2$ lines, where output line $i$ contains a count of the number of lots
having value $i-1$.

SAMPLE INPUT:
7
3 6
17 15
13 15
6 12
9 1
2 7
10 19
SAMPLE OUTPUT: 
28
6
1
0
0


Problem credits: Lewin Gan



\section*{Input Format}
The following $N$ lines contain the $x$ and $y$ coordinates of a single tree;
these are both integers in the range $0 \ldots 1,000,000$.

OUTPUT FORMAT (file triangles.out):Output $N-2$ lines, where output line $i$ contains a count of the number of lots
having value $i-1$.SAMPLE INPUT:7
3 6
17 15
13 15
6 12
9 1
2 7
10 19SAMPLE OUTPUT:28
6
1
0
0Problem credits: Lewin Gan

\section*{Output Format}
SAMPLE INPUT:7
3 6
17 15
13 15
6 12
9 1
2 7
10 19SAMPLE OUTPUT:28
6
1
0
0Problem credits: Lewin Gan

\section*{Sample Input}
\begin{verbatim}
7
3 6
17 15
13 15
6 12
9 1
2 7
10 19
\end{verbatim}

\section*{Sample Output}
\begin{verbatim}
28
6
1
0
0

Problem credits: Lewin Gan

Problem credits: Lewin Gan
\end{verbatim}

\section*{Solution}


\section*{Problem URL}
https://usaco.org/index.php?page=viewproblem2&cpid=672

\section*{Source}
USACO DEC16 Contest, Platinum Division, Problem ID: 672

\end{document}
