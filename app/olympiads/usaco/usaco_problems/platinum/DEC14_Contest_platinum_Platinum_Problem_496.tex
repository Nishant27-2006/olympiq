\documentclass[12pt]{article}
\usepackage{amsmath}
\usepackage{amssymb}
\usepackage{graphicx}
\usepackage{enumitem}
\usepackage{hyperref}
\usepackage{xcolor}

\title{View problem}
\author{USACO Contest: DEC14 Contest - Platinum Division}
\date{\today}

\begin{document}

\maketitle

\section*{Problem Statement}

Problem 3: Cow Jog [Mark Gordon, 2014]



Farmer John's N cows (1 <= N <= 100,000) are out exercising their

hooves again, jogging along an infinite track.  Each cow starts at a

distinct position on the track, and some cows run at different speeds.



The track is divided into lanes so that cows may move past each other.

No two cows in the same lane may ever occupy the same position.

Farmer John doesn't want any cow to have to change lanes or adjust

speed, and he wonders how many lanes he will need to accomplish this

if the cows are going to run for T minutes (1 <= T <= 1,000,000,000).



INPUT: (file cowjog.in)



The first line of input contains N and T.



The following N lines each contain the initial position and speed of a

single cow.  Position is a nonnegative integer and speed is a positive

integer; both numbers are at most 1 billion.  All cows start at

distinct positions, and these will be given in increasing order in the

input.



SAMPLE INPUT:



5 3

0 1

1 2

2 3

3 2

6 1



OUTPUT: (file cowjog.out)



A single integer indicating the minimum number of lanes necessary so

that no two cows in the same lane ever occupy the same location

(including at time T).



SAMPLE OUTPUT:



3




\section*{Input Format}


\section*{Output Format}


\section*{Sample Input}
\begin{verbatim}

\end{verbatim}

\section*{Sample Output}
\begin{verbatim}

\end{verbatim}

\section*{Solution}


\section*{Problem URL}
https://usaco.org/index.php?page=viewproblem2&cpid=496

\section*{Source}
USACO DEC14 Contest, Platinum Division, Problem ID: 496

\end{document}
