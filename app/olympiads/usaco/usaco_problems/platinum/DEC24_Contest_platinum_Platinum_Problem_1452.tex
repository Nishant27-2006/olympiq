\documentclass[12pt]{article}
\usepackage{amsmath}
\usepackage{amssymb}
\usepackage{graphicx}
\usepackage{enumitem}
\usepackage{hyperref}
\usepackage{xcolor}

\title{View problem}
\author{USACO Contest: DEC24 Contest - Platinum Division}
\date{\today}

\begin{document}

\maketitle

\section*{Problem Statement}


**Note: The memory limit for this problem is 512MB, twice the default.**
Farmer John's $N$ ($1\le N\le 5\cdot 10^5$) cows are each assigned a  bitstring
of length $K$ that is not all zero ($1\le K\le 20$). Different cows may be
assigned the same bitstring.

The Jaccard similarity of two bitstrings is defined as the number of set bits in
their bitwise intersection divided by the number of set bits in their bitwise union.
For example, the Jaccard similarity of the bitstrings $\texttt{11001}$ and
$\texttt{11010}$ would be $2/4$.

For each cow, output the sum of her bitstring's Jaccard similarity with each of
the $N$ cows' bitstrings including her own, modulo $10^9+7$. Specifically, if
the sum is equal to a rational number $a/b$ where $a$ and $b$ are integers
sharing no common factors, output the unique integer $x$ in the range
$[0,10^9+7)$ such that $bx-a$ is divisible by $10^9+7$.

INPUT FORMAT (input arrives from the terminal / stdin):
The first line contains $N$ and $K$.

The next $N$ lines each contain an integer $i\in (0,2^K)$, representing a cow
associated with the length-$K$ binary representation of $i$.

INPUT FORMAT (input arrives from the terminal / stdin):
Output the sum modulo $10^9+7$ for each cow on a separate line.

SAMPLE INPUT:
4 2
1
1
2
3
SAMPLE OUTPUT: 
500000006
500000006
500000005
500000006

The cows are associated with the following bitstrings:
$[\texttt{01}, \texttt{01}, \texttt{10}, \texttt{11}]$.

For the first cow, the sum is
$\text{sim}(1,1)+\text{sim}(1,1)+\text{sim}(1,2)+\text{sim}(1,3)=1+1+0+1/2\equiv 500000006\pmod{10^9+7}$.

The second cow's bitstring is the same as the first cow's, so her sum is the
same as above.

For the third cow, the sum is

$\text{sim}(2,1)+\text{sim}(2,1)+\text{sim}(2,2)+\text{sim}(2,3)=0+0+1+1/2\equiv 500000005\pmod{10^9+7}$.

SCORING:
Inputs 2-15: There will be two test cases for each of
$K\in \{10,15,16,17,18,19,20\}$.


Problem credits: Benjamin Qi



\section*{Input Format}
The next $N$ lines each contain an integer $i\in (0,2^K)$, representing a cow
associated with the length-$K$ binary representation of $i$.

INPUT FORMAT (input arrives from the terminal / stdin):Output the sum modulo $10^9+7$ for each cow on a separate line.SAMPLE INPUT:4 2
1
1
2
3SAMPLE OUTPUT:500000006
500000006
500000005
500000006The cows are associated with the following bitstrings:
$[\texttt{01}, \texttt{01}, \texttt{10}, \texttt{11}]$.For the first cow, the sum is
$\text{sim}(1,1)+\text{sim}(1,1)+\text{sim}(1,2)+\text{sim}(1,3)=1+1+0+1/2\equiv 500000006\pmod{10^9+7}$.The second cow's bitstring is the same as the first cow's, so her sum is the
same as above.For the third cow, the sum is$\text{sim}(2,1)+\text{sim}(2,1)+\text{sim}(2,2)+\text{sim}(2,3)=0+0+1+1/2\equiv 500000005\pmod{10^9+7}$.SCORING:Inputs 2-15: There will be two test cases for each of
$K\in \{10,15,16,17,18,19,20\}$.Problem credits: Benjamin Qi

SAMPLE INPUT:4 2
1
1
2
3SAMPLE OUTPUT:500000006
500000006
500000005
500000006The cows are associated with the following bitstrings:
$[\texttt{01}, \texttt{01}, \texttt{10}, \texttt{11}]$.For the first cow, the sum is
$\text{sim}(1,1)+\text{sim}(1,1)+\text{sim}(1,2)+\text{sim}(1,3)=1+1+0+1/2\equiv 500000006\pmod{10^9+7}$.The second cow's bitstring is the same as the first cow's, so her sum is the
same as above.For the third cow, the sum is$\text{sim}(2,1)+\text{sim}(2,1)+\text{sim}(2,2)+\text{sim}(2,3)=0+0+1+1/2\equiv 500000005\pmod{10^9+7}$.SCORING:Inputs 2-15: There will be two test cases for each of
$K\in \{10,15,16,17,18,19,20\}$.Problem credits: Benjamin Qi

\section*{Output Format}


\section*{Sample Input}
\begin{verbatim}
4 2
1
1
2
3
\end{verbatim}

\section*{Sample Output}
\begin{verbatim}
500000006
500000006
500000005
500000006

The cows are associated with the following bitstrings:
$[\texttt{01}, \texttt{01}, \texttt{10}, \texttt{11}]$.For the first cow, the sum is
$\text{sim}(1,1)+\text{sim}(1,1)+\text{sim}(1,2)+\text{sim}(1,3)=1+1+0+1/2\equiv 500000006\pmod{10^9+7}$.The second cow's bitstring is the same as the first cow's, so her sum is the
same as above.For the third cow, the sum is$\text{sim}(2,1)+\text{sim}(2,1)+\text{sim}(2,2)+\text{sim}(2,3)=0+0+1+1/2\equiv 500000005\pmod{10^9+7}$.SCORING:Inputs 2-15: There will be two test cases for each of
$K\in \{10,15,16,17,18,19,20\}$.Problem credits: Benjamin Qi

For the first cow, the sum is
$\text{sim}(1,1)+\text{sim}(1,1)+\text{sim}(1,2)+\text{sim}(1,3)=1+1+0+1/2\equiv 500000006\pmod{10^9+7}$.The second cow's bitstring is the same as the first cow's, so her sum is the
same as above.For the third cow, the sum is$\text{sim}(2,1)+\text{sim}(2,1)+\text{sim}(2,2)+\text{sim}(2,3)=0+0+1+1/2\equiv 500000005\pmod{10^9+7}$.SCORING:Inputs 2-15: There will be two test cases for each of
$K\in \{10,15,16,17,18,19,20\}$.Problem credits: Benjamin Qi

The second cow's bitstring is the same as the first cow's, so her sum is the
same as above.For the third cow, the sum is$\text{sim}(2,1)+\text{sim}(2,1)+\text{sim}(2,2)+\text{sim}(2,3)=0+0+1+1/2\equiv 500000005\pmod{10^9+7}$.SCORING:Inputs 2-15: There will be two test cases for each of
$K\in \{10,15,16,17,18,19,20\}$.Problem credits: Benjamin Qi

For the third cow, the sum is$\text{sim}(2,1)+\text{sim}(2,1)+\text{sim}(2,2)+\text{sim}(2,3)=0+0+1+1/2\equiv 500000005\pmod{10^9+7}$.SCORING:Inputs 2-15: There will be two test cases for each of
$K\in \{10,15,16,17,18,19,20\}$.Problem credits: Benjamin Qi

$\text{sim}(2,1)+\text{sim}(2,1)+\text{sim}(2,2)+\text{sim}(2,3)=0+0+1+1/2\equiv 500000005\pmod{10^9+7}$.SCORING:Inputs 2-15: There will be two test cases for each of
$K\in \{10,15,16,17,18,19,20\}$.Problem credits: Benjamin Qi

SCORING:Inputs 2-15: There will be two test cases for each of
$K\in \{10,15,16,17,18,19,20\}$.Problem credits: Benjamin Qi
\end{verbatim}

\section*{Solution}


\section*{Problem URL}
https://usaco.org/index.php?page=viewproblem2&cpid=1452

\section*{Source}
USACO DEC24 Contest, Platinum Division, Problem ID: 1452

\end{document}
