\documentclass[12pt]{article}
\usepackage{amsmath}
\usepackage{amssymb}
\usepackage{graphicx}
\usepackage{enumitem}
\usepackage{hyperref}
\usepackage{xcolor}

\title{View problem}
\author{USACO Contest: OPEN15 Contest - Platinum Division}
\date{\today}

\begin{document}

\maketitle

\section*{Problem Statement}

Tired of filtering through web search results targeted at other farm animals, 
the cows have decided to launch their own search engine.  Unfortunately, with
their severe lack of experience managing large software projects, the number
of employees $N$ ($1 \le N \le 10^{100}$) in their company ended up being
substantially larger than originally planned.  In order to find a fitting name
for their company, the cows drew inspiration from the upper bound on $N$ and
decided to call it "Googol", being the name for the number $10^{100}$.
In a desperate attempt to improve the managerial structure of their company,
the cows have structured the company in the form of a binary tree, where each
employee is responsible for managing two direct subordinates, designated as the
"left" and "right" subordinates in keeping with the tree structure.  In
order to balance the managerial load on each employee, the cows have arranged
the structure of their company tree so that for each employee E, the
total number of employees in E's left subtree is either equal to or one greater 
than the total number of employees in E's right subtree.
Each employee has a distinct integer-valued ID in the range $1\ldots
N$, with the CEO (the root of the tree) having ID 1.  You can interactively
query any employee to determine the IDs of her two subordinates.  This is done
by writing the ID of the employee on the standard output stream (stdout)
followed by a newline. The response you will get back will be a single line
containing two integers, specifying the IDs of the left and right subordinates
of this employee.  It is possible for both of these to be 0, if the employee has
no subordinates, or for just the right ID to be 0, if the employee only has a
left subordinate (note that it is impossible, due to the balancing condition
above, for an employee to have a right subordinate but no left subordinate).
Unfortunately, the cows have lost track of the exact value of $N$.  Please 
compute this number and print out "Answer N" (followed by a newline) as your 
last line of output.  Your program will be limited to making at most 70,000 queries,
and a running time limit of 4 seconds (8 for Java or Python).
Here is an example of a possible interaction between your program and the
grader:

YOUR PROGRAM: 1
GRADER: 4 3
YOUR PROGRAM: 4
GRADER: 2 0
YOUR PROGRAM: 3
GRADER: 0 0
YOUR PROGRAM: Answer 4

The tree corresponding to this interaction is

     1
   4   3
 2

Note that it was not necessary to ask about the children of node 2, since we
can deduce that there are no such children from the fact that node 4 has no
right child.
Technical note: this problem is being graded by an interactive
grader that is a new part of our grading system.  If you experience
any behavior that appears to be a problem on the technical side with
the grader, please report it to [email protected].  We do not believe
it should be necessary, but if your code seems to be frozen during
grading, try adding commands to flush the output stream (e.g.,
fflush(stdout) or cout.flush()), in case there is unexpected buffering
preventing your output from reaching the grader (if this does seems
necessary to make your code work, please also send a note to
[email protected] so we can fix the issue in the future).
[Problem credits: Brian Dean, 2015]



\section*{Input Format}


\section*{Output Format}


\section*{Sample Input}
\begin{verbatim}

\end{verbatim}

\section*{Sample Output}
\begin{verbatim}

\end{verbatim}

\section*{Solution}


\section*{Problem URL}
https://usaco.org/index.php?page=viewproblem2&cpid=552

\section*{Source}
USACO OPEN15 Contest, Platinum Division, Problem ID: 552

\end{document}
