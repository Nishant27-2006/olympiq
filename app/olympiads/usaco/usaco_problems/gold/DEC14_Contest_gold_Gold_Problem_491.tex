\documentclass[12pt]{article}
\usepackage{amsmath}
\usepackage{amssymb}
\usepackage{graphicx}
\usepackage{enumitem}
\usepackage{hyperref}
\usepackage{xcolor}

\title{View problem}
\author{USACO Contest: DEC14 Contest - Gold Division}
\date{\today}

\begin{document}

\maketitle

\section*{Problem Statement}

Problem 1: Piggy Back [Brian Dean, 2014]

Bessie and her sister Elsie graze in different fields during the day,
and in the evening they both want to walk back to the barn to rest.
Being clever bovines, they come up with a plan to minimize the total
amount of energy they both spend while walking.

Bessie spends B units of energy when walking from a field to an
adjacent field, and Elsie spends E units of energy when she walks to
an adjacent field.  However, if Bessie and Elsie are together in the
same field, Bessie can carry Elsie on her shoulders and both can move
to an adjacent field while spending only P units of energy (where P
might be considerably less than B+E, the amount Bessie and Elsie would
have spent individually walking to the adjacent field).  If P is very
small, the most energy-efficient solution may involve Bessie and Elsie
traveling to a common meeting field, then traveling together piggyback
for the rest of the journey to the barn.  Of course, if P is large, it
may still make the most sense for Bessie and Elsie to travel
separately.  On a side note, Bessie and Elsie are both unhappy with
the term "piggyback", as they don't see why the pigs on the farm
should deserve all the credit for this remarkable form of
transportation.

Given B, E, and P, as well as the layout of the farm, please compute
the minimum amount of energy required for Bessie and Elsie to reach
the barn.

INPUT: (file piggyback.in)

The first line of input contains the positive integers B, E, P, N, and
M.  All of these are at most 40,000.  B, E, and P are described above.
N is the number of fields in the farm (numbered 1..N, where N >= 3),
and M is the number of connections between fields.  Bessie and Elsie
start in fields 1 and 2, respectively.  The barn resides in field N.

The next M lines in the input each describe a connection between a
pair of different fields, specified by the integer indices of the two
fields.  Connections are bi-directional.  It is always possible to
travel from field 1 to field N, and field 2 to field N, along a series
of such connections.  

SAMPLE INPUT:

4 4 5 8 8
1 4
2 3
3 4
4 7
2 5
5 6
6 8
7 8


OUTPUT: (file piggyback.out)

A single integer specifying the minimum amount of energy Bessie and
Elsie collectively need to spend to reach the barn.  In the example
shown here, Bessie travels from 1 to 4 and Elsie travels from 2 to 3
to 4.  Then, they travel together from 4 to 7 to 8.

SAMPLE OUTPUT:

22



\section*{Input Format}


\section*{Output Format}


\section*{Sample Input}
\begin{verbatim}

\end{verbatim}

\section*{Sample Output}
\begin{verbatim}

\end{verbatim}

\section*{Solution}


\section*{Problem URL}
https://usaco.org/index.php?page=viewproblem2&cpid=491

\section*{Source}
USACO DEC14 Contest, Gold Division, Problem ID: 491

\end{document}
