\documentclass[12pt]{article}
\usepackage{amsmath}
\usepackage{amssymb}
\usepackage{graphicx}
\usepackage{enumitem}
\usepackage{hyperref}
\usepackage{xcolor}

\title{View problem}
\author{USACO Contest: OPEN22 Contest - Gold Division}
\date{\today}

\begin{document}

\maketitle

\section*{Problem Statement}

Farmer John has conducted an extensive study of the evolution of different cow
breeds.  The result is a rooted tree with $N$ ($2\le N\le 10^5$) nodes  labeled
$1\ldots N$, each node corresponding to a cow breed.   For each $i\in [2,N]$,
the parent of node $i$ is node $p_i$ ($1\le p_i<i$), meaning that breed $i$
evolved from breed $p_i$. A node $j$ is called an ancestor of node $i$ if
$j=p_i$ or $j$ is an ancestor of $p_i$.

Every node $i$ in the tree is associated with a breed having an integer number
of spots $s_i$.  The "imbalance" of the tree is defined to be the maximum of
$|s_i-s_j|$ over all pairs of nodes $(i,j)$ such that $j$ is an ancestor of $i$.

Farmer John doesn't know the exact value of $s_i$ for each breed, but he knows
lower and upper bounds on these values.  Your  job is to assign an integer value
of $s_i \in [l_i,r_i]$ ($0\le l_i\le r_i\le 10^9$) to each node such that the
imbalance of the tree is minimized. 

INPUT FORMAT (input arrives from the terminal / stdin):
The first line contains $T$ ($1\le T\le 10$), the number of independent test
cases to be solved, and an integer $B\in \{0,1\}$.

Each test case starts with a line containing $N$, followed by $N-1$ integers
$p_2,p_3,\ldots,p_N$.

The next $N$ lines each contain two integers $l_i$ and $r_i$.

It is guaranteed that the sum of $N$ over all test cases does not exceed $10^5$.

OUTPUT FORMAT (print output to the terminal / stdout):
For each test case, output one or two lines, depending on the value of $B$.

The first line for each test case should contain the minimum imbalance.

If $B=1,$ then print an additional line with $N$ space-separated integers
$s_1,s_2,\ldots, s_N$ containing an assignment of spots that achieves the above
imbalance. Any valid assignment will be accepted.

SAMPLE INPUT:
3 0
3
1 1
0 100
1 1
6 7
5
1 2 3 4
6 6
1 6
1 6
1 6
5 5
3
1 1
0 10
0 1
9 10
SAMPLE OUTPUT: 
3
1
4

For the first test case, the minimum imbalance is $3$. One way to achieve
imbalance $3$ is to set $[s_1,s_2,s_3]=[4,1,7]$.

SAMPLE INPUT:
3 1
3
1 1
0 100
1 1
6 7
5
1 2 3 4
6 6
1 6
1 6
1 6
5 5
3
1 1
0 10
0 1
9 10
SAMPLE OUTPUT: 
3
3 1 6
1
6 5 5 5 5
4
5 1 9

This input is the same as the first one aside from the value of $B$. Another way
to achieve imbalance $3$ is to set $[s_1,s_2,s_3]=[3,1,6]$.

SCORING:
Test cases 3-4 satisfy $l_i=r_i$ for all $i$.Test cases 5-6 satisfy $p_i=i-1$ for all $i$.Test cases 7-16 satisfy no additional constraints.
Within each subtask, the first half of the test cases will satisfy $B=0$, and
the rest will satisfy $B=1$.



Problem credits: Andrew Wang



\section*{Input Format}
Each test case starts with a line containing $N$, followed by $N-1$ integers
$p_2,p_3,\ldots,p_N$.The next $N$ lines each contain two integers $l_i$ and $r_i$.It is guaranteed that the sum of $N$ over all test cases does not exceed $10^5$.

The next $N$ lines each contain two integers $l_i$ and $r_i$.It is guaranteed that the sum of $N$ over all test cases does not exceed $10^5$.

It is guaranteed that the sum of $N$ over all test cases does not exceed $10^5$.

OUTPUT FORMAT (print output to the terminal / stdout):For each test case, output one or two lines, depending on the value of $B$.The first line for each test case should contain the minimum imbalance.If $B=1,$ then print an additional line with $N$ space-separated integers
$s_1,s_2,\ldots, s_N$ containing an assignment of spots that achieves the above
imbalance. Any valid assignment will be accepted.SAMPLE INPUT:3 0
3
1 1
0 100
1 1
6 7
5
1 2 3 4
6 6
1 6
1 6
1 6
5 5
3
1 1
0 10
0 1
9 10SAMPLE OUTPUT:3
1
4For the first test case, the minimum imbalance is $3$. One way to achieve
imbalance $3$ is to set $[s_1,s_2,s_3]=[4,1,7]$.SAMPLE INPUT:3 1
3
1 1
0 100
1 1
6 7
5
1 2 3 4
6 6
1 6
1 6
1 6
5 5
3
1 1
0 10
0 1
9 10SAMPLE OUTPUT:3
3 1 6
1
6 5 5 5 5
4
5 1 9This input is the same as the first one aside from the value of $B$. Another way
to achieve imbalance $3$ is to set $[s_1,s_2,s_3]=[3,1,6]$.SCORING:Test cases 3-4 satisfy $l_i=r_i$ for all $i$.Test cases 5-6 satisfy $p_i=i-1$ for all $i$.Test cases 7-16 satisfy no additional constraints.Within each subtask, the first half of the test cases will satisfy $B=0$, and
the rest will satisfy $B=1$.Problem credits: Andrew Wang

\section*{Output Format}
The first line for each test case should contain the minimum imbalance.If $B=1,$ then print an additional line with $N$ space-separated integers
$s_1,s_2,\ldots, s_N$ containing an assignment of spots that achieves the above
imbalance. Any valid assignment will be accepted.

If $B=1,$ then print an additional line with $N$ space-separated integers
$s_1,s_2,\ldots, s_N$ containing an assignment of spots that achieves the above
imbalance. Any valid assignment will be accepted.

SAMPLE INPUT:3 0
3
1 1
0 100
1 1
6 7
5
1 2 3 4
6 6
1 6
1 6
1 6
5 5
3
1 1
0 10
0 1
9 10SAMPLE OUTPUT:3
1
4For the first test case, the minimum imbalance is $3$. One way to achieve
imbalance $3$ is to set $[s_1,s_2,s_3]=[4,1,7]$.SAMPLE INPUT:3 1
3
1 1
0 100
1 1
6 7
5
1 2 3 4
6 6
1 6
1 6
1 6
5 5
3
1 1
0 10
0 1
9 10SAMPLE OUTPUT:3
3 1 6
1
6 5 5 5 5
4
5 1 9This input is the same as the first one aside from the value of $B$. Another way
to achieve imbalance $3$ is to set $[s_1,s_2,s_3]=[3,1,6]$.SCORING:Test cases 3-4 satisfy $l_i=r_i$ for all $i$.Test cases 5-6 satisfy $p_i=i-1$ for all $i$.Test cases 7-16 satisfy no additional constraints.Within each subtask, the first half of the test cases will satisfy $B=0$, and
the rest will satisfy $B=1$.Problem credits: Andrew Wang

\section*{Sample Input}
\begin{verbatim}
3 0
3
1 1
0 100
1 1
6 7
5
1 2 3 4
6 6
1 6
1 6
1 6
5 5
3
1 1
0 10
0 1
9 10

3 1
3
1 1
0 100
1 1
6 7
5
1 2 3 4
6 6
1 6
1 6
1 6
5 5
3
1 1
0 10
0 1
9 10
\end{verbatim}

\section*{Sample Output}
\begin{verbatim}
3
1
4

For the first test case, the minimum imbalance is $3$. One way to achieve
imbalance $3$ is to set $[s_1,s_2,s_3]=[4,1,7]$.SAMPLE INPUT:3 1
3
1 1
0 100
1 1
6 7
5
1 2 3 4
6 6
1 6
1 6
1 6
5 5
3
1 1
0 10
0 1
9 10SAMPLE OUTPUT:3
3 1 6
1
6 5 5 5 5
4
5 1 9This input is the same as the first one aside from the value of $B$. Another way
to achieve imbalance $3$ is to set $[s_1,s_2,s_3]=[3,1,6]$.SCORING:Test cases 3-4 satisfy $l_i=r_i$ for all $i$.Test cases 5-6 satisfy $p_i=i-1$ for all $i$.Test cases 7-16 satisfy no additional constraints.Within each subtask, the first half of the test cases will satisfy $B=0$, and
the rest will satisfy $B=1$.Problem credits: Andrew Wang

SAMPLE INPUT:3 1
3
1 1
0 100
1 1
6 7
5
1 2 3 4
6 6
1 6
1 6
1 6
5 5
3
1 1
0 10
0 1
9 10SAMPLE OUTPUT:3
3 1 6
1
6 5 5 5 5
4
5 1 9This input is the same as the first one aside from the value of $B$. Another way
to achieve imbalance $3$ is to set $[s_1,s_2,s_3]=[3,1,6]$.SCORING:Test cases 3-4 satisfy $l_i=r_i$ for all $i$.Test cases 5-6 satisfy $p_i=i-1$ for all $i$.Test cases 7-16 satisfy no additional constraints.Within each subtask, the first half of the test cases will satisfy $B=0$, and
the rest will satisfy $B=1$.Problem credits: Andrew Wang

3
3 1 6
1
6 5 5 5 5
4
5 1 9

This input is the same as the first one aside from the value of $B$. Another way
to achieve imbalance $3$ is to set $[s_1,s_2,s_3]=[3,1,6]$.SCORING:Test cases 3-4 satisfy $l_i=r_i$ for all $i$.Test cases 5-6 satisfy $p_i=i-1$ for all $i$.Test cases 7-16 satisfy no additional constraints.Within each subtask, the first half of the test cases will satisfy $B=0$, and
the rest will satisfy $B=1$.Problem credits: Andrew Wang

SCORING:Test cases 3-4 satisfy $l_i=r_i$ for all $i$.Test cases 5-6 satisfy $p_i=i-1$ for all $i$.Test cases 7-16 satisfy no additional constraints.Within each subtask, the first half of the test cases will satisfy $B=0$, and
the rest will satisfy $B=1$.Problem credits: Andrew Wang
\end{verbatim}

\section*{Solution}


\section*{Problem URL}
https://usaco.org/index.php?page=viewproblem2&cpid=1235

\section*{Source}
USACO OPEN22 Contest, Gold Division, Problem ID: 1235

\end{document}
