\documentclass[12pt]{article}
\usepackage{amsmath}
\usepackage{amssymb}
\usepackage{graphicx}
\usepackage{enumitem}
\usepackage{hyperref}
\usepackage{xcolor}

\title{View problem}
\author{USACO Contest: OPEN18 Contest - Gold Division}
\date{\today}

\begin{document}

\maketitle

\section*{Problem Statement}

Farmer John is bringing his $N$ cows, conveniently numbered $1 \ldots N$, to the
county fair, to compete in the annual bovine talent show!  His $i$th cow has a
weight $w_i$ and talent level $t_i$, both integers.

Upon arrival, Farmer John is quite surprised by the new rules for this year's 
talent show:

(i) A group of cows of total weight at least $W$ must be entered into the show
(in order to ensure strong teams of cows are competing, not just strong 
individuals), and

(ii) The group with the largest ratio of total talent to total weight shall win.

FJ observes that all of his cows together have weight at least $W$, so he should
be able to enter a team satisfying (i).  Help him determine the optimal ratio of
talent to weight he can achieve for any such team.

INPUT FORMAT (file talent.in):
The first line of input contains $N$ ($1 \leq N \leq 250$) and $W$
($1 \leq W \leq 1000$). The next $N$ lines each describe a cow using two
integers $w_i$ ($1 \leq w_i \leq 10^6$) and $t_i$ ($1 \leq t_i \leq 10^3$).

OUTPUT FORMAT (file talent.out):
Please determine the largest possible ratio of total talent over total weight
Farmer John can achieve using a group of cows of total weight at least $W$.  If
your answer is $A$, please print out the floor of $1000A$ in order to keep the
output integer-valued (the floor operation discards any fractional part by
rounding down to an integer, if the number in question is not already an
integer).

SAMPLE INPUT:
3 15
20 21
10 11
30 31
SAMPLE OUTPUT: 
1066

In this example, the best talent-to-weight ratio overall would be to use just
the single cow with talent 11 and weight 10, but since we need at least  15
units of weight, the optimal solution ends up being to use this cow plus the cow
with talent 21 and weight 20.  This gives a talent-to-weight ratio of
(11+21)/(10+20) = 32/30 = 1.0666666..., which when multiplied by 1000 and
floored gives 1066.


Problem credits: Brian Dean



\section*{Input Format}
OUTPUT FORMAT (file talent.out):Please determine the largest possible ratio of total talent over total weight
Farmer John can achieve using a group of cows of total weight at least $W$.  If
your answer is $A$, please print out the floor of $1000A$ in order to keep the
output integer-valued (the floor operation discards any fractional part by
rounding down to an integer, if the number in question is not already an
integer).SAMPLE INPUT:3 15
20 21
10 11
30 31SAMPLE OUTPUT:1066In this example, the best talent-to-weight ratio overall would be to use just
the single cow with talent 11 and weight 10, but since we need at least  15
units of weight, the optimal solution ends up being to use this cow plus the cow
with talent 21 and weight 20.  This gives a talent-to-weight ratio of
(11+21)/(10+20) = 32/30 = 1.0666666..., which when multiplied by 1000 and
floored gives 1066.Problem credits: Brian Dean

\section*{Output Format}
SAMPLE INPUT:3 15
20 21
10 11
30 31SAMPLE OUTPUT:1066In this example, the best talent-to-weight ratio overall would be to use just
the single cow with talent 11 and weight 10, but since we need at least  15
units of weight, the optimal solution ends up being to use this cow plus the cow
with talent 21 and weight 20.  This gives a talent-to-weight ratio of
(11+21)/(10+20) = 32/30 = 1.0666666..., which when multiplied by 1000 and
floored gives 1066.Problem credits: Brian Dean

\section*{Sample Input}
\begin{verbatim}
3 15
20 21
10 11
30 31
\end{verbatim}

\section*{Sample Output}
\begin{verbatim}
1066

In this example, the best talent-to-weight ratio overall would be to use just
the single cow with talent 11 and weight 10, but since we need at least  15
units of weight, the optimal solution ends up being to use this cow plus the cow
with talent 21 and weight 20.  This gives a talent-to-weight ratio of
(11+21)/(10+20) = 32/30 = 1.0666666..., which when multiplied by 1000 and
floored gives 1066.Problem credits: Brian Dean

Problem credits: Brian Dean

Problem credits: Brian Dean
\end{verbatim}

\section*{Solution}


\section*{Problem URL}
https://usaco.org/index.php?page=viewproblem2&cpid=839

\section*{Source}
USACO OPEN18 Contest, Gold Division, Problem ID: 839

\end{document}
