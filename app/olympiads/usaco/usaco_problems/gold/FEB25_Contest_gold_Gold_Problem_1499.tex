\documentclass[12pt]{article}
\usepackage{amsmath}
\usepackage{amssymb}
\usepackage{graphicx}
\usepackage{enumitem}
\usepackage{hyperref}
\usepackage{xcolor}

\title{View problem}
\author{USACO Contest: FEB25 Contest - Gold Division}
\date{\today}

\begin{document}

\maketitle

\section*{Problem Statement}


Farmer John's $N$ cows are labeled $1$ to $N$ ($2\le N\le 16$). The friendship
relationships between the cows can be modeled as an undirected graph with $M$
($0\le M\le N(N-1)/2$) edges. Two cows are friends if and only if there is an
edge between them in the graph.

In one operation, you can add or remove a single edge from the graph. Count the
minimum number of operations required to ensure that the following property
holds: If cows $a$ and $b$ are friends, then for every other cow $c$, at least
one of $a$ and $b$ is friends with $c$.

INPUT FORMAT (input arrives from the terminal / stdin):
The first line contains $N$ and $M$.

The next $M$ lines each contain a pair of friends $a$ and $b$ ($1\le a<b\le N$).
No pair of friends appears more than once.

OUTPUT FORMAT (print output to the terminal / stdout):
The number of edges you need to add or remove.

SAMPLE INPUT:
3 1
1 2
SAMPLE OUTPUT: 
1

The network violates the property. We can add one of edges $(2,3)$ or $(1,3)$, 
or remove edge $(1,2)$ to fix this.

SAMPLE INPUT:
3 2
1 2
2 3
SAMPLE OUTPUT: 
0

No changes are necessary.

SAMPLE INPUT:
4 4
1 2
1 3
1 4
2 3
SAMPLE OUTPUT: 
1

SCORING:
Inputs 4-13: One input for each $N\in [6, 15]$ in increasing order.Inputs 14-18: $N=16$


Problem credits: Benjamin Qi



\section*{Input Format}
The next $M$ lines each contain a pair of friends $a$ and $b$ ($1\le a<b\le N$).
No pair of friends appears more than once.

OUTPUT FORMAT (print output to the terminal / stdout):The number of edges you need to add or remove.SAMPLE INPUT:3 1
1 2SAMPLE OUTPUT:1The network violates the property. We can add one of edges $(2,3)$ or $(1,3)$, 
or remove edge $(1,2)$ to fix this.SAMPLE INPUT:3 2
1 2
2 3SAMPLE OUTPUT:0No changes are necessary.SAMPLE INPUT:4 4
1 2
1 3
1 4
2 3SAMPLE OUTPUT:1SCORING:Inputs 4-13: One input for each $N\in [6, 15]$ in increasing order.Inputs 14-18: $N=16$Problem credits: Benjamin Qi

\section*{Output Format}
SAMPLE INPUT:3 1
1 2SAMPLE OUTPUT:1The network violates the property. We can add one of edges $(2,3)$ or $(1,3)$, 
or remove edge $(1,2)$ to fix this.SAMPLE INPUT:3 2
1 2
2 3SAMPLE OUTPUT:0No changes are necessary.SAMPLE INPUT:4 4
1 2
1 3
1 4
2 3SAMPLE OUTPUT:1SCORING:Inputs 4-13: One input for each $N\in [6, 15]$ in increasing order.Inputs 14-18: $N=16$Problem credits: Benjamin Qi

\section*{Sample Input}
\begin{verbatim}
3 1
1 2

3 2
1 2
2 3

4 4
1 2
1 3
1 4
2 3
\end{verbatim}

\section*{Sample Output}
\begin{verbatim}
1

The network violates the property. We can add one of edges $(2,3)$ or $(1,3)$, 
or remove edge $(1,2)$ to fix this.SAMPLE INPUT:3 2
1 2
2 3SAMPLE OUTPUT:0No changes are necessary.SAMPLE INPUT:4 4
1 2
1 3
1 4
2 3SAMPLE OUTPUT:1SCORING:Inputs 4-13: One input for each $N\in [6, 15]$ in increasing order.Inputs 14-18: $N=16$Problem credits: Benjamin Qi

SAMPLE INPUT:3 2
1 2
2 3SAMPLE OUTPUT:0No changes are necessary.SAMPLE INPUT:4 4
1 2
1 3
1 4
2 3SAMPLE OUTPUT:1SCORING:Inputs 4-13: One input for each $N\in [6, 15]$ in increasing order.Inputs 14-18: $N=16$Problem credits: Benjamin Qi

0

No changes are necessary.SAMPLE INPUT:4 4
1 2
1 3
1 4
2 3SAMPLE OUTPUT:1SCORING:Inputs 4-13: One input for each $N\in [6, 15]$ in increasing order.Inputs 14-18: $N=16$Problem credits: Benjamin Qi

SAMPLE INPUT:4 4
1 2
1 3
1 4
2 3SAMPLE OUTPUT:1SCORING:Inputs 4-13: One input for each $N\in [6, 15]$ in increasing order.Inputs 14-18: $N=16$Problem credits: Benjamin Qi

1

SCORING:Inputs 4-13: One input for each $N\in [6, 15]$ in increasing order.Inputs 14-18: $N=16$Problem credits: Benjamin Qi
\end{verbatim}

\section*{Solution}


\section*{Problem URL}
https://usaco.org/index.php?page=viewproblem2&cpid=1499

\section*{Source}
USACO FEB25 Contest, Gold Division, Problem ID: 1499

\end{document}
