\documentclass[12pt]{article}
\usepackage{amsmath}
\usepackage{amssymb}
\usepackage{graphicx}
\usepackage{enumitem}
\usepackage{hyperref}
\usepackage{xcolor}

\title{View problem}
\author{USACO Contest: JAN25 Contest - Gold Division}
\date{\today}

\begin{document}

\maketitle

\section*{Problem Statement}


**Note: The time limit for this problem is 4s, twice the default.**
Farmer John has a binary tree with $N$ nodes  where the nodes are numbered from
$1$ to $N$ ($1 \leq N < 2\cdot 10^5$ and $N$ is odd). For $i>1$, the parent of
node $i$  is $\lfloor i/2\rfloor$. Each node has an initial integer value $a_i$,
and a cost $c_i$ to change the initial value to any other integer value
($0\le a_i,c_i\le 10^9$).

He has been tasked by the Federal Bovine Intermediary (FBI) with finding an
approximate median value within this tree, and has devised a clever algorithm to
do so.

He starts at the last node $N$ and works his way backward. At every step of the
algorithm, if a node would not be the median of it and its two children, he
swaps the values of the current node and the child value that would be the
median. At the end of this algorithm, the value at node $1$ (the root) is the
median approximation.

The FBI has also given Farmer John a list of $Q$ $(1 \leq Q \leq 2\cdot 10^5)$
independent queries each specified by a target value $m$ ($0\le m\le 10^9$). For
each query, FJ will first change some of the node's initial values, and then
execute the median approximation algorithm. For each query, determine the
minimum possible total cost for FJ to make the output of the algorithm equal to
$m$.

INPUT FORMAT (input arrives from the terminal / stdin):
The first line of input contains $N$.

The next $N$ lines each contain two integers $a_i$ and $c_i$.

The next line contains $Q$.

The next $Q$ lines each contain a target value $m$.

OUTPUT FORMAT (print output to the terminal / stdout):
Output $Q$ lines, the minimum possible total cost for each target value $m$.

SAMPLE INPUT:
5
10 10000
30 1000
20 100
50 10
40 1
11
55
50
45
40
35
30
25
20
15
10
5
SAMPLE OUTPUT: 
111
101
101
100
100
100
100
0
11
11
111

To make the median approximation equal $40$, FJ can change the value at node 3
to $60$. This costs $c_3=100$.

To make the median approximation equal $45$, FJ can change the value at node 3
to $60$ and the value at node 5 to $45$. This costs $c_3+c_5=100+1=101$.

SCORING:
Inputs 2-4: $N,Q\le 50$Inputs 5-7: $N,Q\le 1000$Inputs 8-16: No additional constraints


Problem credits: Suhas Nagar and Benjamin Qi



\section*{Input Format}
The next $N$ lines each contain two integers $a_i$ and $c_i$.The next line contains $Q$.The next $Q$ lines each contain a target value $m$.

The next line contains $Q$.The next $Q$ lines each contain a target value $m$.

The next $Q$ lines each contain a target value $m$.

OUTPUT FORMAT (print output to the terminal / stdout):Output $Q$ lines, the minimum possible total cost for each target value $m$.SAMPLE INPUT:5
10 10000
30 1000
20 100
50 10
40 1
11
55
50
45
40
35
30
25
20
15
10
5SAMPLE OUTPUT:111
101
101
100
100
100
100
0
11
11
111To make the median approximation equal $40$, FJ can change the value at node 3
to $60$. This costs $c_3=100$.To make the median approximation equal $45$, FJ can change the value at node 3
to $60$ and the value at node 5 to $45$. This costs $c_3+c_5=100+1=101$.SCORING:Inputs 2-4: $N,Q\le 50$Inputs 5-7: $N,Q\le 1000$Inputs 8-16: No additional constraintsProblem credits: Suhas Nagar and Benjamin Qi

\section*{Output Format}
SAMPLE INPUT:5
10 10000
30 1000
20 100
50 10
40 1
11
55
50
45
40
35
30
25
20
15
10
5SAMPLE OUTPUT:111
101
101
100
100
100
100
0
11
11
111To make the median approximation equal $40$, FJ can change the value at node 3
to $60$. This costs $c_3=100$.To make the median approximation equal $45$, FJ can change the value at node 3
to $60$ and the value at node 5 to $45$. This costs $c_3+c_5=100+1=101$.SCORING:Inputs 2-4: $N,Q\le 50$Inputs 5-7: $N,Q\le 1000$Inputs 8-16: No additional constraintsProblem credits: Suhas Nagar and Benjamin Qi

\section*{Sample Input}
\begin{verbatim}
5
10 10000
30 1000
20 100
50 10
40 1
11
55
50
45
40
35
30
25
20
15
10
5
\end{verbatim}

\section*{Sample Output}
\begin{verbatim}
111
101
101
100
100
100
100
0
11
11
111

To make the median approximation equal $40$, FJ can change the value at node 3
to $60$. This costs $c_3=100$.To make the median approximation equal $45$, FJ can change the value at node 3
to $60$ and the value at node 5 to $45$. This costs $c_3+c_5=100+1=101$.SCORING:Inputs 2-4: $N,Q\le 50$Inputs 5-7: $N,Q\le 1000$Inputs 8-16: No additional constraintsProblem credits: Suhas Nagar and Benjamin Qi

To make the median approximation equal $45$, FJ can change the value at node 3
to $60$ and the value at node 5 to $45$. This costs $c_3+c_5=100+1=101$.SCORING:Inputs 2-4: $N,Q\le 50$Inputs 5-7: $N,Q\le 1000$Inputs 8-16: No additional constraintsProblem credits: Suhas Nagar and Benjamin Qi

SCORING:Inputs 2-4: $N,Q\le 50$Inputs 5-7: $N,Q\le 1000$Inputs 8-16: No additional constraintsProblem credits: Suhas Nagar and Benjamin Qi
\end{verbatim}

\section*{Solution}


\section*{Problem URL}
https://usaco.org/index.php?page=viewproblem2&cpid=1473

\section*{Source}
USACO JAN25 Contest, Gold Division, Problem ID: 1473

\end{document}
