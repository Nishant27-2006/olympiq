\documentclass[12pt]{article}
\usepackage{amsmath}
\usepackage{amssymb}
\usepackage{graphicx}
\usepackage{enumitem}
\usepackage{hyperref}
\usepackage{xcolor}

\title{View problem}
\author{USACO Contest: FEB18 Contest - Gold Division}
\date{\today}

\begin{document}

\maketitle

\section*{Problem Statement}

Bessie the cow is surprisingly computer savvy.  On her computer in the barn, she
stores all of her precious files in a collection of directories; for example:


bessie/
  folder1/
    file1
    folder2/
      file2
  folder3/
    file3
  file4

There is a single "top level" directory, called bessie.

Bessie can navigate to be inside any directory she wants. From a given
directory, any file can be referenced by a "relative path".  In a relative path,
the symbol ".." refers to the parent directory. If Bessie were in folder2, she
could refer to the four files as follows:


../file1
file2
../../folder3/file3
../../file4

Bessie would like to choose a directory from which the sum of the lengths of the
relative paths to all the files is minimized.

INPUT FORMAT (file dirtraverse.in):
The first line contains an integer N ($2 \leq N \leq 100,000$), giving the total
number of files and directories.  For the purposes of input, each object (file
or directory) is assigned a unique integer ID between 1 and $N$, where ID 1 
refers to the top level directory.  

Next, there will be $N$ lines. Each line  starts with the name of a file or
directory. The name will have only lower case characters a-z and digits 0-9, and
will be at most 16 characters long. Following the name is an integer, $m$. If
$m$ is 0, then this entity is a file. If $m > 0$, then this entity is a
directory, and it has a total of $m$ files or directories inside it.  Following $m$
there will be $m$ integers giving the IDs of the entities in this directory.

OUTPUT FORMAT (file dirtraverse.out):
Output the minimal possible total length of all relative paths to files. Note
that this value may be too large to fit into a 32-bit integer.

SAMPLE INPUT:
8
bessie 3 2 6 8
folder1 2 3 4
file1 0
folder2 1 5
file2 0
folder3 1 7
file3 0
file4 0
SAMPLE OUTPUT: 
42

This input describes the example directory structure given above.

The best solution is to be in folder1. From this directory, the relative paths
are:


file1
folder2/file2
../folder3/file3
../file4


Problem credits: Mark Gordon



\section*{Input Format}
Next, there will be $N$ lines. Each line  starts with the name of a file or
directory. The name will have only lower case characters a-z and digits 0-9, and
will be at most 16 characters long. Following the name is an integer, $m$. If
$m$ is 0, then this entity is a file. If $m > 0$, then this entity is a
directory, and it has a total of $m$ files or directories inside it.  Following $m$
there will be $m$ integers giving the IDs of the entities in this directory.

OUTPUT FORMAT (file dirtraverse.out):Output the minimal possible total length of all relative paths to files. Note
that this value may be too large to fit into a 32-bit integer.SAMPLE INPUT:8
bessie 3 2 6 8
folder1 2 3 4
file1 0
folder2 1 5
file2 0
folder3 1 7
file3 0
file4 0SAMPLE OUTPUT:42This input describes the example directory structure given above.The best solution is to be in folder1. From this directory, the relative paths
are:file1
folder2/file2
../folder3/file3
../file4Problem credits: Mark Gordon

\section*{Output Format}
SAMPLE INPUT:8
bessie 3 2 6 8
folder1 2 3 4
file1 0
folder2 1 5
file2 0
folder3 1 7
file3 0
file4 0SAMPLE OUTPUT:42This input describes the example directory structure given above.The best solution is to be in folder1. From this directory, the relative paths
are:file1
folder2/file2
../folder3/file3
../file4Problem credits: Mark Gordon

\section*{Sample Input}
\begin{verbatim}
8
bessie 3 2 6 8
folder1 2 3 4
file1 0
folder2 1 5
file2 0
folder3 1 7
file3 0
file4 0
\end{verbatim}

\section*{Sample Output}
\begin{verbatim}
42

This input describes the example directory structure given above.The best solution is to be in folder1. From this directory, the relative paths
are:file1
folder2/file2
../folder3/file3
../file4Problem credits: Mark Gordon

The best solution is to be in folder1. From this directory, the relative paths
are:file1
folder2/file2
../folder3/file3
../file4Problem credits: Mark Gordon

file1
folder2/file2
../folder3/file3
../file4Problem credits: Mark Gordon

file1
folder2/file2
../folder3/file3
../file4

Problem credits: Mark Gordon

Problem credits: Mark Gordon
\end{verbatim}

\section*{Solution}


\section*{Problem URL}
https://usaco.org/index.php?page=viewproblem2&cpid=814

\section*{Source}
USACO FEB18 Contest, Gold Division, Problem ID: 814

\end{document}
