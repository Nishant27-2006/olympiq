\documentclass[12pt]{article}
\usepackage{amsmath}
\usepackage{amssymb}
\usepackage{graphicx}
\usepackage{enumitem}
\usepackage{hyperref}
\usepackage{xcolor}

\title{View problem}
\author{USACO Contest: OPEN25 Contest - Gold Division}
\date{\today}

\begin{document}

\maketitle

\section*{Problem Statement}


Farmer John is trying to make his world's famous OohMoo Milk to sell for a
profit. He has $N$ $(1 \leq N \leq 10^5)$ bottles that he is trying to fill.
Each bottle initially contains some amount of milk $m_i$
$(0 \leq m_i \leq 10^9)$. Every day, he takes $A$ $(1 \le A \le N)$ bottles and
fills each bottle with one unit of milk. 

Unfortunately, Farmer Nhoj, Farmer John's competitor in the business of OohMoo
Milk,  knows about Farmer John's production processes and has a plan to curtail
his business.  Every day, after Farmer John fills his $A$ bottles, Farmer Nhoj
will sneakily steal one unit of milk from each of $B$ $(0 \le B < A)$ different
nonempty bottles. To remain sneaky, Farmer Nhoj chooses $B$ so that it is
strictly less than $A$, so that it is less likely for Farmer John to discover
him.

After $D$ ($1 \leq D \leq 10^9$) days, Farmer John will sell his OohMoo Milk. If
a bottle has $M$ units of milk, it will sell for $M^2$ moonies. 

Let $P$ be the unique profit such that FJ can guarantee that he makes at least
$P$ profit  regardless of how FN behaves, and FN can guarantee that FJ makes at
most $P$ profit regardless of how  FJ behaves. Output the value of $P$ modulo
$10^9+7$.

INPUT FORMAT (input arrives from the terminal / stdin):
The first line of the input contains $N$ and $D$, where $N$ is the number of
bottles and $D$ is the number of days that take place.

The second line of the input contains $A$ and $B$ representing the number of
units of milk that Farmer John fills and Farmer Nhoj steals respectively.

The third line of the input contains $N$ space-separated integers $m_i$
representing the initial amount of milk in each bottle.


OUTPUT FORMAT (print output to the terminal / stdout):
Output the value of $P$ modulo $10^9+7$.

SAMPLE INPUT:
5 4
4 2
4 10 8 10 10
SAMPLE OUTPUT: 
546

On the first day, Farmer John could add milk to the second, third,
fourth, and fifth bottles. Then, Farmer Nhoj could remove milk from the
second and fourth bottles. 

Thus, the new amount of milk in each bottle is
$$[4, 10, 8, 10, 10] \to [4, 11, 9, 11, 11] \to [4, 10, 9, 10, 11].$$
After four days, the amount of milk in each bottle could be
$$[4, 10, 8, 10, 10] \to [4, 10, 9, 10, 11] \to [4, 10, 10, 11, 11] \to [4, 11, 11, 11, 11] \to [4, 11, 11, 12, 12].$$
The total amount of moonies Farmer John would make in this situation is $4^2+11^2+11^2+12^2+12^2 = 546$. It can be shown that this is the value
of $P$.

SAMPLE INPUT:
10 5
5 1
1 2 3 4 5 6 7 8 9 10
SAMPLE OUTPUT: 
777

SAMPLE INPUT:
5 1000000000
3 1
0 1 2 3 4
SAMPLE OUTPUT: 
10

Make sure you output $P$ modulo $10^9+7$.

SCORING:
Inputs 4-6: $N,D\le 1000$. Inputs 7-10: $D\le 10^6$. Inputs 11-20: No additional constraints. 


Problem credits: Suhas Nagar



\section*{Input Format}
The second line of the input contains $A$ and $B$ representing the number of
units of milk that Farmer John fills and Farmer Nhoj steals respectively.The third line of the input contains $N$ space-separated integers $m_i$
representing the initial amount of milk in each bottle.

The third line of the input contains $N$ space-separated integers $m_i$
representing the initial amount of milk in each bottle.

OUTPUT FORMAT (print output to the terminal / stdout):Output the value of $P$ modulo $10^9+7$.SAMPLE INPUT:5 4
4 2
4 10 8 10 10SAMPLE OUTPUT:546On the first day, Farmer John could add milk to the second, third,
fourth, and fifth bottles. Then, Farmer Nhoj could remove milk from the
second and fourth bottles.Thus, the new amount of milk in each bottle is$$[4, 10, 8, 10, 10] \to [4, 11, 9, 11, 11] \to [4, 10, 9, 10, 11].$$After four days, the amount of milk in each bottle could be$$[4, 10, 8, 10, 10] \to [4, 10, 9, 10, 11] \to [4, 10, 10, 11, 11] \to [4, 11, 11, 11, 11] \to [4, 11, 11, 12, 12].$$The total amount of moonies Farmer John would make in this situation is $4^2+11^2+11^2+12^2+12^2 = 546$. It can be shown that this is the value
of $P$.SAMPLE INPUT:10 5
5 1
1 2 3 4 5 6 7 8 9 10SAMPLE OUTPUT:777SAMPLE INPUT:5 1000000000
3 1
0 1 2 3 4SAMPLE OUTPUT:10Make sure you output $P$ modulo $10^9+7$.SCORING:Inputs 4-6: $N,D\le 1000$.Inputs 7-10: $D\le 10^6$.Inputs 11-20: No additional constraints.Problem credits: Suhas Nagar

\section*{Output Format}
SAMPLE INPUT:5 4
4 2
4 10 8 10 10SAMPLE OUTPUT:546On the first day, Farmer John could add milk to the second, third,
fourth, and fifth bottles. Then, Farmer Nhoj could remove milk from the
second and fourth bottles.Thus, the new amount of milk in each bottle is$$[4, 10, 8, 10, 10] \to [4, 11, 9, 11, 11] \to [4, 10, 9, 10, 11].$$After four days, the amount of milk in each bottle could be$$[4, 10, 8, 10, 10] \to [4, 10, 9, 10, 11] \to [4, 10, 10, 11, 11] \to [4, 11, 11, 11, 11] \to [4, 11, 11, 12, 12].$$The total amount of moonies Farmer John would make in this situation is $4^2+11^2+11^2+12^2+12^2 = 546$. It can be shown that this is the value
of $P$.SAMPLE INPUT:10 5
5 1
1 2 3 4 5 6 7 8 9 10SAMPLE OUTPUT:777SAMPLE INPUT:5 1000000000
3 1
0 1 2 3 4SAMPLE OUTPUT:10Make sure you output $P$ modulo $10^9+7$.SCORING:Inputs 4-6: $N,D\le 1000$.Inputs 7-10: $D\le 10^6$.Inputs 11-20: No additional constraints.Problem credits: Suhas Nagar

\section*{Sample Input}
\begin{verbatim}
5 4
4 2
4 10 8 10 10

10 5
5 1
1 2 3 4 5 6 7 8 9 10

5 1000000000
3 1
0 1 2 3 4
\end{verbatim}

\section*{Sample Output}
\begin{verbatim}
546

On the first day, Farmer John could add milk to the second, third,
fourth, and fifth bottles. Then, Farmer Nhoj could remove milk from the
second and fourth bottles.Thus, the new amount of milk in each bottle is$$[4, 10, 8, 10, 10] \to [4, 11, 9, 11, 11] \to [4, 10, 9, 10, 11].$$After four days, the amount of milk in each bottle could be$$[4, 10, 8, 10, 10] \to [4, 10, 9, 10, 11] \to [4, 10, 10, 11, 11] \to [4, 11, 11, 11, 11] \to [4, 11, 11, 12, 12].$$The total amount of moonies Farmer John would make in this situation is $4^2+11^2+11^2+12^2+12^2 = 546$. It can be shown that this is the value
of $P$.SAMPLE INPUT:10 5
5 1
1 2 3 4 5 6 7 8 9 10SAMPLE OUTPUT:777SAMPLE INPUT:5 1000000000
3 1
0 1 2 3 4SAMPLE OUTPUT:10Make sure you output $P$ modulo $10^9+7$.SCORING:Inputs 4-6: $N,D\le 1000$.Inputs 7-10: $D\le 10^6$.Inputs 11-20: No additional constraints.Problem credits: Suhas Nagar

Thus, the new amount of milk in each bottle is$$[4, 10, 8, 10, 10] \to [4, 11, 9, 11, 11] \to [4, 10, 9, 10, 11].$$After four days, the amount of milk in each bottle could be$$[4, 10, 8, 10, 10] \to [4, 10, 9, 10, 11] \to [4, 10, 10, 11, 11] \to [4, 11, 11, 11, 11] \to [4, 11, 11, 12, 12].$$The total amount of moonies Farmer John would make in this situation is $4^2+11^2+11^2+12^2+12^2 = 546$. It can be shown that this is the value
of $P$.SAMPLE INPUT:10 5
5 1
1 2 3 4 5 6 7 8 9 10SAMPLE OUTPUT:777SAMPLE INPUT:5 1000000000
3 1
0 1 2 3 4SAMPLE OUTPUT:10Make sure you output $P$ modulo $10^9+7$.SCORING:Inputs 4-6: $N,D\le 1000$.Inputs 7-10: $D\le 10^6$.Inputs 11-20: No additional constraints.Problem credits: Suhas Nagar

After four days, the amount of milk in each bottle could be$$[4, 10, 8, 10, 10] \to [4, 10, 9, 10, 11] \to [4, 10, 10, 11, 11] \to [4, 11, 11, 11, 11] \to [4, 11, 11, 12, 12].$$The total amount of moonies Farmer John would make in this situation is $4^2+11^2+11^2+12^2+12^2 = 546$. It can be shown that this is the value
of $P$.SAMPLE INPUT:10 5
5 1
1 2 3 4 5 6 7 8 9 10SAMPLE OUTPUT:777SAMPLE INPUT:5 1000000000
3 1
0 1 2 3 4SAMPLE OUTPUT:10Make sure you output $P$ modulo $10^9+7$.SCORING:Inputs 4-6: $N,D\le 1000$.Inputs 7-10: $D\le 10^6$.Inputs 11-20: No additional constraints.Problem credits: Suhas Nagar

The total amount of moonies Farmer John would make in this situation is $4^2+11^2+11^2+12^2+12^2 = 546$. It can be shown that this is the value
of $P$.SAMPLE INPUT:10 5
5 1
1 2 3 4 5 6 7 8 9 10SAMPLE OUTPUT:777SAMPLE INPUT:5 1000000000
3 1
0 1 2 3 4SAMPLE OUTPUT:10Make sure you output $P$ modulo $10^9+7$.SCORING:Inputs 4-6: $N,D\le 1000$.Inputs 7-10: $D\le 10^6$.Inputs 11-20: No additional constraints.Problem credits: Suhas Nagar

SAMPLE INPUT:10 5
5 1
1 2 3 4 5 6 7 8 9 10SAMPLE OUTPUT:777SAMPLE INPUT:5 1000000000
3 1
0 1 2 3 4SAMPLE OUTPUT:10Make sure you output $P$ modulo $10^9+7$.SCORING:Inputs 4-6: $N,D\le 1000$.Inputs 7-10: $D\le 10^6$.Inputs 11-20: No additional constraints.Problem credits: Suhas Nagar

777

SAMPLE INPUT:5 1000000000
3 1
0 1 2 3 4SAMPLE OUTPUT:10Make sure you output $P$ modulo $10^9+7$.SCORING:Inputs 4-6: $N,D\le 1000$.Inputs 7-10: $D\le 10^6$.Inputs 11-20: No additional constraints.Problem credits: Suhas Nagar

10

Make sure you output $P$ modulo $10^9+7$.SCORING:Inputs 4-6: $N,D\le 1000$.Inputs 7-10: $D\le 10^6$.Inputs 11-20: No additional constraints.Problem credits: Suhas Nagar

SCORING:Inputs 4-6: $N,D\le 1000$.Inputs 7-10: $D\le 10^6$.Inputs 11-20: No additional constraints.Problem credits: Suhas Nagar
\end{verbatim}

\section*{Solution}


\section*{Problem URL}
https://usaco.org/index.php?page=viewproblem2&cpid=1523

\section*{Source}
USACO OPEN25 Contest, Gold Division, Problem ID: 1523

\end{document}
