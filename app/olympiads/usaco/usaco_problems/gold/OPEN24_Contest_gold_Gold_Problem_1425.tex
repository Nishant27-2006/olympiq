\documentclass[12pt]{article}
\usepackage{amsmath}
\usepackage{amssymb}
\usepackage{graphicx}
\usepackage{enumitem}
\usepackage{hyperref}
\usepackage{xcolor}

\title{View problem}
\author{USACO Contest: OPEN24 Contest - Gold Division}
\date{\today}

\begin{document}

\maketitle

\section*{Problem Statement}


The cows have formed a dance team, and Farmer John is their choreographer! The
team's latest and greatest dance involves $N$ cows ($2 \le N \le 10^6$) standing
in a line. Each move in the dance involves two cows, up to $K$ positions apart
($1 \le K < N$), gracefully jumping and landing in each other's position.

There are two types of cows in the line – Guernseys and Holsteins. As such,
Farmer John has documented the dance as a sequence of length-$N$ binary
strings, where a $0$ represents a Guernsey, a $1$ represents a Holstein, and
the overall string represents how the cows are arranged in the line.

Unfortunately, Farmer Nhoj (who choreographs for a rival team) has sabotaged the
dance and erased all but the first and last binary strings! With a big
competition quickly approaching, Farmer John must waste no time in
reconstructing the dance.

Given these two binary strings, help Farmer John find the minimum number of
moves in the dance!

INPUT FORMAT (input arrives from the terminal / stdin):
The first line contains $N$ and $K$.

The second line contains the first binary string.

The third line contains the last binary string.

It is guaranteed that both binary strings contain the same number of ones.

OUTPUT FORMAT (print output to the terminal / stdout):
The minimum number of moves in the dance.

SAMPLE INPUT:
4 1
0111
1110
SAMPLE OUTPUT: 
3

One possible dance:


0111 -> 1011 -> 1101 -> 1110

SAMPLE INPUT:
5 2
11000
00011
SAMPLE OUTPUT: 
3

One possible dance:


11000 -> 01100 -> 00110 -> 00011

SAMPLE INPUT:
5 4
11000
00011
SAMPLE OUTPUT: 
2

One possible dance:


11000 -> 10010 -> 00011

SCORING:
Inputs 4-5: $K=1$Inputs 6-7: Both strings have at most $8$ ones.Inputs 8-15: $N\le 5000$Inputs 16-23: No additional constraints.


Problem credits: Benjamin Qi



\section*{Input Format}
The second line contains the first binary string.The third line contains the last binary string.It is guaranteed that both binary strings contain the same number of ones.

The third line contains the last binary string.It is guaranteed that both binary strings contain the same number of ones.

It is guaranteed that both binary strings contain the same number of ones.

OUTPUT FORMAT (print output to the terminal / stdout):The minimum number of moves in the dance.SAMPLE INPUT:4 1
0111
1110SAMPLE OUTPUT:3One possible dance:0111 -> 1011 -> 1101 -> 1110SAMPLE INPUT:5 2
11000
00011SAMPLE OUTPUT:3One possible dance:11000 -> 01100 -> 00110 -> 00011SAMPLE INPUT:5 4
11000
00011SAMPLE OUTPUT:2One possible dance:11000 -> 10010 -> 00011SCORING:Inputs 4-5: $K=1$Inputs 6-7: Both strings have at most $8$ ones.Inputs 8-15: $N\le 5000$Inputs 16-23: No additional constraints.Problem credits: Benjamin Qi

\section*{Output Format}
SAMPLE INPUT:4 1
0111
1110SAMPLE OUTPUT:3One possible dance:0111 -> 1011 -> 1101 -> 1110SAMPLE INPUT:5 2
11000
00011SAMPLE OUTPUT:3One possible dance:11000 -> 01100 -> 00110 -> 00011SAMPLE INPUT:5 4
11000
00011SAMPLE OUTPUT:2One possible dance:11000 -> 10010 -> 00011SCORING:Inputs 4-5: $K=1$Inputs 6-7: Both strings have at most $8$ ones.Inputs 8-15: $N\le 5000$Inputs 16-23: No additional constraints.Problem credits: Benjamin Qi

\section*{Sample Input}
\begin{verbatim}
4 1
0111
1110

5 2
11000
00011

5 4
11000
00011
\end{verbatim}

\section*{Sample Output}
\begin{verbatim}
3

One possible dance:0111 -> 1011 -> 1101 -> 1110SAMPLE INPUT:5 2
11000
00011SAMPLE OUTPUT:3One possible dance:11000 -> 01100 -> 00110 -> 00011SAMPLE INPUT:5 4
11000
00011SAMPLE OUTPUT:2One possible dance:11000 -> 10010 -> 00011SCORING:Inputs 4-5: $K=1$Inputs 6-7: Both strings have at most $8$ ones.Inputs 8-15: $N\le 5000$Inputs 16-23: No additional constraints.Problem credits: Benjamin Qi

0111 -> 1011 -> 1101 -> 1110SAMPLE INPUT:5 2
11000
00011SAMPLE OUTPUT:3One possible dance:11000 -> 01100 -> 00110 -> 00011SAMPLE INPUT:5 4
11000
00011SAMPLE OUTPUT:2One possible dance:11000 -> 10010 -> 00011SCORING:Inputs 4-5: $K=1$Inputs 6-7: Both strings have at most $8$ ones.Inputs 8-15: $N\le 5000$Inputs 16-23: No additional constraints.Problem credits: Benjamin Qi

0111 -> 1011 -> 1101 -> 1110

SAMPLE INPUT:5 2
11000
00011SAMPLE OUTPUT:3One possible dance:11000 -> 01100 -> 00110 -> 00011SAMPLE INPUT:5 4
11000
00011SAMPLE OUTPUT:2One possible dance:11000 -> 10010 -> 00011SCORING:Inputs 4-5: $K=1$Inputs 6-7: Both strings have at most $8$ ones.Inputs 8-15: $N\le 5000$Inputs 16-23: No additional constraints.Problem credits: Benjamin Qi

3

One possible dance:11000 -> 01100 -> 00110 -> 00011SAMPLE INPUT:5 4
11000
00011SAMPLE OUTPUT:2One possible dance:11000 -> 10010 -> 00011SCORING:Inputs 4-5: $K=1$Inputs 6-7: Both strings have at most $8$ ones.Inputs 8-15: $N\le 5000$Inputs 16-23: No additional constraints.Problem credits: Benjamin Qi

11000 -> 01100 -> 00110 -> 00011SAMPLE INPUT:5 4
11000
00011SAMPLE OUTPUT:2One possible dance:11000 -> 10010 -> 00011SCORING:Inputs 4-5: $K=1$Inputs 6-7: Both strings have at most $8$ ones.Inputs 8-15: $N\le 5000$Inputs 16-23: No additional constraints.Problem credits: Benjamin Qi

11000 -> 01100 -> 00110 -> 00011

SAMPLE INPUT:5 4
11000
00011SAMPLE OUTPUT:2One possible dance:11000 -> 10010 -> 00011SCORING:Inputs 4-5: $K=1$Inputs 6-7: Both strings have at most $8$ ones.Inputs 8-15: $N\le 5000$Inputs 16-23: No additional constraints.Problem credits: Benjamin Qi

2

One possible dance:11000 -> 10010 -> 00011SCORING:Inputs 4-5: $K=1$Inputs 6-7: Both strings have at most $8$ ones.Inputs 8-15: $N\le 5000$Inputs 16-23: No additional constraints.Problem credits: Benjamin Qi

11000 -> 10010 -> 00011SCORING:Inputs 4-5: $K=1$Inputs 6-7: Both strings have at most $8$ ones.Inputs 8-15: $N\le 5000$Inputs 16-23: No additional constraints.Problem credits: Benjamin Qi

11000 -> 10010 -> 00011

SCORING:Inputs 4-5: $K=1$Inputs 6-7: Both strings have at most $8$ ones.Inputs 8-15: $N\le 5000$Inputs 16-23: No additional constraints.Problem credits: Benjamin Qi
\end{verbatim}

\section*{Solution}


\section*{Problem URL}
https://usaco.org/index.php?page=viewproblem2&cpid=1425

\section*{Source}
USACO OPEN24 Contest, Gold Division, Problem ID: 1425

\end{document}
