\documentclass[12pt]{article}
\usepackage{amsmath}
\usepackage{amssymb}
\usepackage{graphicx}
\usepackage{enumitem}
\usepackage{hyperref}
\usepackage{xcolor}

\title{View problem}
\author{USACO Contest: JAN16 Contest - Gold Division}
\date{\today}

\begin{document}

\maketitle

\section*{Problem Statement}

Farmer John has installed a fancy new milking machine in his barn, but it draws
so much power that it occasionally causes the power to go out! This happens so
often that Bessie has memorized a map of the barn, making it easier for her to
find the exit of the barn in the dark. She is curious though about the impact of
power loss on her ability to exit the barn quickly. For example, she wonders how
much farther she might need to walk find the exit in the dark.

The barn is described by a simple (non self-intersecting) polygon with integer 
vertices $(x_1, y_1) \ldots (x_n, y_n)$ listed in clockwise order.  Its edges
alternate between horizontal (parallel to the x-axis) and vertical (parallel to
the y-axis); the first edge can be of either type. The exit is located at
$(x_1, y_1)$.  Bessie starts inside the barn located at some vertex 
$(x_i, y_i)$ for $i > 1$.  She can walk only around the perimeter of the barn,
either clockwise or counterclockwise, Her goal is to travel a minimum distance
to reach the exit. This is relatively easy to do with the lights on, of course,
since she will travel either clockwise or counterclockwise from her current
location to the  exit -- whichever direction is shorter.

One day, the lights go out, causing Bessie to panic and forget
which vertex she is standing at.  Fortunately, she still remembers the
exact map of the barn, so she can possibly figure out her position by
walking around and using her sense of touch.  Whenever she is standing
at a vertex (including at her initial vertex), she can feel the exact
interior angle at that vertex, and she can tell if that vertex is the
exit.  When she walks along an edge of the barn, she can determine the
exact length of the edge after walking along the entire edge.  Bessie
decides on the following strategy: she will move clockwise around the
perimeter of the barn until she has felt enough angles and edges to
deduce the vertex at which she is currently located.  At that point,
she can easily figure out how to get to the exit by traveling a
minimum amount of remaining distance, either by continuing to move
clockwise or by switching direction and moving counter-clockwise.

Please help Bessie determine the largest amount by which her travel distance
will increase in the worst case (over all possibilities for her starting vertex)
for travel in the dark versus in a lit barn.

INPUT FORMAT (file lightsout.in):
The first line of the input contains $N$ ($4 \leq N \leq 200$).  Each of the
next $N$ lines contains two integers, describing the points $(x_i, y_i)$ in
clockwise order around the barn.  These integers are in the range
$-100,000 \ldots 100,000$.

OUTPUT FORMAT (file lightsout.out):
Output the largest amount that Bessie's travel distance will increase in the
worst case starting position using the strategy in the problem statement.

SAMPLE INPUT:
4
0 0
0 10
1 10
1 0
SAMPLE OUTPUT: 
2

In this example, Bessie can feel that she is initially standing at a 90-degree
angle, but she cannot tell if she is initially standing at vertex 2, 3, or 4.
After taking a step along one edge in the clockwise direction, Bessie either
reaches the exit or can uniquely determine her location based on the length of
this edge.  The distances she obtains are:
If starting at vertex 2: she travels 12 units in the dark (1 unit to reach
vertex 3, then 11 units to continue to the exit).  She only needs to travel 10
units in a lit barn.  This is an extra distance of 2 for this vertex.
If starting at vertex 3: she travels 11 units in both cases.
If starting at vertex 4: she travels 1 unit in both cases.
The worst-case difference over all starting points is therefore 12 - 10 = 2.  That
is, Bessie can guarantee that using her strategy, no matter where she starts,
she will travel at most 2 extra units of distance farther in the dark than in the light.  

Problem credits: Brian Dean



\section*{Input Format}
OUTPUT FORMAT (file lightsout.out):Output the largest amount that Bessie's travel distance will increase in the
worst case starting position using the strategy in the problem statement.SAMPLE INPUT:4
0 0
0 10
1 10
1 0SAMPLE OUTPUT:2In this example, Bessie can feel that she is initially standing at a 90-degree
angle, but she cannot tell if she is initially standing at vertex 2, 3, or 4.
After taking a step along one edge in the clockwise direction, Bessie either
reaches the exit or can uniquely determine her location based on the length of
this edge.  The distances she obtains are:If starting at vertex 2: she travels 12 units in the dark (1 unit to reach
vertex 3, then 11 units to continue to the exit).  She only needs to travel 10
units in a lit barn.  This is an extra distance of 2 for this vertex.If starting at vertex 3: she travels 11 units in both cases.If starting at vertex 4: she travels 1 unit in both cases.The worst-case difference over all starting points is therefore 12 - 10 = 2.  That
is, Bessie can guarantee that using her strategy, no matter where she starts,
she will travel at most 2 extra units of distance farther in the dark than in the light.Problem credits: Brian Dean

\section*{Output Format}
SAMPLE INPUT:4
0 0
0 10
1 10
1 0SAMPLE OUTPUT:2In this example, Bessie can feel that she is initially standing at a 90-degree
angle, but she cannot tell if she is initially standing at vertex 2, 3, or 4.
After taking a step along one edge in the clockwise direction, Bessie either
reaches the exit or can uniquely determine her location based on the length of
this edge.  The distances she obtains are:If starting at vertex 2: she travels 12 units in the dark (1 unit to reach
vertex 3, then 11 units to continue to the exit).  She only needs to travel 10
units in a lit barn.  This is an extra distance of 2 for this vertex.If starting at vertex 3: she travels 11 units in both cases.If starting at vertex 4: she travels 1 unit in both cases.The worst-case difference over all starting points is therefore 12 - 10 = 2.  That
is, Bessie can guarantee that using her strategy, no matter where she starts,
she will travel at most 2 extra units of distance farther in the dark than in the light.Problem credits: Brian Dean

\section*{Sample Input}
\begin{verbatim}
4
0 0
0 10
1 10
1 0
\end{verbatim}

\section*{Sample Output}
\begin{verbatim}
2

In this example, Bessie can feel that she is initially standing at a 90-degree
angle, but she cannot tell if she is initially standing at vertex 2, 3, or 4.
After taking a step along one edge in the clockwise direction, Bessie either
reaches the exit or can uniquely determine her location based on the length of
this edge.  The distances she obtains are:If starting at vertex 2: she travels 12 units in the dark (1 unit to reach
vertex 3, then 11 units to continue to the exit).  She only needs to travel 10
units in a lit barn.  This is an extra distance of 2 for this vertex.If starting at vertex 3: she travels 11 units in both cases.If starting at vertex 4: she travels 1 unit in both cases.The worst-case difference over all starting points is therefore 12 - 10 = 2.  That
is, Bessie can guarantee that using her strategy, no matter where she starts,
she will travel at most 2 extra units of distance farther in the dark than in the light.Problem credits: Brian Dean

If starting at vertex 2: she travels 12 units in the dark (1 unit to reach
vertex 3, then 11 units to continue to the exit).  She only needs to travel 10
units in a lit barn.  This is an extra distance of 2 for this vertex.If starting at vertex 3: she travels 11 units in both cases.If starting at vertex 4: she travels 1 unit in both cases.The worst-case difference over all starting points is therefore 12 - 10 = 2.  That
is, Bessie can guarantee that using her strategy, no matter where she starts,
she will travel at most 2 extra units of distance farther in the dark than in the light.Problem credits: Brian Dean

If starting at vertex 3: she travels 11 units in both cases.If starting at vertex 4: she travels 1 unit in both cases.The worst-case difference over all starting points is therefore 12 - 10 = 2.  That
is, Bessie can guarantee that using her strategy, no matter where she starts,
she will travel at most 2 extra units of distance farther in the dark than in the light.Problem credits: Brian Dean

If starting at vertex 4: she travels 1 unit in both cases.The worst-case difference over all starting points is therefore 12 - 10 = 2.  That
is, Bessie can guarantee that using her strategy, no matter where she starts,
she will travel at most 2 extra units of distance farther in the dark than in the light.Problem credits: Brian Dean

The worst-case difference over all starting points is therefore 12 - 10 = 2.  That
is, Bessie can guarantee that using her strategy, no matter where she starts,
she will travel at most 2 extra units of distance farther in the dark than in the light.Problem credits: Brian Dean

Problem credits: Brian Dean
\end{verbatim}

\section*{Solution}


\section*{Problem URL}
https://usaco.org/index.php?page=viewproblem2&cpid=599

\section*{Source}
USACO JAN16 Contest, Gold Division, Problem ID: 599

\end{document}
