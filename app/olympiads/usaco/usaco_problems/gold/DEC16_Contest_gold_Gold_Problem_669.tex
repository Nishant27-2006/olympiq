\documentclass[12pt]{article}
\usepackage{amsmath}
\usepackage{amssymb}
\usepackage{graphicx}
\usepackage{enumitem}
\usepackage{hyperref}
\usepackage{xcolor}

\title{View problem}
\author{USACO Contest: DEC16 Contest - Gold Division}
\date{\today}

\begin{document}

\maketitle

\section*{Problem Statement}

Farmer John's $N$ cows ($1 \leq N \leq 1000$) want to organize an emergency
"moo-cast" system for broadcasting important messages among themselves.  

Instead of mooing at each-other over long distances, the cows decide to equip
themselves with walkie-talkies, one for each cow.  These walkie-talkies each
have a limited  transmission radius, but cows can relay messages to one-another
along a path consisting of several hops, so it is not necessary for every cow to
be able to transmit directly to every other cow.

The cows need to decide how much money to spend on their walkie-talkies.  If
they spend \$X, they will each get a walkie-talkie capable of transmitting up to
a distance of $\sqrt{X}$.  That is, the squared distance between two cows must
be at most $X$ for them to be able to communicate.

Please help the cows determine the minimum integer value of $X$ such that a broadcast
from any cow will ultimately be able to reach every other cow.

INPUT FORMAT (file moocast.in):
The first line of input contains $N$.

The next $N$ lines each contain the $x$ and $y$ coordinates of a single cow. 
These are both integers in the range $0 \ldots 25,000$.

OUTPUT FORMAT (file moocast.out):
Write a single line of output containing the integer $X$ giving the minimum
amount the cows must spend on walkie-talkies.

SAMPLE INPUT:
4
1 3
5 4
7 2
6 1
SAMPLE OUTPUT: 
17

Problem credits: Richard Peng



\section*{Input Format}
The next $N$ lines each contain the $x$ and $y$ coordinates of a single cow. 
These are both integers in the range $0 \ldots 25,000$.

OUTPUT FORMAT (file moocast.out):Write a single line of output containing the integer $X$ giving the minimum
amount the cows must spend on walkie-talkies.SAMPLE INPUT:4
1 3
5 4
7 2
6 1SAMPLE OUTPUT:17Problem credits: Richard Peng

\section*{Output Format}
SAMPLE INPUT:4
1 3
5 4
7 2
6 1SAMPLE OUTPUT:17Problem credits: Richard Peng

\section*{Sample Input}
\begin{verbatim}
4
1 3
5 4
7 2
6 1
\end{verbatim}

\section*{Sample Output}
\begin{verbatim}
17

Problem credits: Richard Peng
\end{verbatim}

\section*{Solution}


\section*{Problem URL}
https://usaco.org/index.php?page=viewproblem2&cpid=669

\section*{Source}
USACO DEC16 Contest, Gold Division, Problem ID: 669

\end{document}
