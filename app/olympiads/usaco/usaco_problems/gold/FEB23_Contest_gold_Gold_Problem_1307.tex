\documentclass[12pt]{article}
\usepackage{amsmath}
\usepackage{amssymb}
\usepackage{graphicx}
\usepackage{enumitem}
\usepackage{hyperref}
\usepackage{xcolor}

\title{View problem}
\author{USACO Contest: FEB23 Contest - Gold Division}
\date{\today}

\begin{document}

\maketitle

\section*{Problem Statement}


Farmer John wrote down  $N$ ($1\le N\le 300$) digits on pieces of paper. For
each $i\in [1,N]$, the $i$th piece of paper contains digit $a_i$
($1 \leq a_i \leq 9$). 

The cows have two favorite integers $A$ and $B$ ($1\le A\le B< 10^{18}$), and
would like you to answer $Q$ ($1\le Q\le 5\cdot 10^4$) queries. For the $i$th
query, the cows will move left to right across papers $l_i\dots r_i$
($1\le l_i\le r_i\le N$), maintaining an initially empty pile of papers.  For
each paper, they will either add it to the top of the pile, to the bottom of the
pile, or neither. In the end, they will read the papers in the pile from top to
bottom, forming an integer. Over all $3^{r_i-l_i+1}$ ways for the cows to make
choices during this process,  count the number of ways that result in the cows
reading an integer in $[A,B]$ inclusive, and output this number modulo $10^9+7$.

INPUT FORMAT (input arrives from the terminal / stdin):
The first line contains three space-separated integers $N$, $A$, and $B$. 

The second line contains $N$ space-separated digits $a_1, a_2, \dots, a_N$.

The third line contains an integer $Q$, the number of queries.

The next $Q$ lines each contain two space-separated integers $l_i$ and $r_i$.

OUTPUT FORMAT (print output to the terminal / stdout):
For each query, a single line containing the answer.

SAMPLE INPUT:
5 13 327
1 2 3 4 5
3
1 2
1 3
2 5
SAMPLE OUTPUT: 
2
18
34

For the first query, there are nine ways Bessie can stack papers when reading
the interval $[1, 2]$:
 Bessie can ignore $1$ then ignore $2$, getting $0$.  Bessie can
ignore $1$ then add $2$ to the top of the stack, getting $2$.  Bessie
can ignore $1$ then add $2$ to the bottom of the stack, getting $2$. 
Bessie can add $1$ to the top of the stack then ignore $2$, getting $1$.  Bessie can add $1$ to the top of the stack then add $2$ to the top of
the stack, getting $21$.  Bessie can add $1$ to the top of the stack
then add $2$ to the bottom of the stack, getting $12$.  Bessie
can add $1$ to the bottom of the stack then ignore $2$, getting $1$. 
Bessie can add $1$ to the bottom of the stack then add $2$ to the top of the
stack, getting $21$.  Bessie can add $1$ to the bottom of the stack
then add $2$ to the bottom of the stack, getting $12$. 
Only the $2$ ways that give $21$ yield a number between $13$ and $327$, so the
answer is $2$.

SCORING:
Inputs 2-3: $B<100$Inputs 4-5: $A=B$Inputs 6-13: No additional constraints.


Problem credits: Jesse Choe



\section*{Input Format}
The first line contains three space-separated integers $N$, $A$, and $B$.The second line contains $N$ space-separated digits $a_1, a_2, \dots, a_N$.The third line contains an integer $Q$, the number of queries.The next $Q$ lines each contain two space-separated integers $l_i$ and $r_i$.

The second line contains $N$ space-separated digits $a_1, a_2, \dots, a_N$.The third line contains an integer $Q$, the number of queries.The next $Q$ lines each contain two space-separated integers $l_i$ and $r_i$.

The third line contains an integer $Q$, the number of queries.The next $Q$ lines each contain two space-separated integers $l_i$ and $r_i$.

The next $Q$ lines each contain two space-separated integers $l_i$ and $r_i$.

OUTPUT FORMAT (print output to the terminal / stdout):For each query, a single line containing the answer.SAMPLE INPUT:5 13 327
1 2 3 4 5
3
1 2
1 3
2 5SAMPLE OUTPUT:2
18
34For the first query, there are nine ways Bessie can stack papers when reading
the interval $[1, 2]$:Bessie can ignore $1$ then ignore $2$, getting $0$.Bessie can
ignore $1$ then add $2$ to the top of the stack, getting $2$.Bessie
can ignore $1$ then add $2$ to the bottom of the stack, getting $2$.Bessie can add $1$ to the top of the stack then ignore $2$, getting $1$.Bessie can add $1$ to the top of the stack then add $2$ to the top of
the stack, getting $21$.Bessie can add $1$ to the top of the stack
then add $2$ to the bottom of the stack, getting $12$.Bessie
can add $1$ to the bottom of the stack then ignore $2$, getting $1$.Bessie can add $1$ to the bottom of the stack then add $2$ to the top of the
stack, getting $21$.Bessie can add $1$ to the bottom of the stack
then add $2$ to the bottom of the stack, getting $12$.Only the $2$ ways that give $21$ yield a number between $13$ and $327$, so the
answer is $2$.SCORING:Inputs 2-3: $B<100$Inputs 4-5: $A=B$Inputs 6-13: No additional constraints.Problem credits: Jesse Choe

\section*{Output Format}
SAMPLE INPUT:5 13 327
1 2 3 4 5
3
1 2
1 3
2 5SAMPLE OUTPUT:2
18
34For the first query, there are nine ways Bessie can stack papers when reading
the interval $[1, 2]$:Bessie can ignore $1$ then ignore $2$, getting $0$.Bessie can
ignore $1$ then add $2$ to the top of the stack, getting $2$.Bessie
can ignore $1$ then add $2$ to the bottom of the stack, getting $2$.Bessie can add $1$ to the top of the stack then ignore $2$, getting $1$.Bessie can add $1$ to the top of the stack then add $2$ to the top of
the stack, getting $21$.Bessie can add $1$ to the top of the stack
then add $2$ to the bottom of the stack, getting $12$.Bessie
can add $1$ to the bottom of the stack then ignore $2$, getting $1$.Bessie can add $1$ to the bottom of the stack then add $2$ to the top of the
stack, getting $21$.Bessie can add $1$ to the bottom of the stack
then add $2$ to the bottom of the stack, getting $12$.Only the $2$ ways that give $21$ yield a number between $13$ and $327$, so the
answer is $2$.SCORING:Inputs 2-3: $B<100$Inputs 4-5: $A=B$Inputs 6-13: No additional constraints.Problem credits: Jesse Choe

\section*{Sample Input}
\begin{verbatim}
5 13 327
1 2 3 4 5
3
1 2
1 3
2 5
\end{verbatim}

\section*{Sample Output}
\begin{verbatim}
2
18
34

For the first query, there are nine ways Bessie can stack papers when reading
the interval $[1, 2]$:Bessie can ignore $1$ then ignore $2$, getting $0$.Bessie can
ignore $1$ then add $2$ to the top of the stack, getting $2$.Bessie
can ignore $1$ then add $2$ to the bottom of the stack, getting $2$.Bessie can add $1$ to the top of the stack then ignore $2$, getting $1$.Bessie can add $1$ to the top of the stack then add $2$ to the top of
the stack, getting $21$.Bessie can add $1$ to the top of the stack
then add $2$ to the bottom of the stack, getting $12$.Bessie
can add $1$ to the bottom of the stack then ignore $2$, getting $1$.Bessie can add $1$ to the bottom of the stack then add $2$ to the top of the
stack, getting $21$.Bessie can add $1$ to the bottom of the stack
then add $2$ to the bottom of the stack, getting $12$.Only the $2$ ways that give $21$ yield a number between $13$ and $327$, so the
answer is $2$.SCORING:Inputs 2-3: $B<100$Inputs 4-5: $A=B$Inputs 6-13: No additional constraints.Problem credits: Jesse Choe

SCORING:Inputs 2-3: $B<100$Inputs 4-5: $A=B$Inputs 6-13: No additional constraints.Problem credits: Jesse Choe
\end{verbatim}

\section*{Solution}


\section*{Problem URL}
https://usaco.org/index.php?page=viewproblem2&cpid=1307

\section*{Source}
USACO FEB23 Contest, Gold Division, Problem ID: 1307

\end{document}
