\documentclass[12pt]{article}
\usepackage{amsmath}
\usepackage{amssymb}
\usepackage{graphicx}
\usepackage{enumitem}
\usepackage{hyperref}
\usepackage{xcolor}

\title{View problem}
\author{USACO Contest: OPEN23 Contest - Gold Division}
\date{\today}

\begin{document}

\maketitle

\section*{Problem Statement}


Having just completed a course in graph algorithms, Bessie the cow has begun
coding her very own graph visualizer! Currently, her graph visualizer is only
capable of visualizing rooted trees with nodes of distinct values, and it can
only perform one kind of operation: merging.

In particular, a merging operation takes any two distinct nodes in a tree with
the same parent and merges them into one node, with value equal to the maximum
of the values of the two nodes merged, and children a union of all the children
of the nodes merged (if any).

Unfortunately, after Bessie performed some merging operations on a tree, her
program crashed, losing the history of the merging operations she performed. All
Bessie remembers is the tree she started with and the tree she ended with after
she performed all her merging operations. 

Given her initial and final trees, please determine a sequence of merging
operations Bessie could have performed. It is guaranteed that a sequence exists.

Each input consists of $T$ ($1\le T\le 100$) independent test cases. It is
guaranteed that the sum of $N$  over all test cases does not exceed $1000$.

INPUT FORMAT (input arrives from the terminal / stdin):
The first line contains $T$, the number of independent test cases. Each test
case is formatted as follows.

The first line of each test case contains the number of nodes $N$
($2 \leq N \leq 1000$) in Bessie's initial tree, which have values $1\dots N$. 

Each of the next $N-1$ lines contains two space-separated node values $v_i$ and
$p_i$ ($1 \leq v_i, p_i \leq N$) indicating that the node with value $v_i$ is a
child node of the node with value $p_i$ in Bessie's initial tree.

The next line contains the number of nodes $M$ ($2 \leq M \leq N$) in Bessie's
final tree. 

Each of the next $M-1$ lines contains two space-separated node values $v_i$ and
$p_i$ ($1 \leq v_i, p_i \leq N$) indicating that the node with value $v_i$ is a
child node of the node with value $p_i$ in Bessie's final tree.

OUTPUT FORMAT (print output to the terminal / stdout):
For each test case, output the number of merging operations, followed by an
ordered sequence of merging operations of that length, one per line. 

Each merging operation should be formatted as two distinct space-separated
integers: the values of the two nodes to merge in any order. 

If there are multiple solutions, output any.


SAMPLE INPUT:
1
8
7 5
2 1
4 2
5 1
3 2
8 5
6 2
4
8 5
5 1
6 5
SAMPLE OUTPUT: 
4
2 5
4 8
3 8
7 8

SCORING:
Inputs 2-6: The initial and final trees have the same number of leaves.
Inputs 7-16: No additional constraints.


Problem credits: Aryansh Shrivastava



\section*{Input Format}
The first line of each test case contains the number of nodes $N$
($2 \leq N \leq 1000$) in Bessie's initial tree, which have values $1\dots N$.Each of the next $N-1$ lines contains two space-separated node values $v_i$ and
$p_i$ ($1 \leq v_i, p_i \leq N$) indicating that the node with value $v_i$ is a
child node of the node with value $p_i$ in Bessie's initial tree.The next line contains the number of nodes $M$ ($2 \leq M \leq N$) in Bessie's
final tree.Each of the next $M-1$ lines contains two space-separated node values $v_i$ and
$p_i$ ($1 \leq v_i, p_i \leq N$) indicating that the node with value $v_i$ is a
child node of the node with value $p_i$ in Bessie's final tree.

Each of the next $N-1$ lines contains two space-separated node values $v_i$ and
$p_i$ ($1 \leq v_i, p_i \leq N$) indicating that the node with value $v_i$ is a
child node of the node with value $p_i$ in Bessie's initial tree.The next line contains the number of nodes $M$ ($2 \leq M \leq N$) in Bessie's
final tree.Each of the next $M-1$ lines contains two space-separated node values $v_i$ and
$p_i$ ($1 \leq v_i, p_i \leq N$) indicating that the node with value $v_i$ is a
child node of the node with value $p_i$ in Bessie's final tree.

The next line contains the number of nodes $M$ ($2 \leq M \leq N$) in Bessie's
final tree.Each of the next $M-1$ lines contains two space-separated node values $v_i$ and
$p_i$ ($1 \leq v_i, p_i \leq N$) indicating that the node with value $v_i$ is a
child node of the node with value $p_i$ in Bessie's final tree.

Each of the next $M-1$ lines contains two space-separated node values $v_i$ and
$p_i$ ($1 \leq v_i, p_i \leq N$) indicating that the node with value $v_i$ is a
child node of the node with value $p_i$ in Bessie's final tree.

OUTPUT FORMAT (print output to the terminal / stdout):For each test case, output the number of merging operations, followed by an
ordered sequence of merging operations of that length, one per line.Each merging operation should be formatted as two distinct space-separated
integers: the values of the two nodes to merge in any order.If there are multiple solutions, output any.SAMPLE INPUT:1
8
7 5
2 1
4 2
5 1
3 2
8 5
6 2
4
8 5
5 1
6 5SAMPLE OUTPUT:4
2 5
4 8
3 8
7 8SCORING:Inputs 2-6: The initial and final trees have the same number of leaves.Inputs 7-16: No additional constraints.Problem credits: Aryansh Shrivastava

\section*{Output Format}
Each merging operation should be formatted as two distinct space-separated
integers: the values of the two nodes to merge in any order.If there are multiple solutions, output any.

If there are multiple solutions, output any.

SAMPLE INPUT:1
8
7 5
2 1
4 2
5 1
3 2
8 5
6 2
4
8 5
5 1
6 5SAMPLE OUTPUT:4
2 5
4 8
3 8
7 8SCORING:Inputs 2-6: The initial and final trees have the same number of leaves.Inputs 7-16: No additional constraints.Problem credits: Aryansh Shrivastava

\section*{Sample Input}
\begin{verbatim}
1
8
7 5
2 1
4 2
5 1
3 2
8 5
6 2
4
8 5
5 1
6 5
\end{verbatim}

\section*{Sample Output}
\begin{verbatim}
4
2 5
4 8
3 8
7 8

SCORING:Inputs 2-6: The initial and final trees have the same number of leaves.Inputs 7-16: No additional constraints.Problem credits: Aryansh Shrivastava
\end{verbatim}

\section*{Solution}


\section*{Problem URL}
https://usaco.org/index.php?page=viewproblem2&cpid=1331

\section*{Source}
USACO OPEN23 Contest, Gold Division, Problem ID: 1331

\end{document}
