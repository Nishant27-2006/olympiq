\documentclass[12pt]{article}
\usepackage{amsmath}
\usepackage{amssymb}
\usepackage{graphicx}
\usepackage{enumitem}
\usepackage{hyperref}
\usepackage{xcolor}

\title{View problem}
\author{USACO Contest: JAN22 Contest - Gold Division}
\date{\today}

\begin{document}

\maketitle

\section*{Problem Statement}

Farmer John's cows have decided to offer a programming contest for the cows on
Farmer Nhoj's farm.  In order to make the problems as fun as possible, they have
spent considerable time coming up with challenging input cases. For one problem
in particular, "Haybales", the cows need your help devising challenging inputs. 
This involve solving the following somewhat intriguing problem:

There is an array of sorted integers $x_1 \leq x_2 \leq \dotsb \leq x_N$
($1 \leq N \leq 10^5$), and an integer $K$. You don't know the array or $K$, but
you do know for each index $i$, the largest index $j_i$ such that
$x_{j_i} \leq x_i + K$. It is guaranteed that $i\le j_i$ and
$j_1\le j_2\le \cdots \le j_N\le N$.

Given this information, Farmer John's cows need to construct any array along 
with some integer $K$ that matches that information. The construction needs to
satisfy $0 \leq x_i \leq 10^{18}$ for all $i$ and $1 \leq K \leq 10^{18}$. 

It can be proven that this is always possible. Help Farmer John's cows solve
this problem!

INPUT FORMAT (input arrives from the terminal / stdin):
The first line of input contains $N$.  The next line contains
$j_1,j_2,\ldots,j_N$.

OUTPUT FORMAT (print output to the terminal / stdout):
Print $K$, then $x_1,\ldots,x_N$ on separate lines. Any valid output will be
accepted.

SAMPLE INPUT:
6
2 2 4 5 6 6
SAMPLE OUTPUT: 
6
1
6
17
22
27
32

The sample output is the array $a = [1, 6, 17, 22, 27, 32]$ with $K = 6$.
$j_1 = 2$ is satisfied because $a_2 = 6 \leq 1 + 6 = a_1 + K$ but
$a_3 = 17 > 1 + 6 = a_1 + K$, so $a_2$ is the largest element that is at most
$a_1$. Similarly,
$j_2 = 2$ is satisfied because $a_2 = 6 \leq 6 + 6$ but
$a_3 = 17 > 6 + 6$$j_3 = 4$ is satisfied because
$a_4 = 22 \leq 17 + 6$ but $a_5 = 27 > 17 + 6$$j_4 = 5$ is satisfied
because $a_5 = 27 \leq 22 + 6$ but $a_5 = 32 > 22 + 6$$j_5 = 6$ is
satisfied because $a_6 = 32 \leq 27 + 6$ and $a_6$ is the last element of the
array$j_6 = 6$ is satisfied because $a_6 = 32 \leq 32 + 6$ and $a_6$
is the last element of the array
This is not the only possible correct output for the sample input. For example,
you could instead output the array $[1, 2, 4, 5, 6, 7]$ with $K = 1$.


SCORING:
For 50% of all inputs, $N\le 5000$For the remaining inputs, there are no additional constraints.


Problem credits: Danny Mittal



\section*{Input Format}
OUTPUT FORMAT (print output to the terminal / stdout):Print $K$, then $x_1,\ldots,x_N$ on separate lines. Any valid output will be
accepted.SAMPLE INPUT:6
2 2 4 5 6 6SAMPLE OUTPUT:6
1
6
17
22
27
32The sample output is the array $a = [1, 6, 17, 22, 27, 32]$ with $K = 6$.
$j_1 = 2$ is satisfied because $a_2 = 6 \leq 1 + 6 = a_1 + K$ but
$a_3 = 17 > 1 + 6 = a_1 + K$, so $a_2$ is the largest element that is at most
$a_1$. Similarly,$j_2 = 2$ is satisfied because $a_2 = 6 \leq 6 + 6$ but
$a_3 = 17 > 6 + 6$$j_3 = 4$ is satisfied because
$a_4 = 22 \leq 17 + 6$ but $a_5 = 27 > 17 + 6$$j_4 = 5$ is satisfied
because $a_5 = 27 \leq 22 + 6$ but $a_5 = 32 > 22 + 6$$j_5 = 6$ is
satisfied because $a_6 = 32 \leq 27 + 6$ and $a_6$ is the last element of the
array$j_6 = 6$ is satisfied because $a_6 = 32 \leq 32 + 6$ and $a_6$
is the last element of the arrayThis is not the only possible correct output for the sample input. For example,
you could instead output the array $[1, 2, 4, 5, 6, 7]$ with $K = 1$.SCORING:For 50% of all inputs, $N\le 5000$For the remaining inputs, there are no additional constraints.Problem credits: Danny Mittal

\section*{Output Format}
SAMPLE INPUT:6
2 2 4 5 6 6SAMPLE OUTPUT:6
1
6
17
22
27
32The sample output is the array $a = [1, 6, 17, 22, 27, 32]$ with $K = 6$.
$j_1 = 2$ is satisfied because $a_2 = 6 \leq 1 + 6 = a_1 + K$ but
$a_3 = 17 > 1 + 6 = a_1 + K$, so $a_2$ is the largest element that is at most
$a_1$. Similarly,$j_2 = 2$ is satisfied because $a_2 = 6 \leq 6 + 6$ but
$a_3 = 17 > 6 + 6$$j_3 = 4$ is satisfied because
$a_4 = 22 \leq 17 + 6$ but $a_5 = 27 > 17 + 6$$j_4 = 5$ is satisfied
because $a_5 = 27 \leq 22 + 6$ but $a_5 = 32 > 22 + 6$$j_5 = 6$ is
satisfied because $a_6 = 32 \leq 27 + 6$ and $a_6$ is the last element of the
array$j_6 = 6$ is satisfied because $a_6 = 32 \leq 32 + 6$ and $a_6$
is the last element of the arrayThis is not the only possible correct output for the sample input. For example,
you could instead output the array $[1, 2, 4, 5, 6, 7]$ with $K = 1$.SCORING:For 50% of all inputs, $N\le 5000$For the remaining inputs, there are no additional constraints.Problem credits: Danny Mittal

\section*{Sample Input}
\begin{verbatim}
6
2 2 4 5 6 6
\end{verbatim}

\section*{Sample Output}
\begin{verbatim}
6
1
6
17
22
27
32

The sample output is the array $a = [1, 6, 17, 22, 27, 32]$ with $K = 6$.
$j_1 = 2$ is satisfied because $a_2 = 6 \leq 1 + 6 = a_1 + K$ but
$a_3 = 17 > 1 + 6 = a_1 + K$, so $a_2$ is the largest element that is at most
$a_1$. Similarly,$j_2 = 2$ is satisfied because $a_2 = 6 \leq 6 + 6$ but
$a_3 = 17 > 6 + 6$$j_3 = 4$ is satisfied because
$a_4 = 22 \leq 17 + 6$ but $a_5 = 27 > 17 + 6$$j_4 = 5$ is satisfied
because $a_5 = 27 \leq 22 + 6$ but $a_5 = 32 > 22 + 6$$j_5 = 6$ is
satisfied because $a_6 = 32 \leq 27 + 6$ and $a_6$ is the last element of the
array$j_6 = 6$ is satisfied because $a_6 = 32 \leq 32 + 6$ and $a_6$
is the last element of the arrayThis is not the only possible correct output for the sample input. For example,
you could instead output the array $[1, 2, 4, 5, 6, 7]$ with $K = 1$.SCORING:For 50% of all inputs, $N\le 5000$For the remaining inputs, there are no additional constraints.Problem credits: Danny Mittal

This is not the only possible correct output for the sample input. For example,
you could instead output the array $[1, 2, 4, 5, 6, 7]$ with $K = 1$.SCORING:For 50% of all inputs, $N\le 5000$For the remaining inputs, there are no additional constraints.Problem credits: Danny Mittal

SCORING:For 50% of all inputs, $N\le 5000$For the remaining inputs, there are no additional constraints.Problem credits: Danny Mittal

SCORING:For 50% of all inputs, $N\le 5000$For the remaining inputs, there are no additional constraints.Problem credits: Danny Mittal
\end{verbatim}

\section*{Solution}


\section*{Problem URL}
https://usaco.org/index.php?page=viewproblem2&cpid=1187

\section*{Source}
USACO JAN22 Contest, Gold Division, Problem ID: 1187

\end{document}
