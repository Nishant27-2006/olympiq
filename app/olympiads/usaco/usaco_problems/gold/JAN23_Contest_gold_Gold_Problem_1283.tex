\documentclass[12pt]{article}
\usepackage{amsmath}
\usepackage{amssymb}
\usepackage{graphicx}
\usepackage{enumitem}
\usepackage{hyperref}
\usepackage{xcolor}

\title{View problem}
\author{USACO Contest: JAN23 Contest - Gold Division}
\date{\today}

\begin{document}

\maketitle

\section*{Problem Statement}


Farmer Nhoj dropped Bessie in the middle of nowhere! At time $t=0$, Bessie is
located at $x=0$ on an infinite number line. She frantically searches for an
exit by moving left or right by $1$ unit each second. However, there actually is
no exit and after $T$ seconds, Bessie is back at $x=0$, tired and resigned. 

Farmer Nhoj tries to track Bessie but only knows how many times Bessie crosses
$x=.5, 1.5, 2.5, \ldots, (N-1).5$, given by an array $A_0,A_1,\dots,A_{N-1}$
($1\leq N \leq 10^5$, $1 \leq A_i \leq 10^6$). Bessie never reaches $x>N$ nor
$x<0$.

In particular, Bessie's route can be represented by a string of
$T = \sum_{i=0}^{N-1} A_i$ $L$s and $R$s where the $i$th character represents
the direction Bessie moves in during the $i$th second. The number of direction
changes is defined as the number of occurrences of  $LR$s plus the number of
occurrences of $RL$s. 

Please help Farmer Nhoj count the number of routes Bessie could have taken that
are consistent with $A$ and minimize the number of direction changes. It is
guaranteed that there is at least one valid route.

INPUT FORMAT (input arrives from the terminal / stdin):
The first line contains $N$. The second line contains $A_0,A_1,\dots,A_{N-1}$.

OUTPUT FORMAT (print output to the terminal / stdout):
The number of routes Bessie could have taken, modulo $10^9+7$.

SAMPLE INPUT:
2
4 6
SAMPLE OUTPUT: 
2

Bessie must change direction at least 5 times. There are two routes 
corresponding to Bessie changing direction exactly 5 times:


RRLRLLRRLL
RRLLRRLRLL

SCORING:
Inputs 2-4: $N\le 2$ and $\max(A_i)\le 10^3$Inputs 5-7: $N\le 2$Inputs 8-11: $\max(A_i)\le 10^3$Inputs 12-21: No additional constraints.


Problem credits: Brandon Wang, Claire Zhang, and Benjamin Qi



\section*{Input Format}
OUTPUT FORMAT (print output to the terminal / stdout):The number of routes Bessie could have taken, modulo $10^9+7$.SAMPLE INPUT:2
4 6SAMPLE OUTPUT:2Bessie must change direction at least 5 times. There are two routes 
corresponding to Bessie changing direction exactly 5 times:RRLRLLRRLL
RRLLRRLRLLSCORING:Inputs 2-4: $N\le 2$ and $\max(A_i)\le 10^3$Inputs 5-7: $N\le 2$Inputs 8-11: $\max(A_i)\le 10^3$Inputs 12-21: No additional constraints.Problem credits: Brandon Wang, Claire Zhang, and Benjamin Qi

\section*{Output Format}
SAMPLE INPUT:2
4 6SAMPLE OUTPUT:2Bessie must change direction at least 5 times. There are two routes 
corresponding to Bessie changing direction exactly 5 times:RRLRLLRRLL
RRLLRRLRLLSCORING:Inputs 2-4: $N\le 2$ and $\max(A_i)\le 10^3$Inputs 5-7: $N\le 2$Inputs 8-11: $\max(A_i)\le 10^3$Inputs 12-21: No additional constraints.Problem credits: Brandon Wang, Claire Zhang, and Benjamin Qi

\section*{Sample Input}
\begin{verbatim}
2
4 6
\end{verbatim}

\section*{Sample Output}
\begin{verbatim}
2

Bessie must change direction at least 5 times. There are two routes 
corresponding to Bessie changing direction exactly 5 times:RRLRLLRRLL
RRLLRRLRLLSCORING:Inputs 2-4: $N\le 2$ and $\max(A_i)\le 10^3$Inputs 5-7: $N\le 2$Inputs 8-11: $\max(A_i)\le 10^3$Inputs 12-21: No additional constraints.Problem credits: Brandon Wang, Claire Zhang, and Benjamin Qi

RRLRLLRRLL
RRLLRRLRLLSCORING:Inputs 2-4: $N\le 2$ and $\max(A_i)\le 10^3$Inputs 5-7: $N\le 2$Inputs 8-11: $\max(A_i)\le 10^3$Inputs 12-21: No additional constraints.Problem credits: Brandon Wang, Claire Zhang, and Benjamin Qi

RRLRLLRRLL
RRLLRRLRLL

SCORING:Inputs 2-4: $N\le 2$ and $\max(A_i)\le 10^3$Inputs 5-7: $N\le 2$Inputs 8-11: $\max(A_i)\le 10^3$Inputs 12-21: No additional constraints.Problem credits: Brandon Wang, Claire Zhang, and Benjamin Qi
\end{verbatim}

\section*{Solution}


\section*{Problem URL}
https://usaco.org/index.php?page=viewproblem2&cpid=1283

\section*{Source}
USACO JAN23 Contest, Gold Division, Problem ID: 1283

\end{document}
