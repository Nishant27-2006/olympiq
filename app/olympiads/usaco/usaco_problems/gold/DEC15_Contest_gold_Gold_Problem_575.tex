\documentclass[12pt]{article}
\usepackage{amsmath}
\usepackage{amssymb}
\usepackage{graphicx}
\usepackage{enumitem}
\usepackage{hyperref}
\usepackage{xcolor}

\title{View problem}
\author{USACO Contest: DEC15 Contest - Gold Division}
\date{\today}

\begin{document}

\maketitle

\section*{Problem Statement}

After eating too much fruit in Farmer John's kitchen, Bessie the cow is getting some very
strange dreams!  In her most recent dream, she is trapped in a maze in the shape
of an $N \times M$ grid of tiles ($1 \le N, M \le 1,000$). She starts on the
top-left tile and wants to get to the bottom-right tile. When she is standing on
a tile, she can potentially move to the adjacent tiles in any of the four
cardinal directions.

But wait! Each tile has a color, and each color has a different property!
Bessie's head hurts just thinking about it:


If a tile is red, then it is impassable.

If a tile is pink, then it can be walked on normally.

If a tile is orange, then it can be walked on normally, but will make
Bessie smell like oranges.

If a tile is blue, then it contains piranhas that will only let Bessie
pass if she smells like oranges.

If a tile is purple, then Bessie will slide to the next tile in that
direction (unless she is unable to cross it). If this tile is also a purple
tile, then Bessie will continue to slide until she lands on a non-purple tile or
hits an impassable tile. Sliding through a tile counts as a move. Purple
tiles will also remove Bessie's smell.

(If you're confused about purple tiles, the example will illustrate their use.)

Please help Bessie get from the top-left to the bottom-right in as few moves as
possible.

INPUT FORMAT (file dream.in):

The first line has two integers $N$ and $M$, representing the number of rows and
columns of the maze.

The next $N$ lines have $M$ integers each, representing the maze:


The integer '0' is a red tile

The integer '1' is a pink tile

The integer '2' is an orange tile

The integer '3' is a blue tile

The integer '4' is a purple tile

The top-left and bottom-right integers will always be '1'.


OUTPUT FORMAT (file dream.out):

A single integer, representing the minimum number of moves Bessie must use to
cross the maze, or -1 if it is impossible to do so.


SAMPLE INPUT:
4 4
1 0 2 1
1 1 4 1
1 0 4 0
1 3 1 1

SAMPLE OUTPUT: 
10

In this example, Bessie walks one square down and two squares to the right  (and
then slides one more square to the right). She walks one square up, one square 
left, and one square down (sliding two more squares down) and finishes by
walking one more square right. This is a total of 10 moves (DRRRULDDDR).

Problem credits: Nathan Pinsker, inspired by the game "Undertale".



\section*{Input Format}
The next $N$ lines have $M$ integers each, representing the maze:The integer '0' is a red tileThe integer '1' is a pink tileThe integer '2' is an orange tileThe integer '3' is a blue tileThe integer '4' is a purple tileThe top-left and bottom-right integers will always be '1'.

The integer '0' is a red tileThe integer '1' is a pink tileThe integer '2' is an orange tileThe integer '3' is a blue tileThe integer '4' is a purple tileThe top-left and bottom-right integers will always be '1'.

The top-left and bottom-right integers will always be '1'.

OUTPUT FORMAT (file dream.out):A single integer, representing the minimum number of moves Bessie must use to
cross the maze, or -1 if it is impossible to do so.SAMPLE INPUT:4 4
1 0 2 1
1 1 4 1
1 0 4 0
1 3 1 1SAMPLE OUTPUT:10In this example, Bessie walks one square down and two squares to the right  (and
then slides one more square to the right). She walks one square up, one square 
left, and one square down (sliding two more squares down) and finishes by
walking one more square right. This is a total of 10 moves (DRRRULDDDR).Problem credits: Nathan Pinsker, inspired by the game "Undertale".

\section*{Output Format}
SAMPLE INPUT:4 4
1 0 2 1
1 1 4 1
1 0 4 0
1 3 1 1SAMPLE OUTPUT:10In this example, Bessie walks one square down and two squares to the right  (and
then slides one more square to the right). She walks one square up, one square 
left, and one square down (sliding two more squares down) and finishes by
walking one more square right. This is a total of 10 moves (DRRRULDDDR).Problem credits: Nathan Pinsker, inspired by the game "Undertale".

\section*{Sample Input}
\begin{verbatim}
4 4
1 0 2 1
1 1 4 1
1 0 4 0
1 3 1 1
\end{verbatim}

\section*{Sample Output}
\begin{verbatim}
10

In this example, Bessie walks one square down and two squares to the right  (and
then slides one more square to the right). She walks one square up, one square 
left, and one square down (sliding two more squares down) and finishes by
walking one more square right. This is a total of 10 moves (DRRRULDDDR).Problem credits: Nathan Pinsker, inspired by the game "Undertale".

Problem credits: Nathan Pinsker, inspired by the game "Undertale".
\end{verbatim}

\section*{Solution}


\section*{Problem URL}
https://usaco.org/index.php?page=viewproblem2&cpid=575

\section*{Source}
USACO DEC15 Contest, Gold Division, Problem ID: 575

\end{document}
