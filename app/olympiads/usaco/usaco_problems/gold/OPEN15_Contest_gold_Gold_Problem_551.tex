\documentclass[12pt]{article}
\usepackage{amsmath}
\usepackage{amssymb}
\usepackage{graphicx}
\usepackage{enumitem}
\usepackage{hyperref}
\usepackage{xcolor}

\title{View problem}
\author{USACO Contest: OPEN15 Contest - Gold Division}
\date{\today}

\begin{document}

\maketitle

\section*{Problem Statement}

For Bessie the cow’s birthday, Farmer John has given her free reign over one
of his best fields to eat grass.  
The field is covered in $N$ patches of grass ($1 \le N \le 1000$), conveniently 
numbered $1\ldots N$, that each have a distinct quality value.  If Bessie eats
grass of quality $Q$, she gains $Q$ units of energy.  Each patch is connected to
up to 10 neighboring patches via bi-directional paths, and it takes Bessie $E$
units of energy to move between adjacent patches ($1 \le E \le 1,000,000$). 
Bessie can choose to start grazing in any patch she wishes, and she wants to
stop grazing once she has accumulated a maximum amount of energy.  
Unfortunately, Bessie is a picky bovine, and once she eats grass of a certain 
quality, she’ll never eat grass at or below that quality level again!  She is
still happy to walk through patches without eating their grass; in fact, she 
might find it beneficial to walk through a patch of high-quality grass without
eating it, only to return later for a tasty snack.
Please help determine the maximum amount of energy Bessie can accumulate.
INPUT FORMAT (file buffet.in):
The first line of input contains $N$ and $E$.  Each of the remaining $N$ lines
describe a patch of grass.  They contain two integers $Q$ and $D$ giving the
quality of the patch (in the range $1\ldots 1,000,000$) and its number of 
neighbors.  The remaining $D$ numbers on the line specify the neighbors.

OUTPUT FORMAT (file buffet.out):
Please output the maximum amount of energy Bessie can accumulate.

SAMPLE INPUT:5 2
4 1 2
1 3 1 3 4
6 2 2 5
5 2 2 5
2 2 3 4
SAMPLE OUTPUT: 7

Bessie starts at patch 4 gaining 5 units of energy from the grass there.  She
then takes the path to patch 5 losing 2 units of energy during her travel.
She refuses to eat the lower quality grass at patch 5 and travels to patch 3
again losing 2 units of energy.  Finally she eats the grass at patch 3 gaining 6 units of energy
for a total of 7 energy.
Note tha the sample case above is different from test case 1 when you submit.
[Problem credits: Austin Anderson and Brian Dean, 2015]



\section*{Input Format}
OUTPUT FORMAT (file buffet.out):Please output the maximum amount of energy Bessie can accumulate.SAMPLE INPUT:5 2
4 1 2
1 3 1 3 4
6 2 2 5
5 2 2 5
2 2 3 4SAMPLE OUTPUT:7Bessie starts at patch 4 gaining 5 units of energy from the grass there.  She
then takes the path to patch 5 losing 2 units of energy during her travel.
She refuses to eat the lower quality grass at patch 5 and travels to patch 3
again losing 2 units of energy.  Finally she eats the grass at patch 3 gaining 6 units of energy
for a total of 7 energy.Note tha the sample case above is different from test case 1 when you submit.[Problem credits: Austin Anderson and Brian Dean, 2015]

\section*{Output Format}
SAMPLE INPUT:5 2
4 1 2
1 3 1 3 4
6 2 2 5
5 2 2 5
2 2 3 4SAMPLE OUTPUT:7Bessie starts at patch 4 gaining 5 units of energy from the grass there.  She
then takes the path to patch 5 losing 2 units of energy during her travel.
She refuses to eat the lower quality grass at patch 5 and travels to patch 3
again losing 2 units of energy.  Finally she eats the grass at patch 3 gaining 6 units of energy
for a total of 7 energy.Note tha the sample case above is different from test case 1 when you submit.[Problem credits: Austin Anderson and Brian Dean, 2015]

\section*{Sample Input}
\begin{verbatim}
5 2
4 1 2
1 3 1 3 4
6 2 2 5
5 2 2 5
2 2 3 4
\end{verbatim}

\section*{Sample Output}
\begin{verbatim}
7

Bessie starts at patch 4 gaining 5 units of energy from the grass there.  She
then takes the path to patch 5 losing 2 units of energy during her travel.
She refuses to eat the lower quality grass at patch 5 and travels to patch 3
again losing 2 units of energy.  Finally she eats the grass at patch 3 gaining 6 units of energy
for a total of 7 energy.Note tha the sample case above is different from test case 1 when you submit.[Problem credits: Austin Anderson and Brian Dean, 2015]

Note tha the sample case above is different from test case 1 when you submit.[Problem credits: Austin Anderson and Brian Dean, 2015]

[Problem credits: Austin Anderson and Brian Dean, 2015]
\end{verbatim}

\section*{Solution}


\section*{Problem URL}
https://usaco.org/index.php?page=viewproblem2&cpid=551

\section*{Source}
USACO OPEN15 Contest, Gold Division, Problem ID: 551

\end{document}
