\documentclass[12pt]{article}
\usepackage{amsmath}
\usepackage{amssymb}
\usepackage{graphicx}
\usepackage{enumitem}
\usepackage{hyperref}
\usepackage{xcolor}

\title{View problem}
\author{USACO Contest: OPEN16 Contest - Gold Division}
\date{\today}

\begin{document}

\maketitle

\section*{Problem Statement}

Farmer John's $N$ cows ($3 \leq N \leq 50,000$) are all located at distinct
positions in his two-dimensional field.  FJ wants to enclose all of the cows
with a rectangular fence whose sides are parallel to the x and y axes, and he
wants this fence to be as small as possible so that it contains every cow (cows
on the boundary are allowed).  

FJ is unfortunately on a tight budget due to low milk production last quarter.
He would therefore like to enclose a smaller area to reduce maintenance costs,
and the only way he can see to do this is by building two enclosures instead of
one.  Please help him compute how much less area he needs to enclose, in total,
by using two enclosures instead of one.  Like the original enclosure, the two
enclosures must collectively contain all the cows (with cows on boundaries
allowed), and they must have sides parallel to the x and y axes.  The two
enclosures are not allowed to overlap -- not even on their boundaries. Note that
enclosures of zero area are legal, for example if an enclosure has zero width
and/or zero height.

INPUT FORMAT (file split.in):
The first line of input contains $N$.  The next $N$ lines each contain two
integers specifying the location of a cow.  Cow locations are positive integers
in the range $1 \ldots 1,000,000,000$.

OUTPUT FORMAT (file split.out):
Write a single integer specifying amount of total area FJ can save by using two 
enclosures instead of one.

SAMPLE INPUT:
6
4 2
8 10
1 1
9 12
14 7
2 3
SAMPLE OUTPUT: 
107


Problem credits: Brian Dean



\section*{Input Format}
OUTPUT FORMAT (file split.out):Write a single integer specifying amount of total area FJ can save by using two 
enclosures instead of one.SAMPLE INPUT:6
4 2
8 10
1 1
9 12
14 7
2 3SAMPLE OUTPUT:107Problem credits: Brian Dean

\section*{Output Format}
SAMPLE INPUT:6
4 2
8 10
1 1
9 12
14 7
2 3SAMPLE OUTPUT:107Problem credits: Brian Dean

\section*{Sample Input}
\begin{verbatim}
6
4 2
8 10
1 1
9 12
14 7
2 3
\end{verbatim}

\section*{Sample Output}
\begin{verbatim}
107

Problem credits: Brian Dean

Problem credits: Brian Dean
\end{verbatim}

\section*{Solution}


\section*{Problem URL}
https://usaco.org/index.php?page=viewproblem2&cpid=645

\section*{Source}
USACO OPEN16 Contest, Gold Division, Problem ID: 645

\end{document}
