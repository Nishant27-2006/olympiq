\documentclass[12pt]{article}
\usepackage{amsmath}
\usepackage{amssymb}
\usepackage{graphicx}
\usepackage{enumitem}
\usepackage{hyperref}
\usepackage{xcolor}

\title{View problem}
\author{USACO Contest: JAN25 Contest - Gold Division}
\date{\today}

\begin{document}

\maketitle

\section*{Problem Statement}


Farmer John's farm is full of lush vegetation and every cow wants a photo of its
natural beauty. Unfortunately, Bessie still has places to be, but she doesn't
want to disrupt any  photo ops. 

Bessie is currently standing at $(X,0)$ on the XY-plane and she wants to get to
$(0,Y)$ ($1\le X,Y\le 10^6$). Unfortunately, $N$ ($1 \leq N \leq 3 \cdot 10^5$)
other cows have decided to pose on the $X$ axis. More specifically, cow $i$ will
be positioned at $(x_i,0)$ with a photographer at $(0,y_i)$ where
$(1 \leq x_i,y_i \leq 10^6)$ ready to take their picture. They will begin posing
moments before time $s_i$ ($1 \leq s_i < T$) and they will keep posing for a
very long time (they have to get their picture just right). Here,
$1\le T\le N+1$.

Bessie knows the schedule for every cow's photo op, and she will take the
shortest Euclidean distance to get to her destination, without crossing the line
of sight from any photographer to their respective cow (her path will consist of
one or more line segments).

If Bessie leaves at time $t$, she will avoid the line of sights for all 
photographer/cow pairs that started posing at time $s_i \le t$, and let  the
distance to her final destination be $d_t$. Determine the values of 
$\lfloor d_t\rfloor$ for each integer $t$ from $0$ to $T-1$ inclusive.

INPUT FORMAT (input arrives from the terminal / stdin):
The first line of input contains $N$ and $T$, representing the number of cows
posing on the $x$-axis and the timeframe that Bessie could leave at.

The second line of input contains $X$ and $Y$, representing Bessie's starting $X$
coordinate and her target $Y$ coordinate respectively.

The next $N$ lines contain $s_i$ $x_i$ and $y_i$. It is guaranteed that all
$x_i$ are distinct from each other and $X$, and all $y_i$ are distinct from each
other and $Y$. All $s_i$ will be given in increasing order, where
$s_i \leq s_{i+1}$.

OUTPUT FORMAT (print output to the terminal / stdout):
Print $T$ lines, with the $t$th (0-indexed) line containing
$\lfloor d_t\rfloor$.

SAMPLE INPUT:
4 5
6 7
1 7 5
2 4 4
3 1 6
4 2 9
SAMPLE OUTPUT: 
9
9
9
10
12

SAMPLE INPUT:
2 3
10 7
1 2 10
1 9 1
SAMPLE OUTPUT: 
12
16
16

For $t=0$ the answer is $\lfloor \sqrt{149} \rfloor=12$.

For $t=1$ the answer is $\lfloor 14+\sqrt 5\rfloor=16$.

SAMPLE INPUT:
5 6
8 9
1 3 5
1 4 1
3 10 7
4 9 2
5 6 6
SAMPLE OUTPUT: 
12
12
12
12
14
14

For $t=5$ the answer is $\lfloor 1+\sqrt{9^2+7^2}+2\rfloor=14$. Path:
$(8,0)\to (9,0)\to (0,7)\to (0,9)$

SCORING:
Inputs 4-6: $N\le 100$Inputs 7-9: $N\le 3000$Inputs 10-12: $T\le 10$Inputs 13-18: No additional constraints


Problem credits: Suhas Nagar



\section*{Input Format}
The second line of input contains $X$ and $Y$, representing Bessie's starting $X$
coordinate and her target $Y$ coordinate respectively.The next $N$ lines contain $s_i$ $x_i$ and $y_i$. It is guaranteed that all
$x_i$ are distinct from each other and $X$, and all $y_i$ are distinct from each
other and $Y$. All $s_i$ will be given in increasing order, where
$s_i \leq s_{i+1}$.

The next $N$ lines contain $s_i$ $x_i$ and $y_i$. It is guaranteed that all
$x_i$ are distinct from each other and $X$, and all $y_i$ are distinct from each
other and $Y$. All $s_i$ will be given in increasing order, where
$s_i \leq s_{i+1}$.

OUTPUT FORMAT (print output to the terminal / stdout):Print $T$ lines, with the $t$th (0-indexed) line containing
$\lfloor d_t\rfloor$.SAMPLE INPUT:4 5
6 7
1 7 5
2 4 4
3 1 6
4 2 9SAMPLE OUTPUT:9
9
9
10
12SAMPLE INPUT:2 3
10 7
1 2 10
1 9 1SAMPLE OUTPUT:12
16
16For $t=0$ the answer is $\lfloor \sqrt{149} \rfloor=12$.For $t=1$ the answer is $\lfloor 14+\sqrt 5\rfloor=16$.SAMPLE INPUT:5 6
8 9
1 3 5
1 4 1
3 10 7
4 9 2
5 6 6SAMPLE OUTPUT:12
12
12
12
14
14For $t=5$ the answer is $\lfloor 1+\sqrt{9^2+7^2}+2\rfloor=14$. Path:
$(8,0)\to (9,0)\to (0,7)\to (0,9)$SCORING:Inputs 4-6: $N\le 100$Inputs 7-9: $N\le 3000$Inputs 10-12: $T\le 10$Inputs 13-18: No additional constraintsProblem credits: Suhas Nagar

\section*{Output Format}
SAMPLE INPUT:4 5
6 7
1 7 5
2 4 4
3 1 6
4 2 9SAMPLE OUTPUT:9
9
9
10
12SAMPLE INPUT:2 3
10 7
1 2 10
1 9 1SAMPLE OUTPUT:12
16
16For $t=0$ the answer is $\lfloor \sqrt{149} \rfloor=12$.For $t=1$ the answer is $\lfloor 14+\sqrt 5\rfloor=16$.SAMPLE INPUT:5 6
8 9
1 3 5
1 4 1
3 10 7
4 9 2
5 6 6SAMPLE OUTPUT:12
12
12
12
14
14For $t=5$ the answer is $\lfloor 1+\sqrt{9^2+7^2}+2\rfloor=14$. Path:
$(8,0)\to (9,0)\to (0,7)\to (0,9)$SCORING:Inputs 4-6: $N\le 100$Inputs 7-9: $N\le 3000$Inputs 10-12: $T\le 10$Inputs 13-18: No additional constraintsProblem credits: Suhas Nagar

\section*{Sample Input}
\begin{verbatim}
4 5
6 7
1 7 5
2 4 4
3 1 6
4 2 9

2 3
10 7
1 2 10
1 9 1

5 6
8 9
1 3 5
1 4 1
3 10 7
4 9 2
5 6 6
\end{verbatim}

\section*{Sample Output}
\begin{verbatim}
9
9
9
10
12

SAMPLE INPUT:2 3
10 7
1 2 10
1 9 1SAMPLE OUTPUT:12
16
16For $t=0$ the answer is $\lfloor \sqrt{149} \rfloor=12$.For $t=1$ the answer is $\lfloor 14+\sqrt 5\rfloor=16$.SAMPLE INPUT:5 6
8 9
1 3 5
1 4 1
3 10 7
4 9 2
5 6 6SAMPLE OUTPUT:12
12
12
12
14
14For $t=5$ the answer is $\lfloor 1+\sqrt{9^2+7^2}+2\rfloor=14$. Path:
$(8,0)\to (9,0)\to (0,7)\to (0,9)$SCORING:Inputs 4-6: $N\le 100$Inputs 7-9: $N\le 3000$Inputs 10-12: $T\le 10$Inputs 13-18: No additional constraintsProblem credits: Suhas Nagar

12
16
16

For $t=0$ the answer is $\lfloor \sqrt{149} \rfloor=12$.For $t=1$ the answer is $\lfloor 14+\sqrt 5\rfloor=16$.SAMPLE INPUT:5 6
8 9
1 3 5
1 4 1
3 10 7
4 9 2
5 6 6SAMPLE OUTPUT:12
12
12
12
14
14For $t=5$ the answer is $\lfloor 1+\sqrt{9^2+7^2}+2\rfloor=14$. Path:
$(8,0)\to (9,0)\to (0,7)\to (0,9)$SCORING:Inputs 4-6: $N\le 100$Inputs 7-9: $N\le 3000$Inputs 10-12: $T\le 10$Inputs 13-18: No additional constraintsProblem credits: Suhas Nagar

For $t=1$ the answer is $\lfloor 14+\sqrt 5\rfloor=16$.SAMPLE INPUT:5 6
8 9
1 3 5
1 4 1
3 10 7
4 9 2
5 6 6SAMPLE OUTPUT:12
12
12
12
14
14For $t=5$ the answer is $\lfloor 1+\sqrt{9^2+7^2}+2\rfloor=14$. Path:
$(8,0)\to (9,0)\to (0,7)\to (0,9)$SCORING:Inputs 4-6: $N\le 100$Inputs 7-9: $N\le 3000$Inputs 10-12: $T\le 10$Inputs 13-18: No additional constraintsProblem credits: Suhas Nagar

SAMPLE INPUT:5 6
8 9
1 3 5
1 4 1
3 10 7
4 9 2
5 6 6SAMPLE OUTPUT:12
12
12
12
14
14For $t=5$ the answer is $\lfloor 1+\sqrt{9^2+7^2}+2\rfloor=14$. Path:
$(8,0)\to (9,0)\to (0,7)\to (0,9)$SCORING:Inputs 4-6: $N\le 100$Inputs 7-9: $N\le 3000$Inputs 10-12: $T\le 10$Inputs 13-18: No additional constraintsProblem credits: Suhas Nagar

12
12
12
12
14
14

For $t=5$ the answer is $\lfloor 1+\sqrt{9^2+7^2}+2\rfloor=14$. Path:
$(8,0)\to (9,0)\to (0,7)\to (0,9)$SCORING:Inputs 4-6: $N\le 100$Inputs 7-9: $N\le 3000$Inputs 10-12: $T\le 10$Inputs 13-18: No additional constraintsProblem credits: Suhas Nagar

SCORING:Inputs 4-6: $N\le 100$Inputs 7-9: $N\le 3000$Inputs 10-12: $T\le 10$Inputs 13-18: No additional constraintsProblem credits: Suhas Nagar
\end{verbatim}

\section*{Solution}


\section*{Problem URL}
https://usaco.org/index.php?page=viewproblem2&cpid=1475

\section*{Source}
USACO JAN25 Contest, Gold Division, Problem ID: 1475

\end{document}
