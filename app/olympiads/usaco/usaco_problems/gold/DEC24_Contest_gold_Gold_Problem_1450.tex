\documentclass[12pt]{article}
\usepackage{amsmath}
\usepackage{amssymb}
\usepackage{graphicx}
\usepackage{enumitem}
\usepackage{hyperref}
\usepackage{xcolor}

\title{View problem}
\author{USACO Contest: DEC24 Contest - Gold Division}
\date{\today}

\begin{document}

\maketitle

\section*{Problem Statement}


It's the year $3000$, and Bessie became the first cow in space! During her
journey between the stars, she found a number line with $N$
($2 \leq N \leq 5 \cdot 10^5$) points, numbered from $1$ to $N$. All points are
initially colored white. She can perform the following operation any number of
times.

Choose a position $i$ within the number line and a positive integer $x$.
Then, color all the points in the interval $[i, i + x - 1]$ red and all points
in $[i + x, i + 2x - 1]$ blue. All chosen intervals must be disjoint
(i.e. no points in $[i, i + 2x - 1]$ can be already colored red or blue). The
entire interval must also fall within the number line (i.e.
$1 \leq i \leq i + 2x - 1 \leq N$).
Farmer John gives Bessie a string $s$ of length $N$ consisting of characters
$R$, $B$, and $X$. The string represents Farmer John's color preferences for
each point: $s_i=R$ means the $i$'th point must be colored red, $s_i = B$ means
the $i$'th point must be colored blue, and $s_i = X$ means there is no
constraint on the color for the $i$'th point.

Help Bessie count the number of distinct ways for the number line to be colored
while satisfying Farmer John's preferences. Two colorings are different if there
is at least one corresponding point with a different color. Because the answer
may be large, output it modulo $10^9+7$. 

INPUT FORMAT (input arrives from the terminal / stdin):
The first line contains an integer $N$.

The following line contains string $s$.

OUTPUT FORMAT (print output to the terminal / stdout):
Output the number of distinct ways for the number line to be colored while
satisfying Farmer John's preferences modulo
$10^9+7$.

SAMPLE INPUT:
6
RXXXXB
SAMPLE OUTPUT: 
5

Bessie can choose $i=1,x=1$ (i.e. color point $1$ red and point $2$ blue) and
$i=3,x=2$ (i.e. color points $3,4$ red and points $5,6$ blue) to produce the
coloring $RBRRBB$. 

The other colorings are $RRBBRB$, $RBWWRB$, $RRRBBB$, and
$RBRBRB$.

SAMPLE INPUT:
6
XXRBXX
SAMPLE OUTPUT: 
6

The six colorings are $WWRBWW$, $WWRBRB$, $WRRBBW$, $RBRBWW$, $RBRBRB$, and $RRRBBB$.
SAMPLE INPUT:
12
XBXXXXRXRBXX
SAMPLE OUTPUT: 
18

SCORING:
Input 4: $N\le 500$Inputs 5-6: $N\le 10^4$Inputs 7-13: All but at most $100$ characters in $s$ are
$X$.Inputs 14-23: No additional constraints


Problem credits: Chongtian Ma, Alex Liang



\section*{Input Format}
The following line contains string $s$.

OUTPUT FORMAT (print output to the terminal / stdout):Output the number of distinct ways for the number line to be colored while
satisfying Farmer John's preferences modulo
$10^9+7$.SAMPLE INPUT:6
RXXXXBSAMPLE OUTPUT:5Bessie can choose $i=1,x=1$ (i.e. color point $1$ red and point $2$ blue) and
$i=3,x=2$ (i.e. color points $3,4$ red and points $5,6$ blue) to produce the
coloring $RBRRBB$.The other colorings are $RRBBRB$, $RBWWRB$, $RRRBBB$, and
$RBRBRB$.SAMPLE INPUT:6
XXRBXXSAMPLE OUTPUT:6The six colorings are $WWRBWW$, $WWRBRB$, $WRRBBW$, $RBRBWW$, $RBRBRB$, and $RRRBBB$.SAMPLE INPUT:12
XBXXXXRXRBXXSAMPLE OUTPUT:18SCORING:Input 4: $N\le 500$Inputs 5-6: $N\le 10^4$Inputs 7-13: All but at most $100$ characters in $s$ are
$X$.Inputs 14-23: No additional constraintsProblem credits: Chongtian Ma, Alex Liang

\section*{Output Format}
SAMPLE INPUT:6
RXXXXBSAMPLE OUTPUT:5Bessie can choose $i=1,x=1$ (i.e. color point $1$ red and point $2$ blue) and
$i=3,x=2$ (i.e. color points $3,4$ red and points $5,6$ blue) to produce the
coloring $RBRRBB$.The other colorings are $RRBBRB$, $RBWWRB$, $RRRBBB$, and
$RBRBRB$.SAMPLE INPUT:6
XXRBXXSAMPLE OUTPUT:6The six colorings are $WWRBWW$, $WWRBRB$, $WRRBBW$, $RBRBWW$, $RBRBRB$, and $RRRBBB$.SAMPLE INPUT:12
XBXXXXRXRBXXSAMPLE OUTPUT:18SCORING:Input 4: $N\le 500$Inputs 5-6: $N\le 10^4$Inputs 7-13: All but at most $100$ characters in $s$ are
$X$.Inputs 14-23: No additional constraintsProblem credits: Chongtian Ma, Alex Liang

\section*{Sample Input}
\begin{verbatim}
6
RXXXXB

6
XXRBXX

12
XBXXXXRXRBXX
\end{verbatim}

\section*{Sample Output}
\begin{verbatim}
5

The other colorings are $RRBBRB$, $RBWWRB$, $RRRBBB$, and
$RBRBRB$.SAMPLE INPUT:6
XXRBXXSAMPLE OUTPUT:6The six colorings are $WWRBWW$, $WWRBRB$, $WRRBBW$, $RBRBWW$, $RBRBRB$, and $RRRBBB$.SAMPLE INPUT:12
XBXXXXRXRBXXSAMPLE OUTPUT:18SCORING:Input 4: $N\le 500$Inputs 5-6: $N\le 10^4$Inputs 7-13: All but at most $100$ characters in $s$ are
$X$.Inputs 14-23: No additional constraintsProblem credits: Chongtian Ma, Alex Liang

SAMPLE INPUT:6
XXRBXXSAMPLE OUTPUT:6The six colorings are $WWRBWW$, $WWRBRB$, $WRRBBW$, $RBRBWW$, $RBRBRB$, and $RRRBBB$.SAMPLE INPUT:12
XBXXXXRXRBXXSAMPLE OUTPUT:18SCORING:Input 4: $N\le 500$Inputs 5-6: $N\le 10^4$Inputs 7-13: All but at most $100$ characters in $s$ are
$X$.Inputs 14-23: No additional constraintsProblem credits: Chongtian Ma, Alex Liang

6

18
\end{verbatim}

\section*{Solution}


\section*{Problem URL}
https://usaco.org/index.php?page=viewproblem2&cpid=1450

\section*{Source}
USACO DEC24 Contest, Gold Division, Problem ID: 1450

\end{document}
