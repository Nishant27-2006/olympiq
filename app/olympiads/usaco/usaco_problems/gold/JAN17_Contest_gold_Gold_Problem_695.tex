\documentclass[12pt]{article}
\usepackage{amsmath}
\usepackage{amssymb}
\usepackage{graphicx}
\usepackage{enumitem}
\usepackage{hyperref}
\usepackage{xcolor}

\title{View problem}
\author{USACO Contest: JAN17 Contest - Gold Division}
\date{\today}

\begin{document}

\maketitle

\section*{Problem Statement}

Bessie has gotten herself stuck on the wrong side of Farmer John's barn again,
and since her vision is so poor, she needs your help navigating across the barn.

The barn is described by an $N \times N$ grid of square cells
($2 \leq N \leq 20$), some being empty and some containing impassable haybales. 
Bessie starts in the  lower-left corner (cell 1,1) and wants to move to the
upper-right corner (cell $N,N$).   You can guide her by telling her a sequence
of instructions, each of which is either "forward", "turn left 90 degrees", or
"turn right 90 degrees".  You want to issue the shortest sequence of
instructions that will guide her to her destination.  If you instruct Bessie to
move off the grid (i.e., into the barn wall) or into a haybale, she will not
move and will skip to the next command in your sequence.  

Unfortunately, Bessie doesn't know if she starts out facing up (towards cell
1,2) or right (towards cell 2,1).  You need to give the  shortest sequence of
directions that will guide her to the goal regardless of which case is true. 
Once she reaches the goal she will ignore further commands.

INPUT FORMAT (file cownav.in):
The first line of input contains $N$.

Each of the $N$ following lines contains a string of exactly $N$ characters,
representing the barn. The first character of the last line is cell 1,1. The
last character of the first line is cell N, N. 

Each character will either be an H to represent a haybale or an E to represent
an empty square.

It is guaranteed that cells 1,1 and $N,N$ will be empty, and furthermore it is
guaranteed that there is a path of empty squares from cell 1,1 to cell $N, N$.

OUTPUT FORMAT (file cownav.out):
On a single line of output, output the length of the shortest sequence of
directions that will guide Bessie to the goal, irrespective whether she starts
facing up or right.

SAMPLE INPUT:
3
EHE
EEE
EEE
SAMPLE OUTPUT: 
9

In this example, the instructions "Forward, Right, Forward, Forward, Left,
Forward, Left, Forward, Forward" will guide Bessie to the destination
irrespective of her starting orientation.


Problem credits: Brian Dean



\section*{Input Format}
Each of the $N$ following lines contains a string of exactly $N$ characters,
representing the barn. The first character of the last line is cell 1,1. The
last character of the first line is cell N, N.Each character will either be an H to represent a haybale or an E to represent
an empty square.It is guaranteed that cells 1,1 and $N,N$ will be empty, and furthermore it is
guaranteed that there is a path of empty squares from cell 1,1 to cell $N, N$.

Each character will either be an H to represent a haybale or an E to represent
an empty square.It is guaranteed that cells 1,1 and $N,N$ will be empty, and furthermore it is
guaranteed that there is a path of empty squares from cell 1,1 to cell $N, N$.

It is guaranteed that cells 1,1 and $N,N$ will be empty, and furthermore it is
guaranteed that there is a path of empty squares from cell 1,1 to cell $N, N$.

OUTPUT FORMAT (file cownav.out):On a single line of output, output the length of the shortest sequence of
directions that will guide Bessie to the goal, irrespective whether she starts
facing up or right.SAMPLE INPUT:3
EHE
EEE
EEESAMPLE OUTPUT:9In this example, the instructions "Forward, Right, Forward, Forward, Left,
Forward, Left, Forward, Forward" will guide Bessie to the destination
irrespective of her starting orientation.Problem credits: Brian Dean

\section*{Output Format}
SAMPLE INPUT:3
EHE
EEE
EEESAMPLE OUTPUT:9In this example, the instructions "Forward, Right, Forward, Forward, Left,
Forward, Left, Forward, Forward" will guide Bessie to the destination
irrespective of her starting orientation.Problem credits: Brian Dean

\section*{Sample Input}
\begin{verbatim}
3
EHE
EEE
EEE
\end{verbatim}

\section*{Sample Output}
\begin{verbatim}
9

In this example, the instructions "Forward, Right, Forward, Forward, Left,
Forward, Left, Forward, Forward" will guide Bessie to the destination
irrespective of her starting orientation.Problem credits: Brian Dean

Problem credits: Brian Dean

Problem credits: Brian Dean
\end{verbatim}

\section*{Solution}


\section*{Problem URL}
https://usaco.org/index.php?page=viewproblem2&cpid=695

\section*{Source}
USACO JAN17 Contest, Gold Division, Problem ID: 695

\end{document}
