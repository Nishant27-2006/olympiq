\documentclass[12pt]{article}
\usepackage{amsmath}
\usepackage{amssymb}
\usepackage{graphicx}
\usepackage{enumitem}
\usepackage{hyperref}
\usepackage{xcolor}

\title{View problem}
\author{USACO Contest: FEB23 Contest - Gold Division}
\date{\today}

\begin{document}

\maketitle

\section*{Problem Statement}


**Note: The time limit for this problem is 3s, 1.5x the default.**
FJ gave Bessie an array $a$ of length $N$
($2\le N\le 500, -10^{15}\le a_i\le 10^{15}$) with all $\frac{N(N+1)}{2}$
contiguous subarray sums distinct. For each index $i\in [1,N]$, help Bessie
compute the minimum amount it suffices to change $a_i$ by so that there are two
different contiguous subarrays of $a$ with equal sum.

INPUT FORMAT (input arrives from the terminal / stdin):
The first line contains $N$.

The next line contains $a_1,\dots, a_N$ (the elements of $a$, in order).

OUTPUT FORMAT (print output to the terminal / stdout):
One line for each index $i\in [1,N]$.

SAMPLE INPUT:
2
2 -3
SAMPLE OUTPUT: 
2
3

Decreasing $a_1$ by $2$ would result in $a_1+a_2=a_2$. Similarly, increasing
$a_2$ by $3$ would result in $a_1+a_2=a_1$.

SAMPLE INPUT:
3
3 -10 4
SAMPLE OUTPUT: 
1
6
1

Increasing $a_1$ or decreasing $a_3$ by $1$ would result in $a_1=a_3$.
Increasing $a_2$ by $6$ would result in $a_1=a_1+a_2+a_3$.

SCORING:
Input 3: $N\le 40$Input 4: $N \le 80$Inputs 5-7: $N \le 200$Inputs 8-16: No additional constraints.


Problem credits: Benjamin Qi



\section*{Input Format}
The next line contains $a_1,\dots, a_N$ (the elements of $a$, in order).

OUTPUT FORMAT (print output to the terminal / stdout):One line for each index $i\in [1,N]$.SAMPLE INPUT:2
2 -3SAMPLE OUTPUT:2
3Decreasing $a_1$ by $2$ would result in $a_1+a_2=a_2$. Similarly, increasing
$a_2$ by $3$ would result in $a_1+a_2=a_1$.SAMPLE INPUT:3
3 -10 4SAMPLE OUTPUT:1
6
1Increasing $a_1$ or decreasing $a_3$ by $1$ would result in $a_1=a_3$.
Increasing $a_2$ by $6$ would result in $a_1=a_1+a_2+a_3$.SCORING:Input 3: $N\le 40$Input 4: $N \le 80$Inputs 5-7: $N \le 200$Inputs 8-16: No additional constraints.Problem credits: Benjamin Qi

\section*{Output Format}
SAMPLE INPUT:2
2 -3SAMPLE OUTPUT:2
3Decreasing $a_1$ by $2$ would result in $a_1+a_2=a_2$. Similarly, increasing
$a_2$ by $3$ would result in $a_1+a_2=a_1$.SAMPLE INPUT:3
3 -10 4SAMPLE OUTPUT:1
6
1Increasing $a_1$ or decreasing $a_3$ by $1$ would result in $a_1=a_3$.
Increasing $a_2$ by $6$ would result in $a_1=a_1+a_2+a_3$.SCORING:Input 3: $N\le 40$Input 4: $N \le 80$Inputs 5-7: $N \le 200$Inputs 8-16: No additional constraints.Problem credits: Benjamin Qi

\section*{Sample Input}
\begin{verbatim}
2
2 -3

3
3 -10 4
\end{verbatim}

\section*{Sample Output}
\begin{verbatim}
2
3

Decreasing $a_1$ by $2$ would result in $a_1+a_2=a_2$. Similarly, increasing
$a_2$ by $3$ would result in $a_1+a_2=a_1$.SAMPLE INPUT:3
3 -10 4SAMPLE OUTPUT:1
6
1Increasing $a_1$ or decreasing $a_3$ by $1$ would result in $a_1=a_3$.
Increasing $a_2$ by $6$ would result in $a_1=a_1+a_2+a_3$.SCORING:Input 3: $N\le 40$Input 4: $N \le 80$Inputs 5-7: $N \le 200$Inputs 8-16: No additional constraints.Problem credits: Benjamin Qi

SAMPLE INPUT:3
3 -10 4SAMPLE OUTPUT:1
6
1Increasing $a_1$ or decreasing $a_3$ by $1$ would result in $a_1=a_3$.
Increasing $a_2$ by $6$ would result in $a_1=a_1+a_2+a_3$.SCORING:Input 3: $N\le 40$Input 4: $N \le 80$Inputs 5-7: $N \le 200$Inputs 8-16: No additional constraints.Problem credits: Benjamin Qi

1
6
1

Increasing $a_1$ or decreasing $a_3$ by $1$ would result in $a_1=a_3$.
Increasing $a_2$ by $6$ would result in $a_1=a_1+a_2+a_3$.SCORING:Input 3: $N\le 40$Input 4: $N \le 80$Inputs 5-7: $N \le 200$Inputs 8-16: No additional constraints.Problem credits: Benjamin Qi

SCORING:Input 3: $N\le 40$Input 4: $N \le 80$Inputs 5-7: $N \le 200$Inputs 8-16: No additional constraints.Problem credits: Benjamin Qi
\end{verbatim}

\section*{Solution}


\section*{Problem URL}
https://usaco.org/index.php?page=viewproblem2&cpid=1305

\section*{Source}
USACO FEB23 Contest, Gold Division, Problem ID: 1305

\end{document}
