\documentclass[12pt]{article}
\usepackage{amsmath}
\usepackage{amssymb}
\usepackage{graphicx}
\usepackage{enumitem}
\usepackage{hyperref}
\usepackage{xcolor}

\title{View problem}
\author{USACO Contest: DEC23 Contest - Gold Division}
\date{\today}

\begin{document}

\maketitle

\section*{Problem Statement}


Bessie is going on a trip in Cowland, which has $N$ ($2\le N\le 2\cdot 10^5$)
towns numbered from $1$ to $N$ and $M$ ($1\le M\le 4\cdot 10^5$) one-way roads.
The $i$th road runs from town $a_i$ to town $b_i$ and has label $l_i$
($1\le a_i,b_i\le N$, $1\le l_i\le 10^9$).  

A trip of length $k$ starting at town $x_0$ is a sequence of towns
$x_0, x_1, \ldots, x_k$, such that there is a road from town $x_i$ to town
$x_{i+1}$ for all $0\le i < k$. It is guaranteed that there are no trips of
infinite length in Cowland, and that no two roads connect the same pair of
towns.

For each town, Bessie wants to know the longest possible trip starting at it.
For some starting towns, there are multiple longest trips - out of these, she
prefers the trip with the lexicographically minimum sequence of road labels.  A
sequence is lexicographically smaller than another sequence of the same length
if, at the first position in which they differ, the first sequence has a smaller
element than the second sequence.

Output the length and sum of road labels of Bessie's preferred trip starting at
each town.

INPUT FORMAT (input arrives from the terminal / stdin):
The first line contains $N$ and $M$.

The next $M$ lines each contain three integers $a_i$, $b_i$, and $l_i$, denoting
a road from $a_i$ to $b_i$ with label $l_i$.

OUTPUT FORMAT (print output to the terminal / stdout):
Output $N$ lines. The $i$th should contain two space-separated integers, the
length and sum of road labels of Bessie's preferred trip starting at town
$i$.


SAMPLE INPUT:
4 5
4 3 10
4 2 10
3 1 10
2 1 10
4 1 10
SAMPLE OUTPUT: 
0 0
1 10
1 10
2 20

SAMPLE INPUT:
4 5
4 3 4
4 2 2
3 1 5
2 1 10
4 1 1
SAMPLE OUTPUT: 
0 0
1 10
1 5
2 12

In the following explanation, we let $a_i\overset{l_i}\to b_i$ represent the
road from $a_i$ to $b_i$ with label $l_i$.

There are several trips starting from vertex $4$, including
$4 \overset{4}\to 3\overset{5}\to 1$, $4\overset{1}\to 1$, and
$4\overset{2}\to 2\overset{10}\to 1$. Of these trips,
$4 \overset{4}\to 3\overset{5}\to 1$ and $4\overset{2}\to 2\overset{10}\to 1$
are the longest. These trips each have length 2, and their road label sequences
are $[4,5]$ and $[2,10]$, respectively. $[2,10]$ is the lexicographically
smaller sequence, and its sum is $12$.

SAMPLE INPUT:
4 5
4 3 2
4 2 2
3 1 5
2 1 10
4 1 1
SAMPLE OUTPUT: 
0 0
1 10
1 5
2 7

SAMPLE INPUT:
4 5
4 3 2
4 2 2
3 1 10
2 1 5
4 1 1
SAMPLE OUTPUT: 
0 0
1 5
1 10
2 7

SCORING:
Inputs 5-6: All labels are the same.Inputs 7-8: All labels are distinct.Inputs 9-10: $N,M\le 5000$Inputs 11-20: No additional constraints.


Problem credits: Claire Zhang and Spencer Compton



\section*{Input Format}
The next $M$ lines each contain three integers $a_i$, $b_i$, and $l_i$, denoting
a road from $a_i$ to $b_i$ with label $l_i$.

OUTPUT FORMAT (print output to the terminal / stdout):Output $N$ lines. The $i$th should contain two space-separated integers, the
length and sum of road labels of Bessie's preferred trip starting at town
$i$.SAMPLE INPUT:4 5
4 3 10
4 2 10
3 1 10
2 1 10
4 1 10SAMPLE OUTPUT:0 0
1 10
1 10
2 20SAMPLE INPUT:4 5
4 3 4
4 2 2
3 1 5
2 1 10
4 1 1SAMPLE OUTPUT:0 0
1 10
1 5
2 12In the following explanation, we let $a_i\overset{l_i}\to b_i$ represent the
road from $a_i$ to $b_i$ with label $l_i$.There are several trips starting from vertex $4$, including
$4 \overset{4}\to 3\overset{5}\to 1$, $4\overset{1}\to 1$, and
$4\overset{2}\to 2\overset{10}\to 1$. Of these trips,
$4 \overset{4}\to 3\overset{5}\to 1$ and $4\overset{2}\to 2\overset{10}\to 1$
are the longest. These trips each have length 2, and their road label sequences
are $[4,5]$ and $[2,10]$, respectively. $[2,10]$ is the lexicographically
smaller sequence, and its sum is $12$.SAMPLE INPUT:4 5
4 3 2
4 2 2
3 1 5
2 1 10
4 1 1SAMPLE OUTPUT:0 0
1 10
1 5
2 7SAMPLE INPUT:4 5
4 3 2
4 2 2
3 1 10
2 1 5
4 1 1SAMPLE OUTPUT:0 0
1 5
1 10
2 7SCORING:Inputs 5-6: All labels are the same.Inputs 7-8: All labels are distinct.Inputs 9-10: $N,M\le 5000$Inputs 11-20: No additional constraints.Problem credits: Claire Zhang and Spencer Compton

\section*{Output Format}
SAMPLE INPUT:4 5
4 3 10
4 2 10
3 1 10
2 1 10
4 1 10SAMPLE OUTPUT:0 0
1 10
1 10
2 20SAMPLE INPUT:4 5
4 3 4
4 2 2
3 1 5
2 1 10
4 1 1SAMPLE OUTPUT:0 0
1 10
1 5
2 12In the following explanation, we let $a_i\overset{l_i}\to b_i$ represent the
road from $a_i$ to $b_i$ with label $l_i$.There are several trips starting from vertex $4$, including
$4 \overset{4}\to 3\overset{5}\to 1$, $4\overset{1}\to 1$, and
$4\overset{2}\to 2\overset{10}\to 1$. Of these trips,
$4 \overset{4}\to 3\overset{5}\to 1$ and $4\overset{2}\to 2\overset{10}\to 1$
are the longest. These trips each have length 2, and their road label sequences
are $[4,5]$ and $[2,10]$, respectively. $[2,10]$ is the lexicographically
smaller sequence, and its sum is $12$.SAMPLE INPUT:4 5
4 3 2
4 2 2
3 1 5
2 1 10
4 1 1SAMPLE OUTPUT:0 0
1 10
1 5
2 7SAMPLE INPUT:4 5
4 3 2
4 2 2
3 1 10
2 1 5
4 1 1SAMPLE OUTPUT:0 0
1 5
1 10
2 7SCORING:Inputs 5-6: All labels are the same.Inputs 7-8: All labels are distinct.Inputs 9-10: $N,M\le 5000$Inputs 11-20: No additional constraints.Problem credits: Claire Zhang and Spencer Compton

\section*{Sample Input}
\begin{verbatim}
4 5
4 3 10
4 2 10
3 1 10
2 1 10
4 1 10

4 5
4 3 4
4 2 2
3 1 5
2 1 10
4 1 1

4 5
4 3 2
4 2 2
3 1 5
2 1 10
4 1 1

4 5
4 3 2
4 2 2
3 1 10
2 1 5
4 1 1
\end{verbatim}

\section*{Sample Output}
\begin{verbatim}
0 0
1 10
1 10
2 20

0 0
1 10
1 5
2 12

In the following explanation, we let $a_i\overset{l_i}\to b_i$ represent the
road from $a_i$ to $b_i$ with label $l_i$.There are several trips starting from vertex $4$, including
$4 \overset{4}\to 3\overset{5}\to 1$, $4\overset{1}\to 1$, and
$4\overset{2}\to 2\overset{10}\to 1$. Of these trips,
$4 \overset{4}\to 3\overset{5}\to 1$ and $4\overset{2}\to 2\overset{10}\to 1$
are the longest. These trips each have length 2, and their road label sequences
are $[4,5]$ and $[2,10]$, respectively. $[2,10]$ is the lexicographically
smaller sequence, and its sum is $12$.SAMPLE INPUT:4 5
4 3 2
4 2 2
3 1 5
2 1 10
4 1 1SAMPLE OUTPUT:0 0
1 10
1 5
2 7SAMPLE INPUT:4 5
4 3 2
4 2 2
3 1 10
2 1 5
4 1 1SAMPLE OUTPUT:0 0
1 5
1 10
2 7SCORING:Inputs 5-6: All labels are the same.Inputs 7-8: All labels are distinct.Inputs 9-10: $N,M\le 5000$Inputs 11-20: No additional constraints.Problem credits: Claire Zhang and Spencer Compton

There are several trips starting from vertex $4$, including
$4 \overset{4}\to 3\overset{5}\to 1$, $4\overset{1}\to 1$, and
$4\overset{2}\to 2\overset{10}\to 1$. Of these trips,
$4 \overset{4}\to 3\overset{5}\to 1$ and $4\overset{2}\to 2\overset{10}\to 1$
are the longest. These trips each have length 2, and their road label sequences
are $[4,5]$ and $[2,10]$, respectively. $[2,10]$ is the lexicographically
smaller sequence, and its sum is $12$.SAMPLE INPUT:4 5
4 3 2
4 2 2
3 1 5
2 1 10
4 1 1SAMPLE OUTPUT:0 0
1 10
1 5
2 7SAMPLE INPUT:4 5
4 3 2
4 2 2
3 1 10
2 1 5
4 1 1SAMPLE OUTPUT:0 0
1 5
1 10
2 7SCORING:Inputs 5-6: All labels are the same.Inputs 7-8: All labels are distinct.Inputs 9-10: $N,M\le 5000$Inputs 11-20: No additional constraints.Problem credits: Claire Zhang and Spencer Compton

SAMPLE INPUT:4 5
4 3 2
4 2 2
3 1 5
2 1 10
4 1 1SAMPLE OUTPUT:0 0
1 10
1 5
2 7SAMPLE INPUT:4 5
4 3 2
4 2 2
3 1 10
2 1 5
4 1 1SAMPLE OUTPUT:0 0
1 5
1 10
2 7SCORING:Inputs 5-6: All labels are the same.Inputs 7-8: All labels are distinct.Inputs 9-10: $N,M\le 5000$Inputs 11-20: No additional constraints.Problem credits: Claire Zhang and Spencer Compton

0 0
1 10
1 5
2 7

0 0
1 5
1 10
2 7

SCORING:Inputs 5-6: All labels are the same.Inputs 7-8: All labels are distinct.Inputs 9-10: $N,M\le 5000$Inputs 11-20: No additional constraints.Problem credits: Claire Zhang and Spencer Compton
\end{verbatim}

\section*{Solution}


\section*{Problem URL}
https://usaco.org/index.php?page=viewproblem2&cpid=1354

\section*{Source}
USACO DEC23 Contest, Gold Division, Problem ID: 1354

\end{document}
