\documentclass[12pt]{article}
\usepackage{amsmath}
\usepackage{amssymb}
\usepackage{graphicx}
\usepackage{enumitem}
\usepackage{hyperref}
\usepackage{xcolor}

\title{View problem}
\author{USACO Contest: JAN22 Contest - Gold Division}
\date{\today}

\begin{document}

\maketitle

\section*{Problem Statement}

The grass has dried up in Farmer John's pasture due to a drought. After hours of
despair and contemplation, FJ comes up with the brilliant idea of purchasing
corn to feed his precious cows.

FJ’s $N$ ($1 \leq N \leq 100$) cows are arranged in a line such that the $i$th
cow in line has a non-negative integer hunger level of $h_i$. As FJ’s cows are
social animals and insist on eating together, the only way FJ can decrease the
hunger levels of his cows is to select two adjacent cows $i$ and $i+1$ and feed
each of them a bag of corn, causing each of their hunger levels to decrease by
one. 

FJ wants to feed his cows until all of them have the same non-negative hunger
level. Although he doesn't know his cows' exact hunger levels, he
does know an upper bound on the hunger level of each cow; specifically, the
hunger level $h_i$ of the $i$-th cow is at most $H_i$ ($0\le H_i\le 1000$). 

Your job is to count the number of $N$-tuples of hunger levels
$[h_1,h_2,\ldots,h_N]$ that are consistent with these upper bounds such that it
is possible for FJ to achieve his goal, modulo $10^9+7$.

INPUT FORMAT (input arrives from the terminal / stdin):
The first line contains $N$.

The second line contains $H_1,H_2,\ldots,H_N$.


OUTPUT FORMAT (print output to the terminal / stdout):
The number of $N$-tuples of hunger levels modulo $10^9+7$.

SAMPLE INPUT:
3
9 11 7
SAMPLE OUTPUT: 
241

There are $(9+1)\cdot (11+1)\cdot (7+1)$ $3$-tuples $h$ that are consistent with
$H$.

One of these tuples is $h=[8,10,5]$. In this case, it is possible to make all
cows have equal hunger values: give two bags of corn to both cows $2$ and $3$,
then  give five bags of corn to both cows $1$ and $2$, resulting in each cow
having  a hunger level of $3$.

Another one of these tuples is $h=[0,1,0]$. In this case, it is impossible to 
make the hunger levels of the cows equal.

SAMPLE INPUT:
4
6 8 5 9
SAMPLE OUTPUT: 
137

SCORING:
$N$ is even in even-numbered tests and odd in odd-numbered tests.

Tests 3 and 4 satisfy $N\le 6$ and $H_i \le 10$.Tests 5 through 10
satisfy $N\le 50$ and $H_i \le 100$.Tests 11 through 20 satisfy no
further constraints.


Problem credits: Arpan Banerjee and Benjamin Qi



\section*{Input Format}
The second line contains $H_1,H_2,\ldots,H_N$.

OUTPUT FORMAT (print output to the terminal / stdout):The number of $N$-tuples of hunger levels modulo $10^9+7$.SAMPLE INPUT:3
9 11 7SAMPLE OUTPUT:241There are $(9+1)\cdot (11+1)\cdot (7+1)$ $3$-tuples $h$ that are consistent with
$H$.One of these tuples is $h=[8,10,5]$. In this case, it is possible to make all
cows have equal hunger values: give two bags of corn to both cows $2$ and $3$,
then  give five bags of corn to both cows $1$ and $2$, resulting in each cow
having  a hunger level of $3$.Another one of these tuples is $h=[0,1,0]$. In this case, it is impossible to 
make the hunger levels of the cows equal.SAMPLE INPUT:4
6 8 5 9SAMPLE OUTPUT:137SCORING:$N$ is even in even-numbered tests and odd in odd-numbered tests.Tests 3 and 4 satisfy $N\le 6$ and $H_i \le 10$.Tests 5 through 10
satisfy $N\le 50$ and $H_i \le 100$.Tests 11 through 20 satisfy no
further constraints.Problem credits: Arpan Banerjee and Benjamin Qi

\section*{Output Format}
SAMPLE INPUT:3
9 11 7SAMPLE OUTPUT:241There are $(9+1)\cdot (11+1)\cdot (7+1)$ $3$-tuples $h$ that are consistent with
$H$.One of these tuples is $h=[8,10,5]$. In this case, it is possible to make all
cows have equal hunger values: give two bags of corn to both cows $2$ and $3$,
then  give five bags of corn to both cows $1$ and $2$, resulting in each cow
having  a hunger level of $3$.Another one of these tuples is $h=[0,1,0]$. In this case, it is impossible to 
make the hunger levels of the cows equal.SAMPLE INPUT:4
6 8 5 9SAMPLE OUTPUT:137SCORING:$N$ is even in even-numbered tests and odd in odd-numbered tests.Tests 3 and 4 satisfy $N\le 6$ and $H_i \le 10$.Tests 5 through 10
satisfy $N\le 50$ and $H_i \le 100$.Tests 11 through 20 satisfy no
further constraints.Problem credits: Arpan Banerjee and Benjamin Qi

\section*{Sample Input}
\begin{verbatim}
3
9 11 7

4
6 8 5 9
\end{verbatim}

\section*{Sample Output}
\begin{verbatim}
241

There are $(9+1)\cdot (11+1)\cdot (7+1)$ $3$-tuples $h$ that are consistent with
$H$.One of these tuples is $h=[8,10,5]$. In this case, it is possible to make all
cows have equal hunger values: give two bags of corn to both cows $2$ and $3$,
then  give five bags of corn to both cows $1$ and $2$, resulting in each cow
having  a hunger level of $3$.Another one of these tuples is $h=[0,1,0]$. In this case, it is impossible to 
make the hunger levels of the cows equal.SAMPLE INPUT:4
6 8 5 9SAMPLE OUTPUT:137SCORING:$N$ is even in even-numbered tests and odd in odd-numbered tests.Tests 3 and 4 satisfy $N\le 6$ and $H_i \le 10$.Tests 5 through 10
satisfy $N\le 50$ and $H_i \le 100$.Tests 11 through 20 satisfy no
further constraints.Problem credits: Arpan Banerjee and Benjamin Qi

One of these tuples is $h=[8,10,5]$. In this case, it is possible to make all
cows have equal hunger values: give two bags of corn to both cows $2$ and $3$,
then  give five bags of corn to both cows $1$ and $2$, resulting in each cow
having  a hunger level of $3$.Another one of these tuples is $h=[0,1,0]$. In this case, it is impossible to 
make the hunger levels of the cows equal.SAMPLE INPUT:4
6 8 5 9SAMPLE OUTPUT:137SCORING:$N$ is even in even-numbered tests and odd in odd-numbered tests.Tests 3 and 4 satisfy $N\le 6$ and $H_i \le 10$.Tests 5 through 10
satisfy $N\le 50$ and $H_i \le 100$.Tests 11 through 20 satisfy no
further constraints.Problem credits: Arpan Banerjee and Benjamin Qi

Another one of these tuples is $h=[0,1,0]$. In this case, it is impossible to 
make the hunger levels of the cows equal.SAMPLE INPUT:4
6 8 5 9SAMPLE OUTPUT:137SCORING:$N$ is even in even-numbered tests and odd in odd-numbered tests.Tests 3 and 4 satisfy $N\le 6$ and $H_i \le 10$.Tests 5 through 10
satisfy $N\le 50$ and $H_i \le 100$.Tests 11 through 20 satisfy no
further constraints.Problem credits: Arpan Banerjee and Benjamin Qi

SAMPLE INPUT:4
6 8 5 9SAMPLE OUTPUT:137SCORING:$N$ is even in even-numbered tests and odd in odd-numbered tests.Tests 3 and 4 satisfy $N\le 6$ and $H_i \le 10$.Tests 5 through 10
satisfy $N\le 50$ and $H_i \le 100$.Tests 11 through 20 satisfy no
further constraints.Problem credits: Arpan Banerjee and Benjamin Qi

137

SCORING:$N$ is even in even-numbered tests and odd in odd-numbered tests.Tests 3 and 4 satisfy $N\le 6$ and $H_i \le 10$.Tests 5 through 10
satisfy $N\le 50$ and $H_i \le 100$.Tests 11 through 20 satisfy no
further constraints.Problem credits: Arpan Banerjee and Benjamin Qi
\end{verbatim}

\section*{Solution}


\section*{Problem URL}
https://usaco.org/index.php?page=viewproblem2&cpid=1185

\section*{Source}
USACO JAN22 Contest, Gold Division, Problem ID: 1185

\end{document}
