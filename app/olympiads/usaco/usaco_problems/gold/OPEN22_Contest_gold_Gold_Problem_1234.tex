\documentclass[12pt]{article}
\usepackage{amsmath}
\usepackage{amssymb}
\usepackage{graphicx}
\usepackage{enumitem}
\usepackage{hyperref}
\usepackage{xcolor}

\title{View problem}
\author{USACO Contest: OPEN22 Contest - Gold Division}
\date{\today}

\begin{document}

\maketitle

\section*{Problem Statement}

A program consists of a sequence of instructions, each of which is of one of the
following forms:

$\times d$, where $d$ is a digit in the range $[0,9]$$+s$, where $s$ is a string denoting the name of a variable. Within a
program, all variable names must be distinct.
The result of executing a program is defined to be the expression that results
after applying each instruction in order, starting with $0$. For example, the result of
executing the program  $[\times 3,+x,+y,\times 2,+z]$ is the expression
$(0\times 3+x+y)\times 2+z=2\times x+2\times y+z$. Different programs, when
executed may produce the same expressions; for example, executing
$[+w,\times 0,+y,+x,\times 2,+z, \times 1]$ would also result in the expression
$2\times x+2\times y+z$.

Bessie and Elsie each have programs of $N$ ($1\le N\le 2000$) instructions. They
will interleave these programs to produce a new program of length $2N$.  Note
that there are $\frac{(2N)!}{N!\times N!}$ ways to do this, but not all such
programs, when executed, will produce distinct expressions.

Count the number of distinct expressions that may be produced by executing
Bessie and Elsie's interleaved program, modulo $10^9+7$. 

Each input contains $T$ ($1\le T\le 10$)  test cases that should be solved
independently. It is guaranteed that the sum of $N$ over all test cases does not
exceed $2000$.

INPUT FORMAT (input arrives from the terminal / stdin):
The first line of the input contains $T$, the number of test cases.

The first line of each test case contains $N$.

The second line of each test case contains Bessie's program, represented by a
string of length $N$. Each character is either a digit $d\in [0,9]$,
representing an instruction of type 1, or the character $+$, representing an
instruction of type 2.

The third line of each test case contains Elsie's program in the same format as
Bessie's.

Within a test case, the variable names among all instructions are distinct. Note
that their actual names are not provided, as they do not affect the answer.

OUTPUT FORMAT (print output to the terminal / stdout):
The number of distinct expressions that may be produced by executing  Bessie and
Elsie's interleaved programs, modulo $10^9+7$.

SAMPLE INPUT:
4
1
0
1
3
12+
+02
3
0++
++9
4
5+++
+6+1
SAMPLE OUTPUT: 
1
3
9
9

For the first test case, the two possible interleaved programs are
$[\times 1, \times 0]$  and $[\times 0,\times 1]$. These will both produce the
expression $0$ when executed.

For the second test case, executing an interleaving of $[\times 1,\times 2, +x]$
and $[+y, \times 0,\times 2]$ could produce one of the expressions $0$, $x$, or
$2\times x$.

SCORING:
Input 2 satisfies $N\le 6$.In inputs 3-5, the sum of all $N$ is at most $100$.In inputs 6-8, the sum of all $N$ is at most $500$.Inputs 9-16 satisfy no additional constraints.


Problem credits: Benjamin Qi



\section*{Input Format}
The first line of each test case contains $N$.The second line of each test case contains Bessie's program, represented by a
string of length $N$. Each character is either a digit $d\in [0,9]$,
representing an instruction of type 1, or the character $+$, representing an
instruction of type 2.The third line of each test case contains Elsie's program in the same format as
Bessie's.Within a test case, the variable names among all instructions are distinct. Note
that their actual names are not provided, as they do not affect the answer.

The second line of each test case contains Bessie's program, represented by a
string of length $N$. Each character is either a digit $d\in [0,9]$,
representing an instruction of type 1, or the character $+$, representing an
instruction of type 2.The third line of each test case contains Elsie's program in the same format as
Bessie's.Within a test case, the variable names among all instructions are distinct. Note
that their actual names are not provided, as they do not affect the answer.

The third line of each test case contains Elsie's program in the same format as
Bessie's.Within a test case, the variable names among all instructions are distinct. Note
that their actual names are not provided, as they do not affect the answer.

Within a test case, the variable names among all instructions are distinct. Note
that their actual names are not provided, as they do not affect the answer.

OUTPUT FORMAT (print output to the terminal / stdout):The number of distinct expressions that may be produced by executing  Bessie and
Elsie's interleaved programs, modulo $10^9+7$.SAMPLE INPUT:4
1
0
1
3
12+
+02
3
0++
++9
4
5+++
+6+1SAMPLE OUTPUT:1
3
9
9For the first test case, the two possible interleaved programs are
$[\times 1, \times 0]$  and $[\times 0,\times 1]$. These will both produce the
expression $0$ when executed.For the second test case, executing an interleaving of $[\times 1,\times 2, +x]$
and $[+y, \times 0,\times 2]$ could produce one of the expressions $0$, $x$, or
$2\times x$.SCORING:Input 2 satisfies $N\le 6$.In inputs 3-5, the sum of all $N$ is at most $100$.In inputs 6-8, the sum of all $N$ is at most $500$.Inputs 9-16 satisfy no additional constraints.Problem credits: Benjamin Qi

\section*{Output Format}
SAMPLE INPUT:4
1
0
1
3
12+
+02
3
0++
++9
4
5+++
+6+1SAMPLE OUTPUT:1
3
9
9For the first test case, the two possible interleaved programs are
$[\times 1, \times 0]$  and $[\times 0,\times 1]$. These will both produce the
expression $0$ when executed.For the second test case, executing an interleaving of $[\times 1,\times 2, +x]$
and $[+y, \times 0,\times 2]$ could produce one of the expressions $0$, $x$, or
$2\times x$.SCORING:Input 2 satisfies $N\le 6$.In inputs 3-5, the sum of all $N$ is at most $100$.In inputs 6-8, the sum of all $N$ is at most $500$.Inputs 9-16 satisfy no additional constraints.Problem credits: Benjamin Qi

\section*{Sample Input}
\begin{verbatim}
4
1
0
1
3
12+
+02
3
0++
++9
4
5+++
+6+1
\end{verbatim}

\section*{Sample Output}
\begin{verbatim}
1
3
9
9

For the first test case, the two possible interleaved programs are
$[\times 1, \times 0]$  and $[\times 0,\times 1]$. These will both produce the
expression $0$ when executed.For the second test case, executing an interleaving of $[\times 1,\times 2, +x]$
and $[+y, \times 0,\times 2]$ could produce one of the expressions $0$, $x$, or
$2\times x$.SCORING:Input 2 satisfies $N\le 6$.In inputs 3-5, the sum of all $N$ is at most $100$.In inputs 6-8, the sum of all $N$ is at most $500$.Inputs 9-16 satisfy no additional constraints.Problem credits: Benjamin Qi

For the second test case, executing an interleaving of $[\times 1,\times 2, +x]$
and $[+y, \times 0,\times 2]$ could produce one of the expressions $0$, $x$, or
$2\times x$.SCORING:Input 2 satisfies $N\le 6$.In inputs 3-5, the sum of all $N$ is at most $100$.In inputs 6-8, the sum of all $N$ is at most $500$.Inputs 9-16 satisfy no additional constraints.Problem credits: Benjamin Qi

SCORING:Input 2 satisfies $N\le 6$.In inputs 3-5, the sum of all $N$ is at most $100$.In inputs 6-8, the sum of all $N$ is at most $500$.Inputs 9-16 satisfy no additional constraints.Problem credits: Benjamin Qi
\end{verbatim}

\section*{Solution}


\section*{Problem URL}
https://usaco.org/index.php?page=viewproblem2&cpid=1234

\section*{Source}
USACO OPEN22 Contest, Gold Division, Problem ID: 1234

\end{document}
