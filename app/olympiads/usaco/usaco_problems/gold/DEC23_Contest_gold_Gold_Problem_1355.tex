\documentclass[12pt]{article}
\usepackage{amsmath}
\usepackage{amssymb}
\usepackage{graphicx}
\usepackage{enumitem}
\usepackage{hyperref}
\usepackage{xcolor}

\title{View problem}
\author{USACO Contest: DEC23 Contest - Gold Division}
\date{\today}

\begin{document}

\maketitle

\section*{Problem Statement}


Farmer John is distributing haybales across the farm!

Farmer John's farm has $N$ $(1\le N\le 2\cdot 10^5)$ barns, located at integer
points $x_1,\dots, x_N$ $(0 \le x_i \le 10^6)$ on the number line. Farmer John's
plan is to first have $N$ shipments of haybales delivered to some integer point
$y$ $(0 \le y \le 10^6)$ and then distribute one shipment to each barn.

Unfortunately, Farmer John's distribution service is very wasteful. In
particular, for some $a_i$ and $b_i$ $(1\le a_i, b_i\le 10^6)$, $a_i$ haybales
are wasted per unit of distance left each shipment is transported, and $b_i$
haybales are wasted per unit of distance right each shipment is transported.
Formally, for a shipment being transported from point $y$ to a barn at point
$x$, the number of haybales wasted is given by 

$$\begin{cases}
 a_i\cdot (y-x) & \text{if } y \ge x \\
b_i\cdot (x-y) & \text{if } x > y
\end{cases}.$$
Given $Q$ $(1\le Q\le 2\cdot 10^5)$ independent queries each consisting of
possible values of $(a_i,b_i)$, please help Farmer John determine the fewest
amount of haybales that will be wasted if he chooses $y$ optimally. 

INPUT FORMAT (input arrives from the terminal / stdin):
The first line contains $N$.

The next line contains $x_1\dots x_N$.

The next line contains $Q$.

The next $Q$ lines each contain two integers $a_i$ and $b_i$.

OUTPUT FORMAT (print output to the terminal / stdout):
Output $Q$ lines, the $i$th line containing the answer for the $i$th query.

SAMPLE INPUT:
5
1 4 2 3 10
4
1 1
2 1
1 2
1 4
SAMPLE OUTPUT: 
11
13
18
30

For example, to answer the second query, it is optimal to select $y=2$. Then the
number of wasted haybales is equal to
$2(2-1)+2(2-2)+1(3-2)+1(4-2)+1(10-2)=1+0+1+2+8=13$.

SCORING:
Input 2: $N,Q\le 10$Input 3: $N,Q\le 500$Inputs 4-6: $N,Q\le 5000$Inputs 7-16: No additional constraints.


Problem credits: Benjamin Qi



\section*{Input Format}
The next line contains $x_1\dots x_N$.The next line contains $Q$.The next $Q$ lines each contain two integers $a_i$ and $b_i$.

The next line contains $Q$.The next $Q$ lines each contain two integers $a_i$ and $b_i$.

The next $Q$ lines each contain two integers $a_i$ and $b_i$.

OUTPUT FORMAT (print output to the terminal / stdout):Output $Q$ lines, the $i$th line containing the answer for the $i$th query.SAMPLE INPUT:5
1 4 2 3 10
4
1 1
2 1
1 2
1 4SAMPLE OUTPUT:11
13
18
30For example, to answer the second query, it is optimal to select $y=2$. Then the
number of wasted haybales is equal to
$2(2-1)+2(2-2)+1(3-2)+1(4-2)+1(10-2)=1+0+1+2+8=13$.SCORING:Input 2: $N,Q\le 10$Input 3: $N,Q\le 500$Inputs 4-6: $N,Q\le 5000$Inputs 7-16: No additional constraints.Problem credits: Benjamin Qi

\section*{Output Format}
SAMPLE INPUT:5
1 4 2 3 10
4
1 1
2 1
1 2
1 4SAMPLE OUTPUT:11
13
18
30For example, to answer the second query, it is optimal to select $y=2$. Then the
number of wasted haybales is equal to
$2(2-1)+2(2-2)+1(3-2)+1(4-2)+1(10-2)=1+0+1+2+8=13$.SCORING:Input 2: $N,Q\le 10$Input 3: $N,Q\le 500$Inputs 4-6: $N,Q\le 5000$Inputs 7-16: No additional constraints.Problem credits: Benjamin Qi

\section*{Sample Input}
\begin{verbatim}
5
1 4 2 3 10
4
1 1
2 1
1 2
1 4
\end{verbatim}

\section*{Sample Output}
\begin{verbatim}
11
13
18
30

For example, to answer the second query, it is optimal to select $y=2$. Then the
number of wasted haybales is equal to
$2(2-1)+2(2-2)+1(3-2)+1(4-2)+1(10-2)=1+0+1+2+8=13$.SCORING:Input 2: $N,Q\le 10$Input 3: $N,Q\le 500$Inputs 4-6: $N,Q\le 5000$Inputs 7-16: No additional constraints.Problem credits: Benjamin Qi

SCORING:Input 2: $N,Q\le 10$Input 3: $N,Q\le 500$Inputs 4-6: $N,Q\le 5000$Inputs 7-16: No additional constraints.Problem credits: Benjamin Qi
\end{verbatim}

\section*{Solution}


\section*{Problem URL}
https://usaco.org/index.php?page=viewproblem2&cpid=1355

\section*{Source}
USACO DEC23 Contest, Gold Division, Problem ID: 1355

\end{document}
