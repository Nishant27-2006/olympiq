\documentclass[12pt]{article}
\usepackage{amsmath}
\usepackage{amssymb}
\usepackage{graphicx}
\usepackage{enumitem}
\usepackage{hyperref}
\usepackage{xcolor}

\title{View problem}
\author{USACO Contest: FEB25 Contest - Gold Division}
\date{\today}

\begin{document}

\maketitle

\section*{Problem Statement}


Bessie has a special function $f(x)$ that takes as input an integer in $[1, N]$
and  returns an integer in $[1, N]$ ($1 \le N \le 2 \cdot 10^5$). Her function
$f(x)$ is defined by  $N$ integers $a_1 \ldots a_N$ where $f(x) = a_x$
($1 \le a_i \le N$).  

Bessie wants this function to be idempotent. In other words, it should satisfy
$f(f(x)) = f(x)$  for all integers $x \in [1, N]$. 

For a cost of $c_i$, Bessie can change the value of $a_i$ to any integer in
$[1, N]$ ($1 \le c_i \le 10^9$). Determine the minimum total cost Bessie needs 
to make $f(x)$ idempotent.

INPUT FORMAT (input arrives from the terminal / stdin):
The first line contains $N$.

The second line contains $N$ space-separated integers $a_1,a_2,\dots,a_N$.

The third line contains $N$ space-separated integers $c_1,c_2,\dots,c_N$.


OUTPUT FORMAT (print output to the terminal / stdout):
Output the minimum total cost Bessie needs to make $f(x)$ idempotent.


SAMPLE INPUT:
5
2 4 4 5 3
1 1 1 1 1
SAMPLE OUTPUT: 
3

We can change $a_1 = 4$, $a_4 = 4$, $a_5 = 4$. Since all $c_i$ equal one, the
total cost is equal to $3$, the number of changes. It can be shown that there is
no solution with only $2$ or fewer changes.

SAMPLE INPUT:
8
1 2 5 5 3 3 4 4
9 9 2 5 9 9 9 9
SAMPLE OUTPUT: 
7

We change $a_3 = 3$ and $a_4 = 4$. The total cost is $2+5=7$.

SCORING:
Subtasks:

Input 3: $N\le 20$Inputs 4-9: $a_i\ge i$Inputs 10-15: All $a_i$ are distinct.Inputs 16-21: No additional constraints.
Additionally, in each of the last three subtasks, the first half of tests  will
satisfy $c_i=1$ for all $i$.



Problem credits: Avnith Vijayram



\section*{Input Format}
The second line contains $N$ space-separated integers $a_1,a_2,\dots,a_N$.The third line contains $N$ space-separated integers $c_1,c_2,\dots,c_N$.

The third line contains $N$ space-separated integers $c_1,c_2,\dots,c_N$.

OUTPUT FORMAT (print output to the terminal / stdout):Output the minimum total cost Bessie needs to make $f(x)$ idempotent.SAMPLE INPUT:5
2 4 4 5 3
1 1 1 1 1SAMPLE OUTPUT:3We can change $a_1 = 4$, $a_4 = 4$, $a_5 = 4$. Since all $c_i$ equal one, the
total cost is equal to $3$, the number of changes. It can be shown that there is
no solution with only $2$ or fewer changes.SAMPLE INPUT:8
1 2 5 5 3 3 4 4
9 9 2 5 9 9 9 9SAMPLE OUTPUT:7We change $a_3 = 3$ and $a_4 = 4$. The total cost is $2+5=7$.SCORING:Subtasks:Input 3: $N\le 20$Inputs 4-9: $a_i\ge i$Inputs 10-15: All $a_i$ are distinct.Inputs 16-21: No additional constraints.Additionally, in each of the last three subtasks, the first half of tests  will
satisfy $c_i=1$ for all $i$.Problem credits: Avnith Vijayram

\section*{Output Format}
SAMPLE INPUT:5
2 4 4 5 3
1 1 1 1 1SAMPLE OUTPUT:3We can change $a_1 = 4$, $a_4 = 4$, $a_5 = 4$. Since all $c_i$ equal one, the
total cost is equal to $3$, the number of changes. It can be shown that there is
no solution with only $2$ or fewer changes.SAMPLE INPUT:8
1 2 5 5 3 3 4 4
9 9 2 5 9 9 9 9SAMPLE OUTPUT:7We change $a_3 = 3$ and $a_4 = 4$. The total cost is $2+5=7$.SCORING:Subtasks:Input 3: $N\le 20$Inputs 4-9: $a_i\ge i$Inputs 10-15: All $a_i$ are distinct.Inputs 16-21: No additional constraints.Additionally, in each of the last three subtasks, the first half of tests  will
satisfy $c_i=1$ for all $i$.Problem credits: Avnith Vijayram

\section*{Sample Input}
\begin{verbatim}
5
2 4 4 5 3
1 1 1 1 1

8
1 2 5 5 3 3 4 4
9 9 2 5 9 9 9 9
\end{verbatim}

\section*{Sample Output}
\begin{verbatim}
3

We can change $a_1 = 4$, $a_4 = 4$, $a_5 = 4$. Since all $c_i$ equal one, the
total cost is equal to $3$, the number of changes. It can be shown that there is
no solution with only $2$ or fewer changes.SAMPLE INPUT:8
1 2 5 5 3 3 4 4
9 9 2 5 9 9 9 9SAMPLE OUTPUT:7We change $a_3 = 3$ and $a_4 = 4$. The total cost is $2+5=7$.SCORING:Subtasks:Input 3: $N\le 20$Inputs 4-9: $a_i\ge i$Inputs 10-15: All $a_i$ are distinct.Inputs 16-21: No additional constraints.Additionally, in each of the last three subtasks, the first half of tests  will
satisfy $c_i=1$ for all $i$.Problem credits: Avnith Vijayram

SAMPLE INPUT:8
1 2 5 5 3 3 4 4
9 9 2 5 9 9 9 9SAMPLE OUTPUT:7We change $a_3 = 3$ and $a_4 = 4$. The total cost is $2+5=7$.SCORING:Subtasks:Input 3: $N\le 20$Inputs 4-9: $a_i\ge i$Inputs 10-15: All $a_i$ are distinct.Inputs 16-21: No additional constraints.Additionally, in each of the last three subtasks, the first half of tests  will
satisfy $c_i=1$ for all $i$.Problem credits: Avnith Vijayram

7

We change $a_3 = 3$ and $a_4 = 4$. The total cost is $2+5=7$.SCORING:Subtasks:Input 3: $N\le 20$Inputs 4-9: $a_i\ge i$Inputs 10-15: All $a_i$ are distinct.Inputs 16-21: No additional constraints.Additionally, in each of the last three subtasks, the first half of tests  will
satisfy $c_i=1$ for all $i$.Problem credits: Avnith Vijayram

SCORING:Subtasks:Input 3: $N\le 20$Inputs 4-9: $a_i\ge i$Inputs 10-15: All $a_i$ are distinct.Inputs 16-21: No additional constraints.Additionally, in each of the last three subtasks, the first half of tests  will
satisfy $c_i=1$ for all $i$.Problem credits: Avnith Vijayram
\end{verbatim}

\section*{Solution}


\section*{Problem URL}
https://usaco.org/index.php?page=viewproblem2&cpid=1497

\section*{Source}
USACO FEB25 Contest, Gold Division, Problem ID: 1497

\end{document}
