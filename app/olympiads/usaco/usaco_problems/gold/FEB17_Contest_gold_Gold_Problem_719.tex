\documentclass[12pt]{article}
\usepackage{amsmath}
\usepackage{amssymb}
\usepackage{graphicx}
\usepackage{enumitem}
\usepackage{hyperref}
\usepackage{xcolor}

\title{View problem}
\author{USACO Contest: FEB17 Contest - Gold Division}
\date{\today}

\begin{document}

\maketitle

\section*{Problem Statement}

The layout of Farmer John's farm is quite peculiar, with a large circular road
running around the perimeter of the main field on which his cows graze during
the day. Every morning, the cows cross this road on their way towards the field,
and every evening they all cross again as they leave the field and return to the
barn.

As we know, cows are creatures of habit, and they each cross the road the same
way every day.  Each cow crosses into the field at a different point from where
she crosses out of the field, and all of these crossing points are distinct from
each-other. Farmer John owns $N$ cows, conveniently identified with the integer
IDs $1 \ldots N$, so there are precisely $2N$ crossing points around the road. 
Farmer John records these crossing points concisely by scanning around the
circle clockwise, writing down the ID of the cow for each crossing point,
ultimately forming a sequence with $2N$ numbers in which each number appears
exactly twice.  He does not record which crossing points are entry points and
which are exit points.

Looking at his map of crossing points, Farmer John is curious how many times
various pairs of cows might cross paths during the day.  He calls a pair of cows
$(a,b)$ a "crossing" pair if cow $a$'s path from entry to exit must cross cow
$b$'s path from entry to exit.  Please help Farmer John count the total number
of crossing pairs.

INPUT FORMAT (file circlecross.in):
The first line of input contains $N$ ($1 \leq N \leq 50,000$), and the next $2N$
lines describe the cow IDs for the sequence of entry and exit points around the
field.

OUTPUT FORMAT (file circlecross.out):
Please print the total number of crossing pairs.

SAMPLE INPUT:
4
3
2
4
4
1
3
2
1
SAMPLE OUTPUT: 
3


Problem credits: Brian Dean



\section*{Input Format}
OUTPUT FORMAT (file circlecross.out):Please print the total number of crossing pairs.SAMPLE INPUT:4
3
2
4
4
1
3
2
1SAMPLE OUTPUT:3Problem credits: Brian Dean

\section*{Output Format}
SAMPLE INPUT:4
3
2
4
4
1
3
2
1SAMPLE OUTPUT:3Problem credits: Brian Dean

\section*{Sample Input}
\begin{verbatim}
4
3
2
4
4
1
3
2
1
\end{verbatim}

\section*{Sample Output}
\begin{verbatim}
3

Problem credits: Brian Dean

Problem credits: Brian Dean
\end{verbatim}

\section*{Solution}


\section*{Problem URL}
https://usaco.org/index.php?page=viewproblem2&cpid=719

\section*{Source}
USACO FEB17 Contest, Gold Division, Problem ID: 719

\end{document}
