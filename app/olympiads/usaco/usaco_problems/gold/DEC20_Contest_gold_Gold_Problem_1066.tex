\documentclass[12pt]{article}
\usepackage{amsmath}
\usepackage{amssymb}
\usepackage{graphicx}
\usepackage{enumitem}
\usepackage{hyperref}
\usepackage{xcolor}

\title{View problem}
\author{USACO Contest: DEC20 Contest - Gold Division}
\date{\today}

\begin{document}

\maketitle

\section*{Problem Statement}

After sequencing the genomes of his cows, Farmer John has moved onto genomic
editing! As we know, a genome can be represented by a string consisting of As,
Cs, Gs, and Ts.  The maximum length of a genome under consideration by Farmer
John is $10^5$. 

Farmer John starts with a single genome and edits it by performing the following
steps:

Split the genome between every two consecutive equal characters.Reverse each of the resulting substrings.Concatenate the reversed substrings together in the same order.
For example, if FJ started with the genome AGGCTTT then he would perform the
following steps:

Split between the consecutive Gs and Ts to get AG | GCT | T | T.Reverse each substring to get GA | TCG | T | T.Concatenate the substrings together to get GATCGTT.
Unfortunately, after editing the genome, Farmer John's computer crashed and he
lost the sequence of the genome he originally started with. Furthermore, some 
parts of the edited genome have been damaged, meaning that they have been 
replaced by question marks.

Given the sequence of the edited genome, help FJ determine the number of 
possibilities for the original genome, modulo $10^9+7$.

INPUT FORMAT (input arrives from the terminal / stdin):
A non-empty string where each character is one of A, G, C, T, or ?.

OUTPUT FORMAT (print output to the terminal / stdout):
The number of possible original genomes modulo $10^9+7$.

SAMPLE INPUT:
?
SAMPLE OUTPUT: 
4

The question mark can be any of A, G, C, or T.

SAMPLE INPUT:
GAT?GTT
SAMPLE OUTPUT: 
3

There are two possible original genomes aside from AGGCTTT, which was described
above.


AGGATTT -> AG | GAT | T | T -> GA | TAG | T | T -> GATAGTT

TAGGTTT -> TAG | GT | T | T -> GAT | TG | T | T -> GATTGTT

SCORING:
In test cases 1-4, the genome has length at most $10$.In test cases 5-11, the genome has length at most $10^2$.In test cases 12-20, there are no additional constraints.


Problem credits: Benjamin Qi



\section*{Input Format}
OUTPUT FORMAT (print output to the terminal / stdout):The number of possible original genomes modulo $10^9+7$.SAMPLE INPUT:?SAMPLE OUTPUT:4The question mark can be any of A, G, C, or T.SAMPLE INPUT:GAT?GTTSAMPLE OUTPUT:3There are two possible original genomes aside from AGGCTTT, which was described
above.AGGATTT -> AG | GAT | T | T -> GA | TAG | T | T -> GATAGTT

TAGGTTT -> TAG | GT | T | T -> GAT | TG | T | T -> GATTGTTSCORING:In test cases 1-4, the genome has length at most $10$.In test cases 5-11, the genome has length at most $10^2$.In test cases 12-20, there are no additional constraints.Problem credits: Benjamin Qi

\section*{Output Format}
SAMPLE INPUT:?SAMPLE OUTPUT:4The question mark can be any of A, G, C, or T.SAMPLE INPUT:GAT?GTTSAMPLE OUTPUT:3There are two possible original genomes aside from AGGCTTT, which was described
above.AGGATTT -> AG | GAT | T | T -> GA | TAG | T | T -> GATAGTT

TAGGTTT -> TAG | GT | T | T -> GAT | TG | T | T -> GATTGTTSCORING:In test cases 1-4, the genome has length at most $10$.In test cases 5-11, the genome has length at most $10^2$.In test cases 12-20, there are no additional constraints.Problem credits: Benjamin Qi

\section*{Sample Input}
\begin{verbatim}
?

GAT?GTT
\end{verbatim}

\section*{Sample Output}
\begin{verbatim}
4

The question mark can be any of A, G, C, or T.SAMPLE INPUT:GAT?GTTSAMPLE OUTPUT:3There are two possible original genomes aside from AGGCTTT, which was described
above.AGGATTT -> AG | GAT | T | T -> GA | TAG | T | T -> GATAGTT

TAGGTTT -> TAG | GT | T | T -> GAT | TG | T | T -> GATTGTTSCORING:In test cases 1-4, the genome has length at most $10$.In test cases 5-11, the genome has length at most $10^2$.In test cases 12-20, there are no additional constraints.Problem credits: Benjamin Qi

SAMPLE INPUT:GAT?GTTSAMPLE OUTPUT:3There are two possible original genomes aside from AGGCTTT, which was described
above.AGGATTT -> AG | GAT | T | T -> GA | TAG | T | T -> GATAGTT

TAGGTTT -> TAG | GT | T | T -> GAT | TG | T | T -> GATTGTTSCORING:In test cases 1-4, the genome has length at most $10$.In test cases 5-11, the genome has length at most $10^2$.In test cases 12-20, there are no additional constraints.Problem credits: Benjamin Qi

3

There are two possible original genomes aside from AGGCTTT, which was described
above.AGGATTT -> AG | GAT | T | T -> GA | TAG | T | T -> GATAGTT

TAGGTTT -> TAG | GT | T | T -> GAT | TG | T | T -> GATTGTTSCORING:In test cases 1-4, the genome has length at most $10$.In test cases 5-11, the genome has length at most $10^2$.In test cases 12-20, there are no additional constraints.Problem credits: Benjamin Qi

AGGATTT -> AG | GAT | T | T -> GA | TAG | T | T -> GATAGTT

TAGGTTT -> TAG | GT | T | T -> GAT | TG | T | T -> GATTGTTSCORING:In test cases 1-4, the genome has length at most $10$.In test cases 5-11, the genome has length at most $10^2$.In test cases 12-20, there are no additional constraints.Problem credits: Benjamin Qi

AGGATTT -> AG | GAT | T | T -> GA | TAG | T | T -> GATAGTT

TAGGTTT -> TAG | GT | T | T -> GAT | TG | T | T -> GATTGTT

SCORING:In test cases 1-4, the genome has length at most $10$.In test cases 5-11, the genome has length at most $10^2$.In test cases 12-20, there are no additional constraints.Problem credits: Benjamin Qi
\end{verbatim}

\section*{Solution}


\section*{Problem URL}
https://usaco.org/index.php?page=viewproblem2&cpid=1066

\section*{Source}
USACO DEC20 Contest, Gold Division, Problem ID: 1066

\end{document}
