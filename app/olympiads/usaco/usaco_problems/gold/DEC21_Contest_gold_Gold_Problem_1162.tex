\documentclass[12pt]{article}
\usepackage{amsmath}
\usepackage{amssymb}
\usepackage{graphicx}
\usepackage{enumitem}
\usepackage{hyperref}
\usepackage{xcolor}

\title{View problem}
\author{USACO Contest: DEC21 Contest - Gold Division}
\date{\today}

\begin{document}

\maketitle

\section*{Problem Statement}

Bessie knows a number $x+0.5$ where $x$ is some integer between $0$ to $N,$
inclusive ($1\le N\le 2 \cdot 10^5$).  

Elsie is trying to guess this number. She can ask questions of the form  "is $i$
high or low?" for some integer $i$ between $1$ and $N,$ inclusive.  Bessie
responds by saying "HI" if $i$ is greater than $x+0.5$, or "LO" if $i$ is less
than $x+0.5$.

Elsie comes up with the following strategy for guessing Bessie's number.  Before
making any guesses, she creates a list of $N$ numbers, where every number from
$1$ to $N$ occurs exactly once (in other words, the list is a permutation of
size $N$). Then she goes through the list, guessing numbers that appear in the
list in order.

However, Elsie skips any unnecessary guesses. That is, if Elsie is about to
guess some number $i$ and Elsie previously guessed some $j < i$ and Bessie
responded with "HI", Elsie will not guess $i$ and will move on to the next
number in the list. Similarly, if she is about to guess some number $i$ and she
previously guessed some $j > i$ and Bessie responded with "LO", Elsie will not
guess $i$ and will move on to the next number in the list. It can be proven that
using this strategy, Elsie always uniquely determines $x$ regardless of the
permutation she creates.

If we concatenate all of Bessie's responses of either "HI" or "LO" into a single
string $S$, then the number of times Bessie says "HILO" is the number of length
$4$ substrings of $S$ that are equal to "HILO."

Bessie knows that Elsie will use this strategy; furthermore, she also knows the
exact permutation that Elsie will use. However, Bessie has not decided on what
value of $x$ to choose.

Help Bessie determine how many times she will say "HILO" for each value of $x$.

INPUT FORMAT (input arrives from the terminal / stdin):
The first line contains $N.$ 

The second line contains Elsie's permutation of size $N.$

OUTPUT FORMAT (print output to the terminal / stdout):
For each $x$ from $0$ to $N$, inclusive, output the number of times Bessie will
say HILO on a new line.

SAMPLE INPUT:
5
5 1 2 4 3
SAMPLE OUTPUT: 
0
1
1
2
1
0

For $x=0$, Bessie will say "HIHI," for a total of zero "HILO"s.

For $x=2$, Bessie will say "HILOLOHIHI," for a total of one "HILO".

For $x=3$, Bessie will say "HILOLOHILO", for a total of two "HILO"s.

SCORING:
Tests 1 to 4 satisfy $N \leq 5000$.Tests 5 to 8 have a uniformly
random permutation.Tests 9 to 20 satisfy no further constraints.


Problem credits: Richard Qi



\section*{Input Format}
The second line contains Elsie's permutation of size $N.$

OUTPUT FORMAT (print output to the terminal / stdout):For each $x$ from $0$ to $N$, inclusive, output the number of times Bessie will
say HILO on a new line.SAMPLE INPUT:5
5 1 2 4 3SAMPLE OUTPUT:0
1
1
2
1
0For $x=0$, Bessie will say "HIHI," for a total of zero "HILO"s.For $x=2$, Bessie will say "HILOLOHIHI," for a total of one "HILO".For $x=3$, Bessie will say "HILOLOHILO", for a total of two "HILO"s.SCORING:Tests 1 to 4 satisfy $N \leq 5000$.Tests 5 to 8 have a uniformly
random permutation.Tests 9 to 20 satisfy no further constraints.Problem credits: Richard Qi

\section*{Output Format}
SAMPLE INPUT:5
5 1 2 4 3SAMPLE OUTPUT:0
1
1
2
1
0For $x=0$, Bessie will say "HIHI," for a total of zero "HILO"s.For $x=2$, Bessie will say "HILOLOHIHI," for a total of one "HILO".For $x=3$, Bessie will say "HILOLOHILO", for a total of two "HILO"s.SCORING:Tests 1 to 4 satisfy $N \leq 5000$.Tests 5 to 8 have a uniformly
random permutation.Tests 9 to 20 satisfy no further constraints.Problem credits: Richard Qi

\section*{Sample Input}
\begin{verbatim}
5
5 1 2 4 3
\end{verbatim}

\section*{Sample Output}
\begin{verbatim}
0
1
1
2
1
0

For $x=0$, Bessie will say "HIHI," for a total of zero "HILO"s.For $x=2$, Bessie will say "HILOLOHIHI," for a total of one "HILO".For $x=3$, Bessie will say "HILOLOHILO", for a total of two "HILO"s.SCORING:Tests 1 to 4 satisfy $N \leq 5000$.Tests 5 to 8 have a uniformly
random permutation.Tests 9 to 20 satisfy no further constraints.Problem credits: Richard Qi

For $x=2$, Bessie will say "HILOLOHIHI," for a total of one "HILO".For $x=3$, Bessie will say "HILOLOHILO", for a total of two "HILO"s.SCORING:Tests 1 to 4 satisfy $N \leq 5000$.Tests 5 to 8 have a uniformly
random permutation.Tests 9 to 20 satisfy no further constraints.Problem credits: Richard Qi

For $x=3$, Bessie will say "HILOLOHILO", for a total of two "HILO"s.SCORING:Tests 1 to 4 satisfy $N \leq 5000$.Tests 5 to 8 have a uniformly
random permutation.Tests 9 to 20 satisfy no further constraints.Problem credits: Richard Qi

SCORING:Tests 1 to 4 satisfy $N \leq 5000$.Tests 5 to 8 have a uniformly
random permutation.Tests 9 to 20 satisfy no further constraints.Problem credits: Richard Qi
\end{verbatim}

\section*{Solution}


\section*{Problem URL}
https://usaco.org/index.php?page=viewproblem2&cpid=1162

\section*{Source}
USACO DEC21 Contest, Gold Division, Problem ID: 1162

\end{document}
