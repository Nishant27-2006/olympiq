\documentclass[12pt]{article}
\usepackage{amsmath}
\usepackage{amssymb}
\usepackage{graphicx}
\usepackage{enumitem}
\usepackage{hyperref}
\usepackage{xcolor}

\title{View problem}
\author{USACO Contest: FEB19 Contest - Gold Division}
\date{\today}

\begin{document}

\maketitle

\section*{Problem Statement}

Cow Land is a special amusement park for cows, where they roam around, eat
delicious grass, and visit different cow attractions (the roller cowster is
particularly popular).

There are a total of $N$ different attractions ($2 \leq N \leq 10^5$).  Certain
pairs of attractions are connected by pathways, $N-1$ in total, in such a way
that there is a unique route consisting of various pathways between any two
attractions. Each attraction $i$ has an integer enjoyment value $e_i$, which can
change over the course of a day, since some attractions are more appealing in
the morning and others later in the afternoon.

A cow that travels from attraction $i$ to attraction $j$ gets to experience all
the attractions on the route from $i$ to $j$.  Curiously, the total enjoyment
value of this entire route is given by the bitwise XOR of all the enjoyment
values along the route, including those of attractions $i$ and $j$.  

Please help the cows determine the enjoyment values of the routes they plan to
use during their next trip to Cow Land.

INPUT FORMAT (file cowland.in):
The first line of input contains $N$ and a number of queries $Q$
($1 \leq Q \leq 10^5$). The next line contains $e_1 \ldots e_N$
($0 \leq e_i \leq 10^9$).  The next $N-1$ lines each describe a pathway in terms
of two integer attraction IDs $a$ and $b$ (both in the range $1 \ldots N$). 
Finally, the last $Q$ lines each describe either an update to one of the $e_i$
values or a query for the enjoyment of a route.  A line of the form "1 $i$ $v$"
indicates  that $e_i$ should be updated to value $v$, and a line of the form "2
$i$ $j$" is a query for the enjoyment of the route connecting attractions $i$
and $j$.

In test data worth at most 50% of points, there will be no changes to the values
of the attractions.

OUTPUT FORMAT (file cowland.out):
For each query of the form "2 $i$ $j$", print on a single line the enjoyment of
the route from $i$ to $j$.

SAMPLE INPUT:
5 5
1 2 4 8 16
1 2
1 3
3 4
3 5
2 1 5
1 1 16
2 3 5
2 1 5
2 1 3
SAMPLE OUTPUT: 
21
20
4
20


Problem credits: Charles Bailey



\section*{Input Format}
In test data worth at most 50% of points, there will be no changes to the values
of the attractions.

OUTPUT FORMAT (file cowland.out):For each query of the form "2 $i$ $j$", print on a single line the enjoyment of
the route from $i$ to $j$.SAMPLE INPUT:5 5
1 2 4 8 16
1 2
1 3
3 4
3 5
2 1 5
1 1 16
2 3 5
2 1 5
2 1 3SAMPLE OUTPUT:21
20
4
20Problem credits: Charles Bailey

\section*{Output Format}
SAMPLE INPUT:5 5
1 2 4 8 16
1 2
1 3
3 4
3 5
2 1 5
1 1 16
2 3 5
2 1 5
2 1 3SAMPLE OUTPUT:21
20
4
20Problem credits: Charles Bailey

\section*{Sample Input}
\begin{verbatim}
5 5
1 2 4 8 16
1 2
1 3
3 4
3 5
2 1 5
1 1 16
2 3 5
2 1 5
2 1 3
\end{verbatim}

\section*{Sample Output}
\begin{verbatim}
21
20
4
20

Problem credits: Charles Bailey

Problem credits: Charles Bailey
\end{verbatim}

\section*{Solution}


\section*{Problem URL}
https://usaco.org/index.php?page=viewproblem2&cpid=921

\section*{Source}
USACO FEB19 Contest, Gold Division, Problem ID: 921

\end{document}
