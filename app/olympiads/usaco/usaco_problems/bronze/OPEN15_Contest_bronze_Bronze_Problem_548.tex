\documentclass[12pt]{article}
\usepackage{amsmath}
\usepackage{amssymb}
\usepackage{graphicx}
\usepackage{enumitem}
\usepackage{hyperref}
\usepackage{xcolor}

\title{View problem}
\author{USACO Contest: OPEN15 Contest - Bronze Division}
\date{\today}

\begin{document}

\maketitle

\section*{Problem Statement}

Farmer John's farm is in the shape of an $N \times N$ grid of fields ($2 \le N \le 18$),
each labeled with a letter in the alphabet.  For example:

ABCD
BXZX
CDXB
WCBA

Each day, Bessie the cow walks from the upper-left field to the lower-right
field, each step taking her either one field to the right or one field downward.
Bessie keeps track of the string that she generates during this process, built
from the letters she walks across.  She gets very disoriented, however, if this
string is a palindrome (reading the same forward as backward), since she gets
confused about which direction she had walked.  
Please help Bessie determine the number of different palindromes she can form
during her walk.  Different ways of forming the same palindrome only count once;
for example, there are several routes that yield the palindrome ABXZXBA above,
but there are only four distinct palindromes Bessie can form, ABCDCBA, ABCWCBA,
ABXZXBA, ABXDXBA.
INPUT FORMAT (file palpath.in):
The first line of input contains $N$, and the remaining $N$ lines contain the $N$ rows
of the grid of fields.  Each row contains $N$ characters that are in the range
A..Z.

OUTPUT FORMAT (file palpath.out):
Please output the number of distinct palindromes Bessie can form.

SAMPLE INPUT:4
ABCD
BXZX
CDXB
WCBA
SAMPLE OUTPUT: 4

[Problem credits: Brian Dean, 2015]



\section*{Input Format}
OUTPUT FORMAT (file palpath.out):Please output the number of distinct palindromes Bessie can form.SAMPLE INPUT:4
ABCD
BXZX
CDXB
WCBASAMPLE OUTPUT:4[Problem credits: Brian Dean, 2015]

\section*{Output Format}
SAMPLE INPUT:4
ABCD
BXZX
CDXB
WCBASAMPLE OUTPUT:4[Problem credits: Brian Dean, 2015]

\section*{Sample Input}
\begin{verbatim}
4
ABCD
BXZX
CDXB
WCBA
\end{verbatim}

\section*{Sample Output}
\begin{verbatim}
4

[Problem credits: Brian Dean, 2015]
\end{verbatim}

\section*{Solution}


\section*{Problem URL}
https://usaco.org/index.php?page=viewproblem2&cpid=548

\section*{Source}
USACO OPEN15 Contest, Bronze Division, Problem ID: 548

\end{document}
