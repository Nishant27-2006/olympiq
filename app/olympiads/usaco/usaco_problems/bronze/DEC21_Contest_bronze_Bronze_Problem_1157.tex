\documentclass[12pt]{article}
\usepackage{amsmath}
\usepackage{amssymb}
\usepackage{graphicx}
\usepackage{enumitem}
\usepackage{hyperref}
\usepackage{xcolor}

\title{View problem}
\author{USACO Contest: DEC21 Contest - Bronze Division}
\date{\today}

\begin{document}

\maketitle

\section*{Problem Statement}

Bessie the cow is trying to walk from her favorite pasture back to her barn.

The pasture and farm are on an $N \times N$ grid ($2 \leq N \leq 50$), with her
pasture in the top-left corner and the barn in the bottom-right corner. Bessie
wants to get home as soon as possible, so she will only walk down and to the
right. There are haybales in some locations that Bessie cannot walk through; she
must walk around them.

Bessie is feeling a little tired today, so she wants to change the direction she
walks at most $K$ times ($1 \leq K \leq 3$) .

How many distinct paths can Bessie walk from her favorite pasture to the barn?
Two paths are distinct if Bessie walks in a square in one path but not in the
other.

INPUT FORMAT (input arrives from the terminal / stdin):
The input for each test case contains $T$ sub-test cases, each describing a
different farm and each of which must be  answered correctly to pass the full
test case.  The first line of input  contains $T$ ($1 \leq T \leq 50$).  Each of
the $T$ sub-test cases follow.

Each sub-test case starts with a line containing $N$ and $K$. 

The next $N$ lines each contain a string of $N$ characters. Each character is
either $\texttt{.}$ if it is empty or $\texttt{H}$ if it has a haybale. It is
guaranteed the top-left and bottom-right corners of the farm will not contain
haybales.

OUTPUT FORMAT (print output to the terminal / stdout):
Output $T$ lines, the $i$th line containing the number of distinct paths Bessie
can take in the $i$th sub-test case.

SAMPLE INPUT:
7
3 1
...
...
...
3 2
...
...
...
3 3
...
...
...
3 3
...
.H.
...
3 2
.HH
HHH
HH.
3 3
.H.
H..
...
4 3
...H
.H..
....
H...
SAMPLE OUTPUT: 
2
4
6
2
0
0
6

We'll denote Bessie's possible paths as strings of D's and R's, indicating that
Bessie moved either down or right, respectively.

In the first sub-test case, Bessie's two possible walks are DDRR and RRDD.

In the second sub-test case, Bessie's four possible walks are DDRR, DRRD, RDDR,
and RRDD.

In the third sub-test case, Bessie's six possible walks are DDRR, DRDR, DRRD,
RDDR, RDRD, and RRDD.

In the fourth sub-test case, Bessie's two possible walks are DDRR and RRDD.

In the fifth and sixth sub-test cases, it is impossible for Bessie to walk back
to the barn.

In the seventh sub-test case, Bessie's six possible walks are DDRDRR, DDRRDR,
DDRRRD, RRDDDR, RRDDRD, and RRDRDD.

SCORING:
Test case 2 satisfies $K = 1$.Test cases 3-5 satisfy $K = 2$.Test cases 6-10 satisfy $K = 3$.


Problem credits: Nick Wu



\section*{Input Format}
Each sub-test case starts with a line containing $N$ and $K$.The next $N$ lines each contain a string of $N$ characters. Each character is
either $\texttt{.}$ if it is empty or $\texttt{H}$ if it has a haybale. It is
guaranteed the top-left and bottom-right corners of the farm will not contain
haybales.

The next $N$ lines each contain a string of $N$ characters. Each character is
either $\texttt{.}$ if it is empty or $\texttt{H}$ if it has a haybale. It is
guaranteed the top-left and bottom-right corners of the farm will not contain
haybales.

OUTPUT FORMAT (print output to the terminal / stdout):Output $T$ lines, the $i$th line containing the number of distinct paths Bessie
can take in the $i$th sub-test case.SAMPLE INPUT:7
3 1
...
...
...
3 2
...
...
...
3 3
...
...
...
3 3
...
.H.
...
3 2
.HH
HHH
HH.
3 3
.H.
H..
...
4 3
...H
.H..
....
H...SAMPLE OUTPUT:2
4
6
2
0
0
6We'll denote Bessie's possible paths as strings of D's and R's, indicating that
Bessie moved either down or right, respectively.In the first sub-test case, Bessie's two possible walks are DDRR and RRDD.In the second sub-test case, Bessie's four possible walks are DDRR, DRRD, RDDR,
and RRDD.In the third sub-test case, Bessie's six possible walks are DDRR, DRDR, DRRD,
RDDR, RDRD, and RRDD.In the fourth sub-test case, Bessie's two possible walks are DDRR and RRDD.In the fifth and sixth sub-test cases, it is impossible for Bessie to walk back
to the barn.In the seventh sub-test case, Bessie's six possible walks are DDRDRR, DDRRDR,
DDRRRD, RRDDDR, RRDDRD, and RRDRDD.SCORING:Test case 2 satisfies $K = 1$.Test cases 3-5 satisfy $K = 2$.Test cases 6-10 satisfy $K = 3$.Problem credits: Nick Wu

\section*{Output Format}
SAMPLE INPUT:7
3 1
...
...
...
3 2
...
...
...
3 3
...
...
...
3 3
...
.H.
...
3 2
.HH
HHH
HH.
3 3
.H.
H..
...
4 3
...H
.H..
....
H...SAMPLE OUTPUT:2
4
6
2
0
0
6We'll denote Bessie's possible paths as strings of D's and R's, indicating that
Bessie moved either down or right, respectively.In the first sub-test case, Bessie's two possible walks are DDRR and RRDD.In the second sub-test case, Bessie's four possible walks are DDRR, DRRD, RDDR,
and RRDD.In the third sub-test case, Bessie's six possible walks are DDRR, DRDR, DRRD,
RDDR, RDRD, and RRDD.In the fourth sub-test case, Bessie's two possible walks are DDRR and RRDD.In the fifth and sixth sub-test cases, it is impossible for Bessie to walk back
to the barn.In the seventh sub-test case, Bessie's six possible walks are DDRDRR, DDRRDR,
DDRRRD, RRDDDR, RRDDRD, and RRDRDD.SCORING:Test case 2 satisfies $K = 1$.Test cases 3-5 satisfy $K = 2$.Test cases 6-10 satisfy $K = 3$.Problem credits: Nick Wu

\section*{Sample Input}
\begin{verbatim}
7
3 1
...
...
...
3 2
...
...
...
3 3
...
...
...
3 3
...
.H.
...
3 2
.HH
HHH
HH.
3 3
.H.
H..
...
4 3
...H
.H..
....
H...
\end{verbatim}

\section*{Sample Output}
\begin{verbatim}
2
4
6
2
0
0
6

We'll denote Bessie's possible paths as strings of D's and R's, indicating that
Bessie moved either down or right, respectively.In the first sub-test case, Bessie's two possible walks are DDRR and RRDD.In the second sub-test case, Bessie's four possible walks are DDRR, DRRD, RDDR,
and RRDD.In the third sub-test case, Bessie's six possible walks are DDRR, DRDR, DRRD,
RDDR, RDRD, and RRDD.In the fourth sub-test case, Bessie's two possible walks are DDRR and RRDD.In the fifth and sixth sub-test cases, it is impossible for Bessie to walk back
to the barn.In the seventh sub-test case, Bessie's six possible walks are DDRDRR, DDRRDR,
DDRRRD, RRDDDR, RRDDRD, and RRDRDD.SCORING:Test case 2 satisfies $K = 1$.Test cases 3-5 satisfy $K = 2$.Test cases 6-10 satisfy $K = 3$.Problem credits: Nick Wu

In the first sub-test case, Bessie's two possible walks are DDRR and RRDD.In the second sub-test case, Bessie's four possible walks are DDRR, DRRD, RDDR,
and RRDD.In the third sub-test case, Bessie's six possible walks are DDRR, DRDR, DRRD,
RDDR, RDRD, and RRDD.In the fourth sub-test case, Bessie's two possible walks are DDRR and RRDD.In the fifth and sixth sub-test cases, it is impossible for Bessie to walk back
to the barn.In the seventh sub-test case, Bessie's six possible walks are DDRDRR, DDRRDR,
DDRRRD, RRDDDR, RRDDRD, and RRDRDD.SCORING:Test case 2 satisfies $K = 1$.Test cases 3-5 satisfy $K = 2$.Test cases 6-10 satisfy $K = 3$.Problem credits: Nick Wu

In the second sub-test case, Bessie's four possible walks are DDRR, DRRD, RDDR,
and RRDD.In the third sub-test case, Bessie's six possible walks are DDRR, DRDR, DRRD,
RDDR, RDRD, and RRDD.In the fourth sub-test case, Bessie's two possible walks are DDRR and RRDD.In the fifth and sixth sub-test cases, it is impossible for Bessie to walk back
to the barn.In the seventh sub-test case, Bessie's six possible walks are DDRDRR, DDRRDR,
DDRRRD, RRDDDR, RRDDRD, and RRDRDD.SCORING:Test case 2 satisfies $K = 1$.Test cases 3-5 satisfy $K = 2$.Test cases 6-10 satisfy $K = 3$.Problem credits: Nick Wu

In the third sub-test case, Bessie's six possible walks are DDRR, DRDR, DRRD,
RDDR, RDRD, and RRDD.In the fourth sub-test case, Bessie's two possible walks are DDRR and RRDD.In the fifth and sixth sub-test cases, it is impossible for Bessie to walk back
to the barn.In the seventh sub-test case, Bessie's six possible walks are DDRDRR, DDRRDR,
DDRRRD, RRDDDR, RRDDRD, and RRDRDD.SCORING:Test case 2 satisfies $K = 1$.Test cases 3-5 satisfy $K = 2$.Test cases 6-10 satisfy $K = 3$.Problem credits: Nick Wu

In the fourth sub-test case, Bessie's two possible walks are DDRR and RRDD.In the fifth and sixth sub-test cases, it is impossible for Bessie to walk back
to the barn.In the seventh sub-test case, Bessie's six possible walks are DDRDRR, DDRRDR,
DDRRRD, RRDDDR, RRDDRD, and RRDRDD.SCORING:Test case 2 satisfies $K = 1$.Test cases 3-5 satisfy $K = 2$.Test cases 6-10 satisfy $K = 3$.Problem credits: Nick Wu

In the fifth and sixth sub-test cases, it is impossible for Bessie to walk back
to the barn.In the seventh sub-test case, Bessie's six possible walks are DDRDRR, DDRRDR,
DDRRRD, RRDDDR, RRDDRD, and RRDRDD.SCORING:Test case 2 satisfies $K = 1$.Test cases 3-5 satisfy $K = 2$.Test cases 6-10 satisfy $K = 3$.Problem credits: Nick Wu

In the seventh sub-test case, Bessie's six possible walks are DDRDRR, DDRRDR,
DDRRRD, RRDDDR, RRDDRD, and RRDRDD.SCORING:Test case 2 satisfies $K = 1$.Test cases 3-5 satisfy $K = 2$.Test cases 6-10 satisfy $K = 3$.Problem credits: Nick Wu

SCORING:Test case 2 satisfies $K = 1$.Test cases 3-5 satisfy $K = 2$.Test cases 6-10 satisfy $K = 3$.Problem credits: Nick Wu
\end{verbatim}

\section*{Solution}


\section*{Problem URL}
https://usaco.org/index.php?page=viewproblem2&cpid=1157

\section*{Source}
USACO DEC21 Contest, Bronze Division, Problem ID: 1157

\end{document}
