\documentclass[12pt]{article}
\usepackage{amsmath}
\usepackage{amssymb}
\usepackage{graphicx}
\usepackage{enumitem}
\usepackage{hyperref}
\usepackage{xcolor}

\title{View problem}
\author{USACO Contest: DEC24 Contest - Bronze Division}
\date{\today}

\begin{document}

\maketitle

\section*{Problem Statement}


Farmer John has a block of cheese in the shape of a cube. It lies on the
3-dimensional coordinate plane, extending from $(0,0,0)$ to $(N, N, N)$
($2 \leq N \leq 1000$). Farmer John will perform a series of $Q$
($1 \leq Q \leq 2 \cdot 10^5$) update operations to his cheese block.

For each update operation, FJ will carve out the $1$ by $1$ by $1$ block of
cheese extending from integer coordinates $(x, y, z)$ to $(x+1, y+1, z+1)$,
where $0\le x,y,z<N$. It is guaranteed that there will exist a $1$ by $1$ by $1$
block of cheese at the location FJ carves.  Since FJ is playing Moocraft,
gravity does not cause parts of the cheese to fall if cheese below is carved.

After each update, output the number of distinct configurations that FJ can
stick a $1$ by $1$ by $N$ brick in the cheese block such that no part of the
brick overlaps with any remaining cheese. Every vertex of the brick must have
integer coordinates in the range $[0,N]$ for all three axes. FJ may rotate the
brick however he wants.

INPUT FORMAT (input arrives from the terminal / stdin):
The first line contains $N$ and $Q$.

The following $Q$ lines contain $x$, $y$, and $z$, the coordinates to be carved.

OUTPUT FORMAT (print output to the terminal / stdout):
After each update operation, output an integer, the number of configurations.


SAMPLE INPUT:
2 5
0 0 0
1 1 1
0 1 0
1 0 0
1 1 0
SAMPLE OUTPUT: 
0
0
1
2
5

After the first three updates, the $1\times 2 \times 1$ brick spanning 
$[0, 1]\times [0, 2]\times [0, 1]$ does not overlap with the remaining cheese,
so it contributes toward the answer.


SCORING:
Inputs 2-4: $N\le 10$ and $Q \le 1000$Inputs 5-7:
$N\le 100$ and $Q \le 1000$Inputs 8-16: No additional
constraints


Problem credits: Chongtian Ma, Alex Liang



\section*{Input Format}
The first line contains $N$ and $Q$.The following $Q$ lines contain $x$, $y$, and $z$, the coordinates to be carved.

The following $Q$ lines contain $x$, $y$, and $z$, the coordinates to be carved.

OUTPUT FORMAT (print output to the terminal / stdout):After each update operation, output an integer, the number of configurations.SAMPLE INPUT:2 5
0 0 0
1 1 1
0 1 0
1 0 0
1 1 0SAMPLE OUTPUT:0
0
1
2
5After the first three updates, the $1\times 2 \times 1$ brick spanning 
$[0, 1]\times [0, 2]\times [0, 1]$ does not overlap with the remaining cheese,
so it contributes toward the answer.SCORING:Inputs 2-4: $N\le 10$ and $Q \le 1000$Inputs 5-7:
$N\le 100$ and $Q \le 1000$Inputs 8-16: No additional
constraintsProblem credits: Chongtian Ma, Alex Liang

\section*{Output Format}
After each update operation, output an integer, the number of configurations.

SAMPLE INPUT:2 5
0 0 0
1 1 1
0 1 0
1 0 0
1 1 0SAMPLE OUTPUT:0
0
1
2
5After the first three updates, the $1\times 2 \times 1$ brick spanning 
$[0, 1]\times [0, 2]\times [0, 1]$ does not overlap with the remaining cheese,
so it contributes toward the answer.SCORING:Inputs 2-4: $N\le 10$ and $Q \le 1000$Inputs 5-7:
$N\le 100$ and $Q \le 1000$Inputs 8-16: No additional
constraintsProblem credits: Chongtian Ma, Alex Liang

\section*{Sample Input}
\begin{verbatim}
2 5
0 0 0
1 1 1
0 1 0
1 0 0
1 1 0
\end{verbatim}

\section*{Sample Output}
\begin{verbatim}
0
0
1
2
5

SCORING:Inputs 2-4: $N\le 10$ and $Q \le 1000$Inputs 5-7:
$N\le 100$ and $Q \le 1000$Inputs 8-16: No additional
constraintsProblem credits: Chongtian Ma, Alex Liang

SCORING:Inputs 2-4: $N\le 10$ and $Q \le 1000$Inputs 5-7:
$N\le 100$ and $Q \le 1000$Inputs 8-16: No additional
constraintsProblem credits: Chongtian Ma, Alex Liang
\end{verbatim}

\section*{Solution}


\section*{Problem URL}
https://usaco.org/index.php?page=viewproblem2&cpid=1444

\section*{Source}
USACO DEC24 Contest, Bronze Division, Problem ID: 1444

\end{document}
