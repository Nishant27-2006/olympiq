\documentclass[12pt]{article}
\usepackage{amsmath}
\usepackage{amssymb}
\usepackage{graphicx}
\usepackage{enumitem}
\usepackage{hyperref}
\usepackage{xcolor}

\title{View problem}
\author{USACO Contest: DEC22 Contest - Bronze Division}
\date{\today}

\begin{document}

\maketitle

\section*{Problem Statement}

Farmer John has $N$ ($1 \le {N} \le {10^5}$) cows, the breed of each being
either a Guernsey or a Holstein. They have lined up horizontally with the cows occupying positions 
labeled from
$1\dots N$.

Since all the cows are hungry, FJ decides to plant grassy patches on some
of the positions $1\dots N$. Guernseys and Holsteins prefer different types of
grass, so if Farmer John decides to plant grass at some location, he must choose
to planting either Guernsey-preferred grass or Holstein-preferred grass --- he
cannot plant both at the same location. Each patch of grass planted can feed an
unlimited number of cows of the appropriate breed. 

Each cow is willing to move a maximum of $K$ ($0 \le {K} \le N-1$) positions to
reach a patch. Find the minimum number of patches needed to feed all the cows.
Also, print a configuration of patches that uses the minimum amount of patches
needed to feed the cows. Any configuration that satisfies the above conditions
will be considered correct.

INPUT FORMAT (input arrives from the terminal / stdin):
Each input contains $T$ test cases, each describing an arrangement of cows. The
first line of input contains $T$ ($1 \le T \le 10$). Each of the $T$ test cases
follow.

Each test case starts with a line containing $N$ and $K$. The next line will
contain a string of length $N$, where each character denotes the breed of the
$i$th cow (G meaning Guernsey and H meaning Holstein).

OUTPUT FORMAT (print output to the terminal / stdout):
For each of the $T$ test cases, please write two lines of output. For the first
line, print the minimum number of patches needed to feed the cows. For the
second line, print a string of length $N$ that describes a configuration  that
feeds all the cows with the minimum number of patches. The $i$th character,
describing the $i$th position, should be a '.' if there is no patch, a 'G' if
there is a patch that feeds Guernseys, and a 'H' if it feeds Holsteins. Any
valid configuration will be accepted.

SAMPLE INPUT:
6
5 0
GHHGG
5 1
GHHGG
5 2
GHHGG
5 3
GHHGG
5 4
GHHGG
2 1
GH
SAMPLE OUTPUT: 
5
GHHGG
3
.GH.G
2
..GH.
2
...GH
2
...HG
2
HG

Note that for some test cases, there are multiple acceptable configurations that
manage to feed all cows while using the minimum number of patches. For example,
in the fourth test case, another acceptable answer would be:


.GH..

This corresponds to placing a patch feeding Guernseys on the 2nd position and a
patch feeding Holsteins on the third position. This uses the optimal number of
patches and ensures that all cows are within 3 positions of a patch they prefer.



SCORING:
Inputs 2 through 4 have $N \le 10$. Inputs 5 through 8 have $N \le 40$. Inputs 9 through 12 have $N \le 10^5$. 


Problem credits: Mythreya Dharani



\section*{Input Format}
Each test case starts with a line containing $N$ and $K$. The next line will
contain a string of length $N$, where each character denotes the breed of the
$i$th cow (G meaning Guernsey and H meaning Holstein).

OUTPUT FORMAT (print output to the terminal / stdout):For each of the $T$ test cases, please write two lines of output. For the first
line, print the minimum number of patches needed to feed the cows. For the
second line, print a string of length $N$ that describes a configuration  that
feeds all the cows with the minimum number of patches. The $i$th character,
describing the $i$th position, should be a '.' if there is no patch, a 'G' if
there is a patch that feeds Guernseys, and a 'H' if it feeds Holsteins. Any
valid configuration will be accepted.SAMPLE INPUT:6
5 0
GHHGG
5 1
GHHGG
5 2
GHHGG
5 3
GHHGG
5 4
GHHGG
2 1
GHSAMPLE OUTPUT:5
GHHGG
3
.GH.G
2
..GH.
2
...GH
2
...HG
2
HGNote that for some test cases, there are multiple acceptable configurations that
manage to feed all cows while using the minimum number of patches. For example,
in the fourth test case, another acceptable answer would be:.GH..This corresponds to placing a patch feeding Guernseys on the 2nd position and a
patch feeding Holsteins on the third position. This uses the optimal number of
patches and ensures that all cows are within 3 positions of a patch they prefer.SCORING:Inputs 2 through 4 have $N \le 10$.Inputs 5 through 8 have $N \le 40$.Inputs 9 through 12 have $N \le 10^5$.Problem credits: Mythreya Dharani

\section*{Output Format}
SAMPLE INPUT:6
5 0
GHHGG
5 1
GHHGG
5 2
GHHGG
5 3
GHHGG
5 4
GHHGG
2 1
GHSAMPLE OUTPUT:5
GHHGG
3
.GH.G
2
..GH.
2
...GH
2
...HG
2
HGNote that for some test cases, there are multiple acceptable configurations that
manage to feed all cows while using the minimum number of patches. For example,
in the fourth test case, another acceptable answer would be:.GH..This corresponds to placing a patch feeding Guernseys on the 2nd position and a
patch feeding Holsteins on the third position. This uses the optimal number of
patches and ensures that all cows are within 3 positions of a patch they prefer.SCORING:Inputs 2 through 4 have $N \le 10$.Inputs 5 through 8 have $N \le 40$.Inputs 9 through 12 have $N \le 10^5$.Problem credits: Mythreya Dharani

\section*{Sample Input}
\begin{verbatim}
6
5 0
GHHGG
5 1
GHHGG
5 2
GHHGG
5 3
GHHGG
5 4
GHHGG
2 1
GH
\end{verbatim}

\section*{Sample Output}
\begin{verbatim}
5
GHHGG
3
.GH.G
2
..GH.
2
...GH
2
...HG
2
HG

Note that for some test cases, there are multiple acceptable configurations that
manage to feed all cows while using the minimum number of patches. For example,
in the fourth test case, another acceptable answer would be:.GH..This corresponds to placing a patch feeding Guernseys on the 2nd position and a
patch feeding Holsteins on the third position. This uses the optimal number of
patches and ensures that all cows are within 3 positions of a patch they prefer.SCORING:Inputs 2 through 4 have $N \le 10$.Inputs 5 through 8 have $N \le 40$.Inputs 9 through 12 have $N \le 10^5$.Problem credits: Mythreya Dharani

.GH..This corresponds to placing a patch feeding Guernseys on the 2nd position and a
patch feeding Holsteins on the third position. This uses the optimal number of
patches and ensures that all cows are within 3 positions of a patch they prefer.SCORING:Inputs 2 through 4 have $N \le 10$.Inputs 5 through 8 have $N \le 40$.Inputs 9 through 12 have $N \le 10^5$.Problem credits: Mythreya Dharani

.GH..

This corresponds to placing a patch feeding Guernseys on the 2nd position and a
patch feeding Holsteins on the third position. This uses the optimal number of
patches and ensures that all cows are within 3 positions of a patch they prefer.SCORING:Inputs 2 through 4 have $N \le 10$.Inputs 5 through 8 have $N \le 40$.Inputs 9 through 12 have $N \le 10^5$.Problem credits: Mythreya Dharani

SCORING:Inputs 2 through 4 have $N \le 10$.Inputs 5 through 8 have $N \le 40$.Inputs 9 through 12 have $N \le 10^5$.Problem credits: Mythreya Dharani

SCORING:Inputs 2 through 4 have $N \le 10$.Inputs 5 through 8 have $N \le 40$.Inputs 9 through 12 have $N \le 10^5$.Problem credits: Mythreya Dharani

SCORING:Inputs 2 through 4 have $N \le 10$.Inputs 5 through 8 have $N \le 40$.Inputs 9 through 12 have $N \le 10^5$.Problem credits: Mythreya Dharani
\end{verbatim}

\section*{Solution}


\section*{Problem URL}
https://usaco.org/index.php?page=viewproblem2&cpid=1252

\section*{Source}
USACO DEC22 Contest, Bronze Division, Problem ID: 1252

\end{document}
