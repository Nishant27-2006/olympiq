\documentclass[12pt]{article}
\usepackage{amsmath}
\usepackage{amssymb}
\usepackage{graphicx}
\usepackage{enumitem}
\usepackage{hyperref}
\usepackage{xcolor}

\title{View problem}
\author{USACO Contest: JAN16 Contest - Bronze Division}
\date{\today}

\begin{document}

\maketitle

\section*{Problem Statement}

Bessie the cow is helping Farmer John run the USA Cow Olympiad (USACO), an
on-line contest where participants answer challenging questions to demonstrate
their mastery of bovine trivia.

In response to a wider range of participant backgrounds, Farmer John recently
expanded the contest to include four divisions of difficulty: bronze, silver,
gold, and platinum.  All new participants start in the bronze division, and any
time they score perfectly on a contest they are promoted to the next-higher
division.  It is even possible for a participant to be promoted several times
within the same contest. Farmer John keeps track of a list of all contest
participants and their current divisions, so that he can start everyone out at
the right level any time he holds a contest.

When publishing the results from his most recent contest, Farmer John wants to
include information on the number of participants who were promoted from bronze
to silver, from silver to gold, and from gold to platinum.  However, he
neglected to count promotions as they occurred during the contest.  Bessie,
being the clever bovine she is, realizes however that Farmer John can deduce the
number of promotions that occurred solely from the number of participants at
each level before and after the contest.  Please help her perform this
computation!

INPUT FORMAT (file promote.in):
Input consists of four lines, each containing two integers in the range
0..1,000,000. The first line specifies the number of bronze participants
registered before and after the contest. The second line specifies the number of
silver participants before and after the contest. The third line specifies the
number of gold participants before and after the contest. The last line
specifies the number of platinum participants before and after the contest.

OUTPUT FORMAT (file promote.out):
Please output three lines, each containing a single integer. The first line
should contain the number of participants who were promoted from bronze to
silver. The second line should contain the number of participants who were
promoted from silver to gold. The last line should contain the number of
participants who were promoted from gold to platinum.

SAMPLE INPUT:
1 2
1 1
1 1
1 2
SAMPLE OUTPUT: 
1
1
1

In this example, 1 participant was registered in each division prior to the
contest.  At the end of the contest, 2 participants were registered in bronze
and platinum.  One way this could have happened is that 2 new participants
joined during the contest; one was promoted all the way to platinum, and the
other stayed in bronze.

Problem credits: Brian Dean




\section*{Input Format}
OUTPUT FORMAT (file promote.out):Please output three lines, each containing a single integer. The first line
should contain the number of participants who were promoted from bronze to
silver. The second line should contain the number of participants who were
promoted from silver to gold. The last line should contain the number of
participants who were promoted from gold to platinum.SAMPLE INPUT:1 2
1 1
1 1
1 2SAMPLE OUTPUT:1
1
1In this example, 1 participant was registered in each division prior to the
contest.  At the end of the contest, 2 participants were registered in bronze
and platinum.  One way this could have happened is that 2 new participants
joined during the contest; one was promoted all the way to platinum, and the
other stayed in bronze.Problem credits: Brian Dean

\section*{Output Format}
SAMPLE INPUT:1 2
1 1
1 1
1 2SAMPLE OUTPUT:1
1
1In this example, 1 participant was registered in each division prior to the
contest.  At the end of the contest, 2 participants were registered in bronze
and platinum.  One way this could have happened is that 2 new participants
joined during the contest; one was promoted all the way to platinum, and the
other stayed in bronze.Problem credits: Brian Dean

\section*{Sample Input}
\begin{verbatim}
1 2
1 1
1 1
1 2
\end{verbatim}

\section*{Sample Output}
\begin{verbatim}
1
1
1

In this example, 1 participant was registered in each division prior to the
contest.  At the end of the contest, 2 participants were registered in bronze
and platinum.  One way this could have happened is that 2 new participants
joined during the contest; one was promoted all the way to platinum, and the
other stayed in bronze.Problem credits: Brian Dean

Problem credits: Brian Dean
\end{verbatim}

\section*{Solution}


\section*{Problem URL}
https://usaco.org/index.php?page=viewproblem2&cpid=591

\section*{Source}
USACO JAN16 Contest, Bronze Division, Problem ID: 591

\end{document}
