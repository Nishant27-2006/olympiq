\documentclass[12pt]{article}
\usepackage{amsmath}
\usepackage{amssymb}
\usepackage{graphicx}
\usepackage{enumitem}
\usepackage{hyperref}
\usepackage{xcolor}

\title{View problem}
\author{USACO Contest: FEB17 Contest - Bronze Division}
\date{\today}

\begin{document}

\maketitle

\section*{Problem Statement}

The layout of Farmer John's farm is quite peculiar, with a large circular road
running around the perimeter of the main field on which his cows graze during
the day. Every morning, the cows cross this road on their way towards the field,
and every evening they all cross again as they leave the field and return to the
barn.

As we know, cows are creatures of habit, and they each cross the road the same
way every day.  Each cow crosses into the field at a different point from where
she crosses out of the field, and all of these crossing points are distinct from
each-other. Farmer John owns exactly 26 cows, which he has lazily named A
through Z (he is not sure what he will do if he ever acquires a  27th cow...),
so there are precisely 52 crossing points around the road.  Farmer John records
these crossing points concisely by scanning around the circle clockwise, writing
down the name of the cow for each crossing point, ultimately forming a string
with 52 characters in which each letter of the alphabet appears exactly twice. 
He does not record which crossing points are entry points and which are
exit points.

Looking at his map of crossing points, Farmer John is curious how many times
various pairs of cows might cross paths during the day.  He calls a pair of cows
$(a,b)$ a "crossing" pair if cow $a$'s path from entry to exit must cross cow
$b$'s path from entry to exit.  Please help Farmer John count the total number
of crossing pairs.

INPUT FORMAT (file circlecross.in):
The input consists of a single line containing a string of 52 upper-case
characters.  Each letter of the alphabet appears exactly twice.

OUTPUT FORMAT (file circlecross.out):
Please print the total number of crossing pairs.

SAMPLE INPUT:
ABCCABDDEEFFGGHHIIJJKKLLMMNNOOPPQQRRSSTTUUVVWWXXYYZZ
SAMPLE OUTPUT: 
1

In this example, only cows A and B are a crossing pair.


Problem credits: Brian Dean



\section*{Input Format}
OUTPUT FORMAT (file circlecross.out):Please print the total number of crossing pairs.SAMPLE INPUT:ABCCABDDEEFFGGHHIIJJKKLLMMNNOOPPQQRRSSTTUUVVWWXXYYZZSAMPLE OUTPUT:1In this example, only cows A and B are a crossing pair.Problem credits: Brian Dean

\section*{Output Format}
SAMPLE INPUT:ABCCABDDEEFFGGHHIIJJKKLLMMNNOOPPQQRRSSTTUUVVWWXXYYZZSAMPLE OUTPUT:1In this example, only cows A and B are a crossing pair.Problem credits: Brian Dean

\section*{Sample Input}
\begin{verbatim}
ABCCABDDEEFFGGHHIIJJKKLLMMNNOOPPQQRRSSTTUUVVWWXXYYZZ
\end{verbatim}

\section*{Sample Output}
\begin{verbatim}
1

In this example, only cows A and B are a crossing pair.Problem credits: Brian Dean

Problem credits: Brian Dean

Problem credits: Brian Dean
\end{verbatim}

\section*{Solution}


\section*{Problem URL}
https://usaco.org/index.php?page=viewproblem2&cpid=712

\section*{Source}
USACO FEB17 Contest, Bronze Division, Problem ID: 712

\end{document}
