\documentclass[12pt]{article}
\usepackage{amsmath}
\usepackage{amssymb}
\usepackage{graphicx}
\usepackage{enumitem}
\usepackage{hyperref}
\usepackage{xcolor}

\title{View problem}
\author{USACO Contest: DEC16 Contest - Bronze Division}
\date{\today}

\begin{document}

\maketitle

\section*{Problem Statement}

Farmer John is trying to teach his cows to read by giving them a set of $N$
spelling boards typically used with preschoolers ($1 \leq N \leq 100$). Each
board has a word and an image on each side.  For example, one side might have
the word 'cat' along with a picture of a cat, and the other side might have the
word 'dog' along with a picture of a dog.  When the boards are lying on the
ground, $N$ words are therefore shown.  By flipping over some of the boards, a
different set of $N$ words can be exposed.

To help the cows with their spelling, Farmer John wants to fashion a number of
wooden blocks, each embossed with a single letter of the alphabet. He wants to
make sufficiently many blocks of each letter so that no matter which set of $N$
words is exposed on the upward-facing boards, the cows will be able to spell all
of these words using the blocks.  For example, if $N=3$ and the words 'box',
'cat', and 'car' were facing upward, the cows would need at least one 'b' block,
one 'o' block, one 'x' block, two 'c' blocks, two 'a' blocks, one 't' block, and
one 'r' block.

Please help the Farmer John determine the minimum number of blocks for each
letter of the alphabet that he needs to provide, so that irrespective of which
face of each board is showing, the cows can spell all $N$ visible words.

INPUT FORMAT (file blocks.in):
Line 1 contains the integer $N$. 

The next $N$ lines each contain 2 words separated by a space, giving the two
words on opposite sides of a board.  Each word is a string of at most 10
lowercase letters.

OUTPUT FORMAT (file blocks.out):
Please output 26 lines. The first output line should contain a number specifying
the number of copies of 'a' blocks needed.  The next line should specify the
number of 'b' blocks needed, and so on.

SAMPLE INPUT:
3
fox box
dog cat
car bus
SAMPLE OUTPUT: 
2
2
2
1
0
1
1
0
0
0
0
0
0
0
2
0
0
1
1
1
1
0
0
1
0
0

In this example, there are $N = 3$ boards, giving $2^3 = 8$ possibilities for
the set of upward-facing words:

fox dog car
fox dog bus
fox cat car
fox cat bus
box dog car
box dog bus
box cat car
box cat bus

We need enough blocks for each letter of the alphabet so that we can spell all
three words, irrespective of which of these eight scenarios occurs.


Problem credits: Viktoriia Schwartz



\section*{Input Format}
The next $N$ lines each contain 2 words separated by a space, giving the two
words on opposite sides of a board.  Each word is a string of at most 10
lowercase letters.

OUTPUT FORMAT (file blocks.out):Please output 26 lines. The first output line should contain a number specifying
the number of copies of 'a' blocks needed.  The next line should specify the
number of 'b' blocks needed, and so on.SAMPLE INPUT:3
fox box
dog cat
car busSAMPLE OUTPUT:2
2
2
1
0
1
1
0
0
0
0
0
0
0
2
0
0
1
1
1
1
0
0
1
0
0In this example, there are $N = 3$ boards, giving $2^3 = 8$ possibilities for
the set of upward-facing words:fox dog car
fox dog bus
fox cat car
fox cat bus
box dog car
box dog bus
box cat car
box cat busWe need enough blocks for each letter of the alphabet so that we can spell all
three words, irrespective of which of these eight scenarios occurs.Problem credits: Viktoriia Schwartz

\section*{Output Format}
SAMPLE INPUT:3
fox box
dog cat
car busSAMPLE OUTPUT:2
2
2
1
0
1
1
0
0
0
0
0
0
0
2
0
0
1
1
1
1
0
0
1
0
0In this example, there are $N = 3$ boards, giving $2^3 = 8$ possibilities for
the set of upward-facing words:fox dog car
fox dog bus
fox cat car
fox cat bus
box dog car
box dog bus
box cat car
box cat busWe need enough blocks for each letter of the alphabet so that we can spell all
three words, irrespective of which of these eight scenarios occurs.Problem credits: Viktoriia Schwartz

\section*{Sample Input}
\begin{verbatim}
3
fox box
dog cat
car bus
\end{verbatim}

\section*{Sample Output}
\begin{verbatim}
2
2
2
1
0
1
1
0
0
0
0
0
0
0
2
0
0
1
1
1
1
0
0
1
0
0

In this example, there are $N = 3$ boards, giving $2^3 = 8$ possibilities for
the set of upward-facing words:fox dog car
fox dog bus
fox cat car
fox cat bus
box dog car
box dog bus
box cat car
box cat busWe need enough blocks for each letter of the alphabet so that we can spell all
three words, irrespective of which of these eight scenarios occurs.Problem credits: Viktoriia Schwartz

fox dog car
fox dog bus
fox cat car
fox cat bus
box dog car
box dog bus
box cat car
box cat bus

Problem credits: Viktoriia Schwartz

Problem credits: Viktoriia Schwartz
\end{verbatim}

\section*{Solution}


\section*{Problem URL}
https://usaco.org/index.php?page=viewproblem2&cpid=664

\section*{Source}
USACO DEC16 Contest, Bronze Division, Problem ID: 664

\end{document}
