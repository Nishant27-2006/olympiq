\documentclass[12pt]{article}
\usepackage{amsmath}
\usepackage{amssymb}
\usepackage{graphicx}
\usepackage{enumitem}
\usepackage{hyperref}
\usepackage{xcolor}

\title{View problem}
\author{USACO Contest: DEC15 Contest - Bronze Division}
\date{\today}

\begin{document}

\maketitle

\section*{Problem Statement}

Several seasons of hot summers and cold winters have taken their toll on Farmer
John's fence, and he decides it is time to repaint it, along with the help of
his favorite cow, Bessie.  Unfortunately, while Bessie is actually remarkably
proficient at painting, she is not as good at understanding Farmer John's
instructions.

If we regard the fence as a one-dimensional number line, Farmer
John paints the interval between $x=a$ and $x=b$. For example, if
$a=3$ and $b=5$, then Farmer John paints an interval of length 2.
Bessie, misunderstanding Farmer John's instructions, paints the
interval from $x=c$ to $x=d$, which may possibly overlap with part or
all of Farmer John's interval.  Please determine the total length of
fence that is now covered with paint.  

INPUT FORMAT (file paint.in):

The first line of the input contains the integers $a$ and $b$, separated by a
space ($a < b$). 

The second line contains integers $c$ and $d$, separated by a space ($c < d$). 

The values of $a$, $b$, $c$, and $d$ all lie in the range $0 \ldots 100$, inclusive.


OUTPUT FORMAT (file paint.out):

Please output a single line containing the total length of the fence covered with paint.


SAMPLE INPUT:
7 10
4 8

SAMPLE OUTPUT: 
6

Here, 6 total units of fence are covered with paint, from $x=4$ all
the way through $x=10$.

Problem credits: Brian Dean



\section*{Input Format}
The second line contains integers $c$ and $d$, separated by a space ($c < d$).The values of $a$, $b$, $c$, and $d$ all lie in the range $0 \ldots 100$, inclusive.

The values of $a$, $b$, $c$, and $d$ all lie in the range $0 \ldots 100$, inclusive.

OUTPUT FORMAT (file paint.out):Please output a single line containing the total length of the fence covered with paint.SAMPLE INPUT:7 10
4 8SAMPLE OUTPUT:6Here, 6 total units of fence are covered with paint, from $x=4$ all
the way through $x=10$.Problem credits: Brian Dean

\section*{Output Format}
SAMPLE INPUT:7 10
4 8SAMPLE OUTPUT:6Here, 6 total units of fence are covered with paint, from $x=4$ all
the way through $x=10$.Problem credits: Brian Dean

\section*{Sample Input}
\begin{verbatim}
7 10
4 8
\end{verbatim}

\section*{Sample Output}
\begin{verbatim}
6

Here, 6 total units of fence are covered with paint, from $x=4$ all
the way through $x=10$.Problem credits: Brian Dean

Problem credits: Brian Dean
\end{verbatim}

\section*{Solution}


\section*{Problem URL}
https://usaco.org/index.php?page=viewproblem2&cpid=567

\section*{Source}
USACO DEC15 Contest, Bronze Division, Problem ID: 567

\end{document}
