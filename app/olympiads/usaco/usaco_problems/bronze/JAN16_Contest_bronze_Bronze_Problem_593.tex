\documentclass[12pt]{article}
\usepackage{amsmath}
\usepackage{amssymb}
\usepackage{graphicx}
\usepackage{enumitem}
\usepackage{hyperref}
\usepackage{xcolor}

\title{View problem}
\author{USACO Contest: JAN16 Contest - Bronze Division}
\date{\today}

\begin{document}

\maketitle

\section*{Problem Statement}

Farmer John is quite reliable in all aspects of managing his farm, except one:
he is terrible at mowing the grass in a timely or logical fashion.

The farm is a large 2D grid of square unit cells.  FJ starts in one of these
cells at time $t = 0$, mowing the grass in this cell so that it is initially the
only cell in which the grass is cut.  FJ's remaining mowing pattern is 
described by a sequence of $N$ statements.  For example, if the first  statement
is "W 10", then for times $t = 1$ through $t = 10$ (i.e., the next 10 units of
time), FJ will step one cell to his west, mowing the  grass along the way. 
After completing this sequence of steps, he will end up 10 cells to his west at
time $t = 10$, having mowed the grass in every cell along the way. 

So slow is FJ's progress that some of the grass he mows might grow back before
he is finished with all his mowing. Any section of grass that is cut at time $t$
will reappear at time $t + x$. 

FJ's mowing pattern might have him re-visit the same cell multiple times, but he
remarks that he never encounters a cell in which the grass is already cut.  That
is, every time he visits a cell, his most recent visit to that same cell must
have been at least $x$ units of time earlier, in order for the grass to have
grown back.

Please determine the maximum possible value of $x$ so that FJ's observation
remains valid.

INPUT FORMAT (file mowing.in):
The first line of input contains $N$ ($1 \leq N \leq 100$). Each of the
remaining $N$ lines contains a single statement and is of the form 'D S', where
D is a character describing a direction (N=north, E=east, S=south, W=west) and S
is the number of steps taken in that direction ($1 \leq S \leq 10$).  

OUTPUT FORMAT (file mowing.out):
Please determine the maximum value of $x$ such that FJ never steps on a cell
with cut grass.  If FJ never visits any cell more than once, please
output -1.

SAMPLE INPUT:
6
N 10
E 2
S 3
W 4
S 5
E 8
SAMPLE OUTPUT: 
10

In this example, FJ steps on a cell at time 17 that he stepped on earlier at
time 7; therefore, $x$ must be at most 10 or else the grass from his first visit
would not yet have grown back.  He also steps on a cell at time 26 that he also
visited at time 2; hence $x$ must also be at most 24.  Since the first of these
two constraints is tighter, we see that $x$ can be at most 10.

Problem credits: Brian Dean



\section*{Input Format}
OUTPUT FORMAT (file mowing.out):Please determine the maximum value of $x$ such that FJ never steps on a cell
with cut grass.  If FJ never visits any cell more than once, please
output -1.SAMPLE INPUT:6
N 10
E 2
S 3
W 4
S 5
E 8SAMPLE OUTPUT:10In this example, FJ steps on a cell at time 17 that he stepped on earlier at
time 7; therefore, $x$ must be at most 10 or else the grass from his first visit
would not yet have grown back.  He also steps on a cell at time 26 that he also
visited at time 2; hence $x$ must also be at most 24.  Since the first of these
two constraints is tighter, we see that $x$ can be at most 10.Problem credits: Brian Dean

\section*{Output Format}
SAMPLE INPUT:6
N 10
E 2
S 3
W 4
S 5
E 8SAMPLE OUTPUT:10In this example, FJ steps on a cell at time 17 that he stepped on earlier at
time 7; therefore, $x$ must be at most 10 or else the grass from his first visit
would not yet have grown back.  He also steps on a cell at time 26 that he also
visited at time 2; hence $x$ must also be at most 24.  Since the first of these
two constraints is tighter, we see that $x$ can be at most 10.Problem credits: Brian Dean

\section*{Sample Input}
\begin{verbatim}
6
N 10
E 2
S 3
W 4
S 5
E 8
\end{verbatim}

\section*{Sample Output}
\begin{verbatim}
10

In this example, FJ steps on a cell at time 17 that he stepped on earlier at
time 7; therefore, $x$ must be at most 10 or else the grass from his first visit
would not yet have grown back.  He also steps on a cell at time 26 that he also
visited at time 2; hence $x$ must also be at most 24.  Since the first of these
two constraints is tighter, we see that $x$ can be at most 10.Problem credits: Brian Dean

Problem credits: Brian Dean
\end{verbatim}

\section*{Solution}


\section*{Problem URL}
https://usaco.org/index.php?page=viewproblem2&cpid=593

\section*{Source}
USACO JAN16 Contest, Bronze Division, Problem ID: 593

\end{document}
