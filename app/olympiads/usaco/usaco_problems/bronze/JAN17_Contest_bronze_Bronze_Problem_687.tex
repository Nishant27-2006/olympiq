\documentclass[12pt]{article}
\usepackage{amsmath}
\usepackage{amssymb}
\usepackage{graphicx}
\usepackage{enumitem}
\usepackage{hyperref}
\usepackage{xcolor}

\title{View problem}
\author{USACO Contest: JAN17 Contest - Bronze Division}
\date{\today}

\begin{document}

\maketitle

\section*{Problem Statement}

Farmer John owns 7 dairy cows: Bessie, Elsie, Daisy, Gertie, Annabelle, Maggie,
and Henrietta. He milks them every day and keeps detailed records on the amount
of milk provided by each cow during each milking session.  Not surprisingly,
Farmer John highly prizes cows that provide large amounts of milk.

Cows, being lazy creatures, don't necessarily want to be responsible for
producing too much milk.  If it were up to them, they would each be perfectly
content to be the lowest-producing cow in the entire herd.  However, they keep
hearing Farmer John mentioning the phrase "farm to table" with his human
friends, and while they don't quite understand what this means, they have a
suspicion that it actually may not be the best idea to be the cow producing the
least amount of milk.  Instead, they figure it's safer to be in the position of
producing the second-smallest amount of milk in the herd.  Please help the cows
figure out which of them currently occupies this desirable position.

INPUT FORMAT (file notlast.in):
The input file for this task starts with a line containing the integer $N$
($1 \leq N \leq 100$), giving the number of entries in Farmer John's milking
log.  

Each of the $N$ following lines contains the name of a cow (one of the seven
above) followed by a positive integer (at most 100), indicating the amount of
milk produced by the cow during one of its milking sessions.  

Any cow that does not appear in the log at all is assumed to have produced no
milk.

OUTPUT FORMAT (file notlast.out):
On a single line of output, please print the name of the cow that produces the 
second-smallest amount of milk.  More precisely, if $M$ is the minimum total
amount of milk produced by any cow, please output the name of the cow whose
total production is minimal among all cows that produce more than $M$ units of
milk.  If several cows tie for this designation, or if no cow has this
designation (i.e., if all cows have production equal to $M$), please output the
word "Tie". Don't forget to add a newline character at the end of your line of
output. Note that $M=0$ if one of the seven cows is completely absent from the
milking log, since this cow would have produced no milk.

SAMPLE INPUT:
10
Bessie 1
Maggie 13
Elsie 3
Elsie 4
Henrietta 4
Gertie 12
Daisy 7
Annabelle 10
Bessie 6
Henrietta 5
SAMPLE OUTPUT: 
Henrietta

In this example, Bessie, Elsie, and Daisy all tie for the minimum by each
producing 7 units of milk.  The next-largest production, 9 units, is due to
Henrietta.


Problem credits: Brian Dean



\section*{Input Format}
Each of the $N$ following lines contains the name of a cow (one of the seven
above) followed by a positive integer (at most 100), indicating the amount of
milk produced by the cow during one of its milking sessions.Any cow that does not appear in the log at all is assumed to have produced no
milk.

Any cow that does not appear in the log at all is assumed to have produced no
milk.

OUTPUT FORMAT (file notlast.out):On a single line of output, please print the name of the cow that produces the 
second-smallest amount of milk.  More precisely, if $M$ is the minimum total
amount of milk produced by any cow, please output the name of the cow whose
total production is minimal among all cows that produce more than $M$ units of
milk.  If several cows tie for this designation, or if no cow has this
designation (i.e., if all cows have production equal to $M$), please output the
word "Tie". Don't forget to add a newline character at the end of your line of
output. Note that $M=0$ if one of the seven cows is completely absent from the
milking log, since this cow would have produced no milk.SAMPLE INPUT:10
Bessie 1
Maggie 13
Elsie 3
Elsie 4
Henrietta 4
Gertie 12
Daisy 7
Annabelle 10
Bessie 6
Henrietta 5SAMPLE OUTPUT:HenriettaIn this example, Bessie, Elsie, and Daisy all tie for the minimum by each
producing 7 units of milk.  The next-largest production, 9 units, is due to
Henrietta.Problem credits: Brian Dean

\section*{Output Format}
SAMPLE INPUT:10
Bessie 1
Maggie 13
Elsie 3
Elsie 4
Henrietta 4
Gertie 12
Daisy 7
Annabelle 10
Bessie 6
Henrietta 5SAMPLE OUTPUT:HenriettaIn this example, Bessie, Elsie, and Daisy all tie for the minimum by each
producing 7 units of milk.  The next-largest production, 9 units, is due to
Henrietta.Problem credits: Brian Dean

\section*{Sample Input}
\begin{verbatim}
10
Bessie 1
Maggie 13
Elsie 3
Elsie 4
Henrietta 4
Gertie 12
Daisy 7
Annabelle 10
Bessie 6
Henrietta 5
\end{verbatim}

\section*{Sample Output}
\begin{verbatim}
Henrietta

In this example, Bessie, Elsie, and Daisy all tie for the minimum by each
producing 7 units of milk.  The next-largest production, 9 units, is due to
Henrietta.Problem credits: Brian Dean

Problem credits: Brian Dean

Problem credits: Brian Dean
\end{verbatim}

\section*{Solution}


\section*{Problem URL}
https://usaco.org/index.php?page=viewproblem2&cpid=687

\section*{Source}
USACO JAN17 Contest, Bronze Division, Problem ID: 687

\end{document}
