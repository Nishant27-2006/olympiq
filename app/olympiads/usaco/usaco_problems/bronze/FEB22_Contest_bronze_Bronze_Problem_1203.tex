\documentclass[12pt]{article}
\usepackage{amsmath}
\usepackage{amssymb}
\usepackage{graphicx}
\usepackage{enumitem}
\usepackage{hyperref}
\usepackage{xcolor}

\title{View problem}
\author{USACO Contest: FEB22 Contest - Bronze Division}
\date{\today}

\begin{document}

\maketitle

\section*{Problem Statement}

Bessie the cow was excited to recently return to in-person learning!
Unfortunately, her instructor, Farmer John, is a very boring lecturer, and so
she ends up falling asleep in class often.

Farmer John has noticed that Bessie has not been paying attention in class. He
has asked another student in class, Elsie, to keep track of the number of times
Bessie falls asleep in a given class. There are $N$ class periods
($1\le N\le 10^5$), and Elsie logs that Bessie fell asleep $a_i$ times
($0\le a_i\le 10^6$) in the $i$-th class period. The total number of times Bessie fell
asleep across all class periods is at most $10^6$.

Elsie, feeling very competitive with Bessie, wants to make Farmer John feel like
Bessie is consistently falling asleep the same number of times in every class --
making it appear that the issue is entirely Bessie's fault, with no dependence
on Farmer John's sometimes-boring lectures. The only way Elsie may modify the
log is by combining two adjacent class periods.  For example, if
$a=[1,2,3,4,5],$ then if Elsie combines the second and third class periods the
log will become $[1,5,4,5]$.

Help Elsie compute the minimum number of modifications to the log that she needs
to make so that she can make all the numbers in the log equal.

INPUT FORMAT (input arrives from the terminal / stdin):
Each input will contain $T$ ($1\le T\le 10$) test cases that should be solved
independently.

The first line contains $T$, the number of test cases to be solved. The $T$ test
cases follow, each described by a pair of lines. The first line of each pair
contains $N$, and the second contains $a_1,a_2,\ldots,a_N$. 

It is guaranteed that within each test case, the sum of all values in $a$ is at
most $10^6$. It is also guaranteed that the sum of $N$ over all test cases is at
most
$10^5$.


OUTPUT FORMAT (print output to the terminal / stdout):
Please write $T$ lines of output, giving the minimum number of modifications
Elsie could perform to make all the log entries equal for each case.


SAMPLE INPUT:
3
6
1 2 3 1 1 1
3
2 2 3
5
0 0 0 0 0
SAMPLE OUTPUT: 
3
2
0

For the first test case in this example, Elsie can change her log to consist
solely of 3s with 3 modifications.


   1 2 3 1 1 1
-> 3 3 1 1 1
-> 3 3 2 1
-> 3 3 3

For the second test case, Elsie can change her log to 7 with 2 modifications.


   2 2 3
-> 2 5
-> 7

For the last test case, Elsie doesn’t need to perform any operations; the log
already consists of equal entries.


Problem credits: Jesse Choe



\section*{Input Format}
The first line contains $T$, the number of test cases to be solved. The $T$ test
cases follow, each described by a pair of lines. The first line of each pair
contains $N$, and the second contains $a_1,a_2,\ldots,a_N$.It is guaranteed that within each test case, the sum of all values in $a$ is at
most $10^6$. It is also guaranteed that the sum of $N$ over all test cases is at
most
$10^5$.

It is guaranteed that within each test case, the sum of all values in $a$ is at
most $10^6$. It is also guaranteed that the sum of $N$ over all test cases is at
most
$10^5$.

OUTPUT FORMAT (print output to the terminal / stdout):Please write $T$ lines of output, giving the minimum number of modifications
Elsie could perform to make all the log entries equal for each case.SAMPLE INPUT:3
6
1 2 3 1 1 1
3
2 2 3
5
0 0 0 0 0SAMPLE OUTPUT:3
2
0For the first test case in this example, Elsie can change her log to consist
solely of 3s with 3 modifications.1 2 3 1 1 1
-> 3 3 1 1 1
-> 3 3 2 1
-> 3 3 3For the second test case, Elsie can change her log to 7 with 2 modifications.2 2 3
-> 2 5
-> 7For the last test case, Elsie doesn’t need to perform any operations; the log
already consists of equal entries.Problem credits: Jesse Choe

\section*{Output Format}
Please write $T$ lines of output, giving the minimum number of modifications
Elsie could perform to make all the log entries equal for each case.

SAMPLE INPUT:3
6
1 2 3 1 1 1
3
2 2 3
5
0 0 0 0 0SAMPLE OUTPUT:3
2
0For the first test case in this example, Elsie can change her log to consist
solely of 3s with 3 modifications.1 2 3 1 1 1
-> 3 3 1 1 1
-> 3 3 2 1
-> 3 3 3For the second test case, Elsie can change her log to 7 with 2 modifications.2 2 3
-> 2 5
-> 7For the last test case, Elsie doesn’t need to perform any operations; the log
already consists of equal entries.Problem credits: Jesse Choe

\section*{Sample Input}
\begin{verbatim}
3
6
1 2 3 1 1 1
3
2 2 3
5
0 0 0 0 0
\end{verbatim}

\section*{Sample Output}
\begin{verbatim}
3
2
0

For the first test case in this example, Elsie can change her log to consist
solely of 3s with 3 modifications.1 2 3 1 1 1
-> 3 3 1 1 1
-> 3 3 2 1
-> 3 3 3For the second test case, Elsie can change her log to 7 with 2 modifications.2 2 3
-> 2 5
-> 7For the last test case, Elsie doesn’t need to perform any operations; the log
already consists of equal entries.Problem credits: Jesse Choe

1 2 3 1 1 1
-> 3 3 1 1 1
-> 3 3 2 1
-> 3 3 3For the second test case, Elsie can change her log to 7 with 2 modifications.2 2 3
-> 2 5
-> 7For the last test case, Elsie doesn’t need to perform any operations; the log
already consists of equal entries.Problem credits: Jesse Choe

1 2 3 1 1 1
-> 3 3 1 1 1
-> 3 3 2 1
-> 3 3 3

For the second test case, Elsie can change her log to 7 with 2 modifications.2 2 3
-> 2 5
-> 7For the last test case, Elsie doesn’t need to perform any operations; the log
already consists of equal entries.Problem credits: Jesse Choe

2 2 3
-> 2 5
-> 7For the last test case, Elsie doesn’t need to perform any operations; the log
already consists of equal entries.Problem credits: Jesse Choe

2 2 3
-> 2 5
-> 7

For the last test case, Elsie doesn’t need to perform any operations; the log
already consists of equal entries.Problem credits: Jesse Choe

Problem credits: Jesse Choe

Problem credits: Jesse Choe
\end{verbatim}

\section*{Solution}


\section*{Problem URL}
https://usaco.org/index.php?page=viewproblem2&cpid=1203

\section*{Source}
USACO FEB22 Contest, Bronze Division, Problem ID: 1203

\end{document}
