\documentclass[12pt]{article}
\usepackage{amsmath}
\usepackage{amssymb}
\usepackage{graphicx}
\usepackage{enumitem}
\usepackage{hyperref}
\usepackage{xcolor}

\title{View problem}
\author{USACO Contest: FEB18 Contest - Bronze Division}
\date{\today}

\begin{document}

\maketitle

\section*{Problem Statement}

In preparation for the upcoming hoofball tournament, Farmer John is drilling his
$N$ cows (conveniently numbered $1\dots N$, where $1 \leq N \leq 100$) in
passing the ball. The cows are all standing along a very long line on one side
of the barn, with cow $i$ standing $x_i$ units away from the barn
($1 \leq x_i \leq 1000$).  Each cow is standing at a distinct location.

At the beginning of the drill, Farmer John will pass several balls to different
cows. When cow $i$ receives a ball, either from Farmer John or from another cow,
she will pass the ball to the cow nearest her (and if multiple cows are the same
distance from her, she will pass the ball to the cow farthest to the left among
these).  So that all cows get at least a little bit of practice passing, Farmer
John wants to make sure that every cow will hold a ball at least once. Help him
figure out the minimum number of balls he needs to distribute initially to
ensure this can happen, assuming he hands the balls to an appropriate initial
set of cows.

INPUT FORMAT (file hoofball.in):
The first line of input contains $N$. The second line contains $N$
space-separated integers, where the $i$th integer is $x_i$.

OUTPUT FORMAT (file hoofball.out):
Please output the minimum number of balls Farmer John must initially pass to the
cows, so that every cow can hold a ball at least once.

SAMPLE INPUT:
5
7 1 3 11 4
SAMPLE OUTPUT: 
2

In the above example, Farmer John should pass a ball to the cow at $x=1$ and
pass a ball to the cow at $x=11$. The cow at $x=1$ will pass her ball to the cow
at $x=3$, after which this ball will oscillate between the cow at $x=3$ and the
cow at $x=4$. The cow at $x=11$ will pass her ball to the cow at $x=7$, who will
pass the ball to the cow at $x=4$, after which this ball will also cycle between
the cow at $x=3$ and the cow at $x=4$. In this way, all cows will be passed a
ball at least once (possibly by Farmer John, possibly by another cow).

It can be seen that there is no single cow to whom Farmer John could initially pass a ball
so that every cow would eventually be passed a ball.


Problem credits: Dhruv Rohatgi



\section*{Input Format}
OUTPUT FORMAT (file hoofball.out):Please output the minimum number of balls Farmer John must initially pass to the
cows, so that every cow can hold a ball at least once.SAMPLE INPUT:5
7 1 3 11 4SAMPLE OUTPUT:2In the above example, Farmer John should pass a ball to the cow at $x=1$ and
pass a ball to the cow at $x=11$. The cow at $x=1$ will pass her ball to the cow
at $x=3$, after which this ball will oscillate between the cow at $x=3$ and the
cow at $x=4$. The cow at $x=11$ will pass her ball to the cow at $x=7$, who will
pass the ball to the cow at $x=4$, after which this ball will also cycle between
the cow at $x=3$ and the cow at $x=4$. In this way, all cows will be passed a
ball at least once (possibly by Farmer John, possibly by another cow).It can be seen that there is no single cow to whom Farmer John could initially pass a ball
so that every cow would eventually be passed a ball.Problem credits: Dhruv Rohatgi

\section*{Output Format}
SAMPLE INPUT:5
7 1 3 11 4SAMPLE OUTPUT:2In the above example, Farmer John should pass a ball to the cow at $x=1$ and
pass a ball to the cow at $x=11$. The cow at $x=1$ will pass her ball to the cow
at $x=3$, after which this ball will oscillate between the cow at $x=3$ and the
cow at $x=4$. The cow at $x=11$ will pass her ball to the cow at $x=7$, who will
pass the ball to the cow at $x=4$, after which this ball will also cycle between
the cow at $x=3$ and the cow at $x=4$. In this way, all cows will be passed a
ball at least once (possibly by Farmer John, possibly by another cow).It can be seen that there is no single cow to whom Farmer John could initially pass a ball
so that every cow would eventually be passed a ball.Problem credits: Dhruv Rohatgi

\section*{Sample Input}
\begin{verbatim}
5
7 1 3 11 4
\end{verbatim}

\section*{Sample Output}
\begin{verbatim}
2

In the above example, Farmer John should pass a ball to the cow at $x=1$ and
pass a ball to the cow at $x=11$. The cow at $x=1$ will pass her ball to the cow
at $x=3$, after which this ball will oscillate between the cow at $x=3$ and the
cow at $x=4$. The cow at $x=11$ will pass her ball to the cow at $x=7$, who will
pass the ball to the cow at $x=4$, after which this ball will also cycle between
the cow at $x=3$ and the cow at $x=4$. In this way, all cows will be passed a
ball at least once (possibly by Farmer John, possibly by another cow).It can be seen that there is no single cow to whom Farmer John could initially pass a ball
so that every cow would eventually be passed a ball.Problem credits: Dhruv Rohatgi

It can be seen that there is no single cow to whom Farmer John could initially pass a ball
so that every cow would eventually be passed a ball.Problem credits: Dhruv Rohatgi

Problem credits: Dhruv Rohatgi

Problem credits: Dhruv Rohatgi
\end{verbatim}

\section*{Solution}


\section*{Problem URL}
https://usaco.org/index.php?page=viewproblem2&cpid=808

\section*{Source}
USACO FEB18 Contest, Bronze Division, Problem ID: 808

\end{document}
