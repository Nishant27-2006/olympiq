\documentclass[12pt]{article}
\usepackage{amsmath}
\usepackage{amssymb}
\usepackage{graphicx}
\usepackage{enumitem}
\usepackage{hyperref}
\usepackage{xcolor}

\title{View problem}
\author{USACO Contest: DEC16 Contest - Bronze Division}
\date{\today}

\begin{document}

\maketitle

\section*{Problem Statement}

Bessie and her cow friends are playing as their favorite cow superheroes.  Of
course, everyone knows that any self-respecting superhero needs a signal to call
them to action. Bessie has drawn a special signal on a sheet of $M \times N$
paper ($1 \leq M \leq 10, 1 \leq N \leq 10$), but this is too small, much too
small! Bessie wants to amplify the signal so it is exactly $K$ times bigger
($1 \leq K \leq 10$) in each direction.

The signal will consist only of the '.' and 'X' characters.

INPUT FORMAT (file cowsignal.in):
The first line of input contains $M$, $N$, and $K$, separated by spaces.  

The next $M$ lines each contain a length-$N$ string, collectively describing the
picture of the signal.

OUTPUT FORMAT (file cowsignal.out):
You should output $KM$ lines, each with $KN$ characters, giving a picture of the
enlarged signal.

SAMPLE INPUT:
5 4 2
XXX.
X..X
XXX.
X..X
XXX.
SAMPLE OUTPUT: 
XXXXXX..
XXXXXX..
XX....XX
XX....XX
XXXXXX..
XXXXXX..
XX....XX
XX....XX
XXXXXX..
XXXXXX..


Problem credits: Nathan Pinsker



\section*{Input Format}
The next $M$ lines each contain a length-$N$ string, collectively describing the
picture of the signal.

OUTPUT FORMAT (file cowsignal.out):You should output $KM$ lines, each with $KN$ characters, giving a picture of the
enlarged signal.SAMPLE INPUT:5 4 2
XXX.
X..X
XXX.
X..X
XXX.SAMPLE OUTPUT:XXXXXX..
XXXXXX..
XX....XX
XX....XX
XXXXXX..
XXXXXX..
XX....XX
XX....XX
XXXXXX..
XXXXXX..Problem credits: Nathan Pinsker

\section*{Output Format}
SAMPLE INPUT:5 4 2
XXX.
X..X
XXX.
X..X
XXX.SAMPLE OUTPUT:XXXXXX..
XXXXXX..
XX....XX
XX....XX
XXXXXX..
XXXXXX..
XX....XX
XX....XX
XXXXXX..
XXXXXX..Problem credits: Nathan Pinsker

\section*{Sample Input}
\begin{verbatim}
5 4 2
XXX.
X..X
XXX.
X..X
XXX.
\end{verbatim}

\section*{Sample Output}
\begin{verbatim}
XXXXXX..
XXXXXX..
XX....XX
XX....XX
XXXXXX..
XXXXXX..
XX....XX
XX....XX
XXXXXX..
XXXXXX..

Problem credits: Nathan Pinsker

Problem credits: Nathan Pinsker
\end{verbatim}

\section*{Solution}


\section*{Problem URL}
https://usaco.org/index.php?page=viewproblem2&cpid=665

\section*{Source}
USACO DEC16 Contest, Bronze Division, Problem ID: 665

\end{document}
