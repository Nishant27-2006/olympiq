\documentclass[12pt]{article}
\usepackage{amsmath}
\usepackage{amssymb}
\usepackage{graphicx}
\usepackage{enumitem}
\usepackage{hyperref}
\usepackage{xcolor}

\title{View problem}
\author{USACO Contest: FEB25 Contest - Bronze Division}
\date{\today}

\begin{document}

\maketitle

\section*{Problem Statement}


Bessie is learning to code using a simple programming language. She first
defines a valid program, then executes it to produce some output sequence. 

Defining:
A program is a nonempty sequence of statements.A
statement is either of the form "PRINT $c$" where $c$ is an integer, or "REP
$o$",  followed by a program, followed by "END," where $o$ is an integer that is
at least 1.
Executing:
Executing a program executes its statements in sequence.Executing
the statement "PRINT $c$" appends $c$ to the output sequence.Executing
a statement starting with "REP $o$" executes the inner program a total of $o$
times in sequence.
An example of a program that Bessie knows how to write is as follows.


REP 3
    PRINT 1
    REP 2
        PRINT 2
    END
END

The program outputs the sequence $[1,2,2,1,2,2,1,2,2]$.

Bessie wants to output a sequence of $N$ ($1 \le N \le 100$) positive integers.
Elsie challenges her to use no more than $K$ ($1 \le K \le 3$) "PRINT"
statements. Note that Bessie can use as many "REP" statements as she wants. Also
note that each positive integer in the sequence is no greater than $K$.

For each of $T$ ($1 \le T \le 100$) independent test cases, determine whether
Bessie can write a program that outputs some given sequence using at most $K$
"PRINT" statements.

INPUT FORMAT (input arrives from the terminal / stdin):
The first line contains $T$.

The first line of each test case contains two space-separated integers, $N$ and
$K$.

The second line of each test case contains a sequence of $N$ space-separated
positive integers,  each at most $K$, which is the sequence that Bessie wants to
produce.

OUTPUT FORMAT (print output to the terminal / stdout):
For each test case, output "YES" or "NO" (case sensitive) on a separate line.

SAMPLE INPUT:
2
1 1
1
4 1
1 1 1 1
SAMPLE OUTPUT: 
YES
YES

For the second test case, the following code outputs the sequence $[1,1,1,1]$
with $1$ "PRINT" statement.


REP 4
    PRINT 1
END

SAMPLE INPUT:
11
4 2
1 2 2 2
4 2
1 1 2 1
4 2
1 1 2 2
6 2
1 1 2 2 1 1
10 2
1 1 1 2 2 1 1 1 2 2
8 3
3 3 1 2 2 1 2 2
9 3
1 1 2 2 2 3 3 3 3
16 3
2 2 3 2 2 3 1 1 2 2 3 2 2 3 1 1
24 3
1 1 2 2 3 3 3 2 2 3 3 3 1 1 2 2 3 3 3 2 2 3 3 3
9 3
1 2 2 1 3 3 1 2 2
6 3
1 2 1 2 2 3
SAMPLE OUTPUT: 
YES
NO
YES
NO
YES
YES
YES
YES
YES
NO
NO

For the first test case, the following code outputs the sequence $[1,2,2,2]$
with $2$ "PRINT" statements.


PRINT 1
REP 3
    PRINT 2
END

For the second test case, the answer is "NO" because it is impossible to output
the sequence $[1,1,2,1]$ using at most $2$ "PRINT" statements.

For the sixth test case, the following code outputs the sequence
$[3,3,1,2,2,1,2,2]$ with $3$ "PRINT" statements.


REP 2
    PRINT 3
END
REP 2
    PRINT 1
    REP 2
        PRINT 2
    END
END

SCORING:
Input 3: $K=1$Inputs 4-7: $K \le 2$Inputs 8-13: No
additional constraints.


Problem credits: Alex Liang



\section*{Input Format}
The first line of each test case contains two space-separated integers, $N$ and
$K$.The second line of each test case contains a sequence of $N$ space-separated
positive integers,  each at most $K$, which is the sequence that Bessie wants to
produce.

The second line of each test case contains a sequence of $N$ space-separated
positive integers,  each at most $K$, which is the sequence that Bessie wants to
produce.

OUTPUT FORMAT (print output to the terminal / stdout):For each test case, output "YES" or "NO" (case sensitive) on a separate line.SAMPLE INPUT:2
1 1
1
4 1
1 1 1 1SAMPLE OUTPUT:YES
YESFor the second test case, the following code outputs the sequence $[1,1,1,1]$
with $1$ "PRINT" statement.REP 4
    PRINT 1
ENDSAMPLE INPUT:11
4 2
1 2 2 2
4 2
1 1 2 1
4 2
1 1 2 2
6 2
1 1 2 2 1 1
10 2
1 1 1 2 2 1 1 1 2 2
8 3
3 3 1 2 2 1 2 2
9 3
1 1 2 2 2 3 3 3 3
16 3
2 2 3 2 2 3 1 1 2 2 3 2 2 3 1 1
24 3
1 1 2 2 3 3 3 2 2 3 3 3 1 1 2 2 3 3 3 2 2 3 3 3
9 3
1 2 2 1 3 3 1 2 2
6 3
1 2 1 2 2 3SAMPLE OUTPUT:YES
NO
YES
NO
YES
YES
YES
YES
YES
NO
NOFor the first test case, the following code outputs the sequence $[1,2,2,2]$
with $2$ "PRINT" statements.PRINT 1
REP 3
    PRINT 2
ENDFor the second test case, the answer is "NO" because it is impossible to output
the sequence $[1,1,2,1]$ using at most $2$ "PRINT" statements.For the sixth test case, the following code outputs the sequence
$[3,3,1,2,2,1,2,2]$ with $3$ "PRINT" statements.REP 2
    PRINT 3
END
REP 2
    PRINT 1
    REP 2
        PRINT 2
    END
ENDSCORING:Input 3: $K=1$Inputs 4-7: $K \le 2$Inputs 8-13: No
additional constraints.Problem credits: Alex Liang

\section*{Output Format}
SAMPLE INPUT:2
1 1
1
4 1
1 1 1 1SAMPLE OUTPUT:YES
YESFor the second test case, the following code outputs the sequence $[1,1,1,1]$
with $1$ "PRINT" statement.REP 4
    PRINT 1
ENDSAMPLE INPUT:11
4 2
1 2 2 2
4 2
1 1 2 1
4 2
1 1 2 2
6 2
1 1 2 2 1 1
10 2
1 1 1 2 2 1 1 1 2 2
8 3
3 3 1 2 2 1 2 2
9 3
1 1 2 2 2 3 3 3 3
16 3
2 2 3 2 2 3 1 1 2 2 3 2 2 3 1 1
24 3
1 1 2 2 3 3 3 2 2 3 3 3 1 1 2 2 3 3 3 2 2 3 3 3
9 3
1 2 2 1 3 3 1 2 2
6 3
1 2 1 2 2 3SAMPLE OUTPUT:YES
NO
YES
NO
YES
YES
YES
YES
YES
NO
NOFor the first test case, the following code outputs the sequence $[1,2,2,2]$
with $2$ "PRINT" statements.PRINT 1
REP 3
    PRINT 2
ENDFor the second test case, the answer is "NO" because it is impossible to output
the sequence $[1,1,2,1]$ using at most $2$ "PRINT" statements.For the sixth test case, the following code outputs the sequence
$[3,3,1,2,2,1,2,2]$ with $3$ "PRINT" statements.REP 2
    PRINT 3
END
REP 2
    PRINT 1
    REP 2
        PRINT 2
    END
ENDSCORING:Input 3: $K=1$Inputs 4-7: $K \le 2$Inputs 8-13: No
additional constraints.Problem credits: Alex Liang

\section*{Sample Input}
\begin{verbatim}
2
1 1
1
4 1
1 1 1 1

11
4 2
1 2 2 2
4 2
1 1 2 1
4 2
1 1 2 2
6 2
1 1 2 2 1 1
10 2
1 1 1 2 2 1 1 1 2 2
8 3
3 3 1 2 2 1 2 2
9 3
1 1 2 2 2 3 3 3 3
16 3
2 2 3 2 2 3 1 1 2 2 3 2 2 3 1 1
24 3
1 1 2 2 3 3 3 2 2 3 3 3 1 1 2 2 3 3 3 2 2 3 3 3
9 3
1 2 2 1 3 3 1 2 2
6 3
1 2 1 2 2 3
\end{verbatim}

\section*{Sample Output}
\begin{verbatim}
YES
YES

REP 4
    PRINT 1
ENDSAMPLE INPUT:11
4 2
1 2 2 2
4 2
1 1 2 1
4 2
1 1 2 2
6 2
1 1 2 2 1 1
10 2
1 1 1 2 2 1 1 1 2 2
8 3
3 3 1 2 2 1 2 2
9 3
1 1 2 2 2 3 3 3 3
16 3
2 2 3 2 2 3 1 1 2 2 3 2 2 3 1 1
24 3
1 1 2 2 3 3 3 2 2 3 3 3 1 1 2 2 3 3 3 2 2 3 3 3
9 3
1 2 2 1 3 3 1 2 2
6 3
1 2 1 2 2 3SAMPLE OUTPUT:YES
NO
YES
NO
YES
YES
YES
YES
YES
NO
NOFor the first test case, the following code outputs the sequence $[1,2,2,2]$
with $2$ "PRINT" statements.PRINT 1
REP 3
    PRINT 2
ENDFor the second test case, the answer is "NO" because it is impossible to output
the sequence $[1,1,2,1]$ using at most $2$ "PRINT" statements.For the sixth test case, the following code outputs the sequence
$[3,3,1,2,2,1,2,2]$ with $3$ "PRINT" statements.REP 2
    PRINT 3
END
REP 2
    PRINT 1
    REP 2
        PRINT 2
    END
ENDSCORING:Input 3: $K=1$Inputs 4-7: $K \le 2$Inputs 8-13: No
additional constraints.Problem credits: Alex Liang

REP 4
    PRINT 1
END

SAMPLE INPUT:11
4 2
1 2 2 2
4 2
1 1 2 1
4 2
1 1 2 2
6 2
1 1 2 2 1 1
10 2
1 1 1 2 2 1 1 1 2 2
8 3
3 3 1 2 2 1 2 2
9 3
1 1 2 2 2 3 3 3 3
16 3
2 2 3 2 2 3 1 1 2 2 3 2 2 3 1 1
24 3
1 1 2 2 3 3 3 2 2 3 3 3 1 1 2 2 3 3 3 2 2 3 3 3
9 3
1 2 2 1 3 3 1 2 2
6 3
1 2 1 2 2 3SAMPLE OUTPUT:YES
NO
YES
NO
YES
YES
YES
YES
YES
NO
NOFor the first test case, the following code outputs the sequence $[1,2,2,2]$
with $2$ "PRINT" statements.PRINT 1
REP 3
    PRINT 2
ENDFor the second test case, the answer is "NO" because it is impossible to output
the sequence $[1,1,2,1]$ using at most $2$ "PRINT" statements.For the sixth test case, the following code outputs the sequence
$[3,3,1,2,2,1,2,2]$ with $3$ "PRINT" statements.REP 2
    PRINT 3
END
REP 2
    PRINT 1
    REP 2
        PRINT 2
    END
ENDSCORING:Input 3: $K=1$Inputs 4-7: $K \le 2$Inputs 8-13: No
additional constraints.Problem credits: Alex Liang

YES
NO
YES
NO
YES
YES
YES
YES
YES
NO
NO

PRINT 1
REP 3
    PRINT 2
ENDFor the second test case, the answer is "NO" because it is impossible to output
the sequence $[1,1,2,1]$ using at most $2$ "PRINT" statements.For the sixth test case, the following code outputs the sequence
$[3,3,1,2,2,1,2,2]$ with $3$ "PRINT" statements.REP 2
    PRINT 3
END
REP 2
    PRINT 1
    REP 2
        PRINT 2
    END
ENDSCORING:Input 3: $K=1$Inputs 4-7: $K \le 2$Inputs 8-13: No
additional constraints.Problem credits: Alex Liang

PRINT 1
REP 3
    PRINT 2
END

For the second test case, the answer is "NO" because it is impossible to output
the sequence $[1,1,2,1]$ using at most $2$ "PRINT" statements.For the sixth test case, the following code outputs the sequence
$[3,3,1,2,2,1,2,2]$ with $3$ "PRINT" statements.REP 2
    PRINT 3
END
REP 2
    PRINT 1
    REP 2
        PRINT 2
    END
ENDSCORING:Input 3: $K=1$Inputs 4-7: $K \le 2$Inputs 8-13: No
additional constraints.Problem credits: Alex Liang

For the sixth test case, the following code outputs the sequence
$[3,3,1,2,2,1,2,2]$ with $3$ "PRINT" statements.REP 2
    PRINT 3
END
REP 2
    PRINT 1
    REP 2
        PRINT 2
    END
ENDSCORING:Input 3: $K=1$Inputs 4-7: $K \le 2$Inputs 8-13: No
additional constraints.Problem credits: Alex Liang

REP 2
    PRINT 3
END
REP 2
    PRINT 1
    REP 2
        PRINT 2
    END
ENDSCORING:Input 3: $K=1$Inputs 4-7: $K \le 2$Inputs 8-13: No
additional constraints.Problem credits: Alex Liang

REP 2
    PRINT 3
END
REP 2
    PRINT 1
    REP 2
        PRINT 2
    END
END
\end{verbatim}

\section*{Solution}


\section*{Problem URL}
https://usaco.org/index.php?page=viewproblem2&cpid=1493

\section*{Source}
USACO FEB25 Contest, Bronze Division, Problem ID: 1493

\end{document}
