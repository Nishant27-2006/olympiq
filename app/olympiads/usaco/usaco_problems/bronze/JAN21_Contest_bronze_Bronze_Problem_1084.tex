\documentclass[12pt]{article}
\usepackage{amsmath}
\usepackage{amssymb}
\usepackage{graphicx}
\usepackage{enumitem}
\usepackage{hyperref}
\usepackage{xcolor}

\title{View problem}
\author{USACO Contest: JAN21 Contest - Bronze Division}
\date{\today}

\begin{document}

\maketitle

\section*{Problem Statement}

Farmer John is yet again trying to take a photograph of his $N$ cows
($2 \leq N \leq 1000$).

Each cow has an integer "breed ID" number in the range $1 \ldots 100$.  Farmer
John has a very peculiar idea in mind for his photo: he wants to partition all
the cows into disjoint groups (in other words, place each cow in exactly one group) and then
line up the groups so the sum of the breed IDs of  the cows in the first group
is even, the sum of the IDs in the second group is odd, and so on, alternating
between even and odd.  

What is the maximum possible number of groups Farmer John can form?  

INPUT FORMAT (input arrives from the terminal / stdin):
The first line of input contains $N$.  The next line contains $N$ 
space-separated integers giving the breed IDs of the $N$ cows.

OUTPUT FORMAT (print output to the terminal / stdout):
The maximum possible number of groups in Farmer John's photo. It can be shown
that at least one feasible grouping exists.

SAMPLE INPUT:
7
1 3 5 7 9 11 13
SAMPLE OUTPUT: 
3

In this example, one way to form the maximum number of three groups is as
follows. Place 1 and 3 in the first group, 5, 7, and 9 in the second group, and
11 and 13 in the third group.

SAMPLE INPUT:
7
11 2 17 13 1 15 3
SAMPLE OUTPUT: 
5

In this example, one way to form the maximum number of five groups is as
follows. Place 2 in the first group, 11 in the second group, 13 and 1 in the
third group, 15 in the fourth group, and 17 and 3 in the fifth group.


Problem credits: Nick Wu



\section*{Input Format}
OUTPUT FORMAT (print output to the terminal / stdout):The maximum possible number of groups in Farmer John's photo. It can be shown
that at least one feasible grouping exists.SAMPLE INPUT:7
1 3 5 7 9 11 13SAMPLE OUTPUT:3In this example, one way to form the maximum number of three groups is as
follows. Place 1 and 3 in the first group, 5, 7, and 9 in the second group, and
11 and 13 in the third group.SAMPLE INPUT:7
11 2 17 13 1 15 3SAMPLE OUTPUT:5In this example, one way to form the maximum number of five groups is as
follows. Place 2 in the first group, 11 in the second group, 13 and 1 in the
third group, 15 in the fourth group, and 17 and 3 in the fifth group.Problem credits: Nick Wu

\section*{Output Format}
SAMPLE INPUT:7
1 3 5 7 9 11 13SAMPLE OUTPUT:3In this example, one way to form the maximum number of three groups is as
follows. Place 1 and 3 in the first group, 5, 7, and 9 in the second group, and
11 and 13 in the third group.SAMPLE INPUT:7
11 2 17 13 1 15 3SAMPLE OUTPUT:5In this example, one way to form the maximum number of five groups is as
follows. Place 2 in the first group, 11 in the second group, 13 and 1 in the
third group, 15 in the fourth group, and 17 and 3 in the fifth group.Problem credits: Nick Wu

\section*{Sample Input}
\begin{verbatim}
7
1 3 5 7 9 11 13

7
11 2 17 13 1 15 3
\end{verbatim}

\section*{Sample Output}
\begin{verbatim}
3

In this example, one way to form the maximum number of three groups is as
follows. Place 1 and 3 in the first group, 5, 7, and 9 in the second group, and
11 and 13 in the third group.SAMPLE INPUT:7
11 2 17 13 1 15 3SAMPLE OUTPUT:5In this example, one way to form the maximum number of five groups is as
follows. Place 2 in the first group, 11 in the second group, 13 and 1 in the
third group, 15 in the fourth group, and 17 and 3 in the fifth group.Problem credits: Nick Wu

SAMPLE INPUT:7
11 2 17 13 1 15 3SAMPLE OUTPUT:5In this example, one way to form the maximum number of five groups is as
follows. Place 2 in the first group, 11 in the second group, 13 and 1 in the
third group, 15 in the fourth group, and 17 and 3 in the fifth group.Problem credits: Nick Wu

5

In this example, one way to form the maximum number of five groups is as
follows. Place 2 in the first group, 11 in the second group, 13 and 1 in the
third group, 15 in the fourth group, and 17 and 3 in the fifth group.Problem credits: Nick Wu

Problem credits: Nick Wu

Problem credits: Nick Wu
\end{verbatim}

\section*{Solution}


\section*{Problem URL}
https://usaco.org/index.php?page=viewproblem2&cpid=1084

\section*{Source}
USACO JAN21 Contest, Bronze Division, Problem ID: 1084

\end{document}
