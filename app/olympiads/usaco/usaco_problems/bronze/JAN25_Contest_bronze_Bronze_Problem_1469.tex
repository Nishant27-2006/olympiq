\documentclass[12pt]{article}
\usepackage{amsmath}
\usepackage{amssymb}
\usepackage{graphicx}
\usepackage{enumitem}
\usepackage{hyperref}
\usepackage{xcolor}

\title{View problem}
\author{USACO Contest: JAN25 Contest - Bronze Division}
\date{\today}

\begin{document}

\maketitle

\section*{Problem Statement}


**Note: We suggest using a language other than Python to earn full credit on this problem.**
Farmer John's $N$ ($1 \leq N \leq 7500$) cows are standing in a line, with cow
$1$ at the front of the line and cow $N$ at the back of the line. FJ's cows also
come in many different species. He denotes each species with an integer from $1$
to $N$. The $i$'th cow from the front of the line is of species $a_i$
($1 \leq a_i \leq N$). 

FJ is taking his cows to a checkup at a local bovine hospital. However, the
bovine veterinarian is very picky and wants to perform a checkup on the $i$'th
cow in the line, only if it is of species $b_i$ ($1 \leq b_i \leq N$). 

FJ is lazy and does not want to completely reorder his cows.  He will perform
the following operation exactly once.

 Select two integers $l$ and $r$ such that $1 \leq l \le r \leq N$. Reverse
the order of the cows that are between the $l$-th cow and the $r$-th cow in the
line, inclusive. 
FJ wants to measure how effective this approach is. For each $c=0 \ldots N$,
help FJ find the number of distinct operations ($l,r$) that result in exactly
$c$ cows being checked. Two operations ($l_1,r_1$) and ($l_2,r_2$) are different
if $l_1 \neq l_2$ or $r_1 \neq r_2$.

INPUT FORMAT (input arrives from the terminal / stdin):
The first line contains an integer $N$.

The second line contains $a_1, a_2, \ldots, a_N$.

The third line contains $b_1, b_2, \ldots, b_N$.

OUTPUT FORMAT (print output to the terminal / stdout):
Output $N+1$ lines with the $i$-th line containing the number of distinct
operations ($l,r$) that result in $i-1$ cows being checked.

SAMPLE INPUT:
3
1 3 2
3 2 1
SAMPLE OUTPUT: 
3
3
0
0

If FJ chooses ($l=1,r=1$), ($l=2,r=2$), or ($l=3,r=3$) then no cows will be
checked. Note that those operations do not modify any of the cows' locations.

The following operations result in one cow being checked.
$l=1,r=2$: FJ reverses the order of the first and second cows so the species
of each cow in the new lineup will be $[3,1,2]$. The first cow will be checked.
$l=2,r=3$: FJ reverses the order of the second and third cows so the species
of each cow in the new lineup will be $[1,2,3]$. The second cow will be checked.
$l=1,r=3$: FJ reverses the order of the first, second, and third cows so the
species of each cow in the new lineup will be $[2,3,1]$. The third cow will be
checked. 
SAMPLE INPUT:
3
1 2 3
1 2 3
SAMPLE OUTPUT: 
0
3
0
3

The three possible operations that cause $3$ cows to be checked are ($l=1,r=1$),
($l=2,r=2$), and ($l=3,r=3$).
SAMPLE INPUT:
7
1 3 2 2 1 3 2
3 2 2 1 2 3 1
SAMPLE OUTPUT: 
0
6
14
6
2
0
0
0

The two possible operations that cause $4$ cows to be checked are ($l=4,r=5$)
and ($l=5,r=7$).
SCORING:
Inputs 4-6: $N\le 100$Inputs 7-13: No additional constraints


Problem credits: Chongtian Ma and Haokai Ma



\section*{Input Format}
The second line contains $a_1, a_2, \ldots, a_N$.The third line contains $b_1, b_2, \ldots, b_N$.

The third line contains $b_1, b_2, \ldots, b_N$.

OUTPUT FORMAT (print output to the terminal / stdout):Output $N+1$ lines with the $i$-th line containing the number of distinct
operations ($l,r$) that result in $i-1$ cows being checked.SAMPLE INPUT:3
1 3 2
3 2 1SAMPLE OUTPUT:3
3
0
0If FJ chooses ($l=1,r=1$), ($l=2,r=2$), or ($l=3,r=3$) then no cows will be
checked. Note that those operations do not modify any of the cows' locations.The following operations result in one cow being checked.$l=1,r=2$: FJ reverses the order of the first and second cows so the species
of each cow in the new lineup will be $[3,1,2]$. The first cow will be checked.$l=2,r=3$: FJ reverses the order of the second and third cows so the species
of each cow in the new lineup will be $[1,2,3]$. The second cow will be checked.$l=1,r=3$: FJ reverses the order of the first, second, and third cows so the
species of each cow in the new lineup will be $[2,3,1]$. The third cow will be
checked.SAMPLE INPUT:3
1 2 3
1 2 3SAMPLE OUTPUT:0
3
0
3The three possible operations that cause $3$ cows to be checked are ($l=1,r=1$),
($l=2,r=2$), and ($l=3,r=3$).SAMPLE INPUT:7
1 3 2 2 1 3 2
3 2 2 1 2 3 1SAMPLE OUTPUT:0
6
14
6
2
0
0
0The two possible operations that cause $4$ cows to be checked are ($l=4,r=5$)
and ($l=5,r=7$).SCORING:Inputs 4-6: $N\le 100$Inputs 7-13: No additional constraintsProblem credits: Chongtian Ma and Haokai Ma

\section*{Output Format}
SAMPLE INPUT:3
1 3 2
3 2 1SAMPLE OUTPUT:3
3
0
0If FJ chooses ($l=1,r=1$), ($l=2,r=2$), or ($l=3,r=3$) then no cows will be
checked. Note that those operations do not modify any of the cows' locations.The following operations result in one cow being checked.$l=1,r=2$: FJ reverses the order of the first and second cows so the species
of each cow in the new lineup will be $[3,1,2]$. The first cow will be checked.$l=2,r=3$: FJ reverses the order of the second and third cows so the species
of each cow in the new lineup will be $[1,2,3]$. The second cow will be checked.$l=1,r=3$: FJ reverses the order of the first, second, and third cows so the
species of each cow in the new lineup will be $[2,3,1]$. The third cow will be
checked.SAMPLE INPUT:3
1 2 3
1 2 3SAMPLE OUTPUT:0
3
0
3The three possible operations that cause $3$ cows to be checked are ($l=1,r=1$),
($l=2,r=2$), and ($l=3,r=3$).SAMPLE INPUT:7
1 3 2 2 1 3 2
3 2 2 1 2 3 1SAMPLE OUTPUT:0
6
14
6
2
0
0
0The two possible operations that cause $4$ cows to be checked are ($l=4,r=5$)
and ($l=5,r=7$).SCORING:Inputs 4-6: $N\le 100$Inputs 7-13: No additional constraintsProblem credits: Chongtian Ma and Haokai Ma

\section*{Sample Input}
\begin{verbatim}
3
1 3 2
3 2 1

3
1 2 3
1 2 3

7
1 3 2 2 1 3 2
3 2 2 1 2 3 1
\end{verbatim}

\section*{Sample Output}
\begin{verbatim}
3
3
0
0

The following operations result in one cow being checked.$l=1,r=2$: FJ reverses the order of the first and second cows so the species
of each cow in the new lineup will be $[3,1,2]$. The first cow will be checked.$l=2,r=3$: FJ reverses the order of the second and third cows so the species
of each cow in the new lineup will be $[1,2,3]$. The second cow will be checked.$l=1,r=3$: FJ reverses the order of the first, second, and third cows so the
species of each cow in the new lineup will be $[2,3,1]$. The third cow will be
checked.SAMPLE INPUT:3
1 2 3
1 2 3SAMPLE OUTPUT:0
3
0
3The three possible operations that cause $3$ cows to be checked are ($l=1,r=1$),
($l=2,r=2$), and ($l=3,r=3$).SAMPLE INPUT:7
1 3 2 2 1 3 2
3 2 2 1 2 3 1SAMPLE OUTPUT:0
6
14
6
2
0
0
0The two possible operations that cause $4$ cows to be checked are ($l=4,r=5$)
and ($l=5,r=7$).SCORING:Inputs 4-6: $N\le 100$Inputs 7-13: No additional constraintsProblem credits: Chongtian Ma and Haokai Ma

0
3
0
3

0
6
14
6
2
0
0
0
\end{verbatim}

\section*{Solution}


\section*{Problem URL}
https://usaco.org/index.php?page=viewproblem2&cpid=1469

\section*{Source}
USACO JAN25 Contest, Bronze Division, Problem ID: 1469

\end{document}
