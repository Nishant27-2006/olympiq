\documentclass[12pt]{article}
\usepackage{amsmath}
\usepackage{amssymb}
\usepackage{graphicx}
\usepackage{enumitem}
\usepackage{hyperref}
\usepackage{xcolor}

\title{View problem}
\author{USACO Contest: FEB16 Contest - Bronze Division}
\date{\today}

\begin{document}

\maketitle

\section*{Problem Statement}

Farmer John's $N$ cows are each standing at distinct locations
$(x_1, y_1) \ldots (x_n, y_n)$ on his two-dimensional farm ($1 \leq N \leq 100$,
and the $x_i$'s and $y_i$'s are positive odd integers of size at most $B$).  FJ
wants to partition his field by building a  long (effectively infinite-length)
north-south fence with equation $x=a$ ($a$ will be an even integer, thus
ensuring that he does not build the fence through the position of any cow). He
also wants to build a long (effectively infinite-length) east-west fence with
equation $y=b$, where $b$ is an even integer.  These two fences cross at the
point $(a,b)$, and together they partition his field into four regions.  

FJ wants to choose $a$ and $b$ so that the cows appearing in the four resulting
regions are reasonably "balanced", with no region containing too many cows. 
Letting $M$ be the maximum number of cows appearing in one of the four regions,
FJ wants to make $M$ as small as possible.  Please help him determine this
smallest possible value for $M$.

For the first five test cases, $B$ is guaranteed to be at most 100.  In all test
cases, $B$ is guaranteed to be at most 1,000,000.

INPUT FORMAT (file balancing.in):
The first line of the input contains two integers, $N$ and $B$. The next $n$
lines each contain the location of a single cow, specifying its $x$ and $y$
coordinates.

OUTPUT FORMAT (file balancing.out):
You should output the smallest possible value of $M$ that FJ can achieve by
positioning his fences optimally.

SAMPLE INPUT:
7 10
7 3
5 5
9 7
3 1
7 7
5 3
9 1
SAMPLE OUTPUT: 
2

Problem credits: Brian Dean



\section*{Input Format}
OUTPUT FORMAT (file balancing.out):You should output the smallest possible value of $M$ that FJ can achieve by
positioning his fences optimally.SAMPLE INPUT:7 10
7 3
5 5
9 7
3 1
7 7
5 3
9 1SAMPLE OUTPUT:2Problem credits: Brian Dean

\section*{Output Format}
SAMPLE INPUT:7 10
7 3
5 5
9 7
3 1
7 7
5 3
9 1SAMPLE OUTPUT:2Problem credits: Brian Dean

\section*{Sample Input}
\begin{verbatim}
7 10
7 3
5 5
9 7
3 1
7 7
5 3
9 1
\end{verbatim}

\section*{Sample Output}
\begin{verbatim}
2

Problem credits: Brian Dean
\end{verbatim}

\section*{Solution}


\section*{Problem URL}
https://usaco.org/index.php?page=viewproblem2&cpid=617

\section*{Source}
USACO FEB16 Contest, Bronze Division, Problem ID: 617

\end{document}
