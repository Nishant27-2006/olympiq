\documentclass[12pt]{article}
\usepackage{amsmath}
\usepackage{amssymb}
\usepackage{graphicx}
\usepackage{enumitem}
\usepackage{hyperref}
\usepackage{xcolor}

\title{View problem}
\author{USACO Contest: OPEN23 Contest - Bronze Division}
\date{\today}

\begin{document}

\maketitle

\section*{Problem Statement}


Bessie and Elsie are plotting to overthrow Farmer John at last! They plan it out
over $N$ ($1\le N\le 2\cdot 10^5$) text messages. Their conversation can be
represented by a string $S$ of length $N$ where $S_i$ is either $B$ or $E$,
meaning the $i$th message was sent by Bessie or Elsie, respectively. 

However, Farmer John hears of the plan and attempts to intercept their
conversation. Thus, some letters of $S$ are $F$, meaning Farmer John obfuscated
the message and the sender is unknown.

The excitement level of a non-obfuscated conversation is the number of
times a cow double-sends - that is, the number of occurrences of substring $BB$
or $EE$ in $S$. You want to find the excitement level of the original message,
but you don’t know which of Farmer John’s messages were actually Bessie’s
/ Elsie’s. Over all possibilities, output all possible excitement levels of
$S$.

INPUT FORMAT (input arrives from the terminal / stdin):
The first line will consist of one integer $N$.

The next line contains $S$.

OUTPUT FORMAT (print output to the terminal / stdout):
First output $K$, the number of distinct excitement levels possible. On the next
$K$ lines, output the excitement levels, in increasing order.

SAMPLE INPUT:
4
BEEF
SAMPLE OUTPUT: 
2
1
2

SAMPLE INPUT:
9
FEBFEBFEB
SAMPLE OUTPUT: 
2
2
3

SAMPLE INPUT:
10
BFFFFFEBFE
SAMPLE OUTPUT: 
3
2
4
6

SCORING:
Inputs 4-8: $N\le 10$Inputs 9-20: No additional
constraints.


Problem credits: William Yue and Claire Zhang



\section*{Input Format}
The next line contains $S$.

OUTPUT FORMAT (print output to the terminal / stdout):First output $K$, the number of distinct excitement levels possible. On the next
$K$ lines, output the excitement levels, in increasing order.SAMPLE INPUT:4
BEEFSAMPLE OUTPUT:2
1
2SAMPLE INPUT:9
FEBFEBFEBSAMPLE OUTPUT:2
2
3SAMPLE INPUT:10
BFFFFFEBFESAMPLE OUTPUT:3
2
4
6SCORING:Inputs 4-8: $N\le 10$Inputs 9-20: No additional
constraints.Problem credits: William Yue and Claire Zhang

\section*{Output Format}
SAMPLE INPUT:4
BEEFSAMPLE OUTPUT:2
1
2SAMPLE INPUT:9
FEBFEBFEBSAMPLE OUTPUT:2
2
3SAMPLE INPUT:10
BFFFFFEBFESAMPLE OUTPUT:3
2
4
6SCORING:Inputs 4-8: $N\le 10$Inputs 9-20: No additional
constraints.Problem credits: William Yue and Claire Zhang

\section*{Sample Input}
\begin{verbatim}
4
BEEF

9
FEBFEBFEB

10
BFFFFFEBFE
\end{verbatim}

\section*{Sample Output}
\begin{verbatim}
2
1
2

SAMPLE INPUT:9
FEBFEBFEBSAMPLE OUTPUT:2
2
3SAMPLE INPUT:10
BFFFFFEBFESAMPLE OUTPUT:3
2
4
6SCORING:Inputs 4-8: $N\le 10$Inputs 9-20: No additional
constraints.Problem credits: William Yue and Claire Zhang

2
2
3

SAMPLE INPUT:10
BFFFFFEBFESAMPLE OUTPUT:3
2
4
6SCORING:Inputs 4-8: $N\le 10$Inputs 9-20: No additional
constraints.Problem credits: William Yue and Claire Zhang

3
2
4
6

SCORING:Inputs 4-8: $N\le 10$Inputs 9-20: No additional
constraints.Problem credits: William Yue and Claire Zhang
\end{verbatim}

\section*{Solution}


\section*{Problem URL}
https://usaco.org/index.php?page=viewproblem2&cpid=1323

\section*{Source}
USACO OPEN23 Contest, Bronze Division, Problem ID: 1323

\end{document}
