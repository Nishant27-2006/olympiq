\documentclass[12pt]{article}
\usepackage{amsmath}
\usepackage{amssymb}
\usepackage{graphicx}
\usepackage{enumitem}
\usepackage{hyperref}
\usepackage{xcolor}

\title{View problem}
\author{USACO Contest: FEB24 Contest - Bronze Division}
\date{\today}

\begin{document}

\maketitle

\section*{Problem Statement}


Bessie and Elsie are playing a game with a pile of stones that initially
contains $S$ stones ($1\le S<10^{10^5}$). The two cows alternate turns,  with
Bessie going first. When it is a cow's turn, she must remove $x$ stones from the
pile, where $x$ is any positive integer palindrome of the cow's choosing. If the
pile is empty when a cow's turn starts, that cow loses.

Definition: A positive integer is a palindrome if it reads the same
forward and backward; examples of palindromes include 1, 121, and 9009. Leading
zeros are not allowed; e.g., 990 is *not* a palindrome.

There are $T$ ($1\le T\le 10$) independent test cases. For each test case, 
print who wins the game if both cows play optimally.

INPUT FORMAT (input arrives from the terminal / stdin):
The first line contains $T$, the number of test cases. The next $T$ lines
describe the test cases, one line per test case.

Each test case is specified by a single integer $S$.

OUTPUT FORMAT (print output to the terminal / stdout):
For each test case, output B if Bessie wins the game under optimal play starting
with a pile of stones of size $S$, or E otherwise, on a new line.

SAMPLE INPUT:
3
8
10
12
SAMPLE OUTPUT: 
B
E
B

For the first test case, Bessie can remove all the stones on her first move,
since $8$ is a palindrome, guaranteeing her win.

For the second test case, $10$ is not a palindrome, so Bessie cannot remove all
the stones on her first move. Regardless of how many stones  Bessie removes on
her first move, Elsie can always remove all remaining stones on her second move,
guaranteeing her win.

For the third test case, it can be proven that Bessie wins under optimal play.

SCORING:
Inputs 2-4: $S<100$Inputs 5-7: $S<10^6$Inputs 8-10: $S<10^9$Inputs 11-13: No additional constraints.


Problem credits: Nick Wu



\section*{Input Format}
Each test case is specified by a single integer $S$.

OUTPUT FORMAT (print output to the terminal / stdout):For each test case, output B if Bessie wins the game under optimal play starting
with a pile of stones of size $S$, or E otherwise, on a new line.SAMPLE INPUT:3
8
10
12SAMPLE OUTPUT:B
E
BFor the first test case, Bessie can remove all the stones on her first move,
since $8$ is a palindrome, guaranteeing her win.For the second test case, $10$ is not a palindrome, so Bessie cannot remove all
the stones on her first move. Regardless of how many stones  Bessie removes on
her first move, Elsie can always remove all remaining stones on her second move,
guaranteeing her win.For the third test case, it can be proven that Bessie wins under optimal play.SCORING:Inputs 2-4: $S<100$Inputs 5-7: $S<10^6$Inputs 8-10: $S<10^9$Inputs 11-13: No additional constraints.Problem credits: Nick Wu

\section*{Output Format}
SAMPLE INPUT:3
8
10
12SAMPLE OUTPUT:B
E
BFor the first test case, Bessie can remove all the stones on her first move,
since $8$ is a palindrome, guaranteeing her win.For the second test case, $10$ is not a palindrome, so Bessie cannot remove all
the stones on her first move. Regardless of how many stones  Bessie removes on
her first move, Elsie can always remove all remaining stones on her second move,
guaranteeing her win.For the third test case, it can be proven that Bessie wins under optimal play.SCORING:Inputs 2-4: $S<100$Inputs 5-7: $S<10^6$Inputs 8-10: $S<10^9$Inputs 11-13: No additional constraints.Problem credits: Nick Wu

\section*{Sample Input}
\begin{verbatim}
3
8
10
12
\end{verbatim}

\section*{Sample Output}
\begin{verbatim}
B
E
B

For the first test case, Bessie can remove all the stones on her first move,
since $8$ is a palindrome, guaranteeing her win.For the second test case, $10$ is not a palindrome, so Bessie cannot remove all
the stones on her first move. Regardless of how many stones  Bessie removes on
her first move, Elsie can always remove all remaining stones on her second move,
guaranteeing her win.For the third test case, it can be proven that Bessie wins under optimal play.SCORING:Inputs 2-4: $S<100$Inputs 5-7: $S<10^6$Inputs 8-10: $S<10^9$Inputs 11-13: No additional constraints.Problem credits: Nick Wu

For the second test case, $10$ is not a palindrome, so Bessie cannot remove all
the stones on her first move. Regardless of how many stones  Bessie removes on
her first move, Elsie can always remove all remaining stones on her second move,
guaranteeing her win.For the third test case, it can be proven that Bessie wins under optimal play.SCORING:Inputs 2-4: $S<100$Inputs 5-7: $S<10^6$Inputs 8-10: $S<10^9$Inputs 11-13: No additional constraints.Problem credits: Nick Wu

For the third test case, it can be proven that Bessie wins under optimal play.SCORING:Inputs 2-4: $S<100$Inputs 5-7: $S<10^6$Inputs 8-10: $S<10^9$Inputs 11-13: No additional constraints.Problem credits: Nick Wu

SCORING:Inputs 2-4: $S<100$Inputs 5-7: $S<10^6$Inputs 8-10: $S<10^9$Inputs 11-13: No additional constraints.Problem credits: Nick Wu
\end{verbatim}

\section*{Solution}


\section*{Problem URL}
https://usaco.org/index.php?page=viewproblem2&cpid=1395

\section*{Source}
USACO FEB24 Contest, Bronze Division, Problem ID: 1395

\end{document}
