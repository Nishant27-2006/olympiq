\documentclass[12pt]{article}
\usepackage{amsmath}
\usepackage{amssymb}
\usepackage{graphicx}
\usepackage{enumitem}
\usepackage{hyperref}
\usepackage{xcolor}

\title{View problem}
\author{USACO Contest: FEB18 Contest - Bronze Division}
\date{\today}

\begin{document}

\maketitle

\section*{Problem Statement}

One of the farming chores Farmer John dislikes the most is hauling around lots
of cow manure.  In order to streamline this process, he comes up with a
brilliant invention: the manure teleporter!  Instead of hauling manure between
two points in a cart behind his tractor, he can use the manure teleporter to
instantly transport manure from  one location to another.

Farmer John's farm is built along a single long straight road, so any location
on his farm can be described simply using its position along this road
(effectively a point on the number line).  A teleporter is described by two
numbers $x$ and $y$, where manure brought to location $x$ can be instantly
transported to location $y$, or vice versa. 

Farmer John wants to transport manure from location $a$ to location $b$, and he
has built a teleporter that might be helpful during this process (of course, he
doesn't need to use the teleporter if it doesn't help).  Please help him
determine the minimum amount of total distance he needs to haul the manure using
his tractor.

INPUT FORMAT (file teleport.in):
The first and only line of input contains four space-separated integers: $a$ and $b$, describing the start and end locations, followed by $x$ and $y$, describing the teleporter. All positions are integers in the range
$0 \ldots 100$, and they are not necessarily distinct from each-other.

OUTPUT FORMAT (file teleport.out):
Print a single integer giving the minimum distance Farmer John needs to haul
manure in his tractor.

SAMPLE INPUT:
3 10 8 2
SAMPLE OUTPUT: 
3

In this example, the best strategy is to haul the manure from position 3 to
position 2, teleport it to position 8, then haul it to position 10.   The total
distance requiring the tractor is therefore 1 + 2 = 3.


Problem credits: Brian Dean



\section*{Input Format}
OUTPUT FORMAT (file teleport.out):Print a single integer giving the minimum distance Farmer John needs to haul
manure in his tractor.SAMPLE INPUT:3 10 8 2SAMPLE OUTPUT:3In this example, the best strategy is to haul the manure from position 3 to
position 2, teleport it to position 8, then haul it to position 10.   The total
distance requiring the tractor is therefore 1 + 2 = 3.Problem credits: Brian Dean

\section*{Output Format}
SAMPLE INPUT:3 10 8 2SAMPLE OUTPUT:3In this example, the best strategy is to haul the manure from position 3 to
position 2, teleport it to position 8, then haul it to position 10.   The total
distance requiring the tractor is therefore 1 + 2 = 3.Problem credits: Brian Dean

\section*{Sample Input}
\begin{verbatim}
3 10 8 2
\end{verbatim}

\section*{Sample Output}
\begin{verbatim}
3

In this example, the best strategy is to haul the manure from position 3 to
position 2, teleport it to position 8, then haul it to position 10.   The total
distance requiring the tractor is therefore 1 + 2 = 3.Problem credits: Brian Dean

Problem credits: Brian Dean

Problem credits: Brian Dean
\end{verbatim}

\section*{Solution}


\section*{Problem URL}
https://usaco.org/index.php?page=viewproblem2&cpid=807

\section*{Source}
USACO FEB18 Contest, Bronze Division, Problem ID: 807

\end{document}
