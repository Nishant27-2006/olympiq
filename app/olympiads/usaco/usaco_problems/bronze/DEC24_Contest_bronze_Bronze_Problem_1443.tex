\documentclass[12pt]{article}
\usepackage{amsmath}
\usepackage{amssymb}
\usepackage{graphicx}
\usepackage{enumitem}
\usepackage{hyperref}
\usepackage{xcolor}

\title{View problem}
\author{USACO Contest: DEC24 Contest - Bronze Division}
\date{\today}

\begin{document}

\maketitle

\section*{Problem Statement}


Bessie the cow is back in school!  She has started doing her math homework  in
which she is tasked to round positive integers to powers of $10$.

To round a positive integer $a$ to the nearest $10^b$, where $b$ is a positive
integer, Bessie first locates the $b$'th digit from the right. Let $x$ denote
this digit.

If $x \geq 5$, Bessie adds $10^b$ to $a$.

Then, Bessie sets all the digits including and to the right of the $b$'th digit
from the right to be $0$.

For instance, if Bessie wanted to round $456$ to the nearest $10^2$ (hundred),
Bessie would first locate the $2$nd digit from the right which is $5$. This
means $x = 5$. Then since $x \geq 5$, Bessie adds $100$ to $a$. Finally, Bessie
sets all the digits in $a$ to the right of and including the $2$nd digit from
the right to be $0$, resulting in $500$. 

However, if Bessie were to round $446$ to the nearest $10^2$, she would end up
with $400$.

After looking at Bessie's homework, Elsie thinks she has invented a new type of
rounding: chain rounding. To chain round to the nearest $10^b$, Elsie will first
round to the nearest $10^1$, then the nearest $10^2$, and so on until the
nearest $10^b$. 

Bessie thinks Elsie is wrong, but is too busy with math homework to confirm her
suspicions. She tasks you to count how many integers $x$ at least $2$ and at
most $N$ ($1 \leq N \leq 10^{9}$) exist such that rounding $x$ to the nearest
$10^P$ is different than chain rounding to the nearest $10^P$, where $P$ is the
smallest integer such that $10^P \geq x$.

INPUT FORMAT (input arrives from the terminal / stdin):
You have to answer multiple test cases.

The first line of input contains a single integer $T$ ($1 \leq T \leq 10^5$)
denoting the number of test cases. $T$ test cases follow.

The first and only line of input in every test case contains a single integer
$N$. All $N$ within the same input file are guaranteed to be distinct.


OUTPUT FORMAT (print output to the terminal / stdout):
Output $T$ lines, the $i$'th line containing the answer to the $i$'th test case.
Each line should be an integer denoting how many integers at least $2$ and at
most $N$ exist that are different when using the two rounding methods.


SAMPLE INPUT:
4
1
100
4567
3366
SAMPLE OUTPUT: 
0
5
183
60

Consider the second test case in the sample. $48$ should be counted because $48$
chain rounded to the nearest $10^2$ is $100$ ($48\to 50\to 100$), but $48$
rounded to the nearest $10^2$ is
$0$.

In the third test case, two integers counted are $48$ and $480$. $48$ chain 
rounds to $100$ instead of to $0$ and $480$ chain rounds to $1000$ instead of
$0$.  However, $67$ is not counted since it chain rounds to $100$ which is $67$
rounded to the nearest $10^2$.

SCORING:
Inputs 2-4: $N\le 10^3$Inputs 5-7: $N\le 10^6$Inputs 8-13: No additional constraints.


Problem credits: Weiming Zhou



\section*{Input Format}
You have to answer multiple test cases.The first line of input contains a single integer $T$ ($1 \leq T \leq 10^5$)
denoting the number of test cases. $T$ test cases follow.The first and only line of input in every test case contains a single integer
$N$. All $N$ within the same input file are guaranteed to be distinct.

The first line of input contains a single integer $T$ ($1 \leq T \leq 10^5$)
denoting the number of test cases. $T$ test cases follow.The first and only line of input in every test case contains a single integer
$N$. All $N$ within the same input file are guaranteed to be distinct.

The first and only line of input in every test case contains a single integer
$N$. All $N$ within the same input file are guaranteed to be distinct.

OUTPUT FORMAT (print output to the terminal / stdout):Output $T$ lines, the $i$'th line containing the answer to the $i$'th test case.
Each line should be an integer denoting how many integers at least $2$ and at
most $N$ exist that are different when using the two rounding methods.SAMPLE INPUT:4
1
100
4567
3366SAMPLE OUTPUT:0
5
183
60Consider the second test case in the sample. $48$ should be counted because $48$
chain rounded to the nearest $10^2$ is $100$ ($48\to 50\to 100$), but $48$
rounded to the nearest $10^2$ is
$0$.In the third test case, two integers counted are $48$ and $480$. $48$ chain 
rounds to $100$ instead of to $0$ and $480$ chain rounds to $1000$ instead of
$0$.  However, $67$ is not counted since it chain rounds to $100$ which is $67$
rounded to the nearest $10^2$.SCORING:Inputs 2-4: $N\le 10^3$Inputs 5-7: $N\le 10^6$Inputs 8-13: No additional constraints.Problem credits: Weiming Zhou

\section*{Output Format}
Output $T$ lines, the $i$'th line containing the answer to the $i$'th test case.
Each line should be an integer denoting how many integers at least $2$ and at
most $N$ exist that are different when using the two rounding methods.

SAMPLE INPUT:4
1
100
4567
3366SAMPLE OUTPUT:0
5
183
60Consider the second test case in the sample. $48$ should be counted because $48$
chain rounded to the nearest $10^2$ is $100$ ($48\to 50\to 100$), but $48$
rounded to the nearest $10^2$ is
$0$.In the third test case, two integers counted are $48$ and $480$. $48$ chain 
rounds to $100$ instead of to $0$ and $480$ chain rounds to $1000$ instead of
$0$.  However, $67$ is not counted since it chain rounds to $100$ which is $67$
rounded to the nearest $10^2$.SCORING:Inputs 2-4: $N\le 10^3$Inputs 5-7: $N\le 10^6$Inputs 8-13: No additional constraints.Problem credits: Weiming Zhou

\section*{Sample Input}
\begin{verbatim}
4
1
100
4567
3366
\end{verbatim}

\section*{Sample Output}
\begin{verbatim}
0
5
183
60

Consider the second test case in the sample. $48$ should be counted because $48$
chain rounded to the nearest $10^2$ is $100$ ($48\to 50\to 100$), but $48$
rounded to the nearest $10^2$ is
$0$.In the third test case, two integers counted are $48$ and $480$. $48$ chain 
rounds to $100$ instead of to $0$ and $480$ chain rounds to $1000$ instead of
$0$.  However, $67$ is not counted since it chain rounds to $100$ which is $67$
rounded to the nearest $10^2$.SCORING:Inputs 2-4: $N\le 10^3$Inputs 5-7: $N\le 10^6$Inputs 8-13: No additional constraints.Problem credits: Weiming Zhou

In the third test case, two integers counted are $48$ and $480$. $48$ chain 
rounds to $100$ instead of to $0$ and $480$ chain rounds to $1000$ instead of
$0$.  However, $67$ is not counted since it chain rounds to $100$ which is $67$
rounded to the nearest $10^2$.SCORING:Inputs 2-4: $N\le 10^3$Inputs 5-7: $N\le 10^6$Inputs 8-13: No additional constraints.Problem credits: Weiming Zhou

SCORING:Inputs 2-4: $N\le 10^3$Inputs 5-7: $N\le 10^6$Inputs 8-13: No additional constraints.Problem credits: Weiming Zhou
\end{verbatim}

\section*{Solution}


\section*{Problem URL}
https://usaco.org/index.php?page=viewproblem2&cpid=1443

\section*{Source}
USACO DEC24 Contest, Bronze Division, Problem ID: 1443

\end{document}
