\documentclass[12pt]{article}
\usepackage{amsmath}
\usepackage{amssymb}
\usepackage{graphicx}
\usepackage{enumitem}
\usepackage{hyperref}
\usepackage{xcolor}

\title{View problem}
\author{USACO Contest: DEC24 Contest - Bronze Division}
\date{\today}

\begin{document}

\maketitle

\section*{Problem Statement}


Farmer John is trying to describe his favorite USACO contest to Elsie, but she
is having trouble understanding why he likes it so much. He says "My favorite
part of the contest was when Bessie said 'It's Mooin' Time' and mooed all over
the contest."

Elsie still doesn't understand so Farmer John downloads the contest as a text
file and  tries to explain what he means. The contest is defined as a string of
lowercase letters of length $N$ ($3 \leq N \leq 20\,000$). A moo is generally
defined as the substring $c_ic_jc_j$ where some character $c_i$ followed
directly by $2$ occurrences of some  character $c_j$ where $c_i \neq c_j$.
According to Farmer John, Bessie moos a lot, so if some moo appears at least $F$
($1\le F\le N$) times in the contest, that might be from Bessie.  

However, Farmer John's download might have been corrupted, and the text file
might have up to one character that differs from the original file. Print all
possible moos that  Bessie could have made taking the potential error into
account, sorted in alphabetical order.

INPUT FORMAT (input arrives from the terminal / stdin):
The first line contains $N$ and $F$, representing the length of the string and
the frequency threshold for a moo by Bessie. 

The second line contains a string of lowercase letters of length $N$,
representing the contest.

OUTPUT FORMAT (print output to the terminal / stdout):
Print out the number of possible moos that Bessie makes, followed by a
lexicographically sorted list of the moos. Each moo should appear on a separate
line.

SAMPLE INPUT:
10 2
zzmoozzmoo
SAMPLE OUTPUT: 
1
moo

In this case, no character change affects the answer. The only moo Bessie made
was "moo".

SAMPLE INPUT:
17 2
momoobaaaaaqqqcqq
SAMPLE OUTPUT: 
3
aqq
baa
cqq

In this case, the $a$ at position $8$ (zero-indexed) could have been corrupted from a $b$ which
would have resulted in "baa" being a moo that Bessie made twice. Alternatively,
the $q$ at position $11$ could have been corrupted from a $c$ which would have
resulted in "cqq" being a possible moo that Bessie made. "aqq" can be made by
swapping the $c$ with an $a$.

SAMPLE INPUT:
3 1
ooo
SAMPLE OUTPUT: 
25
aoo
boo
coo
doo
eoo
foo
goo
hoo
ioo
joo
koo
loo
moo
noo
poo
qoo
roo
soo
too
uoo
voo
woo
xoo
yoo
zoo

SCORING:
Inputs 4-8: $N \le 100$ Inputs 9-13: No additional constraints.



Problem credits: Suhas Nagar



\section*{Input Format}
The second line contains a string of lowercase letters of length $N$,
representing the contest.

OUTPUT FORMAT (print output to the terminal / stdout):Print out the number of possible moos that Bessie makes, followed by a
lexicographically sorted list of the moos. Each moo should appear on a separate
line.SAMPLE INPUT:10 2
zzmoozzmooSAMPLE OUTPUT:1
mooIn this case, no character change affects the answer. The only moo Bessie made
was "moo".SAMPLE INPUT:17 2
momoobaaaaaqqqcqqSAMPLE OUTPUT:3
aqq
baa
cqqIn this case, the $a$ at position $8$ (zero-indexed) could have been corrupted from a $b$ which
would have resulted in "baa" being a moo that Bessie made twice. Alternatively,
the $q$ at position $11$ could have been corrupted from a $c$ which would have
resulted in "cqq" being a possible moo that Bessie made. "aqq" can be made by
swapping the $c$ with an $a$.SAMPLE INPUT:3 1
oooSAMPLE OUTPUT:25
aoo
boo
coo
doo
eoo
foo
goo
hoo
ioo
joo
koo
loo
moo
noo
poo
qoo
roo
soo
too
uoo
voo
woo
xoo
yoo
zooSCORING:Inputs 4-8: $N \le 100$Inputs 9-13: No additional constraints.Problem credits: Suhas Nagar

\section*{Output Format}
SAMPLE INPUT:10 2
zzmoozzmooSAMPLE OUTPUT:1
mooIn this case, no character change affects the answer. The only moo Bessie made
was "moo".SAMPLE INPUT:17 2
momoobaaaaaqqqcqqSAMPLE OUTPUT:3
aqq
baa
cqqIn this case, the $a$ at position $8$ (zero-indexed) could have been corrupted from a $b$ which
would have resulted in "baa" being a moo that Bessie made twice. Alternatively,
the $q$ at position $11$ could have been corrupted from a $c$ which would have
resulted in "cqq" being a possible moo that Bessie made. "aqq" can be made by
swapping the $c$ with an $a$.SAMPLE INPUT:3 1
oooSAMPLE OUTPUT:25
aoo
boo
coo
doo
eoo
foo
goo
hoo
ioo
joo
koo
loo
moo
noo
poo
qoo
roo
soo
too
uoo
voo
woo
xoo
yoo
zooSCORING:Inputs 4-8: $N \le 100$Inputs 9-13: No additional constraints.Problem credits: Suhas Nagar

\section*{Sample Input}
\begin{verbatim}
10 2
zzmoozzmoo

17 2
momoobaaaaaqqqcqq

3 1
ooo
\end{verbatim}

\section*{Sample Output}
\begin{verbatim}
1
moo

SAMPLE INPUT:17 2
momoobaaaaaqqqcqqSAMPLE OUTPUT:3
aqq
baa
cqqIn this case, the $a$ at position $8$ (zero-indexed) could have been corrupted from a $b$ which
would have resulted in "baa" being a moo that Bessie made twice. Alternatively,
the $q$ at position $11$ could have been corrupted from a $c$ which would have
resulted in "cqq" being a possible moo that Bessie made. "aqq" can be made by
swapping the $c$ with an $a$.SAMPLE INPUT:3 1
oooSAMPLE OUTPUT:25
aoo
boo
coo
doo
eoo
foo
goo
hoo
ioo
joo
koo
loo
moo
noo
poo
qoo
roo
soo
too
uoo
voo
woo
xoo
yoo
zooSCORING:Inputs 4-8: $N \le 100$Inputs 9-13: No additional constraints.Problem credits: Suhas Nagar

3
aqq
baa
cqq

SAMPLE INPUT:3 1
oooSAMPLE OUTPUT:25
aoo
boo
coo
doo
eoo
foo
goo
hoo
ioo
joo
koo
loo
moo
noo
poo
qoo
roo
soo
too
uoo
voo
woo
xoo
yoo
zooSCORING:Inputs 4-8: $N \le 100$Inputs 9-13: No additional constraints.Problem credits: Suhas Nagar

25
aoo
boo
coo
doo
eoo
foo
goo
hoo
ioo
joo
koo
loo
moo
noo
poo
qoo
roo
soo
too
uoo
voo
woo
xoo
yoo
zoo

SCORING:Inputs 4-8: $N \le 100$Inputs 9-13: No additional constraints.Problem credits: Suhas Nagar
\end{verbatim}

\section*{Solution}


\section*{Problem URL}
https://usaco.org/index.php?page=viewproblem2&cpid=1445

\section*{Source}
USACO DEC24 Contest, Bronze Division, Problem ID: 1445

\end{document}
