\documentclass[12pt]{article}
\usepackage{amsmath}
\usepackage{amssymb}
\usepackage{graphicx}
\usepackage{enumitem}
\usepackage{hyperref}
\usepackage{xcolor}

\title{View problem}
\author{USACO Contest: DEC23 Contest - Bronze Division}
\date{\today}

\begin{document}

\maketitle

\section*{Problem Statement}


Farmer John's cows have quite the sweet tooth, and they especially enjoy eating
candy canes!  FJ has $N$ total cows, each with a certain initial height and he
wants to feed them  $M$ candy canes, each also of varying height
($1\le N,M\le 2\cdot 10^5$).  

FJ plans to feed the candy canes one by one to the cows, in the order they are
given in the input.  To feed a candy cane to his cows, he will hang the candy
cane so that initially the candy cane is just touching the ground. The cows will
then line up one by one, in the order given by the input, and go up to the candy
cane, each eating up to their height (since they cannot reach any higher). The
candy cane stays suspended in place where it is initially set up  and is not
lowered to the ground, even after cows eat the bottom of the candy cane. It is 
possible a cow may eat nothing during her turn, if the base of the candy cane is
already above that cow's height.  After every cow has had their turn, the cows
grow in height by how many units of candy cane they ate, and Farmer John hangs
the next candy cane and the cows repeat the process again (with cow 1
again being the first to start eating the next candy cane).

INPUT FORMAT (input arrives from the terminal / stdin):
The first line contains $N$ and $M$.

The next line contains the initial heights of the $N$ cows, each in the range
$[1,10^9]$.

The next line contains the heights of the $M$ candy canes, each in the range
$[1,10^9]$.

OUTPUT FORMAT (print output to the terminal / stdout):
The final heights of each of the $N$ cows on separate lines.

Note that the large size of integers involved in this problem may require the
use of 64-bit integer data types (e.g., a "long long" in C/C++).

SAMPLE INPUT:
3 2
3 2 5
6 1
SAMPLE OUTPUT: 
7
2
7

The first candy cane is $6$ units tall. 
The first cow eats the portion of the first candy cane up to height $3$,
after which the remaining portion of the first candy cane occupies heights
$[3,6]$. The second cow is not tall enough to eat any of the remaining portion of the
first candy cane.
The third cow eats two additional units of the first candy cane. The
remaining portion of the first candy cane, occupying heights $[5,6]$, is not
eaten.
Next, each cow grows by the amount she ate, so the heights of the cows  become
$[3+3, 2+0, 5+2]=[6, 2, 7]$.

The second candy cane is $1$ unit tall, and the first cow eats all of it.

SCORING:
Inputs 2-10: $N, M \le 10^3$Inputs 11-14: No additional constraints.


Problem credits: Agastya Goel



\section*{Input Format}
The next line contains the initial heights of the $N$ cows, each in the range
$[1,10^9]$.The next line contains the heights of the $M$ candy canes, each in the range
$[1,10^9]$.

The next line contains the heights of the $M$ candy canes, each in the range
$[1,10^9]$.

OUTPUT FORMAT (print output to the terminal / stdout):The final heights of each of the $N$ cows on separate lines.Note that the large size of integers involved in this problem may require the
use of 64-bit integer data types (e.g., a "long long" in C/C++).SAMPLE INPUT:3 2
3 2 5
6 1SAMPLE OUTPUT:7
2
7The first candy cane is $6$ units tall.The first cow eats the portion of the first candy cane up to height $3$,
after which the remaining portion of the first candy cane occupies heights
$[3,6]$.The second cow is not tall enough to eat any of the remaining portion of the
first candy cane.The third cow eats two additional units of the first candy cane. The
remaining portion of the first candy cane, occupying heights $[5,6]$, is not
eaten.Next, each cow grows by the amount she ate, so the heights of the cows  become
$[3+3, 2+0, 5+2]=[6, 2, 7]$.The second candy cane is $1$ unit tall, and the first cow eats all of it.SCORING:Inputs 2-10: $N, M \le 10^3$Inputs 11-14: No additional constraints.Problem credits: Agastya Goel

\section*{Output Format}
Note that the large size of integers involved in this problem may require the
use of 64-bit integer data types (e.g., a "long long" in C/C++).

SAMPLE INPUT:3 2
3 2 5
6 1SAMPLE OUTPUT:7
2
7The first candy cane is $6$ units tall.The first cow eats the portion of the first candy cane up to height $3$,
after which the remaining portion of the first candy cane occupies heights
$[3,6]$.The second cow is not tall enough to eat any of the remaining portion of the
first candy cane.The third cow eats two additional units of the first candy cane. The
remaining portion of the first candy cane, occupying heights $[5,6]$, is not
eaten.Next, each cow grows by the amount she ate, so the heights of the cows  become
$[3+3, 2+0, 5+2]=[6, 2, 7]$.The second candy cane is $1$ unit tall, and the first cow eats all of it.SCORING:Inputs 2-10: $N, M \le 10^3$Inputs 11-14: No additional constraints.Problem credits: Agastya Goel

\section*{Sample Input}
\begin{verbatim}
3 2
3 2 5
6 1
\end{verbatim}

\section*{Sample Output}
\begin{verbatim}
7
2
7

The first candy cane is $6$ units tall.The first cow eats the portion of the first candy cane up to height $3$,
after which the remaining portion of the first candy cane occupies heights
$[3,6]$.The second cow is not tall enough to eat any of the remaining portion of the
first candy cane.The third cow eats two additional units of the first candy cane. The
remaining portion of the first candy cane, occupying heights $[5,6]$, is not
eaten.Next, each cow grows by the amount she ate, so the heights of the cows  become
$[3+3, 2+0, 5+2]=[6, 2, 7]$.The second candy cane is $1$ unit tall, and the first cow eats all of it.SCORING:Inputs 2-10: $N, M \le 10^3$Inputs 11-14: No additional constraints.Problem credits: Agastya Goel

Next, each cow grows by the amount she ate, so the heights of the cows  become
$[3+3, 2+0, 5+2]=[6, 2, 7]$.The second candy cane is $1$ unit tall, and the first cow eats all of it.SCORING:Inputs 2-10: $N, M \le 10^3$Inputs 11-14: No additional constraints.Problem credits: Agastya Goel

The second candy cane is $1$ unit tall, and the first cow eats all of it.SCORING:Inputs 2-10: $N, M \le 10^3$Inputs 11-14: No additional constraints.Problem credits: Agastya Goel

SCORING:Inputs 2-10: $N, M \le 10^3$Inputs 11-14: No additional constraints.Problem credits: Agastya Goel
\end{verbatim}

\section*{Solution}


\section*{Problem URL}
https://usaco.org/index.php?page=viewproblem2&cpid=1347

\section*{Source}
USACO DEC23 Contest, Bronze Division, Problem ID: 1347

\end{document}
