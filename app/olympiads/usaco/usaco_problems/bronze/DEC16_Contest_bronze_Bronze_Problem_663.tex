\documentclass[12pt]{article}
\usepackage{amsmath}
\usepackage{amssymb}
\usepackage{graphicx}
\usepackage{enumitem}
\usepackage{hyperref}
\usepackage{xcolor}

\title{View problem}
\author{USACO Contest: DEC16 Contest - Bronze Division}
\date{\today}

\begin{document}

\maketitle

\section*{Problem Statement}

Farmer John has decided to update his farm to simplify its geometry. 
Previously, his cows grazed in two rectangular fenced-in pastures.  Farmer John
would like to replace these with a single square fenced-in pasture of minimum
size that still covers all the regions of his farm that were previously enclosed
by the former two fences.

Please help Farmer John figure out the minimum area he needs to make his new
square pasture so that if he places it appropriately, it can still cover all the
area formerly covered by the two older rectangular pastures.  The square
pasture should have its sides parallel to the x and y axes.

INPUT FORMAT (file square.in):
The first line in the input file specifies one of the original rectangular
pastures with four space-separated integers $x_1$ $y_1$ $x_2$ $y_2$,  each in
the range $0 \ldots 10$.  The lower-left corner of the pasture is at the point
$(x_1, y_1)$, and the upper-right corner is at the point $(x_2, y_2)$, where 
$x_2 > x_1$ and $y_2 > y_1$.

The second line of input has the same 4-integer format as the first line, and
specifies the second original rectangular pasture.  This pasture will not
overlap or touch the first pasture.

OUTPUT FORMAT (file square.out):
The output should consist of one line containing the minimum area required of a
square pasture that would cover all the regions originally enclosed by the two
rectangular pastures.  

SAMPLE INPUT:
6 6 8 8
1 8 4 9
SAMPLE OUTPUT: 
49

In the example above, the first original rectangle has corners $(6,6)$ and
$(8,8)$. The second has corners at $(1,8)$ and $(4,9)$.  By drawing a square fence
of side length 7 with corners $(1,6)$ and $(8,13)$, the original areas can still be enclosed; moreover, this is the best possible, since it is impossible to enclose the original areas with a square of side length only 6.  Note that there are several different possible valid placements for the square of side length 7, as it could have been shifted vertically a bit.


Problem credits: Brian Dean



\section*{Input Format}
The second line of input has the same 4-integer format as the first line, and
specifies the second original rectangular pasture.  This pasture will not
overlap or touch the first pasture.

OUTPUT FORMAT (file square.out):The output should consist of one line containing the minimum area required of a
square pasture that would cover all the regions originally enclosed by the two
rectangular pastures.SAMPLE INPUT:6 6 8 8
1 8 4 9SAMPLE OUTPUT:49In the example above, the first original rectangle has corners $(6,6)$ and
$(8,8)$. The second has corners at $(1,8)$ and $(4,9)$.  By drawing a square fence
of side length 7 with corners $(1,6)$ and $(8,13)$, the original areas can still be enclosed; moreover, this is the best possible, since it is impossible to enclose the original areas with a square of side length only 6.  Note that there are several different possible valid placements for the square of side length 7, as it could have been shifted vertically a bit.Problem credits: Brian Dean

\section*{Output Format}
SAMPLE INPUT:6 6 8 8
1 8 4 9SAMPLE OUTPUT:49In the example above, the first original rectangle has corners $(6,6)$ and
$(8,8)$. The second has corners at $(1,8)$ and $(4,9)$.  By drawing a square fence
of side length 7 with corners $(1,6)$ and $(8,13)$, the original areas can still be enclosed; moreover, this is the best possible, since it is impossible to enclose the original areas with a square of side length only 6.  Note that there are several different possible valid placements for the square of side length 7, as it could have been shifted vertically a bit.Problem credits: Brian Dean

\section*{Sample Input}
\begin{verbatim}
6 6 8 8
1 8 4 9
\end{verbatim}

\section*{Sample Output}
\begin{verbatim}
49

In the example above, the first original rectangle has corners $(6,6)$ and
$(8,8)$. The second has corners at $(1,8)$ and $(4,9)$.  By drawing a square fence
of side length 7 with corners $(1,6)$ and $(8,13)$, the original areas can still be enclosed; moreover, this is the best possible, since it is impossible to enclose the original areas with a square of side length only 6.  Note that there are several different possible valid placements for the square of side length 7, as it could have been shifted vertically a bit.Problem credits: Brian Dean

Problem credits: Brian Dean

Problem credits: Brian Dean
\end{verbatim}

\section*{Solution}


\section*{Problem URL}
https://usaco.org/index.php?page=viewproblem2&cpid=663

\section*{Source}
USACO DEC16 Contest, Bronze Division, Problem ID: 663

\end{document}
