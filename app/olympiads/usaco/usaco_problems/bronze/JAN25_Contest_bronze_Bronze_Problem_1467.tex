\documentclass[12pt]{article}
\usepackage{amsmath}
\usepackage{amssymb}
\usepackage{graphicx}
\usepackage{enumitem}
\usepackage{hyperref}
\usepackage{xcolor}

\title{View problem}
\author{USACO Contest: JAN25 Contest - Bronze Division}
\date{\today}

\begin{document}

\maketitle

\section*{Problem Statement}


**Note: The time limit for this problem is 4s, twice the default.**
Bessie is using her nifty telescope to take photos of all the stars in the night
sky. Her telescope can capture an $N \times N$ ($1 \leq N \leq 1000$) photo of
the stars where each pixel is either a star or empty sky. Each star will be
represented by exactly one pixel, and no two distinct stars share the same
pixel.

Overnight, something strange happens to the stars in the sky. Every star either
disappears or moves $A$ pixels to the right, and $B$ pixels downwards
($0 \leq A,B \leq N$). If a star disappears or moves beyond the photo boundary,
it no longer appears in the second photo.

Bessie took photos before and after the shifting positions, but after
experimenting in  Mootoshop, she accidentally superimposed one photo onto the
other. Now, she can see white pixels where both photos were empty, gray pixels
where stars existed in exactly one photo, and black pixels where there was a
star in both photos. Bessie also remembers that no new stars moved into the
frame of the second photo, so her first photo contains all of the stars in the
night sky.

Given the final photo, determine the minimum possible number of stars in the sky
before the shifting incident for $T$ ($1 \leq T \leq 1000$) independent test
cases. If no arrangement of stars can produce the given final photo, output
$-1$.

INPUT FORMAT (input arrives from the terminal / stdin):
The first line of input contains $T$ and $T$ test cases will follow.

The first line of each test case will contain $N$ $A$ $B$.

Then follow $N$ lines each representing one row of the superimposed photo.  The
$i$th row from the top is represented by a string $c_{i,1}c_{i,2}\dots c_{i,N}$,
where each $c_{i,j} \in \{W,G,B\}$, representing the colors white, gray, and
black respectively.

It is guaranteed that the sum of $N^2$ over all test cases will not exceed
$10^7$.

OUTPUT FORMAT (print output to the terminal / stdout):
For each test case, output the minimum number of stars that existed before the
shifting, or $-1$ if impossible.

SAMPLE INPUT:
1
3 0 0
WWB
BBB
GGG
SAMPLE OUTPUT: 
7

In this example, there is no shifting. The first photo is as follows: (. for
sky, * for star)


..*
***
***

The second photo, where the stars on the bottom row disappeared, is as follows:


..*
***
...

This is the only way to produce the superimposed photo, so the minimum possible
number of initial stars is $7$.

SAMPLE INPUT:
3
5 1 2
GWGWW
WGWWW
WBWGW
WWWWW
WWGWW
3 1 1
WWW
WBW
WWW
3 1 0
GGB
GGW
WWW
SAMPLE OUTPUT: 
4
-1
4

In the first case, there were at least $4$ stars at the start. If we let $(r,c)$
denote the intersection of the $r$th row from the top and $c$th column from the
left, one possibility is that they were initially at $(1,1), (3,2), (2,2),$  and
$(1,3)$. All the stars shifted, except for the one at $(2,2)$ which disappeared.

In the second case, there is no arrangement of stars in the initial photo that
can produce the middle black pixel given the shift.

In the third case, there were at least $4$ stars at the start. One possibility
is that they were initially at $(1,1), (1,2), (1,3),$ and $(2,1)$. In the second
photo, the star originally at $(1,1)$ disappeared and the star originally at
$(1,3)$ moved off frame. The other two stars shifted to the right by 1.

SCORING:
Input 3: $A=B=0$Inputs 4-7: $A=1, B=0, N\le 10$Inputs 8-9: $A=1, B=0$Inputs 10-12: No additional constraints.


Problem credits: Suhas Nagar



\section*{Input Format}
The first line of each test case will contain $N$ $A$ $B$.Then follow $N$ lines each representing one row of the superimposed photo.  The
$i$th row from the top is represented by a string $c_{i,1}c_{i,2}\dots c_{i,N}$,
where each $c_{i,j} \in \{W,G,B\}$, representing the colors white, gray, and
black respectively.It is guaranteed that the sum of $N^2$ over all test cases will not exceed
$10^7$.

Then follow $N$ lines each representing one row of the superimposed photo.  The
$i$th row from the top is represented by a string $c_{i,1}c_{i,2}\dots c_{i,N}$,
where each $c_{i,j} \in \{W,G,B\}$, representing the colors white, gray, and
black respectively.It is guaranteed that the sum of $N^2$ over all test cases will not exceed
$10^7$.

It is guaranteed that the sum of $N^2$ over all test cases will not exceed
$10^7$.

OUTPUT FORMAT (print output to the terminal / stdout):For each test case, output the minimum number of stars that existed before the
shifting, or $-1$ if impossible.SAMPLE INPUT:1
3 0 0
WWB
BBB
GGGSAMPLE OUTPUT:7In this example, there is no shifting. The first photo is as follows: (. for
sky, * for star)..*
***
***The second photo, where the stars on the bottom row disappeared, is as follows:..*
***
...This is the only way to produce the superimposed photo, so the minimum possible
number of initial stars is $7$.SAMPLE INPUT:3
5 1 2
GWGWW
WGWWW
WBWGW
WWWWW
WWGWW
3 1 1
WWW
WBW
WWW
3 1 0
GGB
GGW
WWWSAMPLE OUTPUT:4
-1
4In the first case, there were at least $4$ stars at the start. If we let $(r,c)$
denote the intersection of the $r$th row from the top and $c$th column from the
left, one possibility is that they were initially at $(1,1), (3,2), (2,2),$  and
$(1,3)$. All the stars shifted, except for the one at $(2,2)$ which disappeared.In the second case, there is no arrangement of stars in the initial photo that
can produce the middle black pixel given the shift.In the third case, there were at least $4$ stars at the start. One possibility
is that they were initially at $(1,1), (1,2), (1,3),$ and $(2,1)$. In the second
photo, the star originally at $(1,1)$ disappeared and the star originally at
$(1,3)$ moved off frame. The other two stars shifted to the right by 1.SCORING:Input 3: $A=B=0$Inputs 4-7: $A=1, B=0, N\le 10$Inputs 8-9: $A=1, B=0$Inputs 10-12: No additional constraints.Problem credits: Suhas Nagar

\section*{Output Format}
SAMPLE INPUT:1
3 0 0
WWB
BBB
GGGSAMPLE OUTPUT:7In this example, there is no shifting. The first photo is as follows: (. for
sky, * for star)..*
***
***The second photo, where the stars on the bottom row disappeared, is as follows:..*
***
...This is the only way to produce the superimposed photo, so the minimum possible
number of initial stars is $7$.SAMPLE INPUT:3
5 1 2
GWGWW
WGWWW
WBWGW
WWWWW
WWGWW
3 1 1
WWW
WBW
WWW
3 1 0
GGB
GGW
WWWSAMPLE OUTPUT:4
-1
4In the first case, there were at least $4$ stars at the start. If we let $(r,c)$
denote the intersection of the $r$th row from the top and $c$th column from the
left, one possibility is that they were initially at $(1,1), (3,2), (2,2),$  and
$(1,3)$. All the stars shifted, except for the one at $(2,2)$ which disappeared.In the second case, there is no arrangement of stars in the initial photo that
can produce the middle black pixel given the shift.In the third case, there were at least $4$ stars at the start. One possibility
is that they were initially at $(1,1), (1,2), (1,3),$ and $(2,1)$. In the second
photo, the star originally at $(1,1)$ disappeared and the star originally at
$(1,3)$ moved off frame. The other two stars shifted to the right by 1.SCORING:Input 3: $A=B=0$Inputs 4-7: $A=1, B=0, N\le 10$Inputs 8-9: $A=1, B=0$Inputs 10-12: No additional constraints.Problem credits: Suhas Nagar

\section*{Sample Input}
\begin{verbatim}
1
3 0 0
WWB
BBB
GGG

3
5 1 2
GWGWW
WGWWW
WBWGW
WWWWW
WWGWW
3 1 1
WWW
WBW
WWW
3 1 0
GGB
GGW
WWW
\end{verbatim}

\section*{Sample Output}
\begin{verbatim}
7

..*
***
***The second photo, where the stars on the bottom row disappeared, is as follows:..*
***
...This is the only way to produce the superimposed photo, so the minimum possible
number of initial stars is $7$.SAMPLE INPUT:3
5 1 2
GWGWW
WGWWW
WBWGW
WWWWW
WWGWW
3 1 1
WWW
WBW
WWW
3 1 0
GGB
GGW
WWWSAMPLE OUTPUT:4
-1
4In the first case, there were at least $4$ stars at the start. If we let $(r,c)$
denote the intersection of the $r$th row from the top and $c$th column from the
left, one possibility is that they were initially at $(1,1), (3,2), (2,2),$  and
$(1,3)$. All the stars shifted, except for the one at $(2,2)$ which disappeared.In the second case, there is no arrangement of stars in the initial photo that
can produce the middle black pixel given the shift.In the third case, there were at least $4$ stars at the start. One possibility
is that they were initially at $(1,1), (1,2), (1,3),$ and $(2,1)$. In the second
photo, the star originally at $(1,1)$ disappeared and the star originally at
$(1,3)$ moved off frame. The other two stars shifted to the right by 1.SCORING:Input 3: $A=B=0$Inputs 4-7: $A=1, B=0, N\le 10$Inputs 8-9: $A=1, B=0$Inputs 10-12: No additional constraints.Problem credits: Suhas Nagar

..*
***
***

The second photo, where the stars on the bottom row disappeared, is as follows:..*
***
...This is the only way to produce the superimposed photo, so the minimum possible
number of initial stars is $7$.SAMPLE INPUT:3
5 1 2
GWGWW
WGWWW
WBWGW
WWWWW
WWGWW
3 1 1
WWW
WBW
WWW
3 1 0
GGB
GGW
WWWSAMPLE OUTPUT:4
-1
4In the first case, there were at least $4$ stars at the start. If we let $(r,c)$
denote the intersection of the $r$th row from the top and $c$th column from the
left, one possibility is that they were initially at $(1,1), (3,2), (2,2),$  and
$(1,3)$. All the stars shifted, except for the one at $(2,2)$ which disappeared.In the second case, there is no arrangement of stars in the initial photo that
can produce the middle black pixel given the shift.In the third case, there were at least $4$ stars at the start. One possibility
is that they were initially at $(1,1), (1,2), (1,3),$ and $(2,1)$. In the second
photo, the star originally at $(1,1)$ disappeared and the star originally at
$(1,3)$ moved off frame. The other two stars shifted to the right by 1.SCORING:Input 3: $A=B=0$Inputs 4-7: $A=1, B=0, N\le 10$Inputs 8-9: $A=1, B=0$Inputs 10-12: No additional constraints.Problem credits: Suhas Nagar

..*
***
...This is the only way to produce the superimposed photo, so the minimum possible
number of initial stars is $7$.SAMPLE INPUT:3
5 1 2
GWGWW
WGWWW
WBWGW
WWWWW
WWGWW
3 1 1
WWW
WBW
WWW
3 1 0
GGB
GGW
WWWSAMPLE OUTPUT:4
-1
4In the first case, there were at least $4$ stars at the start. If we let $(r,c)$
denote the intersection of the $r$th row from the top and $c$th column from the
left, one possibility is that they were initially at $(1,1), (3,2), (2,2),$  and
$(1,3)$. All the stars shifted, except for the one at $(2,2)$ which disappeared.In the second case, there is no arrangement of stars in the initial photo that
can produce the middle black pixel given the shift.In the third case, there were at least $4$ stars at the start. One possibility
is that they were initially at $(1,1), (1,2), (1,3),$ and $(2,1)$. In the second
photo, the star originally at $(1,1)$ disappeared and the star originally at
$(1,3)$ moved off frame. The other two stars shifted to the right by 1.SCORING:Input 3: $A=B=0$Inputs 4-7: $A=1, B=0, N\le 10$Inputs 8-9: $A=1, B=0$Inputs 10-12: No additional constraints.Problem credits: Suhas Nagar

..*
***
...

This is the only way to produce the superimposed photo, so the minimum possible
number of initial stars is $7$.SAMPLE INPUT:3
5 1 2
GWGWW
WGWWW
WBWGW
WWWWW
WWGWW
3 1 1
WWW
WBW
WWW
3 1 0
GGB
GGW
WWWSAMPLE OUTPUT:4
-1
4In the first case, there were at least $4$ stars at the start. If we let $(r,c)$
denote the intersection of the $r$th row from the top and $c$th column from the
left, one possibility is that they were initially at $(1,1), (3,2), (2,2),$  and
$(1,3)$. All the stars shifted, except for the one at $(2,2)$ which disappeared.In the second case, there is no arrangement of stars in the initial photo that
can produce the middle black pixel given the shift.In the third case, there were at least $4$ stars at the start. One possibility
is that they were initially at $(1,1), (1,2), (1,3),$ and $(2,1)$. In the second
photo, the star originally at $(1,1)$ disappeared and the star originally at
$(1,3)$ moved off frame. The other two stars shifted to the right by 1.SCORING:Input 3: $A=B=0$Inputs 4-7: $A=1, B=0, N\le 10$Inputs 8-9: $A=1, B=0$Inputs 10-12: No additional constraints.Problem credits: Suhas Nagar

SAMPLE INPUT:3
5 1 2
GWGWW
WGWWW
WBWGW
WWWWW
WWGWW
3 1 1
WWW
WBW
WWW
3 1 0
GGB
GGW
WWWSAMPLE OUTPUT:4
-1
4In the first case, there were at least $4$ stars at the start. If we let $(r,c)$
denote the intersection of the $r$th row from the top and $c$th column from the
left, one possibility is that they were initially at $(1,1), (3,2), (2,2),$  and
$(1,3)$. All the stars shifted, except for the one at $(2,2)$ which disappeared.In the second case, there is no arrangement of stars in the initial photo that
can produce the middle black pixel given the shift.In the third case, there were at least $4$ stars at the start. One possibility
is that they were initially at $(1,1), (1,2), (1,3),$ and $(2,1)$. In the second
photo, the star originally at $(1,1)$ disappeared and the star originally at
$(1,3)$ moved off frame. The other two stars shifted to the right by 1.SCORING:Input 3: $A=B=0$Inputs 4-7: $A=1, B=0, N\le 10$Inputs 8-9: $A=1, B=0$Inputs 10-12: No additional constraints.Problem credits: Suhas Nagar

4
-1
4

In the first case, there were at least $4$ stars at the start. If we let $(r,c)$
denote the intersection of the $r$th row from the top and $c$th column from the
left, one possibility is that they were initially at $(1,1), (3,2), (2,2),$  and
$(1,3)$. All the stars shifted, except for the one at $(2,2)$ which disappeared.In the second case, there is no arrangement of stars in the initial photo that
can produce the middle black pixel given the shift.In the third case, there were at least $4$ stars at the start. One possibility
is that they were initially at $(1,1), (1,2), (1,3),$ and $(2,1)$. In the second
photo, the star originally at $(1,1)$ disappeared and the star originally at
$(1,3)$ moved off frame. The other two stars shifted to the right by 1.SCORING:Input 3: $A=B=0$Inputs 4-7: $A=1, B=0, N\le 10$Inputs 8-9: $A=1, B=0$Inputs 10-12: No additional constraints.Problem credits: Suhas Nagar

In the second case, there is no arrangement of stars in the initial photo that
can produce the middle black pixel given the shift.In the third case, there were at least $4$ stars at the start. One possibility
is that they were initially at $(1,1), (1,2), (1,3),$ and $(2,1)$. In the second
photo, the star originally at $(1,1)$ disappeared and the star originally at
$(1,3)$ moved off frame. The other two stars shifted to the right by 1.SCORING:Input 3: $A=B=0$Inputs 4-7: $A=1, B=0, N\le 10$Inputs 8-9: $A=1, B=0$Inputs 10-12: No additional constraints.Problem credits: Suhas Nagar

In the third case, there were at least $4$ stars at the start. One possibility
is that they were initially at $(1,1), (1,2), (1,3),$ and $(2,1)$. In the second
photo, the star originally at $(1,1)$ disappeared and the star originally at
$(1,3)$ moved off frame. The other two stars shifted to the right by 1.SCORING:Input 3: $A=B=0$Inputs 4-7: $A=1, B=0, N\le 10$Inputs 8-9: $A=1, B=0$Inputs 10-12: No additional constraints.Problem credits: Suhas Nagar

SCORING:Input 3: $A=B=0$Inputs 4-7: $A=1, B=0, N\le 10$Inputs 8-9: $A=1, B=0$Inputs 10-12: No additional constraints.Problem credits: Suhas Nagar
\end{verbatim}

\section*{Solution}


\section*{Problem URL}
https://usaco.org/index.php?page=viewproblem2&cpid=1467

\section*{Source}
USACO JAN25 Contest, Bronze Division, Problem ID: 1467

\end{document}
