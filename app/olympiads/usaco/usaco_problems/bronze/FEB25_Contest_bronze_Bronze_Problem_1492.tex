\documentclass[12pt]{article}
\usepackage{amsmath}
\usepackage{amssymb}
\usepackage{graphicx}
\usepackage{enumitem}
\usepackage{hyperref}
\usepackage{xcolor}

\title{View problem}
\author{USACO Contest: FEB25 Contest - Bronze Division}
\date{\today}

\begin{document}

\maketitle

\section*{Problem Statement}


You are given an array $a$ of $N$ non-negative integers $a_1, a_2, \dots, a_N$
($1\le N\le 2\cdot 10^5, 0\le a_i\le N$). In one operation, you can change any
element of $a$ to any non-negative integer.

The mex of an array is the minimum non-negative integer that it does not
contain. For each $i$ in the range $0$ to $N$ inclusive, compute the minimum
number of operations you need  in order to make the mex of $a$ equal $i$.

INPUT FORMAT (input arrives from the terminal / stdin):
The first line contains $N$.

The next line contains $a_1,a_2,\dots, a_N$.


OUTPUT FORMAT (print output to the terminal / stdout):
For each $i$ in the range $0$ to $N$, output the minimum number of operations
for $i$ on a new line. Note that it is always possible to make the mex of $a$
equal to any $i$ in the range $0$ to $N$.

SAMPLE INPUT:
4
2 2 2 0
SAMPLE OUTPUT: 
1
0
3
1
2

To make the mex of $a$ equal to $0$, we can change $a_4$ to $3$ (or any
positive integer). In the resulting array, $[2, 2, 2, 3]$, $0$ is the smallest
non-negative integer that the array does not contain, so $0$ is the mex of the
array.To make the mex of $a$ equal to $1$, we don't need to make any changes since
$1$ is already the smallest non-negative integer that $a = [2, 2, 2, 0]$ does
not contain.To make the mex of $a$ equal to $2$, we need to change the first three
elements of $a$. For example, we can change $a$ to be $[3, 1, 1, 0]$.
SCORING:
Inputs 2-6: $N\le 10^3$Inputs 7-11: No additional constraints.


Problem credits: Benjamin Qi



\section*{Input Format}
The next line contains $a_1,a_2,\dots, a_N$.

OUTPUT FORMAT (print output to the terminal / stdout):For each $i$ in the range $0$ to $N$, output the minimum number of operations
for $i$ on a new line. Note that it is always possible to make the mex of $a$
equal to any $i$ in the range $0$ to $N$.SAMPLE INPUT:4
2 2 2 0SAMPLE OUTPUT:1
0
3
1
2To make the mex of $a$ equal to $0$, we can change $a_4$ to $3$ (or any
positive integer). In the resulting array, $[2, 2, 2, 3]$, $0$ is the smallest
non-negative integer that the array does not contain, so $0$ is the mex of the
array.To make the mex of $a$ equal to $1$, we don't need to make any changes since
$1$ is already the smallest non-negative integer that $a = [2, 2, 2, 0]$ does
not contain.To make the mex of $a$ equal to $2$, we need to change the first three
elements of $a$. For example, we can change $a$ to be $[3, 1, 1, 0]$.SCORING:Inputs 2-6: $N\le 10^3$Inputs 7-11: No additional constraints.Problem credits: Benjamin Qi

\section*{Output Format}
SAMPLE INPUT:4
2 2 2 0SAMPLE OUTPUT:1
0
3
1
2To make the mex of $a$ equal to $0$, we can change $a_4$ to $3$ (or any
positive integer). In the resulting array, $[2, 2, 2, 3]$, $0$ is the smallest
non-negative integer that the array does not contain, so $0$ is the mex of the
array.To make the mex of $a$ equal to $1$, we don't need to make any changes since
$1$ is already the smallest non-negative integer that $a = [2, 2, 2, 0]$ does
not contain.To make the mex of $a$ equal to $2$, we need to change the first three
elements of $a$. For example, we can change $a$ to be $[3, 1, 1, 0]$.SCORING:Inputs 2-6: $N\le 10^3$Inputs 7-11: No additional constraints.Problem credits: Benjamin Qi

\section*{Sample Input}
\begin{verbatim}
4
2 2 2 0
\end{verbatim}

\section*{Sample Output}
\begin{verbatim}
1
0
3
1
2

To make the mex of $a$ equal to $0$, we can change $a_4$ to $3$ (or any
positive integer). In the resulting array, $[2, 2, 2, 3]$, $0$ is the smallest
non-negative integer that the array does not contain, so $0$ is the mex of the
array.To make the mex of $a$ equal to $1$, we don't need to make any changes since
$1$ is already the smallest non-negative integer that $a = [2, 2, 2, 0]$ does
not contain.To make the mex of $a$ equal to $2$, we need to change the first three
elements of $a$. For example, we can change $a$ to be $[3, 1, 1, 0]$.SCORING:Inputs 2-6: $N\le 10^3$Inputs 7-11: No additional constraints.Problem credits: Benjamin Qi

SCORING:Inputs 2-6: $N\le 10^3$Inputs 7-11: No additional constraints.Problem credits: Benjamin Qi
\end{verbatim}

\section*{Solution}


\section*{Problem URL}
https://usaco.org/index.php?page=viewproblem2&cpid=1492

\section*{Source}
USACO FEB25 Contest, Bronze Division, Problem ID: 1492

\end{document}
