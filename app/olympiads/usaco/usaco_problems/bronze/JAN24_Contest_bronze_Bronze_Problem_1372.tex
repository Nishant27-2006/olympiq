\documentclass[12pt]{article}
\usepackage{amsmath}
\usepackage{amssymb}
\usepackage{graphicx}
\usepackage{enumitem}
\usepackage{hyperref}
\usepackage{xcolor}

\title{View problem}
\author{USACO Contest: JAN24 Contest - Bronze Division}
\date{\today}

\begin{document}

\maketitle

\section*{Problem Statement}


Bessie has mastered the art of turning into a cannonball and bouncing along a
number line of length $N$  $(1 \leq N \leq 10^5)$ with locations numbered
$1,2,\dots,N$ from left to right.  She starts at some integer location $S$
$(1 \leq S \leq N)$ bouncing to the right with a starting power of $1$. If
Bessie has power $k$, her next bounce will be at a distance $k$ forward from her
current location.  

Every integer location from $1$ to $N$ is either a target or a jump pad. Each
target and jump pad has an integer value in the range $0$ to $N$ inclusive. A
jump pad with a value of $v$ increases Bessie's power by $v$ and reverses her
direction.  A target with a value of $v$ will be broken if landed on with a
power of at least $v$.  Landing on a target does not change Bessie's power or
direction.  A target that is broken will remain broken and Bessie can still
bounce on it, also without changing power or direction.

If Bessie bounces for an infinite amount of time or until she leaves the number
line, how many targets will she break?

If Bessie starts on a target that she can break, she will immediately do so.
Similarly, if Bessie starts on a jump pad, the pad's effects will be applied
before her first jump.

INPUT FORMAT (input arrives from the terminal / stdin):
The first line of the input contains $N$ and $S$, where $N$ is the length of the
number line and $S$ is Bessie's starting location.

The next $N$ lines describe each of the locations. The $i$th of these lines
contains integers $q_i$ and $v_i$, where $q_i = 0$ if location $i$ is a jump pad
and  $q_i = 1$ if location $i$ is a target, and where $v_i$ is the value of
location $i$.

OUTPUT FORMAT (print output to the terminal / stdout):
Output one number representing the number of targets that will be broken.

SAMPLE INPUT:
5 2
0 1
1 1
1 2
0 1
1 1
SAMPLE OUTPUT: 
1

Bessie starts at coordinate $2$, which is a target of value $1$, so she
immediately breaks it. She then bounces to coordinate $3$, which is a target of
value $2$, so she can't break it. She continues to coordinate $4$, which
switches her direction and increases her power by $1$ to $2$. She bounces back
to coordinate $2$, which is an already broken target, so she continues. At this
point, she bounces to coordinate $0$, so she stops. She breaks exactly one
target at located at $2$.

SAMPLE INPUT:
6 4
0 3
1 1
1 2
1 1
0 1
1 1
SAMPLE OUTPUT: 
3

The path Bessie takes is $4\to 5\to 3\to 1\to 6$, where the next bounce would
take her out of the number line ($11$). She breaks targets $4, 3, 6$ in that
order.

SCORING:
Inputs 3-5: $N \le 100$ Inputs 6-10: $N \le 1000$ Inputs
11-20: No additional constraints. 


Problem credits: Suhas Nagar



\section*{Input Format}
The next $N$ lines describe each of the locations. The $i$th of these lines
contains integers $q_i$ and $v_i$, where $q_i = 0$ if location $i$ is a jump pad
and  $q_i = 1$ if location $i$ is a target, and where $v_i$ is the value of
location $i$.

OUTPUT FORMAT (print output to the terminal / stdout):Output one number representing the number of targets that will be broken.SAMPLE INPUT:5 2
0 1
1 1
1 2
0 1
1 1SAMPLE OUTPUT:1Bessie starts at coordinate $2$, which is a target of value $1$, so she
immediately breaks it. She then bounces to coordinate $3$, which is a target of
value $2$, so she can't break it. She continues to coordinate $4$, which
switches her direction and increases her power by $1$ to $2$. She bounces back
to coordinate $2$, which is an already broken target, so she continues. At this
point, she bounces to coordinate $0$, so she stops. She breaks exactly one
target at located at $2$.SAMPLE INPUT:6 4
0 3
1 1
1 2
1 1
0 1
1 1SAMPLE OUTPUT:3The path Bessie takes is $4\to 5\to 3\to 1\to 6$, where the next bounce would
take her out of the number line ($11$). She breaks targets $4, 3, 6$ in that
order.SCORING:Inputs 3-5: $N \le 100$Inputs 6-10: $N \le 1000$Inputs
11-20: No additional constraints.Problem credits: Suhas Nagar

\section*{Output Format}
SAMPLE INPUT:5 2
0 1
1 1
1 2
0 1
1 1SAMPLE OUTPUT:1Bessie starts at coordinate $2$, which is a target of value $1$, so she
immediately breaks it. She then bounces to coordinate $3$, which is a target of
value $2$, so she can't break it. She continues to coordinate $4$, which
switches her direction and increases her power by $1$ to $2$. She bounces back
to coordinate $2$, which is an already broken target, so she continues. At this
point, she bounces to coordinate $0$, so she stops. She breaks exactly one
target at located at $2$.SAMPLE INPUT:6 4
0 3
1 1
1 2
1 1
0 1
1 1SAMPLE OUTPUT:3The path Bessie takes is $4\to 5\to 3\to 1\to 6$, where the next bounce would
take her out of the number line ($11$). She breaks targets $4, 3, 6$ in that
order.SCORING:Inputs 3-5: $N \le 100$Inputs 6-10: $N \le 1000$Inputs
11-20: No additional constraints.Problem credits: Suhas Nagar

\section*{Sample Input}
\begin{verbatim}
5 2
0 1
1 1
1 2
0 1
1 1

6 4
0 3
1 1
1 2
1 1
0 1
1 1
\end{verbatim}

\section*{Sample Output}
\begin{verbatim}
1

Bessie starts at coordinate $2$, which is a target of value $1$, so she
immediately breaks it. She then bounces to coordinate $3$, which is a target of
value $2$, so she can't break it. She continues to coordinate $4$, which
switches her direction and increases her power by $1$ to $2$. She bounces back
to coordinate $2$, which is an already broken target, so she continues. At this
point, she bounces to coordinate $0$, so she stops. She breaks exactly one
target at located at $2$.SAMPLE INPUT:6 4
0 3
1 1
1 2
1 1
0 1
1 1SAMPLE OUTPUT:3The path Bessie takes is $4\to 5\to 3\to 1\to 6$, where the next bounce would
take her out of the number line ($11$). She breaks targets $4, 3, 6$ in that
order.SCORING:Inputs 3-5: $N \le 100$Inputs 6-10: $N \le 1000$Inputs
11-20: No additional constraints.Problem credits: Suhas Nagar

SAMPLE INPUT:6 4
0 3
1 1
1 2
1 1
0 1
1 1SAMPLE OUTPUT:3The path Bessie takes is $4\to 5\to 3\to 1\to 6$, where the next bounce would
take her out of the number line ($11$). She breaks targets $4, 3, 6$ in that
order.SCORING:Inputs 3-5: $N \le 100$Inputs 6-10: $N \le 1000$Inputs
11-20: No additional constraints.Problem credits: Suhas Nagar

3

The path Bessie takes is $4\to 5\to 3\to 1\to 6$, where the next bounce would
take her out of the number line ($11$). She breaks targets $4, 3, 6$ in that
order.SCORING:Inputs 3-5: $N \le 100$Inputs 6-10: $N \le 1000$Inputs
11-20: No additional constraints.Problem credits: Suhas Nagar

SCORING:Inputs 3-5: $N \le 100$Inputs 6-10: $N \le 1000$Inputs
11-20: No additional constraints.Problem credits: Suhas Nagar
\end{verbatim}

\section*{Solution}


\section*{Problem URL}
https://usaco.org/index.php?page=viewproblem2&cpid=1372

\section*{Source}
USACO JAN24 Contest, Bronze Division, Problem ID: 1372

\end{document}
