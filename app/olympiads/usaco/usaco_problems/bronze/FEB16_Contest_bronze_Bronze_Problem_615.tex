\documentclass[12pt]{article}
\usepackage{amsmath}
\usepackage{amssymb}
\usepackage{graphicx}
\usepackage{enumitem}
\usepackage{hyperref}
\usepackage{xcolor}

\title{View problem}
\author{USACO Contest: FEB16 Contest - Bronze Division}
\date{\today}

\begin{document}

\maketitle

\section*{Problem Statement}

Farmer John has received an order for exactly $M$ units of milk
($1 \leq M \leq 1,000$) that he needs to fill right away.  Unfortunately, his
fancy milking machine has just become broken, and all he has are three milk
pails of integer sizes $X$, $Y$, and $M$ ($1 \leq X < Y < M$). All three pails
are initially empty.  Using these three pails, he can perform any number of the
following two types of operations:

- He can fill the smallest pail (of size $X$) completely to the top with $X$
units of milk and pour it into the size-$M$ pail, as long as this will not cause
the size-$M$ pail to overflow.

- He can fill the medium-sized pail (of size $Y$) completely to the top with $Y$
units of milk and pour it into the size-$M$ pail, as long as this will not cause
the size-$M$ pail to overflow.

Although FJ realizes he may not be able to completely fill the size-$M$ pail,
please help him determine the maximum amount of milk he can possibly add to this
pail.

INPUT FORMAT (file pails.in):
The first, and only line of input, contains $X$, $Y$, and $M$, separated by spaces.

OUTPUT FORMAT (file pails.out):
Output the maximum amount of milk FJ can possibly add to the size-$M$ pail.

SAMPLE INPUT:
17 25 77
SAMPLE OUTPUT: 
76

In this example, FJ fills the pail of size 17 three times and the pail of size
25 once, accumulating a total of 76 units of milk.

Problem credits: Brian Dean



\section*{Input Format}
OUTPUT FORMAT (file pails.out):Output the maximum amount of milk FJ can possibly add to the size-$M$ pail.SAMPLE INPUT:17 25 77SAMPLE OUTPUT:76In this example, FJ fills the pail of size 17 three times and the pail of size
25 once, accumulating a total of 76 units of milk.Problem credits: Brian Dean

\section*{Output Format}
SAMPLE INPUT:17 25 77SAMPLE OUTPUT:76In this example, FJ fills the pail of size 17 three times and the pail of size
25 once, accumulating a total of 76 units of milk.Problem credits: Brian Dean

\section*{Sample Input}
\begin{verbatim}
17 25 77
\end{verbatim}

\section*{Sample Output}
\begin{verbatim}
76

In this example, FJ fills the pail of size 17 three times and the pail of size
25 once, accumulating a total of 76 units of milk.Problem credits: Brian Dean

Problem credits: Brian Dean
\end{verbatim}

\section*{Solution}


\section*{Problem URL}
https://usaco.org/index.php?page=viewproblem2&cpid=615

\section*{Source}
USACO FEB16 Contest, Bronze Division, Problem ID: 615

\end{document}
