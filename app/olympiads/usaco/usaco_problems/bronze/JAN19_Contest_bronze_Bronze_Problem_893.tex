\documentclass[12pt]{article}
\usepackage{amsmath}
\usepackage{amssymb}
\usepackage{graphicx}
\usepackage{enumitem}
\usepackage{hyperref}
\usepackage{xcolor}

\title{View problem}
\author{USACO Contest: JAN19 Contest - Bronze Division}
\date{\today}

\begin{document}

\maketitle

\section*{Problem Statement}

When bored of playing their usual shell game, Bessie the cow and her friend
Elsie like to play another common game called "guess the animal".

Initially, Bessie thinks of some animal (most of the time, this animal is a cow,
making the game rather boring, but occasionally Bessie is  creative and thinks
of something else).  Then Elsie proceeds to ask a series of questions to figure
out what animal Bessie has selected.  Each question asks whether the animal has
some specific characteristic, and Bessie answers each question with "yes" or
"no". For example:


Elsie: "Does the animal fly?" 
Bessie: "No" 
Elsie: "Does the animal eat grass?" 
Bessie: "Yes" 
Elsie: "Does the animal make milk?"
Bessie: "Yes" 
Elsie: "Does the animal go moo?"
Bessie: "Yes" 
Elsie: "In that case I think the animal is a cow." 
Bessie: "Correct!"

If we call the "feasible set" the set of all animals with characteristics
consistent with Elsie's questions so far, then Elsie keeps asking questions
until the feasible set contains only one animal, after which she announces this
animal as her answer.  In each question, Elsie picks a characteristic of some
animal in the feasible set to ask about (even if this characteristic might not
help her narrow down the feasible set any further).  She never asks about the
same  characteristic twice.

Given all of the animals that Bessie and Elsie know as well as their 
characteristics, please determine the maximum number of "yes" answers  Elsie
could possibly receive before she knows the right animal.

INPUT FORMAT (file guess.in):
The first line of input contains the number of animals, $N$
($2 \leq N \leq 100$).   Each of the next $N$ lines describes an animal.  The
line starts with the animal name,  then an integer $K$ ($1 \leq K \leq 100$),
then $K$ characteristics of that animal. Animal names and characteristics are
strings of up to 20 lowercase characters (a..z).  No two animals have  exactly
the same characteristics.

OUTPUT FORMAT (file guess.out):
Please output the maximum number of "yes" answers Elsie could receive before the
game ends.

SAMPLE INPUT:
4
bird 2 flies eatsworms
cow 4 eatsgrass isawesome makesmilk goesmoo
sheep 1 eatsgrass
goat 2 makesmilk eatsgrass
SAMPLE OUTPUT: 
3

In this example, it is possible for Elsie to generate a transcript with  3 "yes"
answers (the one above), and it is not possible to generate a transcript with
more than 3 "yes" answers.


Problem credits: Brian Dean



\section*{Input Format}
OUTPUT FORMAT (file guess.out):Please output the maximum number of "yes" answers Elsie could receive before the
game ends.SAMPLE INPUT:4
bird 2 flies eatsworms
cow 4 eatsgrass isawesome makesmilk goesmoo
sheep 1 eatsgrass
goat 2 makesmilk eatsgrassSAMPLE OUTPUT:3In this example, it is possible for Elsie to generate a transcript with  3 "yes"
answers (the one above), and it is not possible to generate a transcript with
more than 3 "yes" answers.Problem credits: Brian Dean

\section*{Output Format}
SAMPLE INPUT:4
bird 2 flies eatsworms
cow 4 eatsgrass isawesome makesmilk goesmoo
sheep 1 eatsgrass
goat 2 makesmilk eatsgrassSAMPLE OUTPUT:3In this example, it is possible for Elsie to generate a transcript with  3 "yes"
answers (the one above), and it is not possible to generate a transcript with
more than 3 "yes" answers.Problem credits: Brian Dean

\section*{Sample Input}
\begin{verbatim}
4
bird 2 flies eatsworms
cow 4 eatsgrass isawesome makesmilk goesmoo
sheep 1 eatsgrass
goat 2 makesmilk eatsgrass
\end{verbatim}

\section*{Sample Output}
\begin{verbatim}
3

In this example, it is possible for Elsie to generate a transcript with  3 "yes"
answers (the one above), and it is not possible to generate a transcript with
more than 3 "yes" answers.Problem credits: Brian Dean

Problem credits: Brian Dean

Problem credits: Brian Dean
\end{verbatim}

\section*{Solution}


\section*{Problem URL}
https://usaco.org/index.php?page=viewproblem2&cpid=893

\section*{Source}
USACO JAN19 Contest, Bronze Division, Problem ID: 893

\end{document}
