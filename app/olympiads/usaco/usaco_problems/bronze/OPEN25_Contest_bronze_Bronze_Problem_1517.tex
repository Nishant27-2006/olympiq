\documentclass[12pt]{article}
\usepackage{amsmath}
\usepackage{amssymb}
\usepackage{graphicx}
\usepackage{enumitem}
\usepackage{hyperref}
\usepackage{xcolor}

\title{View problem}
\author{USACO Contest: OPEN25 Contest - Bronze Division}
\date{\today}

\begin{document}

\maketitle

\section*{Problem Statement}


Elsie is trying to describe her favorite USACO contest to Bessie, but Bessie is
having trouble understanding why Elsie likes it so much. Elsie says "And It's
mooin' time! Who wants a mooin'? Please, I just want to do USACO". 

Bessie still doesn't understand, so she transcribes Elsie's description in a
string of length $N$ ($3 \leq N \leq 10^5$)  containing lowercase alphabetic
characters $s_1s_2 \ldots s_N$.  Elsie considers a string $t$ containing three
characters a moo if $t_2 = t_3$ and $t_2 \neq t_1$.

A triplet $(i, j, k)$ is valid if $i < j < k$ and string $s_i s_j s_k$ forms a
moo. For the triplet, FJ performs the following to calculate its value:

 FJ bends string $s$ 90-degrees at index $j$  The value of the
triplet is twice the area of $\Delta ijk$. 
In other words, the value of the triplet is $(j-i)(k-j)$.

Bessie asks you $Q$ ($1 \leq Q \leq 3 \cdot 10^4$) queries. In each query, she
gives you two integers $l$ and $r$ ($1 \leq l \leq r \leq N$, $r-l+1 \ge 3$) and
ask you for the maximum value among valid triplets $(i, j, k)$ such that
$l \leq i$ and $k \leq r$. If no valid triplet exists, output $-1$.  

Note that the large size of integers involved in this problem may require the
use of 64-bit integer data types (e.g., a "long long" in C/C++).
INPUT FORMAT (input arrives from the terminal / stdin):
The first line contains two integers $N$ and $Q$. 

The following line contains $s_1 s_2, \ldots s_N$.

The following $Q$ lines contain two integers $l$ and $r$, denoting each query.

OUTPUT FORMAT (print output to the terminal / stdout):
Output the answer for each query on a new line.

SAMPLE INPUT:
12 5
abcabbacabac
1 12
2 7
4 8
2 5
3 10
SAMPLE OUTPUT: 
28
6
1
-1
12

For the first query, ($i,j,k$) must satisfy $1 \le i < j < k \le 12$. It can be
shown that the maximum area of $\Delta ijk$ for some valid ($i,j,k$) is achieved
when $i=1$, $j=8$, and $k=12$. Note that $s_1 s_8 s_{12}$ is the string "acc"
which is  a moo according to the definitions above. $\Delta ijk$ will have legs
of lengths $7$ and $4$ so two times the area of it will be $28$. 

For the third query, ($i,j,k$) must satisfy $4 \le i < j < k \le 8$. It can be
shown that the maximum area of $\Delta ijk$ for some valid ($i,j,k$) is achieved
when $i=4$, $j=5$, and $k=6$.

For the fourth query, there exists no ($i,j,k$) satisfying
$2 \le i < j < k \le 5$ in which $s_i s_j s_k$ is a moo so the output to that
query is $-1$.
SCORING:
Inputs 2-3: $N,Q\le 50$Inputs 4-6: $Q=1$ and the singular query
satisfies $l=1$ and $r=N$Inputs 7-11: No additional constraints


Problem credits: Chongtian Ma



\section*{Input Format}
The following line contains $s_1 s_2, \ldots s_N$.The following $Q$ lines contain two integers $l$ and $r$, denoting each query.

The following $Q$ lines contain two integers $l$ and $r$, denoting each query.

\section*{Output Format}


\section*{Sample Input}
\begin{verbatim}
12 5
abcabbacabac
1 12
2 7
4 8
2 5
3 10
\end{verbatim}

\section*{Sample Output}
\begin{verbatim}
28
6
1
-1
12

For the third query, ($i,j,k$) must satisfy $4 \le i < j < k \le 8$. It can be
shown that the maximum area of $\Delta ijk$ for some valid ($i,j,k$) is achieved
when $i=4$, $j=5$, and $k=6$.For the fourth query, there exists no ($i,j,k$) satisfying
$2 \le i < j < k \le 5$ in which $s_i s_j s_k$ is a moo so the output to that
query is $-1$.SCORING:Inputs 2-3: $N,Q\le 50$Inputs 4-6: $Q=1$ and the singular query
satisfies $l=1$ and $r=N$Inputs 7-11: No additional constraintsProblem credits: Chongtian Ma

For the fourth query, there exists no ($i,j,k$) satisfying
$2 \le i < j < k \le 5$ in which $s_i s_j s_k$ is a moo so the output to that
query is $-1$.SCORING:Inputs 2-3: $N,Q\le 50$Inputs 4-6: $Q=1$ and the singular query
satisfies $l=1$ and $r=N$Inputs 7-11: No additional constraintsProblem credits: Chongtian Ma
\end{verbatim}

\section*{Solution}


\section*{Problem URL}
https://usaco.org/index.php?page=viewproblem2&cpid=1517

\section*{Source}
USACO OPEN25 Contest, Bronze Division, Problem ID: 1517

\end{document}
