\documentclass[12pt]{article}
\usepackage{amsmath}
\usepackage{amssymb}
\usepackage{graphicx}
\usepackage{enumitem}
\usepackage{hyperref}
\usepackage{xcolor}

\title{View problem}
\author{USACO Contest: JAN16 Contest - Bronze Division}
\date{\today}

\begin{document}

\maketitle

\section*{Problem Statement}

Bessie the cow has designed what she thinks will be the next big hit video game:
"Angry Cows".  The premise, which she believes is completely original, is that
the player shoots a cow with a slingshot  into a one-dimensional scene
consisting of a set of hay bales located at various points on a number line; the
cow lands on a hay bale with sufficient force to cause the bale to explode,
which in turn might set of a chain reaction that causes  additional nearby hay
bales to explode.  The goal is to use a single cow to start a chain reaction
that detonates as many hay bales as possible.

There are $N$ hay bales located at distinct integer positions
$x_1, x_2, \ldots, x_N$ on the number line.  If a cow is launched onto a hay
bale at position $x$, this hay bale explodes with a "blast radius" of 1, meaning
that any other hay bales within 1 unit of distance are also engulfed by the
explosion. These neighboring bales then themselves explode (all simultaneously),
each with a blast radius of 2, so these explosions may engulf additional
yet-unexploded bales up to 2 units of distance away.  In the next time step,
these bales also  explode (all simultaneously) with blast radius 3.  In general,
at time $t$ a  set of hay bales will explode, each with blast radius $t$.  Bales
engulfed by these explosions will themselves explode at time $t+1$ with blast
radius $t+1$, and so on.

Please determine the maximum number of hay bales that can explode if a single
cow is launched onto the best possible hay bale to start a chain reaction.

INPUT FORMAT (file angry.in):
The first line of input contains $N$ ($1 \leq N \leq 100$).  The remaining $N$
lines all  contain integers $x_1 \ldots x_N$ (each in the range
$0 \ldots 1,000,000,000$).

OUTPUT FORMAT (file angry.out):
Please output the maximum number of hay bales that a single cow can cause to
explode.

SAMPLE INPUT:
6
8
5
6
13
3
4
SAMPLE OUTPUT: 
5

In this example, launching a cow onto the hay bale at position 5 will cause the
bales at positions 4 and 6 to explode, each with blast radius 2.  These
explosions in turn cause the bales at positions 3 and 8 to explode, each with
blast radius 3.   However, these final explosions are not strong enough to reach
the bale at position 13.

Problem credits: Brian Dean



\section*{Input Format}
OUTPUT FORMAT (file angry.out):Please output the maximum number of hay bales that a single cow can cause to
explode.SAMPLE INPUT:6
8
5
6
13
3
4SAMPLE OUTPUT:5In this example, launching a cow onto the hay bale at position 5 will cause the
bales at positions 4 and 6 to explode, each with blast radius 2.  These
explosions in turn cause the bales at positions 3 and 8 to explode, each with
blast radius 3.   However, these final explosions are not strong enough to reach
the bale at position 13.Problem credits: Brian Dean

\section*{Output Format}
SAMPLE INPUT:6
8
5
6
13
3
4SAMPLE OUTPUT:5In this example, launching a cow onto the hay bale at position 5 will cause the
bales at positions 4 and 6 to explode, each with blast radius 2.  These
explosions in turn cause the bales at positions 3 and 8 to explode, each with
blast radius 3.   However, these final explosions are not strong enough to reach
the bale at position 13.Problem credits: Brian Dean

\section*{Sample Input}
\begin{verbatim}
6
8
5
6
13
3
4
\end{verbatim}

\section*{Sample Output}
\begin{verbatim}
5

In this example, launching a cow onto the hay bale at position 5 will cause the
bales at positions 4 and 6 to explode, each with blast radius 2.  These
explosions in turn cause the bales at positions 3 and 8 to explode, each with
blast radius 3.   However, these final explosions are not strong enough to reach
the bale at position 13.Problem credits: Brian Dean

Problem credits: Brian Dean
\end{verbatim}

\section*{Solution}


\section*{Problem URL}
https://usaco.org/index.php?page=viewproblem2&cpid=592

\section*{Source}
USACO JAN16 Contest, Bronze Division, Problem ID: 592

\end{document}
